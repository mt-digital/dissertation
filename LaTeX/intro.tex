% Author: Matthew Turner

\documentclass[12pt,letterpaper]{article}
% \documentclass[11pt]{report}
% \documentclass{report}
% \documentclass{book}
\usepackage[bookmarks,hidelinks]{hyperref}
\usepackage{amssymb,amsmath}
% \usepackage{fullpage}
\usepackage{tabulary}
\usepackage{tabularx}
\usepackage{float}
\usepackage[margin=1.00in]{geometry}
% \usepackage[margin=0.90in]{geometry}

\usepackage{caption}
\usepackage{booktabs}
\usepackage{pslatex}
\usepackage{apacite}
\usepackage{subcaption}
\usepackage{pgfplots}
\usepackage{wrapfig}
\usepackage[english]{babel}
\usepackage{lmodern}
\usepackage{setspace}
\doublespace
% \usepackage{url}
\usepackage{bigfoot}
\usepackage[export]{adjustbox}
\setlength\intextsep{0pt}

\usepackage{graphicx}

\title{Introduction to the dissertation}

% \author{{Matthew A.~Turner}}
\date{}

\begin{document}
% \vspace{-4in}
% \maketitle

\vspace{-2in}
\textbf{Communicative, cognitive, and social factors in
extremism and polarization}
\begin{abstract}
  Rising extremism and polarization threaten democratic institutions worldwide.
  As opposing factions become more extreme in their opinions, polarization
  widens the chasm between fellow citizens, and common ground erodes, washed
  away down a river of vitriol, bitterness, and hate. What causes increased
  extremism and polarization? Due to the highly complex nature of human
  societies, this problem of explaining polarization must be broken down into
  many sub-problems, which themselves require complex systems thinking to
  address. Only through ``walling off'' smaller components of social systems,
  and rigorously modeling and analyzing empirical data,
  can we build a thorough, coherent understanding of social behavior.

  In this dissertation I present my findings from studying three
  sub-problems in explaining why and how extremism and polarization emerge.
  First, I focus narrowly on a communication strategy shown in behavioral
  studies to increase extremism, \emph{metaphorical violence}, such as 
  ``Biden hit Trump over his tax returns in yesterday's debate.'' While we know
  the effects of violence metaphors, we do not understand their distribution
  in the wild, or what causes their usage to increase and decrease. I found
  that metaphorical violence use increased around the time of presidential
  debates and elections in the United States, and was correlated with 
  presidentical candidates' tweets. 

  Second, I show that rising extremism
  in isolated social gropus may be simply explained by the fact that
  extremists are more stubborn than centrists---however existing data on the
  subject is unavailable and behavioral studies on the subject may
  contain ubiquitous false detections of rising extremism. 

  Finally, I developed
  and analyzed an empirically-motivated, network-theoretic, agent-based model of 
  social influence at the societal level to understand how well we can 
  predict polarization, and the effect of initial conditions, network structure,
  communication noise, and random chance on predictions of polarization.

  Taken together these studies advance our understanding of communicative,
  cognitive, and social factors in the emergence of extremism and polarization.
\end{abstract}


This dissertation takes on the widely studied topic of social polarization. 
Polarization is a phenomenon where opposing social groups become more extreme
in opposite directions~\cite{Bramson2016,Bramson2017,Jung2019,Klein2020}. 
Although polarization has been widely studied by a diversity of scientists,
the topic is scientifically unwieldy because human societies are the things
that become polarized, and human societies are highly complex systems. As the
saying goes, the human brain-body is the single most complex thing we know of.
How much more complex are thousands, millions, or billions of interacting 
humans?

It is necessary to ``wall off'' small, specific subsystems within human
societies in order to make the study of polarization (or any complex social
phenomena) tractable~\cite{Cartwright1999}.
This process requires us to simultaneously specify and generalize to develop models that
are simple enough to be interpreted but specific enough to be understood
mechanically~\cite{Craver2006,Wimsatt2007,Smaldino2017,Turner2021}. In the end
we develop studies that are very specific to the phenomena of interest
in carefully controlled conditions, despite the unfortunate common practice of
overgeneralizing results in the behavioral and social sciences~\cite{Yarkoni2021}.
When we develop models of social subsystems, we assume certain factors to be
important while ignoring others that may actually influence real world systems
of interest. This is necessary to understand how certain, but in the long run
it is also necessary to investigate the effect of other important factors
theoretically and empirically.

(BEGIN ~5 PARAGRAPH PRELIMINARY OVERVIEW---ONE BROAD OVERVIEW/INTRO AND 
ONE FOR EACH CHAPTER)
In this dissertation I want to explain three studies that walled off different
components of human society to understand various communicative, cognitive, and
social factors in the emergence of polarization. (EXPLAIN HOW WE THEORETICALLY
WALL OFF FOR EACH OF THE STUDIES I DID)

Metaphors are theoretically important for cognitive science and practically
important for understanding and predicting the consequences of political
speech. Metaphor is ubiquitous in political discourse~\cite{Burnes2011,Charteris-Black2009,Charteris-Black2005,Charteris-Black2004,Lakoff1996,Lakoff2008}.

To complement our 30,000 foot view of how mass media affects public opinion,
win the second study I focus on explaining and predicting \emph{group polarization},
which refers to the tendency of small groups to become more extreme in their
opinions on some topic or in their decision making. For example, a jury that
has to decide damages for a civil liability lawsuit might increase the amount
they are willing to award a victim of some wrongdoing~\cite{Schkade2000}.

Unfortunately many or most of the empirical detections of group polarization
are false positives, i.e., Type I errors or just ``false detections''. 

These focused studies where we narrowly walled off social subsystems do not
address whether and when polarization can be reliably predicted at the
societal scale. In my fourth and final study, I answered these questions
based on general factors that affect predictions of political polarization, 
including initial extremism and polarization in society, how reliably individuals
can communicate, and social network structure. 


\section{Rising extremism and polarization}

The basics.

\citeA{Rollwage2019} among many others I will add to here.


\subsection{How real is polarization, and who is becoming polarized?}

Some believe polarization is actually a myth, and increasingly 
extreme ideological commitments arise among a minority of the population,
with a majority independently aligned either out of an honest belief in 
evaluating individuals and not parties, or who are casual political
observers who begin to pay attention only around election time~\cite{Kinder2017}.


\section{Language and communication in polarization}



\section{Cognitive and social factors in polarization}


\subsection{Group polarization: socially isolated groups become more extreme over time}


\subsection{A neurobiological view on group dynamics}


\section{Overview}

\subsection{Violence metaphors foster extremism and polarization}


\subsection{Opportunities and limits for predicting polarization}

\section{Plan}

So far we have introduced the complex problem of polarization, 

\bibliographystyle{apacite}

\setlength{\bibleftmargin}{.125in}
\setlength{\bibindent}{-\bibleftmargin}

\bibliography{/Users/mt/workspace/papers/library.bib}

\end{document}
