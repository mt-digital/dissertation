% Author: Matthew Turner

\documentclass[12pt,letterpaper]{article}
% \documentclass[11pt]{report}
% \documentclass{report}
% \documentclass{book}
\usepackage[bookmarks,hidelinks]{hyperref}
\usepackage{amssymb,amsmath}
% \usepackage{fullpage}
\usepackage{tabulary}
\usepackage{tabularx}
\usepackage{float}
\usepackage[margin=1.00in]{geometry}
% \usepackage[margin=0.90in]{geometry}

\usepackage{datetime2}
\usepackage{caption}
\usepackage{booktabs}
\usepackage{pslatex}
\usepackage{apacite}
\usepackage{subcaption}
\usepackage{pgfplots}
\usepackage{wrapfig}
\usepackage[english]{babel}
\usepackage{lmodern}
\usepackage{setspace}
\doublespace
% \usepackage{url}
\usepackage{bigfoot}
\usepackage[export]{adjustbox}
\setlength\intextsep{0pt}

\usepackage{graphicx}

\title{Introduction to the dissertation}

% \author{{Matthew A.~Turner}}
\date{}

\begin{document}
% \vspace{-4in}
% \maketitle

\vspace{-2.5in}
\textbf{Four studies of communicative, cognitive, and social factors in
extremism and polarization} 
\begin{abstract}
  Rising extremism and polarization threaten democratic institutions worldwide.
  As opposing factions become more extreme in their opinions, polarization
  widens the chasm between fellow citizens, and common ground erodes, washed
  away down a river of vitriol, bitterness, and hate. What causes increased
  extremism and polarization? Due to the highly complex nature of human
  societies, this problem of explaining polarization must be broken down into
  many sub-problems, which themselves require complex systems thinking to
  address. Only through ``walling off'' smaller components of social systems,
  and rigorously modeling and analyzing empirical data,
  can we build a thorough, coherent understanding of social behavior.

  In this dissertation I present my findings from studying three
  sub-problems in explaining why and how extremism and polarization emerge.
  First, I focus narrowly on a communication strategy shown in behavioral
  studies to increase extremism, \emph{metaphorical violence}, such as 
  ``Biden hit Trump over his tax returns in yesterday's debate.'' While we know
  the effects of violence metaphors, we do not understand their distribution
  in the wild, or what causes their usage to increase and decrease. I found
  that metaphorical violence use increased around the time of presidential
  debates and elections in the United States, and was correlated with 
  presidentical candidates' tweets. 

  Second, I show that rising extremism
  in isolated social gropus may be simply explained by the fact that
  extremists are more stubborn than centrists---however existing data on the
  subject is unavailable and behavioral studies on the subject may
  contain ubiquitous false detections of rising extremism. 

  Finally, I developed
  and analyzed an empirically-motivated, network-theoretic, agent-based model of 
  social influence at the societal level to understand how well we can 
  predict polarization, and the effect of initial conditions, network structure,
  communication noise, and random chance on predictions of polarization.

  Taken together these studies advance our understanding of communicative,
  cognitive, and social factors in the emergence of extremism and polarization.
\end{abstract}

Compiled \today

\section{Introduction}

\begin{itemize}
  \item 
    High levels of political polarization seem to bring about or go along with
    hardening of partisan identites~\cite{Lee2015};
    spilling over of political disagreement into collaborative social behavior
    more generally~\cite{Iyengar2019}, and even make violent responses to verbal 
    communication more likely~\cite{Kalmoe2014,Kalmoe2018,Mason2018UncivilAgreementBook}.
    Why does extremism and polarization increase and decrease? These simple
    questions bear no simple answers. These questions
    have been studied for some decades now by researchers across perhaps a dozen
    diverse disciplines and sub-disciplines including political science, 
    sociology, economics, and cognitive science~\cite{Dixit2007,Rollwage2019}. 

  \item
    In this dissertation I focus
    on three more specific questions about increasing extremism and polarization.
    These questions help us understand rising extremism and polarization, and
    predict what situations will foster rising extremism and polarization.
    First, what is the prevalence of communication strategies on mass media
    known to increase extremism, specifically the use of \emph{violence metaphors}
    to describe non-violent political events? Second, what causes observed
    increases in extremism among ideologically similar groups over time?
    Third, and finally, what are some fundamental cognitive capacities and
    social factors that are required for polarization to occur, and what
    role do random chance and miscommunication play in the emergence of
    polarization? These three questions are answered by Study 1, Study 2, and
    Study 4, respectively. In the course of this work, I identified a statistical
    problem that undermines possibly many or most published results on rising
    extremism among ideologically biased groups, demonstrated in Study 3.
    The Studies I present in this dissertation help delineate and explore points of contact between
    the various disciplinary approaches to studying polarization. As such, they
    required diverse theoretical, modeling, and computational methods, 
    including structured corpus building and analysis for studying metaphor
    use on cable TV news, statistical modeling of the opinion generation and
    measurement process, and agent-based modeling of social influence processes.

  \item
    To answer these research questions, this dissertation focuses on the communicative, cognitive,
    and social factors that provide the human substrate for rising extremism
    and polarization. (INTRO THE STUDIES' SCOPES IN ONE PARAGRAPH---1-2 SENTENCES ON
    ALTERNATIVE APPROACHES, E.G., \cite{Zmigrod2018}, \cite{Mason2018}, \cite{Iyengar2019},
    \cite{Wood2017}, \cite{Dixit2007}, \cite{Jern2014,OConnor2018})

  \item
    INTRO METAPHOR AND COMMUNICATION FACTORS

  \item
    INTRO COGNITIVE FACTORS

  \item
    INTRO SOCIAL FACTORS

  \item
    STUDY 1 SUMMARY

  \item
    STUDY 2 SUMMARY

  \item
    STUDY 3 SUMMARY

  \item
    STUDY 4 SUMMARY

  \item
    PLAN FOR THE PAPER

\end{itemize}


\section{Rising extremism and polarization}

Among the popular press and politically-involved citizens, it seems obvious
that polarization is increasing, and that this increase is a 
dangerous problem that needs to be solved~\cite{Klein2020}. 
For example, when in 2014 the Pew Research Center found polarization
in 2014 to be the highest in decades, journalist and Vox founder Ezra Klein 
(2014) found the statement obvious---``(E)veryone already knew that,'' Klein
wrote. However, whether or not polarization occurs depends on how polarization 
is defined and which people are being studied. 
In fact, there is a debate among political scientists whether
polarization really occurs, but this seems to be a matter of definition~\cite{Mason2015,Lelkes2016,Kinder2017}.
Under some senses of the word, polarization can stay constant even when extremism increases (or vice-versa);
in the sense of polarization studied in this dissertation,
when one or both of two opposed group's opinions becomes more
extreme on average, polarization increases by definition. Extremism must also
be defined to study it scientifically. Some think of extremism in
terms of real world actions, such as political violence; others, including
myself in this dissertation, think of extremism in terms of opinions. This complication of defining
extremism and polarization hint at the complexity of humans, and their groups
and societies, Such complexity requires multidisciplinary, multimethod
scientific approaches to understand. Disciplinary results must then
be examined to understand how results from different disciplines and sub-disciplines, obtained through
diverse scientific methodologies, may either be understood as complementary or 
possibly flagged as discrepent, requiring further 
investigation~\cite{Cartwright1999,Brewer2013}.

At worst, extremism and political
polarization leads to political violence and even civil war~\cite{Epstein2013,Freeman2018}.
This seems true no matter what definition of extremism and polarization one
uses. However, different types of extremism and polarization may have different
effects on society and governance~\cite{Lelkes2016}. \citeA{Lee2015}, for example,
found that although partisan sorting had occurred in recent decades, there
had been little degradation in legislative and other government outcomes.
In this dissertation, \emph{extremism} is defined by how extreme one's opinion is on some opinion 
scale, which represents how intensely or confidently someone believes in their
own opinion on some topic. \emph{Polarization} in this dissertation is 
conceptualized as the bimodality in opinions among a population, ignoring 
whether individuals are members of political parties or not. Opinion bimodality
can quantify ideological divergence among all members of society~\cite{Bramson2016,Lelkes2016}. 

Alternatives to measuring polarization as bimodality of opinion distributions
start by assuming the existence of political parties and ideologies---the 
Republican and Democratic parties, and conservative and liberal ideologies.
Then, polarization can be measured by the degree of sorting of ideologies
into political parties~\cite{Mason2015}. 
In the United States, this means polarization is measured
as the combined degree to which liberals are also Democrats, and to which 
conservatives are also Republicans. Polarization is certainly on the rise,
but this may be a correction to normal from several previous decades of 
unnaturally low levels of political sorting~\cite{Lee2015,Wood2017b}.
What may be most concerning in modern developments is the rise of
\emph{affective polarization} between political groups has been found to increase as non-political
preferences align among partisans as well---for example, preferences for
leisure activities and entertainment are becoming increasingly correlated
with ideology and party membership in the United States~\cite{Pew2014,DellaPosta2015}.
Affective polarization includes the increasing dislike and distrust between opposing political parties, 
which spills over into non-political areas of life~\cite{Iyengar2019}. 
While we focus on extremism and polarization in the United States, similar trends and concerns can be observed 
worldwide~\cite{Borge-Holthoefer2015,Morales2015,Romenskyy2017,Zmigrod2018}. 
Comparative study is necessary to understand the fundamental human factors underlying
extremism and polarization in societies that are not Western or democratic,
with possibly lower standards of living~\cite{Henrich2010}.

Among all the ways in which individual preferences and opinions are sorted and polarized, a
major one one that both reflects and drives rising polarization is the split in
where partisans get their news~\cite{Pew2014PolarizationAndMediaHabits,Martin2017}. 
What is said on cable TV news and other mass media is extremely important, given
the reach of mass media and the way mass media frames the terms of debate~\cite{Chong2007}.
One important communication strategy is the use of different 
metaphors to frame different political
messages, processes, and events. These framings influence the way politics is understood by 
news consumers. One's opinions about immigrants, for example, may depend
on whether one has been exposed to metahpors relating immigrants to 
``indigestible food, conquering hordes,'' or ``waste materials''~\cite{OBrien2003}.
Those who had been exposed to such metaphors may later tend to favor stricter limits
on immigration and harsher treatment for, e.g., undocumented immigrants.
In Study 1, I study the change in frequency over time of a 
specific type of metaphor use, violence metaphors, across cable news channels
MSNBC, CNN, and Fox News, around the time of the United States presidential
debates and elections. Violence metaphors are important because they have
been observed to push individuals to more extreme political opinions, even
increasing support for real world political violence~\cite{Kalmoe2014,Kalmoe2018}.
Metaphorical violence is a prime strategy for inflaming partisan passions 
through statements such as ``Trump has been getting \emph{attacked} by the liberal
democrats on Capital Hill,'' which one might hear by a commentator or anchor on Fox News.

While mass media are highly influential in framing the terms of political
discourse, interpersonal influence between citizens is at least equally important in
shaping political opinions and behavior~\cite{Katz1955}. But how does 
interpersonal social influence work, and how do we know which social 
influence processes are essential for rising extremism and polarization to
emerge? We break down interpersonal influence into a verbal, formal, and computational
model and see if polarization emerges from that model, without assuming anything
about polarization itself. Interpersonal influence of opinions can be broken down into
four important cognitive components: (1) attractive and repulsive influence of
opinions~\cite{French1956,Cikara2014,Bail2018}; (2) homophily, i.e., preferential assortment with like others~\cite{McPherson2001};
(3) biased assimilation, i.e., heightened influence by similar others~\cite{Dandekar2013}; and
(4) a correlation between stubbornness and extremity of opinions~\cite{Reiss2019,Zmigrod2019a}. 
Other cognitive components may include, e.g., personality traits that may be predictive of 
ideological or other opinion, attitude, belief, etc., preference~\cite{Zmigrod2018}.
Social factors that modulate social influence processes are: (1) social
networks, theoretical entities that represent a person's social relationships
that structure who in a society interacts, when; and (2) the stochasticity
of interpersonal influence, e.g., three people may frequent a certain bar
and talk regularly on Fridays, but who attends varies depending on essentially
random factors like other obligations or obstacles to attending. I show in
Study 2 that by incorporating these cognitive and social factors into a computational
model results in the emergence of ``group polarization'', the empirical 
observation that socially isolated, initially biased groups tend to become
more extreme in their opinions over time. In Study 4 I examine critical
tipping points of the model that might guarantee polarization or consensus and
explore the limits of using this model (or any model) for predicting 
rising extremism and polarization.

Like the field of psychology generally, understanding and predicting extremism
and polarization requires cross-disciplinary understanding that sometimes
involves scientists working focusing on one system component and sometimes has 
scientists exploring the interfaces between system components~\cite{Brewer2013,Rollwage2019}.
Taken as a whole, this dissertation explores the interface between social
system components that, when combined, produce the many sorts of polarization
observed these days.  I conceptualize polarization as an emergent property or phenomenon of society. By assuming
extremely general things about how social influence works in society, it is
possible to encapsulate many of the dimensions along which individuals are
separated and how social influence regarding opinions (or beliefs, attitudes, etc.) on one dimension
is correlated with social influence along other dimensions. We can add 
details to such a general model as necessary to understand, for example, 
how framing strategies change over time on cable TV news (Study 1) or how
extremism rises in initially biased, socially isolated groups (Study 2). 
In Study 4, I use a general model of social influence to 
investigate the role of social network structure, initial extremism/polarization,
communication noise, and random ``path dependence'' on the timing of 
interpersonal interactions on the emergence of polarization.

In the remainder of this chapter I will develop some theoretical and modeling
tools I used in my dissertation Studies to understand different components
of the complex social system that gives rise to extremism and polarization.
First, I will introduce metaphor and conceptual framing more generally, and
their role in social influence, extremism, and polarization. Next, I will
introduce a framework of cognitive and social factors that, when present,
provide the social substrate for the emergence of rising extremism and polarization.
Next I will discuss how I think about empirically-motivated models of human 
behavior and their application to understanding emergent social phenomena
such as extremism and polarization. I will close this introductory chapter with
a review of the four studies of communicative, cognitive, and social factors
in rising extremism and polarization that I present in this dissertation.


\section{Metaphor and framing in politics and polarization}

In the first study of this dissertation, I analyze violence metaphor use on
cable news around the times of the 2012 and 2016 United States presidential
debates and elections. But what is metaphor? How is it used and what are the
effects of metaphor use in political communication, especially regarding
political polarization? Metaphor has long been recognized as an important 
element in the study of political 
communication at least since Aristotle's time if not before, 
who saw metaphor as a special 
feature of especially talented orators' rhetoric~\cite{Aristotle1965,Kirby1997}. Our
modern cognitive understanding of metaphor, in contrast, recognizes the
ubiquity of metaphor as a critical cognitive tool evolved for 
conceptual scaffolding used for abstract thought.  
People gain intuition about abstract concepts through
\emph{embodied metaphors}~\cite{Gibbs2006,Gibbs2010}, which map intangible
concepts such as relationships and the order of events in time to sensorimotor
experience such as navigation and directionality. 
People have been observed to use embodied metaphors for conceptualizing
\emph{navigating} the Internet which amounts to typing or clicking links and buttons
in the real world~\cite{Matlock2014};
for the passage of time~\cite{Matlock2005,Nunez2012,Flusberg2017a}; 
mathematics~\cite{Lakoff1997,Marghetis2013}, 
and, the focus of this dissertation, politics~\cite{Lakoff2008,Charteris-Black2009}.
Metaphor is one of several forms of linguistic \emph{framing} that can 
powerfully influence our understanding of real world 
events through strategic, pragmatic choice of 
words~\cite{Fillmore1982,Chong2007,Fausey2011,Matlock2012,Sagi2013a,Cacciatore2016}.

Metaphor is a cognitive tool to understand abstract concepts in terms of more
concrete, physically-experienced, or \emph{embodied}, 
concepts~\cite{Lakoff1980,Gibbs2006,Gibbs2008,Gibbs2010}. The word metaphor
comes from the ancient Greek word \emph{metaphora} 
($\mu \epsilon \tau \alpha \phi o \rho \acute{\alpha}$), meaning
\emph{transferrence}. Metaphor works by ``transferring'', or mapping in
the mathematical sense, conceptual entailments
from a more concrete concept, such as a fight, onto a more abstract concept,
such as politics~\cite{Regier1996,Kovecses2010a,Lakoff2014}. 
Politics is an abstract concept because it can describe many
different situations, events, and processes. One never truly, directly sees or feels
politics, only indirectly are the outcomes of political decision making felt. 
On the other hand, either being engaged in or observing physical conflict
results in a cascade of bodily effects, including body-to-body contact and
possibly injury for fight participants. Thus the ``conceptual entailments'' in
the generic ``conceptual metaphor'' \textsc{politics is a fight} include the
fact that politics generates similar feelings to being in a fight, including
the physical sensations of elation or depression following a political win
or a political defeat, and the adrenaline (and other biophysical repsonses)
of the fight itself~\cite{Gallese2005,David2016}. 

Politicians and commentators have long used metaphor, in many cases to motivate
supporters and villify opponents. To take a current example, Fox News has
recently been covering what they call ``Classroom Warfare'' around the issue
of Critical Race Theory~\footnote{Humorously covered by The Daily Show here:}
Calling this a war first inflames passions. But if we want to understand the
entailments of this metaphorical framing, we have to consider what else goes
along with the \textsc{war} conceptual frame. Wars have at least two combatants,
literally mortal enemies. 
\begin{itemize}
  \item
    Review of results on metaphor in politics and mass media~\cite{Lakoff1991,Charteris-Black2009,Burnes2011,Flusberg2017,Flusberg2017a,Flusberg2018,Burgers2019}
  \item
    Metaphor and framing effects in extremism and polarization~\cite{Flusberg2017,Flusberg2018,Kalmoe2014,Kalmoe2018}
\end{itemize}

% In this section, I zoom out to discuss metaphor's
% importance in cognitive science and to discuss how metaphor is one of several
% methods studied by communication and political scientists as a method of
% \emph{framing} more generally~\cite{Fillmore1982}---though use of that 
% word may be misleading due to being used too freely for too many 
% diverse situations~\cite{Cacciatore2016}.

% Metaphor is important in cognitive science because metaphor enables people 
% to scaffold their understanding of abstract concepts in terms of more concrete
% concepts~\cite{Lakoff1980,Gibbs2011,Gibbs2012a,Kovecses2017}. 

% Violence metaphors, which I focus on in this disseration, 
% are particularly important, especially as regards political
% polarization because they have been shown 
% to siginificantly affect political opinions and influence the likelihood of
% real world violent behavior. 

% Violence metaphors are just one of a great number of metaphorical constructions
% \cite{Goldberg2003,Goldberg2006,Dodge2015}. 
% In politics, for example, another important class of metaphors is
% metaphors for refugees and immigrants. It changes our behavior towards 
% immigrants if we cast them in a negative light as something destructive, 
% such as an infestation or a flood, compared to finding more positive 
% metaphors~\cite{OBrien2003,Cisneros2008}.

% Metaphor is a more general instance of a way to frame a particular topic. 
% Linguistic frames are ``instantiated'' based on words being used to describe the
% topic, along with the emotional valence and other factors of the words. To take
% a general example consider the Instagram post from Comedy Central's 
% \emph{The Daily Show} host Trevor Noah identifying how
% framing can drastically change the way ideological presented in Figure~\ref{fig:TrevorNoahFraming}. In this example, an anchor from the right-wing outlet NewsMax
% reports that a high school yearbook was banned because it contained a feature
% on Black Lives Matter, but did not contain any features on ``opposing views
% like Blue Lives Matter and All Lives Matter''. Trevor Noah notes this is a 
% false dichotomy---the opposite of Black Lives Matter is Black Lives \emph{do not}
% matter. This example also serves as an example of how media strategies can 
% foster either consensus or polarized opinions in society. By definition, if
% those advocating for policing reforms to stop police killing of black people
% are understood to be \emph{against} police officers themselves and policing
% in general, then one will have to choose one of these polar opposites or 
% another. Consensus becomes more likely when false dichotomies such as this one
% are abandoned for a more nuanced, accurate representation of reality.

% \begin{figure}
%   \centering
%     \includegraphics[width=0.5\textwidth]{Figures/TrevorNoahFraming.png}
%   \caption{Political and social issues look different depending on what language
% is used to describe them. Furthermore, the way a person frames a certain issue
% can be used to infer their opinions or beliefs regarding political and social
% topics. Here, the NewsMax anchor reveals he and his organization believe 
% Black Lives Matter stands in opposition to other movements, groups, or 
% slogans such as Blue Lives Matter or All Lives Matter. Noah identifies the
% illogic of the NewsMax anchor's statement, revealing a misleading dichotomy.
% Subtle framing strategies such as this one build on one another to develop
% the sorts of polarization widely observed to day, in combination with other
% factors.}
% \label{fig:TrevorNoahFraming}
% \end{figure}

% In addition to the sort of conceptual blending and analogy-making that
% characterizes metaphorical constructions, grammatical variations
% and gesture or other multi-modal communication 
% can result in different responses~\cite{Bennett2008,Matlock2012}. 
% \citeA{Fausey2011} found that a fictional politician's personal failings and
% corruption were judged more harshly if it was described using the past 
% imperfect tense (i.e. \emph{VERB} + ing) compared to the past perfect tense
% (\emph{VERB} + ed), since the imperfect implies such problems are ongoing.
% (READ BENNETT AND ADD)



\section{Cognitive and social factors in polarization and other emergent social phenomena}

Extremism and polarization are emergent phenomena of social systems composed
of individuals. Human beings are the most fundamental components in social
systems, which we theoretically assume to have a small set of relevant
capacities essential for social interaction and 
influence~\cite{Cartwright1989,Smaldino2017}. 
Human beings are complex systems themselves,
composed of simpler cells properly organized to have the capacity for
social influence of and by others~\cite{Kello2007,Spivey2020}. Understanding
the capacities for social interaction and influence requires multi-modal
investigation spanning several disciplines undertaking computational,
behavioral, and neurobiological studies. Also important for understanding
the emergence of extremism and polarization are social factors, such as
understanding the importance of relationship networks on emergent
social phenomena. While communication is essential to increasing extremism and
polarization, much can be explained when we study cognitive and social factors
indpenedntly of specific communications.



\subsection{Cognitive factors in polarization}

Cognitive factors in polarization I focus on here are some essential
individual- and dyad-level capacities and processes (or ``activities'' in
\citeA{Machamer2000}) that
enable and lead to social influence of one individual by others. I review
these essential capacities and processes now. The first essential capacity
is the capacity for one's opinions to become more similar to others' opinions.
The second essential capacity is the ability to become more different from those
with whom we disagree. Whether two individuals are attracted to or repulsed
from one anothers' opinions is often determined by their group membership---people
tend to be attracted to in-group members' opinions and repulsed by out-group
members' opinions. Therefore determining one's and others' group membership is also an essential
cognitive capacity. Social influence can be modulated by one's degree of
similarity or dissimilarity to in-group or out-group members---more similar
views may be more attractive, e.g., or more different opinions more repulsive.
Individuals may also vary in their susceptibility to social influence---for
example in the model used in Studies 2 and 4 I assume those with more
extreme opinions are less susceptible to social influence, i.e., 
they are more stubborn.

Rather uncontroversially, we know that humans tend to find agreement with
one another and consensus often 
emerges within groups~\cite{Festinger1954,Cartwright1956,French1956}. 
Consensus with (or conformity to) others' opinions has been shown to emerge 
even when direct evidence contradicts those opinions, as \citeA{Asch1955,Asch1956}
found in his classic studies in which participants were fooled by confederates
into going along with the crowd despite their own direct perception that
the crowd was obviously wrong. Consensus can be problematic when
consensus occurs around, e.g., 
false scientific beliefs and 
misinformation~\cite{Zollman2007,Zollman2013,OConnor2018,OConnor2019e}.

Often in intergroup social influence, there is no drive to unite in
opinions, but instead a drive towards differencing~\cite{Tajfel1979,Sherif1988,Flache2011},
especially when confronted with out-group opinions~\cite{Bail2018}.
Group membership may be determined by observable traits such as race, language,
or style of dress, but it need not be. The ``novel group'' experimental design in
social psychology has revealed in a number of cases that group membership
as specified in a given experiment by the experimenters can override observable
indicators of group membership~\cite{Tajfel1971,Billig1973,Tajfel1982}.
These quick changes in behavior are reflected by equally quick changes in 
brain activity that apparently supports intergroup 
behavior~\cite{Cikara2014,Cikara2017}.

People often are more strongly influenced on one issue if they are already
similar with their interaction partners. Conversely, biased assimilation
also leads individuals to be more repulsed by opposing views the more 
different other views are. This is
known as \emph{biased assimilation}, somewhat equivalent to 
the principle that birds of a feather flock together~\cite{Lord1979,McPherson2001}.
In politics, individuals have been observed to be more influenced by presidential
candidates in a debate who are perceived as similar to themselves.
On large scales, it has been observed that food, hobby, and other
preferences are becoming increasingly correlated with political ideologies such
as conservativism and liberalism~\cite{DellaPosta2015}.
\citeA{Suhay2018} found that emotions may be critical: they found that
anger especially, along with other emotional states, ``fuel(ed) biased reactions
to issue arguments'' in an online behavioral study. 

The final cognitive factor in social influence I consider is that those with
more extreme opinions tend to be more stubborn, i.e.\ less susceptible to
social influence, than centrists. On the one hand, longitudinal survey studies
have found that a large portion of the population are centrists demonstrating
poor ``opinion stability over time''~\cite{Converse1964,Zaller1992,Kinder2017}. 
Extremists in the United States and United Kingdom were observed to be 
more cognitively inflexible than their centrist counterparts~\cite{Zmigrod2019a}.
Centrist opinions tend to be more susceptible to framing effects~\cite{Chong2007}
and to question ordering. Extremism has also been electrophysiologically linked
to differences in responses to stimuli. In an EEG study, \citeA{Reiss2019} found that ERP
responses to anomolies in experimental stimuli were muted among participants
with more extreme socio-political opinions compared to centrist participants.

Other approaches to studying the cognitive factors in extremism and polarization
include analyzing the correlation between personality traits and ideological
alignment~\cite{Rollwage2019}, and considering cognitive factors of social influence 
in the context of exchanging information, instead of influencing opinions~\cite{Carley1990,Carley1991,Bala1998}.
In the UK, for instance, \citeA{Zmigrod2018} found that dependence on routines was
positively correlated with subscribing to conservativism, nationalisim, and
authoritarianism, which in turn were positively correlated with support for
Brexit from the European Union. My work complements these approaches in that
it considers simpler cognitive factors than personality traits, which I see
as a composite of opinions and beliefs. Due to the complexity of the
personality trait construct, it is difficult to tell whether personality traits,
and their relationship to national-scale ideologies and policies,
are biologically or culturally determined~\cite{Claidiere2012c,Smaldino2019d},
and so possibly subject to change along with cultural context.
Some approaches to social influence of knowledge assume mechanisms for generating and
sharing knowledge, possibly under an assumption of biased 
assimilation~\cite{Mark1998,Mark1998a,Mark2003}. In other approaches, 
knowledge that one behavior is better than another is based on two
belief channels: one channel is the observation of stochastic payoffs 
received from taking one action or another; the other belief channel is 
the social one, where individuals influence one anothers' knowledge about
the system in terms of influencing their beliefs about which behavior is
more beneficial~\cite{Zollman2007,OConnor2019e}.

% Also these (?): \cite{Toner2013,Jern2014,Zmigrod2018,Zmigrod2019a,Kruglanski2019a}


\subsection{Social factors contributing to polarization}

If we want to understand how opinions change under social influence in the
media and societal system outlined above, we need a way to think about
social relationships. Social polarization and its opposites, such as consensus and cohesiveness, strongly
depend on who interacts with whom, and when~\cite{Flache2008,Turner2018}. 
But what determines these social relationships and how do these relationships
change over time? Furthermore, how do we think about this relationships 
scientifically---what model should we use? Who we interact with is somewhat
random and out of our control: it depends on our family membership, geographic location, 
participating in social activities (e.g.\ attending school, getting
groceries, going to a restaurant, etc.). In addition to these random factors, 
we also adjust our social relationships based on interpersonal affinity and similarity, 
i.e., we tend to prefer to interact with people we like and avoid people we dislike.
Social activity is the source of the complexity of social polarization and
other emergent social phenomena---how do we represent the combinatorial
possibilities of social interaction in a tractable way? Thinking of these
evolving relationships as a social network, modeled by a mathematical graph,
enables us to formally represent social relationships and harness graph
theory to calculate and predict social facts and behavior. For example, graph
theory can help us predict how quickly information~\cite{Milgram1967,Travers1969}, 
disease~\cite{Salathe2010,Block2020}, violence~\cite{Epstein2002}, etc., 
spreads in groups and in society~\cite{Milgram1967,Travers1969,Watts1999,Palla2007,Backstrom2012,Wohlgemuth2014}.

In social systems there are several ways people are influenced by one
another. People tend to choose social interaction partners who 
are similar to themselves, known as \emph{homophily}. As homophily
increases among a population, this
increases the chance that individuals interact with similar others, 
and decreases the chance that individuals will act with dissimilar
others~\cite{McPherson2001}. Homophily, then, amplifies the cognitive factor of biased assimilation,
since increased homophily tends to further insulate individuals from
exposure to opposing viewpoints as biased assimilation causes individuals to 
ignore or reflexively dislike opposing viewpoints and uncritically incorporate
information that supports their pre-existing opinions~\cite{Mark2003,Dandekar2013}.
Another social factor that affects social outcomes are power structures
that result in one person having a greater social influence than others.
This may be represented as having a greater number of relationships
with others, so that their opinions are more widely shared~\cite{French1956,Friedkin1986}.

To think about social influence and social relationships, we can use a social 
network to represent the relationships between individuals
in groups or societies. 
Social networks are based on mathematical graph theory. Study 4 uses a network-theoretic
model of social influence to understand how social networks contribute to
the emergence of extremism and polarization---the model is formally 
introduced there. Individuals in a social network are represented by \emph{nodes},
often drawn as dots or some other marker. Nodes can encapsulate an individual's
identity in addition to traits such as group
membership, accumulated resources (i.e. ``payoffs''), etc. Sometimes these traits are visualized
by changing the marker size, color, or shape of the node in network visualizations.  Relationships
between individuals are represented by \emph{edges}, drawn as lines that connect
Homophily and power differentials between individuals 
may also be represented in terms of edge \emph{weights} on the graph. Any graph
may have weighted edges which could stand for many different things; in
navigation applications edge weight might represent the time it takes to reach 
one location from another. \emph{Temporal} or \emph{dynamic} graphs are graphs that change 
over time, which in social situations could result from changing affinities,
geographic relocation, or new communication technologies~\cite{Li2017}. 
In our application to understanding social influence and , we might see
social networks change when, for example, social ties are abandoned when interpersonal
similarity drops below some threshold~\cite{Axelrod1997,Hegselmann2002,Centola2007,Kossinets2009}.

Whabout these? \cite{Watts1999,Macy2003,Baldassarri2007, Flache2011,DellaPosta2015,Turner2018,Stewart2020b}

\section{Mechanistic models of emergent social phenomena}

A hallmark of the scientific explanation of some phenomenon is that the explanation
only posits the existence of theoretical entities and entity capacities or 
relationships~\cite{Kauffman1970,Cartwright1989,Craver2006,Turner2021}. If the phenomenon of interest 
emerges from system dynamics specified by the entities and their capacities,
then the model and its theoretical basis has some explanatory power. 
A phenomenon \emph{emerges} when a statistical pattern is detected that is 
associated with that phenomenon, e.g., polarization is often measured as the
variance of individual opinions (i.e.\ attitudes, beliefs, etc.) in a society.
The patterns of interest in this dissertation are static (polarization at a given
point in time) or dynamic~\cite{Kelso1995} (rising extremism and collective changes in 
violence metaphor use on cable news).
It is not valid to assume in advance the existence of the phenomenon. Identifying
an explanatory mechanistic model of some emergent phenomenon is a necessary step
in developing a valid, robust theory of the phenomenon. Theories of emergent
social phenomena must be supported by real world empirical studies that observe predicted patterns and 
relationships between independent variables/model parameters and dependent variables/model outcome variables.
All studies in this dissertation are driven by a mechanistic model-based theoretical
approach designed to explain observed patterns in mass media metaphorical violence use (Study 1) and
rising extremism in socially isolated groups (Study 2); false detections of
rising extremism in socially isolated groups (Study 3); and the determinants social polarization (Study 4).


\subsection{Emergent social phenomena}

In this dissertation I focus on the emergence of rising extremism and polarization,
which I theorize is influenced by the emergent dynamics of metaphorical violence
use on cable news (one of many influential mass media communication strategies). 
Emergent social phenomena are identified by finding patterns in
the distributions of individual-level behaviors, opinions, traits, etc., 
among a population~\cite{Blau1974,Schelling2006}. It is challenging to
explain, with scientific rigor, how emergent social phenomena such as
rising extremism and political polarization actually 
emerge from repeated instances of social influence. Social systems are
complex systems of groups of various sizes, and individual humans themselves
are complex emergent phenomena~\cite{Kello2007,Lazer2009}. 

In this work we have assumed that individuals as have opinions. Polarization is 
In the case of opinion polarization, 
individual-level opinions are reported, and polarization is calculated 
as the variance (or similar) of opinions~\cite{Bramson2016}. A totally polarized society has
exactly half of the population holding one of two extreme opinions, and the
other half holding the opposing view. 
Other behaviors that lead to emergent social phenomena 
include choosing where to live based on racial preferences (not racial animosity), 
which can result in emergent racial segregation~\cite{Schelling1971}; publishing journal 
articles of differing validity which leads to systemic scientific problems~\cite{Smaldino2019};
or writing statements and documents online that together form a system of 
cultural frames including harmful ethnic, gender, and racial biases and 
stereotypes~\cite{Caliskan2017,Garg2018}. 

There are also emergent phenomena
at smaller social scales, such as dyads and other small groups~\cite{Abney2014a}. 
For example, dyads were found to synchronize with one another when in collaborative
tasks, and ``asynchronize'' when in an adversarial 
relationship~\cite{Abney2014,Ramirez-Aristizabal2018,Schloesser2019,Schneider2020,Abney2021}.
In turn, individual humans are emergent properties of a complex electrochemical
interaction of individual, differentiated cells~\cite{Schrodinger19??,Kello2007,Lazer2009}.
It is for this reason that I believe it is best to avoid thinking about
``micro'' and ``macro'' scales as seems to be popular among 
sociologists~\cite{Macy2002,Schelling2006}. I prefer to think about the assumptions
we must make about individual cognition and social interaction between dyads
as ``individual-level'' assumptions, operationally equivalent to ``micro-motives''.
In the science of human social influence, the concept of an \emph{emergent phenomenon} is operationally equivalent to 
the ``macrobehavior'', as \citeA{Schelling2006} and others have called it.  
My reasons to think about \emph{individuals} and \emph{activities} 
are partly stylistic, but mostly logical~\cite{Machamer2000}.
I prefer speaking of the emergent phenomena arising from interacting
individuals because in social systems an ``individual'' need not be a single
person---it may be groups, companies, TV news outlets, or other aggregate 
body. A group, or even a single person, is hardly ``micro'' as in the proper sense of requiring
a microscope to view it, or occurring on length scales of 1 micrometer.

Variations in social network structure can lead to significantly different
emergent social structure. 
For instance, in a classic model of cultural differentiation~\cite{Axelrod1997},
it seemed at first that cultural homogeneity was impossible---there seemed to
always be at least a few subcultures co-existing within a larger simulated
society. However, when . Finally, when homophily was represented
as the ability of individuals to sever ties with dissimilar others, cultural
heterogeneity was again preserved as in the original study. In another series
of studies, \citeA{Flache2011} and \citeA{Turner2018} found that small world
networks bias societies to become more polarized over time, on average, 
as long as the initial polarization is large enough. 

\subsubsection{Collective violence metaphor usage on cable TV news}

The first emergent phenomenon we consider is the frequency of usage across
cable TV news outlets, which we assume varies depending on how soon there will
be or how recently there has been a presidential debate or the presidential 
election. Using counts of metaphor use as an observed variable 
is not a standard one like rising polarization. However, it does 
complement similar emerging approaches to studying time series of semantic content in
mass media and social media in order to understand how cognitive, cultural, and
communicative frames covary with historical 
events~\cite{Nunn2012,Klingenstein2014,Hamilton2016c,Caliskan2017,Barron2018,Garg2018}. 
Partisan polarization can be identified through such semantic differences~\cite{Gentzkow2019}.

From the complex systems perspective, ``pragmatic choice'', i.e. what words
to use when, is the result of many ongoing subprocesses, which occur within
different contexts~\cite{Gibbs2012a}. The collective attention of society becomes entrained on
shared cultural events~\cite{Fusaroli2015}. The utterances of news anchors, commentators, and 
pundits cannot be separated from their pragmatic purpose and societal context~\cite{Kovesces2010}.



\subsubsection{``Group polarization'': rising extremism in small, socially isolated groups}

When social psychologists observed socailly isolated groups becoming more extreme
in their opinions after deliberating some topic in a small group of likeminded
individuals, they called it \emph{group polarization}. 
This clashes with how I believe our culture generally understands the word
``polarization'', which refers to the extent to which opposing groups disagree with
and dislike one another. For better or worse, we seem to be stuck with 
\emph{group polarization} due to decades of 
use among social psychologists~\cite{Brown1986,Sieber2019},
sociologists~\cite{Friedkin1999a}, political scientists~\cite{Schkade2010}, and legal scholars~\cite{Sunstein2002}.

\subsubsection{Polarization}

\begin{itemize}
  \item
    Polarization through repulsive influence~\cite{Baldassarri2007,Flache2011,Turner2018}
  \item
    Polarization only through attractive influence~\cite{Mas2013,Turner2020}
\end{itemize}


\subsection{Model-based theoretical approach}

Models are simplified versions of reality we use to identify which components of
complex systems are most important in the emergence of collective larger-scale
phenomena~\cite{Kauffman1970,Wimsatt1972,Wimsatt1997,Machamer2000,Wimsatt2007,Smaldino2017}.
The best models are \emph{mechanistic models}, ones that explicitly identify the atomic theoretical 
entities in a system and how those entities influence one another~\cite{Machamer2000,Craver2006,Turner2021}. 
In our case a verbal mechanistic model of a societal system would be that
human individuals communicate with and influence those with whom they share
social connections, represented as being social network neighbors. Above we listed
assumptions about how individuals process social influence and how social
interactions are structured, which make up further details in our verbal model
of social influence. Mechanistic models are stronger still when they are 
formalized into mathematical notation and implemented computationally to
make quantitative predictions of how different social phenomena emerge based
on model assumptions. The Studies I present in this dissertation all develop
mechanistic models of social influence and mass communications. Studies 1 and
3 implement mechanistic models as generative, computational statistical models, while
Studies 2 and 4 implement their social influence models as agent-based models
of social influence incorporating the cognitive and social factors listed above,
theorized to be important in the emergence of rising extremism and polarization.
The dissertation studies are organized based on their approach
and findings in the study of how rising extremism and polarization emerge
under mass communications and social influence, not their modeling approach.

(
Insert abridged and expanded version of Mech Modeling for the Masses
here, possibly in a few paragraphs, or maybe just summarize in one.
)

This dissertation uses two major modeling 
approaches: statistical modeling and agent-based modeling. Statistical
models are often fit to data, but they need not be. To show that
many group polarization results are false, I used a statistical model
to generate counterexamples demonstrating that many published results are
plausibly false. These statistical models are mechanical in that they measure
the dynamics of violence metaphor use under the influence of presidential
debates in one case, and in the other case the statistical model encapsulates
the measurement-deliberation-measurement process in group polarization studies. 
The other modeling approach is the use of agent-based
models. Agent-based models are a way to directly specify the mechanisms
identified in a mechanistic model. Agents are simulated individuals whose
capacities within the model determine how it updates its own opinion based on
the opinions of interaction partners. Interaction structure in these models,
Who interacts with whom is encoded in and structured by a social network, that
is also a model component.


\subsubsection{Models in the dissertation Studies}

All four Studies presented here use some form of mechanistic modeling to 
represent system dynamics that give rise to emergent phenomena. Mechanistic
models may be expressed and implemented in a variety ways. In addition to 
developing detailed verbal models of how social influence and mass
communication work, Studies 1 and 3 implement statistical models and 
fitting procedures to empirically
determine inflection points in violence metaphor dynamics (Study 1) and 
to demonstrate that a high rate of experimental detections of rising extremism
are plausibly false (Study 3); Studies 2 and 4 use 
agent-based models to understand which cognitive and social factors best explain
and predict rising extremism and polarization, respectively.

Study 1 and Study 3 both use statistical models---in Study 1 the model is fit
to observations, and in Study 3 the model generates simulated counterfactual
data. In Study 1, to partially explain the dynamics of metaphorical violence use on cable TV news,
I developed a dynamical model expressed in terms of a statistical regression
model where each news channel is in either a normal state or a 
transient excited or depressed state. I used the Akaike Information 
Criterion to identify the statistical model
that best partitions daily timestamps that most likely belong in either the
normal or excited/depressed state~\cite{Burnham2004,Burnham2011}. 
In Study 3, we use a generative 
statistical model to simulate experimental group polarization data where
pre- and post-deliberation opinions are drawn from distributions with the
same mean. By also simulating the measurement of these opinions, I show
that ceiling effects lead to a false detection of an opinion shift due to
the process of consensus that reduces group opinion variance from
pre- to post-deliberation.

Study 2 and Study 4 model different systems using the same underlying
agent-based social influence model that incorporates the cognitive and social 
factors outlined in Section~\ref{sec:cognitiveSocialFactors} above. Agent-based
models start by defining a computational representation of a person, called
an \emph{agent}. I wrote computer code that created a world in which agents
were created, made to interact with other agents according to rules and
assumptions based on the cognitive and social factors outlined above, and, 
after thousands or millions of rounds of social interaction, we can measure
the distribution of opinions to calculate either a rise in extremism or
increased polarization.
     

\section{Overview}

Now we have reviewed the overarching problem and goals of this work, the
theoretical foundation we draw on to address specific subproblems, and
the analytical approach I take to studying different emergent social
phenomena, I will now give an overview of the four Studies presented in this
dissertation.

The four Studies examine three specific subsystems in the greater social system
of elite and interpersonal communication. First, Study 1 studies and 
calculates the influence of political events, namely three US presidential
debates and the presidential election, on metaphorical violence use across
three cable TV news channels in 2012 and 2016. Study 2 focuses down from
mass communications to group-level social influence and shifts in extremism.
In that study I use agent-based modeling to show that groups may become 
more extreme when they are dragged to extreme opinions by stubborn extremists.
However, in the course of Study 2 I found that many behavioral studies of
group polarization used a problematic method for measuring group polarization.
Study 3 uses a generative statistical model to show that, indeed, over 90\% of published detections of group polarization
are plausibly false detections. Finally, in Study 4, I explore how well the
model from Study 2 could be used to explain and predict society-level
polarization. I found that long-term polarization increases with the initial
polarization; that realistic small-world networks result in higher levels of
polarization than common alternative configurations; that polarization
increases with miscommunication probability; and that
polarization is highly stochastic---the same initial configurations and
parameter settings can result in a range of predictions from low to high
levels of polarization solely due to the path dependence of interpersonal
interactions over many time steps.


\subsection{Study 1: Violence metaphors foster extremism and polarization}

(FOUR IMRAD PARAGRAPHS PER STUDY)

\subsection{Study 2: Stubborn extremism explanation of group polarization}

\subsection{Study 3: Many observations group polarization are plausibly false positives}

\subsection{Study 4: Opportunities and limits for predicting polarization}



\bibliographystyle{apacite}

\setlength{\bibleftmargin}{.125in}
\setlength{\bibindent}{-\bibleftmargin}

\bibliography{/Users/mt/workspace/papers/library.bib}

\end{document}
