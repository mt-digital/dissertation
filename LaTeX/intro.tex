% Author: Matthew Turner

\documentclass[11pt,letterpaper]{article}
% \documentclass[11pt]{report}
% \documentclass{report}
% \documentclass{book}
\usepackage[bookmarks,hidelinks]{hyperref}
\usepackage{amssymb,amsmath}
% \usepackage{fullpage}
\usepackage{tabulary}
\usepackage{tabularx}
\usepackage{float}
% \usepackage[margin=1.00in]{geometry}
\usepackage[margin=0.90in]{geometry}

\usepackage{caption}
\usepackage{booktabs}
\usepackage{pslatex}
\usepackage{apacite}
\usepackage{subcaption}
\usepackage{pgfplots}
\usepackage{wrapfig}
\usepackage[english]{babel}
\usepackage{lmodern}
\usepackage{setspace}
\doublespace
% \usepackage{url}
\usepackage{bigfoot}
\usepackage[export]{adjustbox}
\setlength\intextsep{0pt}

\usepackage{graphicx}

\title{Introduction to the dissertation}

\author{{Matthew A.~Turner}}

\begin{document}
\maketitle

In this dissertation I present solutions to modest subproblems in
the following complex, computational social science problem. 
The big problem from 30,000 feet is to calculate the societal impact of a common
communication strategy used by mass media outlets, namely cable TV news. 
This communication strategy, metaphorical violence, is known to increase 
extremism including support for real world violence to achieve political goals. 
This increased support for political violence is even more pronounced among the most trait 
aggressive individuals in a population. To explain shifts in opinions due to
interpersonal influence via communication it is necessary to accurately measure 
differences in opinions, which I will show is no trivial task---in fact, several
dozen studies on rising extremism among isolated groups are unreliable due to
a mismatch between the psychological model of opinions assumed in these 
studies and ordinal scale measurement schemes, e.g., Likert scale measurements.
In the most basic case, it is not clear how extremism and polarization depend
on social structure such as social influence networks or initial extremism
and polarization in a society. Similarly, it is not clear how miscommunication
and random chance would affect extremism and polarization outcomes either.
All these factors, and likely others not covered here, must be accounted for
in order to make predictions about the emergence of extremism and polarization.

To begin examining this complex social problem it is important to understand
the nature of metaphorical violence use on cable news. Metaphors in
political discourse are generally important to understand because metaphors
connect the abstract conceptual domain of politics to any number of more
embodied physical domains, such a footrace, a sinking ship, or a fight~\cite{Charteris-Black2009}.
We limit our focus to particularly important times in the United
States: around the time of the presidential debates and election in the 
USA in 2012 and 2016. This lays the foundation for predicting how much
more likely political violence is around the time of the elections, just due
to the seemingly insignificant effect of metaphorical violence on opinions.
We developed several cyberinfrastructure tools to 
acquire, annotate, and analyze cable TV transcripts provided by the TV News
Archive\footnote{https://archive.org/tv/}. I also developed a dynamic, explanatory 
statistical modeling approach used to calculate when and by how much violence metaphor 
usage changes around the debates and election.

In order to measure changes in opinions (or attitudes, beliefs, etc.) one must
provide experiment participants a measurement instrument on which to 
indicate their opinion value. 

Before we develop predictive models with complex mechanisms such as 
metaphorical violence use, we should understand how and when simple, empirically motivated
models predict the emergence of extremism and polarization, or their opposite,
consensus. For this I developed an agent-based model of social influence and
netowrk structure effects...~\cite{Turner2018}


(THEN WHAT?)

% \begin{itemize}
%   \item
%     This dissertation presents multi-method tools and multi-scale and -dimensional
%     insights into the cognitive, social, and communicative foundations of rising extremism and
%     polarization.
%     \begin{itemize}
%       \item 
%         Cognitive factors include mechanisms of social influence within
%         individuals, such as one's openness or stubbornness and how one
%         is attracted to or repulsed from others' opinions.
%       \item
%         Social factors in this work are mainly considerations of how social
%         networks and societal norms of computational and real-world actors
%         influence those actors' behaviors.
%       \item
%         Communicative factors in this case are those features of language or other media that
%         have an effect on how 
%       \item
%         Because human groups and societies are complex systems, we can only
%         ever learn a little at a time through specialized methods and approaches.
%     \end{itemize}
%   \item
%     In this dissertation I first use empirically-motivated computational modeling 
%     to explain and predict rising extremism among small, socially isolated 
%     groups and show that existing methods for studying such rising extremism
%     have resulted in plausibly false detections of group polarization.
%     \begin{itemize}
%       \item
%         In the next part of the paper I analyze how extremism and polarization
%         emerge in small societies based on the same model used to explain
%         and predict rising extremism in small, isolated, biased groups.
%         In this part I also examine the hypothetical effects of increasingly
%         connected social networks and miscommunication on the emergence of
%         extremism and polarization.
%       \item
%         In a complementary analysis of a different target system with different
%         methods, I developed new cyberinfrastructure tools to build and analyze
%         a corpus of US cable news to analyze metaphorical violence use,
%         which an important communication strategy
%         that has been shown to increase extremism and the likelihood of
%         real world violence~\cite{Kalmoe2014,Kalmoe2018}.
%       \item
%         Compiling my dissertation research has opened several routes for future
%         work, including developing new tests of my explanation for rising extremism
%         in isolated groups, and more general investigations of the roles for,
%         and schema that describe, different kinds of models and analyses 
%         in the study of complex human social systems.
%     \end{itemize}

% \subsection{Notes on studying extremism and polarization as emergent dynamics}

% Rising extremism and social and political polarization across a wide range
% of issues are emergent properties of extremely complex systems---human
% social systems. I do this partly to develop theoretical intuition into the
% problem of rising extremism and polarization. I also do this to explain why,
% despite their very different methodologies, the computational modeling work
% and the corpus-based metaphorical violence work both indeed contribute to 
% understanding extremism and polarization.

%   \begin{itemize}
%     \item
%       When we think about trying to find solutions to these problems
%       we have to first recognize that our knowledge of complex systems must
%       necessarily be built up piecemeal. 
%     \item
%       Complex systems are characterized by the operation of processes
%       occurring at different scales of complexity, where processes at one 
%       scale can have an effect on processes at other scales. 
%     \item
%       Therefore it is necessary to perform a sort of dimensionality reduction
%       that has the additional effect of theoretically walling out potentially
%       many important cognitive and social factors in order to learn more about
%       some small subset of cognitive and social factors.
%     \item
%       One must not only determine theoretical limits of experiments and
%       analyses, but also the scale to study a particular effect. Social
%       neuroscientists may study neural correlates of attitude change that
%       occur over milliseconds or seconds, while anthropologists and
%       historians might study attitude change over several years, up to centuries and
%       millenia.
%     \item
%       One must also choose what sort of knowledge one hopes to gain about a 
%       system, which depends on how much is already known about systems of
%       interest. Models can vary on several important dimensions in this regard,
%       most importantly for us is distinguishing ``how-possibly'' from
%       ``how-actually'' models. Statistical models exist mainly to process
%       data, but also express important, possibly unexplained relationships
%       between observables. Explanations must follow for us to trust and
%       utilize the results~\cite{Craver2006}.
%   \end{itemize}

% \end{itemize}

% \subsection{Real world polarization and extremism}

% \subsection{``Group polarization'': initially biased groups become more extreme}

% \subsection{Simulation as a tool for studying social behavior and outcomes}

% \subsection{The communication strategy of metaphorical violence}

% \begin{itemize}
%   \item
%     In the theoretical model I developed for studying extremism and polarization,
%     communication was assumed to be highly idealized: model agents communicated
%     their opinions in numerical form to other agents, possibly modulated by
%     random noise. In this section we explain an important detail such an 
%     approach obscures: the effect of different rhetorical strategies on
%     extremism and polarization. Among the many possible rhetorical strategies
%     we focus on metaphor. This is sensible for two reasons: first because 
%     Second, understanding metaphor is theoretically important in cognitive science 
%     because we seem to scaffold our knowledge about the world in terms of
%     metaphor~\cite{Lakoff1980,Lakoff2000,Gibbs2006,Kovecses2010a,Gibbs2011,Nunez2012,Matlock2012}
%   \item 
%     First we need to know what a metaphor is and why metaphors are important
%     generally---briefly, metaphor is a rhetorical device that maps some more
%     abstract concept, such as love, onto a more concrete object, such as 
%     a flower. One cannot hold love, but one can hold, see, and smell a flower.
%   \item
%     Then we can understand what metaphorical violence is (or, equivalently, what
%     violence metaphors are). They are metaphors that map some abstract concept
%     to some form of violence. Specifically, since I am most interested in 
%     extremism and polarization in politics, I study metaphorical violence in 
%     political discourse on cable news. In this case, the abstract concepts
%     to be mapped on to violent behaviors, such as ``attacking'', ``fighting'',
%     ``punching'', etc., including war metaphors \cite{Flusberg2018}.
%   \item
%     Metaphorical violence has been linked to increased support for violence
%     to achieve political goals~\cite{Kalmoe2014,2018}. 
%   \item
%     More generally, whether or not metaphors are used in communication affects
%     outcomes for speakers and listeners (or writers and readers)~\cite{Gann2011,Matlock2012a,Magaña2018}.
% \end{itemize}


% \subsection{Other cognitive, social, and communicative factors}

% As discussed, one or a few studies can only examine a small fraction of the
% entities and processes in complex human social systems.
% \begin{itemize}
%   \item
%     Other cognitive factors:
%     \begin{itemize}
%       \item 
%         Neural correlates of social influence~\cite{Cikara2014,Cikara2017}
%     \end{itemize}
%   \item
%     Other sociological factors: 
%     \begin{itemize}
%       \item
%         payoff maximization under unfair power dynamics~\cite{OConnor2019a}
%       \item
%         multiplex networks~\cite{Weatherall2020} 
%         of large-scale social networks, e.g., on social media
%     \end{itemize}
%   \item 
%     Other communication strategies:
%     \begin{itemize}
%       \item 
%         Misinformation~\cite{OConnor2019e,OConnor2020b}.
%       \item
%         The effect of the affordances provided by social media
%     \end{itemize}
% \end{itemize}


\bibliographystyle{apacite}

\setlength{\bibleftmargin}{.125in}
\setlength{\bibindent}{-\bibleftmargin}

\bibliography{/Users/mt/workspace/papers/library.bib}

\end{document}
