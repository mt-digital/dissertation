% Author: Matthew Turner

\documentclass[12pt,letterpaper]{article}
% \documentclass[11pt]{report}
% \documentclass{report}
% \documentclass{book}
\usepackage[bookmarks,hidelinks]{hyperref}
\usepackage{amssymb,amsmath}
% \usepackage{fullpage}
\usepackage{tabulary}
\usepackage{tabularx}
\usepackage{float}
\usepackage[margin=1.00in]{geometry}
% \usepackage[margin=0.90in]{geometry}

\usepackage{caption}
\usepackage{booktabs}
\usepackage{pslatex}
\usepackage{apacite}
\usepackage{subcaption}
\usepackage{pgfplots}
\usepackage{wrapfig}
\usepackage[english]{babel}
\usepackage{lmodern}
\usepackage{setspace}
\doublespace
% \usepackage{url}
\usepackage{bigfoot}
\usepackage[export]{adjustbox}
\setlength\intextsep{0pt}

\usepackage{graphicx}

\title{Introduction to the dissertation}

% \author{{Matthew A.~Turner}}
\date{}

\begin{document}
% \vspace{-4in}
% \maketitle

\vspace{-2.5in}
\textbf{Communicative, cognitive, and social factors in
extremism and polarization}
\begin{abstract}
  Rising extremism and polarization threaten democratic institutions worldwide.
  As opposing factions become more extreme in their opinions, polarization
  widens the chasm between fellow citizens, and common ground erodes, washed
  away down a river of vitriol, bitterness, and hate. What causes increased
  extremism and polarization? Due to the highly complex nature of human
  societies, this problem of explaining polarization must be broken down into
  many sub-problems, which themselves require complex systems thinking to
  address. Only through ``walling off'' smaller components of social systems,
  and rigorously modeling and analyzing empirical data,
  can we build a thorough, coherent understanding of social behavior.

  In this dissertation I present my findings from studying three
  sub-problems in explaining why and how extremism and polarization emerge.
  First, I focus narrowly on a communication strategy shown in behavioral
  studies to increase extremism, \emph{metaphorical violence}, such as 
  ``Biden hit Trump over his tax returns in yesterday's debate.'' While we know
  the effects of violence metaphors, we do not understand their distribution
  in the wild, or what causes their usage to increase and decrease. I found
  that metaphorical violence use increased around the time of presidential
  debates and elections in the United States, and was correlated with 
  presidentical candidates' tweets. 

  Second, I show that rising extremism
  in isolated social gropus may be simply explained by the fact that
  extremists are more stubborn than centrists---however existing data on the
  subject is unavailable and behavioral studies on the subject may
  contain ubiquitous false detections of rising extremism. 

  Finally, I developed
  and analyzed an empirically-motivated, network-theoretic, agent-based model of 
  social influence at the societal level to understand how well we can 
  predict polarization, and the effect of initial conditions, network structure,
  communication noise, and random chance on predictions of polarization.

  Taken together these studies advance our understanding of communicative,
  cognitive, and social factors in the emergence of extremism and polarization.
\end{abstract}
\hrule
\vspace{2em}

This dissertation takes on the widely studied topic of social polarization. 
Polarization is a phenomenon where opposing social groups become more extreme
in opposite directions~\cite{Bramson2016,Bramson2017,Jung2019,Klein2020,StewartMcCBryson2020}. 
Although polarization has been widely studied by a diversity of scientists,
the topic is scientifically unwieldy because human societies are the things
that become polarized, and human societies are highly complex systems. As the
saying goes, the human brain-body is the single most complex thing we know of.
How much more complex are thousands, millions, or billions of interacting 
humans?

It is necessary to ``wall off'' small, specific subsystems within human
societies in order to make the study of polarization (or any complex social
phenomena) tractable~\cite{Cartwright1999}.
This process requires us to simultaneously specify and generalize to 
develop models that
are simple enough to be interpreted but specific enough to be understood
mechanically~\cite{Craver2006,Wimsatt2007,Smaldino2017,Turner2021}. In the end
we develop studies that are very specific to the phenomena of interest
in carefully controlled conditions, despite the unfortunate common practice of
overgeneralizing results in the behavioral and social sciences~\cite{Yarkoni2021}.
When we develop models of social subsystems, we assume certain factors to be
important while ignoring others that may actually influence real world systems
of interest. This is necessary to understand how certain, but in the long run
it is also necessary to investigate the effect of other important factors
theoretically and empirically.

(BEGIN ~5 PARAGRAPH PRELIMINARY OVERVIEW---ONE BROAD OVERVIEW/INTRO AND 
ONE FOR EACH CHAPTER)
In this dissertation I want to explain three studies that walled off different
components of human society to understand various communicative, cognitive, and
social factors in the emergence of polarization. (EXPLAIN HOW WE THEORETICALLY
WALL OFF FOR EACH OF THE STUDIES I DID)

Metaphors are theoretically important for cognitive science and practically
important for understanding and predicting the consequences of political
speech. Metaphor is ubiquitous in political discourse~\cite{Burnes2011,Charteris-Black2009,Charteris-Black2005,Charteris-Black2004,Lakoff1996,Lakoff2008}.

To complement our 30,000 foot view of how mass media affects public opinion,
win the second study I focus on explaining and predicting \emph{group polarization},
which refers to the tendency of small groups to become more extreme in their
opinions on some topic or in their decision making. For example, a jury that
has to decide damages for a civil liability lawsuit might increase the amount
they are willing to award a victim of some wrongdoing~\cite{Schkade2000}.
Group-level behaviors and traits emerge only when individuals interact
and so must be studied independently from individual- and societal-level
phenomena~\cite{Goldstone2008,Smaldino2014}.

Unfortunately many or most of the empirical detections of group polarization
are false positives, i.e., Type I errors or just
``false detections''~\cite{Turner2021a}. This is due to a mismatch between 
the assumption that opinions are drawn from a continuous, latent psychological
distribution (e.g., a normal distribution) but reported on an ordinal
scale (e.g., a Likert-type attitude/opinion scale).

These focused studies where we narrowly walled off social subsystems do not
address whether and when polarization can be reliably predicted at the
societal scale. In my fourth and final study, I answered these questions
based on general factors that affect predictions of political polarization, 
including initial extremism and polarization in society, how reliably individuals
can communicate, and social network structure. These models provide a
\emph{how possibly} account of rising extremism and polarization in society,
and can frame future empirical, computational, and theoretical work 
studying rising extremism and polarization.

In the course of my dissertation work, I ended up studying three very 
different systems. Each required different computational modeling and analytical
tools. I believe an important contribution of this dissertation are the
cyberinfrastructure tools and approaches I have developed. In studying
metaphorical violence use on cable news, I developed a custom Python API
for searching, ingesting, and converting cable news transcripts from the
Internet Archive's TV News Archive\footnote{\url{https://archive.org/tv}}. 
I also developed a custom web application for finding, tracking, and annotating 
potential instances of metaphor in TV News Archive transcripts. 
In my work demonstrating false detections of group polarization, I adapted
a general computational model of false detections for modeling why and how
group polarization studies generate false discoveries. Finally, I developed and
supported a simpler, more parsimonious and explanatory model of group polarization,
and ``stress tested'' that model to see how reliably it could predict 
extremism \emph{and} polarization under a variety of empirically motivated
conditions.  These contributions are open
source, available on GitHub. These tools can be used and extended by myself or
others to potentially multiply the impact of the studies presented here.
I was inspired by the emerging \emph{computational social science} approach
to studying social systems, that focuses on using computational tools for
modeling and analysis in order to understand emergent social 
phenomena~\cite{Lazer2009}.


\section{Rising extremism and polarization}

Most broadly, the projects in this dissertation advance our understanding
of extremism and polarization---specifically ``affective polarization'' where
the distribution of opinions in society come to be increasingly 
bifurcated~\cite{Bramson2016,Iyengar2019}. 
It lays some groundwork for predicting 
rising extremism and polarization. Explaining and predicting 
extremism and polarization is a problem studied by researchers across disciplines
and subdisciplines, with many nuanced approaches. Understanding
polarization, even or especially among scientists~\cite{OConnor2018}, 
is critical as the world population
grows and our polarized responses to hazards 
including global warming~\cite{Cook2016} and pandemics~\cite{Green2020} 
put the world at risk. 
In this section I introduce a selection of related, relevant work on the 
problem of explaining and predicting extremism and polarization.

The primary motivation and basic data on polarization are long-term 


There are several approaches to explaining rising extremism and polarization.
Empirical studies investigate various factors and communication media to understand
how polarization is affected by, e.g., social media use and 
sentiment~\cite{Bail2018}, mass media effects~\cite{Prior2013,Pew2014,Martin2017a}, 
or pragmatic communicative 
strategies~\cite{Charteris-Black2009,Thibodeau2015,Flusberg2017,Flusberg2018,Turner2021a}.


Computational and formal modeling complements, evaluates, and motivates, evaluates 
empirical work investigating the emergence of extremism and polarization.
Formal computational models allow us to prove out mechanistically what might
be assumed or claimed verbally. Computational models test the logic of 
verbal models. In this dissertation I will claim that a few simple empirically
motivated assumptions about individual-level psychology and behavior result
in the emergent group- and society-level phenomena of rising extremism and
polarization. It is one thing to develop this claim verbally. When
Observing that a set of \emph{how possibly} assumptions foster extremism and
polarization in a society of computational simulated agents 
(think of them as ``sims''), this strengthens the claim of causality. While
agent-based models of social influence are powerful tools in proving out 
theories of emergent social phenomena, such model results rarely match any
specific observations, exactly. Improving modeling efforts so they can predict
data exactly is a subject of ongoing work (REF).

Polarization is not just a problem in the United States, but indeed countries
around the world are feeling stretched by widening gaps in . For example,
X in Venezuela Y~\cite{Morales2015}. In Ukraine, \citeA{Romenskyy2018} found , 
which they showed resulted
in Z based on a bounded confidence computational social influence model. 
In Egypt, \cite{Borge-Holthoefer2014}.


\subsection{A simple visual model of societal extremism and polarization}

Polarization refers to a particular distribution of opinions in a 
society~\cite{Blau1974,Bramson2016,Turner2018}. Opinion polarization is maximum when
a society holds one extreme opinion or the other, with each group having half
the society's population. At the same time, we 

First, let us consider how social structure can be represented as
the distribution of people's opinions in a geometric space~\cite{Blau1974}.

In representing  means we use mathematical graph theory to represent 
individuals as graph ``nodes''; relationships between individuals
are represented by edges~\cite{Friedkin1998,Barabasi2016}. 
Edges may be uni-directional or bi-directional.


(FIGURE HERE OF CONNECTED CAVEMAN AND ONE HISTOGRAM OF RESPONSES TO LIKERT SCALE)

\subsection{How real is polarization, and who is becoming polarized?}

Some believe polarization is actually a myth, and increasingly 
extreme ideological commitments arise among a minority of the population,
with a majority independently aligned either out of an honest belief in 
evaluating individuals and not parties, or who are casual political
observers who begin to pay attention only around election time~\cite{Kinder2017}.

\section{Metaphor and framing in politics and polarization}

In the first study of this dissertation, I analyze violence metaphor use on
cable news around the times of the 2012 and 2016 United States presidential
debates and elections. In this section, I zoom out to discuss metaphor's
importance in cognitive science and to discuss how metaphor is one of several
methods studied by communication and political scientists as a method of
\emph{framing} more generally~\cite{Fillmore1982}---though use of that 
word may be misleading due to being used too freely for too many 
diverse situations~\cite{Cacciatore2016}.

Metaphor is important in cognitive science because metaphor enables people 
to scaffold their understanding of abstract concepts in terms of more concrete
concepts~\cite{Lakoff1986}. In this dissertation, we focus on the use of metaphor
to understand political events, actions, and debates in terms of physical
violence. This is just one of hundreds or thousands of ways metaphor is 
used~\cite{Dodge2015}. In politics, another important class of metaphors is
metaphors for refugees and immigrants. It changes our behavior towards 
immigrants if we cast them in a negative light as something destructive, 
such as an infestation or a flood, compared to finding more positive 
metaphors~\cite{Magaña2018}.

Metaphor is a more general instance of a way to frame a particular topic. 
Linguistic frames are ``instantiated'' based on words being used to describe the
topic, along with the emotional valence and other factors of the words. (MORE)


\section{Cognitive and social factors in polarization}


\subsection{Group polarization: socially isolated groups become more extreme over time}


\subsection{A neurobiological view on group dynamics}


\section{Model-based theoretical approach}

Explain how models are used throughout
\begin{enumerate}
  \item 
    Dynamical impulse model of violent metaphor use 
    increase near debates and elections
  \item
    Social influence model (could get into other such models, e.g., bounded confidence)
  \item
    Generative statistical model of false group polarization detections.
    Formalizing the unformalizable.
\end{enumerate}

\section{Overview}

\subsection{Violence metaphors foster extremism and polarization}

\subsection{Stubborn extremism explanation of group polarization}

\subsection{Many observations group polarization are plausibly false positives}

\subsection{Opportunities and limits for predicting polarization}

\section{Plan}

So far we have introduced the complex problem of polarization, 

\bibliographystyle{apacite}

\setlength{\bibleftmargin}{.125in}
\setlength{\bibindent}{-\bibleftmargin}

\bibliography{/Users/mt/workspace/papers/library.bib}

\end{document}
