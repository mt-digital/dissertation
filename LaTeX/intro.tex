% Author: Matthew Turner

\documentclass[12pt,letterpaper]{article}
% \documentclass[11pt]{report}
% \documentclass{report}
% \documentclass{book}
\usepackage[bookmarks,hidelinks]{hyperref}
\usepackage{amssymb,amsmath}
% \usepackage{fullpage}
\usepackage{tabulary}
\usepackage{tabularx}
\usepackage{float}
\usepackage[margin=1.00in]{geometry}
% \usepackage[margin=0.90in]{geometry}

\usepackage{datetime2}
\usepackage{caption}
\usepackage{booktabs}
\usepackage{pslatex}
\usepackage{apacite}
\usepackage{subcaption}
\usepackage{pgfplots}
\usepackage{wrapfig}
\usepackage[english]{babel}
\usepackage{lmodern}
\usepackage{setspace}
\doublespace
% \usepackage{url}
\usepackage{bigfoot}
\usepackage[export]{adjustbox}
\setlength\intextsep{0pt}

\usepackage{graphicx}

\title{Introduction to the dissertation}

% \author{{Matthew A.~Turner}}
\date{}

\begin{document}
% \vspace{-4in}
% \maketitle

\vspace{-2.5in}
\textbf{Communicative, cognitive, and social factors in
extremism and polarization} 
\begin{abstract}
  Rising extremism and polarization threaten democratic institutions worldwide.
  As opposing factions become more extreme in their opinions, polarization
  widens the chasm between fellow citizens, and common ground erodes, washed
  away down a river of vitriol, bitterness, and hate. What causes increased
  extremism and polarization? Due to the highly complex nature of human
  societies, this problem of explaining polarization must be broken down into
  many sub-problems, which themselves require complex systems thinking to
  address. Only through ``walling off'' smaller components of social systems,
  and rigorously modeling and analyzing empirical data,
  can we build a thorough, coherent understanding of social behavior.

  In this dissertation I present my findings from studying three
  sub-problems in explaining why and how extremism and polarization emerge.
  First, I focus narrowly on a communication strategy shown in behavioral
  studies to increase extremism, \emph{metaphorical violence}, such as 
  ``Biden hit Trump over his tax returns in yesterday's debate.'' While we know
  the effects of violence metaphors, we do not understand their distribution
  in the wild, or what causes their usage to increase and decrease. I found
  that metaphorical violence use increased around the time of presidential
  debates and elections in the United States, and was correlated with 
  presidentical candidates' tweets. 

  Second, I show that rising extremism
  in isolated social gropus may be simply explained by the fact that
  extremists are more stubborn than centrists---however existing data on the
  subject is unavailable and behavioral studies on the subject may
  contain ubiquitous false detections of rising extremism. 

  Finally, I developed
  and analyzed an empirically-motivated, network-theoretic, agent-based model of 
  social influence at the societal level to understand how well we can 
  predict polarization, and the effect of initial conditions, network structure,
  communication noise, and random chance on predictions of polarization.

  Taken together these studies advance our understanding of communicative,
  cognitive, and social factors in the emergence of extremism and polarization.
\end{abstract}

Compiled \today

\section{Introduction}

This dissertation takes on the widely studied topic of social polarization. 
Polarization is a phenomenon where opposing social groups become more extreme
in opposite directions~\cite{Baldassarri2007,Bramson2016,Bramson2017,Turner2018,Jung2019,Klein2020,StewartMcCBryson2020}. 
Although polarization has been widely studied by a diversity of scientists,
the topic is scientifically unwieldy because human societies are the things
that become polarized, and human societies are highly complex systems. As the
saying goes, the human brain-body is the single most complex thing we know of.
How much more complex is a system of thousands, millions, or billions of interacting 
humans?

It is necessary to ``wall off'' small, specific subsystems within human
societies in order to make the study of polarization (or any complex social
phenomena) tractable~\cite{Cartwright1999}.
This process requires us to simultaneously specify and generalize to 
develop models that
are simple enough to be interpreted but specific enough to be understood
mechanically~\cite{Craver2006,Wimsatt2007,Smaldino2017,Turner2021}. In the end
we develop studies that are very specific to the phenomena of interest
in carefully controlled conditions, despite the unfortunate common practice of
overgeneralizing results in the behavioral and social sciences~\cite{Yarkoni2021}.
When we develop models of social subsystems, we assume certain factors to be
important while ignoring others that may actually influence real world systems
of interest. This is necessary to understand how certain, but in the long run
it is also necessary to investigate the effect of other important factors
theoretically and empirically.

(BEGIN ~5 PARAGRAPH PRELIMINARY OVERVIEW---ONE BROAD OVERVIEW/INTRO AND 
ONE FOR EACH CHAPTER)
In this dissertation I want to explain three studies that walled off different
components of human society to understand various communicative, cognitive, and
social factors in the emergence of polarization. (EXPLAIN HOW WE THEORETICALLY
WALL OFF FOR EACH OF THE STUDIES I DID)

Metaphors are theoretically important for cognitive science and practically
important for understanding and predicting the consequences of political
speech. Metaphor is ubiquitous in political discourse~\cite{Burnes2011,Charteris-Black2009,Charteris-Black2005,Charteris-Black2004,Lakoff1996,Lakoff2008}.

To complement our 30,000 foot view of how mass media affects public opinion,
win the second study I focus on explaining and predicting \emph{group polarization},
which refers to the tendency of small groups to become more extreme in their
opinions on some topic or in their decision making. For example, a jury that
has to decide damages for a civil liability lawsuit might increase the amount
they are willing to award a victim of some wrongdoing~\cite{Schkade2000}.
Group-level behaviors and traits emerge only when individuals interact
and so must be studied independently from individual- and societal-level
phenomena~\cite{Goldstone2008,Smaldino2014}.

Unfortunately many or most of the empirical detections of group polarization
are false positives, i.e., Type I errors or just
``false detections''~\cite{Turner2021a}. This is due to a mismatch between 
the assumption that opinions are drawn from a continuous, latent psychological
distribution (e.g., a normal distribution) but reported on an ordinal
scale (e.g., a Likert-type attitude/opinion scale).

These focused studies where we narrowly walled off social subsystems do not
address whether and when polarization can be reliably predicted at the
societal scale. In my fourth and final study, I answered these questions
based on general factors that affect predictions of political polarization, 
including initial extremism and polarization in society, how reliably individuals
can communicate, and social network structure. These models provide a
\emph{how possibly} account of rising extremism and polarization in society,
and can frame future empirical, computational, and theoretical work 
studying rising extremism and polarization.

In the course of my dissertation work, I ended up studying three very 
different systems. Each required different computational modeling and analytical
tools. I believe an important contribution of this dissertation are the
cyberinfrastructure tools and approaches I have developed. In studying
metaphorical violence use on cable news, I developed a custom Python API
for searching, ingesting, and converting cable news transcripts from the
Internet Archive's TV News Archive\footnote{\url{https://archive.org/tv}}. 
I also developed a custom web application for finding, tracking, and annotating 
potential instances of metaphor in TV News Archive transcripts. 
In my work demonstrating false detections of group polarization, I adapted
a general computational model of false detections for modeling why and how
group polarization studies generate false discoveries. Finally, I developed and
supported a simpler, more parsimonious and explanatory model of group polarization,
and ``stress tested'' that model to see how reliably it could predict 
extremism \emph{and} polarization under a variety of empirically motivated
conditions.  These contributions are open
source, available on GitHub. These tools can be used and extended by myself or
others to potentially multiply the impact of the studies presented here.
I was inspired by the emerging \emph{computational social science} approach
to studying social systems, that focuses on using computational tools for
modeling and analysis in order to understand emergent social 
phenomena~\cite{Lazer2009}.

In the remainder of this Introduction I will (1) outline the related problems of
rising social and political extremism and polarization; (2) introduce the 
individual- and dyad-level cognitive, communicative, and social factors 
underlying emergent rising extremism and polarization; (3) explain why and how
these factors lead to rising extremism and polarization, and why a mechanistic
modeling approach is required; and finally, (4) outline the major results 
presented in each dissertation Study.



\section{Rising extremism and polarization}

Most broadly, the projects in this dissertation advance our understanding
of extremism and polarization---specifically ``affective polarization'' where
the distribution of opinions in society come to be increasingly 
bifurcated~\cite{Bramson2016,Iyengar2019}. 
It lays some groundwork for predicting 
rising extremism and polarization. Explaining and predicting 
extremism and polarization is a problem studied by researchers across disciplines
and subdisciplines, with many nuanced approaches. Understanding
polarization, even or especially among scientists~\cite{OConnor2018}, 
is critical as the world population
grows and our polarized responses to hazards 
including global warming~\cite{Cook2016} and pandemics~\cite{Green2020} 
put the world at risk. 
In this section I introduce a selection of related, relevant work on the 
problem of explaining and predicting extremism and polarization.

The primary motivation and basic data on polarization are long-term 


There are several approaches to explaining rising extremism and polarization.
Empirical studies investigate various factors and communication media to understand
how polarization is affected by, e.g., social media use and 
sentiment~\cite{Bail2018}, mass media effects~\cite{Prior2013,Pew2014,Martin2017a}, 
or pragmatic communicative 
strategies~\cite{Charteris-Black2009,Thibodeau2015,Flusberg2017,Flusberg2018,Turner2021a}.


Computational and formal modeling complements, evaluates, and motivates, evaluates 
empirical work investigating the emergence of extremism and polarization.
Formal computational models allow us to prove out mechanistically what might
be assumed or claimed verbally. Computational models test the logic of 
verbal models. In this dissertation I will claim that a few simple empirically
motivated assumptions about individual-level psychology and behavior result
in the emergent group- and society-level phenomena of rising extremism and
polarization. It is one thing to develop this claim verbally. When
Observing that a set of \emph{how possibly} assumptions foster extremism and
polarization in a society of computational simulated agents 
(think of them as ``sims''), this strengthens the claim of causality. While
agent-based models of social influence are powerful tools in proving out 
theories of emergent social phenomena, such model results rarely match any
specific observations, exactly. Improving modeling efforts so they can predict
data exactly is a subject of ongoing work (REF).

Polarization is not just a problem in the United States, but indeed countries
around the world are feeling stretched by widening gaps in . For example,
X in Venezuela Y~\cite{Morales2015}. In Ukraine, \citeA{Romenskyy2018} found , 
which they showed resulted
in Z based on a bounded confidence computational social influence model. 
In Egypt, \cite{Borge-Holthoefer2014}.

Polarization, understood as the variance in opinions or attitudes among
individuals in a population, increases whenever one group in society 
becomes more extremely opposed to the other group(s). Therefore, understanding
the factors that lead to extremism are important to understand, since these
contribute to polarization. Behavioral studies have found that when groups share a valence of
opinion to some degree (e.g., they all ``agree'' more or less with some
statement) they tend to become more extreme in the direction they were already
leaning---social psychologists have named this \emph{group polarization}.
However, in Study 3 here I actually shows that many of the detections of 
increased extremism in these studies are plausibly false detections, but that
does not necessarily mean the phenomenon does not occur. Study 2 will show
that group polarization could be caused by the simple cognitive fact that
those with more extreme opinions are more stubborn, i.e., less susceptible
to social influence.


\subsection{A simple visual model of societal extremism and polarization}

Polarization refers to a particular distribution of opinions in a 
society~\cite{Blau1974,Bramson2016,Turner2018}. Opinion polarization is maximum when
a society holds one extreme opinion or the other, with each group having half
the society's population. At the same time, we 

First, let us consider how social structure can be represented as
the distribution of people's opinions in a geometric space~\cite{Blau1974}.

In representing  means we use mathematical graph theory to represent 
individuals as graph ``nodes''; relationships between individuals
are represented by edges~\cite{Friedkin1998,Barabasi2016}. 
Edges may be uni-directional or bi-directional.


(FIGURE HERE OF CONNECTED CAVEMAN AND ONE HISTOGRAM OF RESPONSES TO LIKERT SCALE)

\subsection{How real is polarization, and who is becoming polarized?}

Some believe polarization is actually a myth, and increasingly 
extreme ideological commitments arise among a minority of the population,
with a majority independently aligned either out of an honest belief in 
evaluating individuals and not parties, or who are casual political
observers who begin to pay attention only around election time~\cite{Kinder2017}.

\section{Metaphor and framing in politics and polarization}

In the first study of this dissertation, I analyze violence metaphor use on
cable news around the times of the 2012 and 2016 United States presidential
debates and elections. In this section, I zoom out to discuss metaphor's
importance in cognitive science and to discuss how metaphor is one of several
methods studied by communication and political scientists as a method of
\emph{framing} more generally~\cite{Fillmore1982}---though use of that 
word may be misleading due to being used too freely for too many 
diverse situations~\cite{Cacciatore2016}.

Metaphor is important in cognitive science because metaphor enables people 
to scaffold their understanding of abstract concepts in terms of more concrete
concepts~\cite{Lakoff1980,Gibbs2011,Gibbs2012a,Kovecses2017}. 

Violence metaphors, which I focus on in this disseration, 
are particularly important, especially as regards political
polarization because they have been shown 
to siginificantly affect political opinions and influence the likelihood of
real world violent behavior. 

Violence metaphors are just one of a great number of metaphorical constructions
\cite{Goldberg2003,Goldberg2006,Dodge2015}. 
In politics, for example, another important class of metaphors is
metaphors for refugees and immigrants. It changes our behavior towards 
immigrants if we cast them in a negative light as something destructive, 
such as an infestation or a flood, compared to finding more positive 
metaphors~\cite{OBrien2003,Cisneros2008}.

Metaphor is a more general instance of a way to frame a particular topic. 
Linguistic frames are ``instantiated'' based on words being used to describe the
topic, along with the emotional valence and other factors of the words. To take
a general example consider the Instagram post from Comedy Central's 
\emph{The Daily Show} host Trevor Noah identifying how
framing can drastically change the way ideological presented in Figure~\ref{fig:TrevorNoahFraming}. In this example, an anchor from the right-wing outlet NewsMax
reports that a high school yearbook was banned because it contained a feature
on Black Lives Matter, but did not contain any features on ``opposing views
like Blue Lives Matter and All Lives Matter''. Trevor Noah notes this is a 
false dichotomy---the opposite of Black Lives Matter is Black Lives \emph{do not}
matter. This example also serves as an example of how media strategies can 
foster either consensus or polarized opinions in society. By definition, if
those advocating for policing reforms to stop police killing of black people
are understood to be \emph{against} police officers themselves and policing
in general, then one will have to choose one of these polar opposites or 
another. Consensus becomes more likely when false dichotomies such as this one
are abandoned for a more nuanced, accurate representation of reality.

\begin{figure}
  \centering
    \includegraphics[width=0.5\textwidth]{Figures/TrevorNoahFraming.png}
  \caption{Political and social issues look different depending on what language
is used to describe them. Furthermore, the way a person frames a certain issue
can be used to infer their opinions or beliefs regarding political and social
topics. Here, the NewsMax anchor reveals he and his organization believe 
Black Lives Matter stands in opposition to other movements, groups, or 
slogans such as Blue Lives Matter or All Lives Matter. Noah identifies the
illogic of the NewsMax anchor's statement, revealing a misleading dichotomy.
Subtle framing strategies such as this one build on one another to develop
the sorts of polarization widely observed to day, in combination with other
factors.}
\label{fig:TrevorNoahFraming}
\end{figure}

In addition to the sort of conceptual blending and analogy-making that
characterizes metaphorical constructions, grammatical variations
and gesture or other multi-modal communication 
can result in different responses~\cite{Bennett2008,Matlock2012}. 
\citeA{Fausey2011} found that a fictional politician's personal failings and
corruption were judged more harshly if it was described using the past 
imperfect tense (i.e. \emph{VERB} + ing) compared to the past perfect tense
(\emph{VERB} + ed), since the imperfect implies such problems are ongoing.
(READ BENNETT AND ADD)



\section{Cognitive and social factors in polarization and other emergent social phenomena}

Extremism and polarization are emergent phenomena of social systems composed
of individuals. Human beings are the most fundamental components in social
systems, which we theoretically assume to have a small set of relevant
capacities essential for social interaction and 
influence~\cite{Cartwright1989,Smaldino2017}. 
Human beings are complex systems themselves,
composed of simpler cells properly organized to have the capacity for
social influence of and by others~\cite{Kello2007,Spivey2020}. Understanding
the capacities for social interaction and influence requires multi-modal
investigation spanning several disciplines undertaking computational,
behavioral, and neurobiological studies. Also important for understanding
the emergence of extremism and polarization are social factors, such as
understanding the importance of relationship networks on emergent
social phenomena.

The focus of this 
section and dissertation is understanding the consequences of the presence of
these capacities for emergent social phenomena, specifically rising extremism
and polarization. We take some time now to describe the individual-level
cognitive factors, the social factors, and then verbally sketch a model
of social influence based on the combination of these factors, including
reasoning out some of the consequences of these combined assumptions. I formalize
this model in Studies 2 and 4 to analyze group polarization and societal 
polarization, respectively. In those studies, the cognitive and social factors
below become behavioral rules for simulated individuals (``agents'') who
interact over many time steps. After several time steps group polarization
and societal polarization emerge in the form of increased extremism among
social groups, and in the form of a separation into two or more 
groups with opposing opinions.  Later in this Introduction I will define
terms such as individuals, capacities, and emergence more clearly. I will also
explain our general approach to agent-based modeling, which allows us to
explain and predict social phenomena mechanistically, a powerful tool in
scientific explanation~\cite{Machamer2000,Craver2006,Turner2020}.


\subsection{Cognitive factors in polarization}

Cognitive factors in polarization I focus on here are some essential
individual- and dyad-level capacities and processes (or ``activities'' in
\citeA{Machamer2000}) that
enable and lead to social influence of one individual by others. I review
these essential capacities and processes now. The first essential capacity
is the capacity for one's opinions to become more similar to others' opinions.
The second essential capacity is the ability to become more different from those
with whom we disagree. Whether two individuals are attracted to or repulsed
from one anothers' opinions is often determined by their group membership---people
tend to be attracted to in-group members' opinions and repulsed by out-group
members' opinions. Therefore determining one's and others' group membership is also an essential
cognitive capacity. Social influence can be modulated by one's degree of
similarity or dissimilarity to in-group or out-group members---more similar
views may be more attractive, e.g., or more different opinions more repulsive.
Individuals may also vary in their susceptibility to social influence---for
example in the model used in Studies 2 and 4 I assume those with more
extreme opinions are less susceptible to social influence, i.e., 
they are more stubborn.

Rather uncontroversially, we know that humans tend to find agreement with
one another and consensus often 
emerges within groups~\cite{Festinger1954,Cartwright1956,French1956}. 
Consensus with (or conformity to) others' opinions has been shown to emerge 
even when direct evidence contradicts those opinions, as \citeA{Asch1955,Asch1956}
found in his classic studies in which participants were fooled by confederates
into going along with the crowd despite their own direct perception that
the crowd was obviously wrong. Consensus can be problematic when
consensus occurs around, e.g., 
false scientific beliefs and 
misinformation~\cite{Zollman2007,Zollman2013,OConnor2018,OConnor2019e}.

Often in intergroup social influence, there is no drive to unite in
opinions, but instead a drive towards differencing~\cite{Tajfel1979,Sherif1988,Flache2011}. 
Group membership may be determined by observable traits such as race, language,
or style of dress, but it need not be. The ``novel group'' experimental design in
social psychology has revealed in a number of cases that group membership
as specified in a given experiment by the experimenters can override observable
indicators of group membership~\cite{Tajfel1971,Billig1973,Tajfel1982}.
These quick changes in behavior are reflected by equally quick changes in 
brain activity that apparently supports intergroup 
behavior~\cite{Cikara2014,Cikara2017}.

People often are more strongly influenced on one issue if they are already
similar with their interaction partners. Conversely, biased assimilation
also leads individuals to be more repulsed by opposing views the more 
different other views are. This is
known as \emph{biased assimilation}, somewhat equivalent to 
the principle that birds of a feather flock together~\cite{Lord1979,McPherson2001}.
In politics, individuals have been observed to be more influenced by presidential
candidates in a debate who are perceived as similar to themselves.
On large scales, it has been observed that food, hobby, and other
preferences are becoming increasingly correlated with political ideologies such
as conservativism and liberalism~\cite{DellaPosta2015}.
\citeA{Suhay2018} found that emotions may be critical: they found that
anger especially, along with other emotional states, ``fuel(ed) biased reactions
to issue arguments'' in an online behavioral study. 

The final cognitive factor in social influence I consider is that those with
more extreme opinions tend to be more stubborn, i.e.\ less susceptible to
social influence, than centrists. On the one hand, longitudinal survey studies
have found that a large portion of the population are centrists demonstrating
poor ``opinion stability over time''~\cite{Converse1964,Zaller1992,Kinder2017}. 
Extremists in the United States and United Kingdom were observed to be 
more cognitively inflexible than their centrist counterparts~\cite{Zmigrod2019a}.
Centrist opinions tend to be more susceptible to framing effects~\cite{Chong2007}
and to question ordering. Extremism has also been electrophysiologically linked
to differences in responses to stimuli. In an EEG study, \citeA{Reiss2019} found that ERP
responses to anomolies in experimental stimuli were muted among participants
with more extreme socio-political opinions compared to centrist participants.

Other approaches to studying the cognitive factors in extremism and polarization
include analyzing the correlation between personality traits and ideological
alignment~\cite{Rollwage2019}, and considering cognitive factors of social influence 
in the context of exchanging information, instead of influencing opinions~\cite{Carley1990,Carley1991,Bala1998}.
In the UK, for instance, \citeA{Zmigrod2018} found that dependence on routines was
positively correlated with subscribing to conservativism, nationalisim, and
authoritarianism, which in turn were positively correlated with support for
Brexit from the European Union. My work complements these approaches in that
it considers simpler cognitive factors than personality traits, which I see
as a composite of opinions and beliefs. Due to the complexity of the
personality trait construct, it is difficult to tell whether personality traits,
and their relationship to national-scale ideologies and policies,
are biologically or culturally determined~\cite{Claidiere2012c,Smaldino2019d},
and so possibly subject to change along with cultural context.
Some approaches to social influence of knowledge assume mechanisms for generating and
sharing knowledge, possibly under an assumption of biased 
assimilation~\cite{Mark1998,Mark1998a,Mark2003}. In other approaches, 
knowledge that one behavior is better than another is based on two
belief channels: one channel is the observation of stochastic payoffs 
received from taking one action or another; the other belief channel is 
the social one, where individuals influence one anothers' knowledge about
the system in terms of influencing their beliefs about which behavior is
more beneficial~\cite{Zollman2007,OConnor2019e}.




\subsection{Social factors in polarization}

\begin{itemize}
  \item 
    Important/relevant social factors
    \begin{itemize}
      \item 
        Unless we know someone or are in close physical or virtual proximity to someone, it is
        unlikely we will interact with them. Thus, unlike molecules in an
        ideal gas, we cannot in general assume that people in society are
        ``well mixed'', i.e., we cannot assume 
        that every person will interact with everyone else within some 
        relevant timeframe. In fact understanding
        the spread of pandemcis or predicting the resilience of
        a power grid under attack, and other important cases,
        cannot be understood unless the relationships
        between individuals are modeled as structured, as opposed to unstructured
        or random.
      \item
        In social systems there are several ways people are influenced by one
        another. People tend to choose social interaction partners who 
        are similar to themselves, known as \emph{homophily}. As homophily
        increases among a population, this
        increases the chance that individuals interact with similar others, 
        and decreases the chance that individuals will act with dissimilar
        others.
      \item
        The structure of social connections determines how long it takes, on average, for an opinion
        or other information to percolate through a society as one person
        influences another. The classic emprical study on this revealed the
        first glimpse of the ``six degrees'' phenomenon, where, on average,
        it appears anyone is related to anyone else in the world by only
        six or so intermediaries. \citeA{Milgram1967} and \citeA{Travers1969}
        sent\ldots
      \item
        Another social factor that affects social outcomes are power structures
        that result in one person having a greater social influence than others.
        This may be represented as having a greater number of relationships
        with others, so that their opinions are more widely shared~\cite{French1956,Friedkin1986}.
    \end{itemize}

  \item
    These various social factors are often represented by a social network.
    \begin{itemize}
      \item 
        Basic concepts in social network theory
        \begin{itemize}
          \item 
            Node
          \item
            Edge
          \item
            Weighted edges
          \item
            Temporal networks
        \end{itemize}
    \end{itemize}

  \item
    Abstracting social interaction structure in terms of a social network enables
    us to use mathematics and computational modeling to understand and predict subtle 
    differences between different interaction structures and emergent social
    phenomena such as extremism and polarization, infection rates, and 

  \item


    
\end{itemize}

Variations in social network structure can lead to significantly different
outcomes. For instance, in a classic model of cultural differentiation~\cite{Axelrod1997},
it seemed at first that cultural homogeneity was impossible---there seemed to
always be at least a few subcultures co-existing within a larger simulated
society. However, when . Finally, when homophily was represented
as the ability of individuals to sever ties with dissimilar others, cultural
heterogeneity was again preserved as in the original study. In another series
of studies, \citeA{Flache2011} and \citeA{Turner2018} found that small world
networks bias societies to become more polarized over time, on average, 
as long as the initial polarization is large enough. 


\subsection{A combined model of social influence based on cognitive and social factors}

Emergent social phenomena occur when individuals repeatedly interact with
partner individuals, relying on their cognitive capacities
with interactions being structured by social relationships. Before
reviewing the emergent social phenomena in the next section I 
will here first describe how social influence is hypothesized to proceed over
time assuming the cognitive capacities and social influence structures 
explained above. (SHOULD) I present a formal model based on these assumptions
that is used in Study 2 and Study 4 to understand and explain the emergent 
social phenomena of ``group polarization'' (rising extremism in small groups) 
and societal and political polarization (in larger populations). In those studies
I re-introduce the model in the context of the study. I present the model
here as a preview and as a means to conceptually integrate the cognitive and
social factors involved in emergent social phenomena of interest. Models are
simplified versions of reality we use to identify which components of
complex systems are most important in the emergence of collective larger-scale
phenomena~\cite{Wimsatt1972,Wimsatt1997,Machamer2000,Wimsatt2007,Smaldino2017}.

First, we have to establish how social influence progresses \emph{over time}.
Social structure such as polarization does not emerge after one day, for example.
It takes many days with many social interactions per day, or even per minute.
Therefore one fundamental auxiliary model component required is to propose
a time step and to decide how social influence works within each time step.


In conceptualizing social influence, at one ``time step'', every agent may
possibly interact with those whom they have a relationship. The relationships
could change over time, and could be highly transitory, such as having a 
random, thirty-second conversation with a stranger in public. One way or
another, if one could stop time and was omniscient, one could assign 
probabilities that an individual would interact with any other individual
in society, up to the global scale. For example, it is highly unlikely that
I will interact with anyone living in Mongolia because it is far away, I do not
know the language, I have no connections with any Mongolians, etc. It is
highly likely that my close friends will influence my opinions, on the other
hand.

We represent 


\section{Emergent social phenomena}

(MAKE IT ABOUT THE PHENOMENA FIRST, EXPLANATION OF IT SECOND)

Emergent social phenomena are identified by finding patterns in some 
individual-level behavior among a population~\cite{Blau1974,Schelling2006}.
The behavior in this dissertation's case of opinion (or belief, attitude, etc.) 
polarization is, e.g., the reporting of one's opinions in response to a survey
instrument. 
In the case of opinion polarization, 
individual-level opinions are reported, and polarization is calculated 
as the variance (or similar) of opinions~\cite{Bramson2016}. A totally polarized society has
exactly half of the population holding one of two extreme opinions, and the
other half holding the opposing view. 
Other behaviors that lead to emergent social phenomena 
include choosing where to live based on racial preferences (not racial animosity), 
which can result in emergent racial segregation~\cite{Schelling1971}; publishing journal 
articles of differing validity which leads to systemic scientific problems~\cite{Smaldino2019};
or writing statements and documents online that together form a system of 
cultural frames including harmful ethnic, gender, and racial biases and 
stereotypes~\cite{Caliskan2017,Garg2018}. 

There are also emergent phenomena
at smaller social scales, such as dyads and other small groups~\cite{Abney2014a}. 
For example, dyads were found to synchronize with one another when in collaborative
tasks, and ``asynchronize'' when in an adversarial 
relationship~\cite{Abney2014,Ramirez-Aristizabal2018,Schloesser2019,Schneider2020,Abney2021}.



\subsection{Collective violence metaphor usage on cable TV news}

The first emergent phenomenon we consider is the frequency of usage across
cable TV news outlets, which we assume varies depending on how soon there will
be or how recently there has been a presidential debate or the presidential 
election. Using counts of metaphor use as an observed variable 
is not a standard one like rising polarization. However, it does 
complement similar emerging approaches to studying time series of semantic content in
mass media and social media in order to understand how cognitive, cultural, and
communicative frames covary with historical 
events~\cite{Nunn2012,Klingenstein2014,Hamilton2016c,Caliskan2017,Barron2018,Garg2018}. 


\subsection{``Group polarization'': rising extremism in small, socially isolated groups}

When social psychologists observed socailly isolated groups becoming more extreme
in their opinions after deliberating some topic in a small group of likeminded
individuals, they called it \emph{group polarization}. 
This clashes with how I believe our culture generally understands the word
``polarization'', which refers to the extent to which opposing groups disagree with
and dislike one another. For better or worse, we seem to be stuck with 
\emph{group polarization} due to decades of 
use among social psychologists~\cite{Brown1986,Sieber2019},
sociologists~\cite{Friedkin1999a}, political scientists~\cite{Schkade2010}, and legal scholars~\cite{Sunstein2002}.

\subsection{Polarization}

\begin{itemize}
  \item
    Measuring polarization as variance in opinions, or in pairwise opinion differences
  \item 
    Who becomes polarized (cable TV news viewers, politicians, voters).
  \item
    Review of polarization in groups identified in previous point.
\end{itemize}

\section{Model-based theoretical approach}

One cornerstone of this dissertation is the use of formal, computational models
for explaining and predicting emergent social phenomena. We have outlined
some basic cognitive, communicative, and social factors that, together,
form the substrate from which emergent social phenomena arise, such as rising extremism
and polarization. But some questions remain that will be useful
to address before presenting the dissertation studies, since all four
studies use this model-based approach. I will try to explain 
just what mechanistic models are and how they are necessary for a robust
explanation of any object of scientific study, not just emergent social
phenomena. 

Insert abridged and expanded version of Mech Modeling for the Masses
here, possibly in a few paragraphs, or maybe just summarize in one.

This dissertation uses two major modeling 
approaches: statistical modeling and agent-based modeling. Statistical
models are often fit to data, but they need not be. To show that
many group polarization results are false, I used a statistical model
to generate counterexamples demonstrating that many published results are
plausibly false. These statistical models are mechanical in that they measure
the dynamics of violence metaphor use under the influence of presidential
debates in one case, and in the other case the statistical model encapsulates
the measurement-deliberation-measurement process in group polarization studies. 
The other modeling approach is the use of agent-based
models. Agent-based models are a way to directly specify the mechanisms
identified in a mechanistic model. Agents are simulated individuals whose
capacities within the model determine how it updates its own opinion based on
the opinions of interaction partners. Interaction structure in these models,
Who interacts with whom is encoded in and structured by a social network, that
is also a model component.

Study 1 and Study 3 both use statistical models---in Study 1 the model is fit
to observations, and in Study 3 the model generates simulated counterfactual
data. In Study 1, to partially explain the dynamics of metaphorical violence use on cable TV news,
I developed a dynamical model expressed in terms of a statistical regression
model where each news channel is in either a normal state or a 
transient excited or depressed state. I used the Akaike Information 
Criterion to identify the statistical model
that best partitions daily timestamps that most likely belong in either the
normal or excited/depressed state~\cite{Burnham2004,Burnham2011}. 
In Study 3, we use a generative 
statistical model to simulate experimental group polarization data where
pre- and post-deliberation opinions are drawn from distributions with the
same mean. By also simulating the measurement of these opinions, I show
that ceiling effects lead to a false detection of an opinion shift due to
the process of consensus that reduces group opinion variance from
pre- to post-deliberation.

Study 2 and Study 4 model different systems using the same underlying
agent-based social influence model that incorporates the cognitive and social 
factors outlined in Section~\ref{sec:cognitiveSocialFactors} above. Agent-based
models start by defining a computational representation of a person, called
an \emph{agent}. I wrote computer code that created a world in which agents
were created, made to interact with other agents according to rules and
assumptions based on the cognitive and social factors outlined above, and, 
after thousands or millions of rounds of social interaction, we can measure
the distribution of opinions to calculate either a rise in extremism or
increased polarization.
     


\section{Overview}

Now we have reviewed the overarching problem and goals of this work, the
theoretical foundation we draw on to address specific subproblems, and
the analytical approach I take to studying different emergent social
phenomena, I will now give an overview of the four Studies presented in this
dissertation.

The four Studies examine three specific subsystems in the greater social system
of elite and interpersonal communication. First, Study 1 studies and 
calculates the influence of political events, namely three US presidential
debates and the presidential election, on metaphorical violence use across
three cable TV news channels in 2012 and 2016. Study 2 focuses down from
mass communications to group-level social influence and shifts in extremism.
In that study I use agent-based modeling to show that groups may become 
more extreme when they are dragged to extreme opinions by stubborn extremists.
However, in the course of Study 2 I found that many behavioral studies of
group polarization used a problematic method for measuring group polarization.
Study 3 uses a generative statistical model to show that, indeed, over 90\% of published detections of group polarization
are plausibly false detections. Finally, in Study 4, I explore how well the
model from Study 2 could be used to explain and predict society-level
polarization. I found that long-term polarization increases with the initial
polarization; that realistic small-world networks result in higher levels of
polarization than common alternative configurations; that polarization
increases with miscommunication probability; and that
polarization is highly stochastic---the same initial configurations and
parameter settings can result in a range of predictions from low to high
levels of polarization solely due to the path dependence of interpersonal
interactions over many time steps.


\subsection{Study 1: Violence metaphors foster extremism and polarization}

(FOUR IMRAD PARAGRAPHS PER STUDY)

\subsection{Study 2: Stubborn extremism explanation of group polarization}

\subsection{Study 3: Many observations group polarization are plausibly false positives}

\subsection{Study 4: Opportunities and limits for predicting polarization}



\bibliographystyle{apacite}

\setlength{\bibleftmargin}{.125in}
\setlength{\bibindent}{-\bibleftmargin}

\bibliography{/Users/mt/workspace/papers/library.bib}

\end{document}
