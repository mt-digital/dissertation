% UC Merced  PhD Dissertation Template
% (UCSD Mathematics Dissertation Template is modified according to the UC Merced guidelines by Lasith Adhikari. Not responsible for any issues regarding modifications. You are welcome to add/update if there is any missing requirernment)
%
% Please read the comments in this file and make appropriate edits.
% NOTE: Always refer to the ``Preperation and Submission Manual for 
% Doctoral Dissertations and Masters Theses for 20**'', where 20** is 
% the year of your graduation, for officiation preparations guidelines.
%
% If you desire more control, please see the attached files:
%   * ucsd.cls -- Class file
%   * uct10.clo, uct11.clo, uct12.clo -- Configuration files for font sizes 10pt,11pt,12pt
%
% CHANGELOG:
%   * Original file adapted from brockman.tex by JRB and RMR
%     to work with ucsd.cls


\documentclass[12pt,chapterheads]{UCMerced}
% documentclass options: default is 11pt, oneside, final.
% fonts: 10pt, 11pt, 12pt -- are valid for UCSD dissertations (now UC Merced).
% sides: oneside, twoside -- note that two-sided theses are not accepted by OGS
% mode: draft, final -- draft mode switches to single spacing, removes hyperlinks,
%                       and places a black box at every overfull hbox (check these before submission).
% chapterheads -- include this if you want your chapters to read:
% Chapter 1
% Title of Chapter
%
% instead of
%
% 1 Title of Chapter


% Include all packages you need here.  Some standard options are suggested below.

% GEOMETRY - This will force the use of Letter paper.
% Many TeX installations default to A4 paper.  The formatting
% of the thesis class file requires Letter, else the margins
% will be wrong when you go to print it (and OGS will complain).
% If your TeX implementation is not setup for Letter paper, and
% you cannot change it, uncommenting the following line may fix 
% problem.
% \usepackage[paper=letterpaper]{geometry}

\usepackage{graphicx}

%% AMS PACKAGES - Chances are you will want some or all of these if writing a math dissertation.
\usepackage{amsmath, amscd, amssymb, amsthm}
% \usepackage{amssymb,amsmath}
\usepackage{multicol}
\usepackage{tabulary}
\usepackage{tabularx}
\usepackage{pgfplots}
\usepackage{longtable}
% \usepackage{url}
\usepackage{float}
\usepackage{subcaption}
\usepackage{booktabs}

\usepackage{gb4e}  % linguistic examples


%% LATIN MODERN FONTS (replacements for Computer Modern)
\usepackage{lmodern}
\usepackage[T1]{fontenc}
\usepackage{adjustbox}

%% INDEX
% Uncomment the following two lines to create an index: 
% \usepackage{makeidx}
% \makeindex
% You will need to uncomment the \printindex line near the
% bibliography to display the index.  Use the command
% \index{keyword} within the text to create an entry in the index
% for keyword.

%% HYPERLINKS
% To create a PDF with hyperlinks, you need to include the hyperref package.
% THIS HAS TO BE THE LAST PACKAGE INCLUDED!
% Note that the options plainpages=false and pdfpagelabels exist
% to fix indexing associated with having both (ii) and (2) as pages.
% Also, all links must be black according to OGS.
% See: http://www.tex.ac.uk/cgi-bin/texfaq2html?label=hyperdupdest
% Note: This may not work correctly with all DVI viewers (i.e. Yap breaks).
\usepackage[colorlinks=true, pdfstartview=FitV, linkcolor=black, citecolor=black, urlcolor=black,plainpages=false,pdfpagelabels]{hyperref}
\hypersetup{ pdfauthor = {Your Name Here}, pdftitle = {The Title of The Dissertation}, pdfkeywords = {Keywords for Searching}, pdfcreator = {pdfLaTeX with hyperref package}, pdfproducer = {pdfLaTeX}}

\usepackage{cleveref}
%% GRAPHICX - This is the standard package for including graphics for latex/pdflatex.
\usepackage{apacite}

\DeclareMathOperator*{\argmin}{arg\,min}
\DeclareMathOperator*{\obspre}{$\langle o_{pre} \rangle$}
\DeclareMathOperator*{\obspost}{$\langle o_{post} \rangle$}

\usepackage{url}
\noautomath
\begin{document}

%% REQUIRED FIELDS -- Replace with the values appropriate to you
\title{Four studies of communicative, cognitive, and social factors in extremism and polarization}
% No symbols, formulas, superscripts, or Greek letters are allowed
% in your title.

\author{Matthew A. Turner}
\degreeyear{2021}
\degree{Doctor of Philosophy} 
% Master's Degree theses will NOT be formatted properly with this
% file.

\field{Cognitive and Information Sciences}

\numberofmembers{4} % |chair|  + |othermembers| (do not count co-chair) % change here

\chair{Paul E. Smaldino}
% Uncomment the next line iff you have a Co-Chair
%\cochair{Professor Cochair Semimaster} 

\memberone{Teenie Matlock}
\membertwo{Christopher T. Kello}
\memberthree{Jeffrey Yoshimi}


\begin{frontmatter}
\makefrontmatter % The title, copyright, and signature pages.

%% DEDICATION
% You have three choices here:
%   1. Use the ``dedication'' environment.   Put in the text you want,
%   and you'll get a perfectly respectable dedication page.
%
%   2. Use the ``mydedication'' environment.  If you don't like the
%   formatting of option 1, use this environment and format things
%   however you wish.
%
%   3. If you don't want a dedication, it's not required.


\begin{dedication} % The style file will format this for you.
  To Phoebe Ruth Bernacchi-Turner.  \\[2em]
  As Daniel Tiger and his parents sing, ``We gotta look a little 
  closer to see just how things go.'' Taking care of you makes me happy, 
  too---and caring for you has pushed me to try to 
  understand what is going on between people 
  in this crazy world, which
  will maybe contribute a little bit to making the world better for you and
  any future family members. Thank you for getting me out of my own
  head and back into thinking about what I can do to make the world a better
  place. Maybe some day this will inspire you in your own pursuits of new skills and
  a deeper understanding of the world.
\end{dedication}

% \begin{mydedication} % You are responsible for formatting here.
%   \vspace{1in}
%   \begin{flushleft}
% 	To me.
%   \end{flushleft}
%   
%   \vspace{2in}
%   \begin{center}
% 	And you.
%   \end{center}
% 
%   \vspace{2in}
%   \begin{flushright}
% 	Which equals us.
%   \end{flushright}
% \end{mydedication}


%% EPIGRAPH
%  The same choices that applied to the dedication apply here.

\begin{epigraph} % The style file will position the text for you.
  \emph{Toute v\'{e}rit\'{e} et toute action \\
  impliquent un milieu et une subjectivit\'{e} humaine.}\\[2em]
  Every truth and every action \\
  emerges in the context of human culture and human subjectivity.\\[2em]

  ---Jean Paul Sartre, ``L'Existentialisme est un Humanisme'', 1946
\end{epigraph}

% \begin{myepigraph} % You position the text yourself.
%   \vfil
%   \begin{center}
%     {\bf Think! It ain't illegal yet.}
% 
% 	\emph{---George Clinton}
%   \end{center}
% \end{myepigraph}

\tableofcontents
\listoffigures  % Uncomment if you have any figures
\listoftables   % Uncomment if you have any tables


%% ACKNOWLEDGEMENTS
%  While technically optional, you probably have someone to thank.
%  Also, a paragraph acknowledging all coauthors and publishers (if
%  you have any) is required in the acknowledgements page and as the
%  last paragraph of text at the end of each respective chapter. See
%  the OGS Formatting Manual for more information.

\begin{acknowledgements} 
First, I thank my parents for providing me with many rich experiences throughout
childhood that set me on the path to being a scientist,
and for their generous help and encouragement while completing this 
PhD program.

I sincerely thank Paul Smaldino for pushing me to improve with 
more patience than could be reasonably expected of anyone. 
I entered this program unaware of my poor
sense of career direction and lack of discipline.
I thought I could find success freewheeling across
many different subdisciplines. Paul patiently peeled back the layers of
insulation over my tender ego to help me see that I had a lot of work to do
to improve to be the expert I wanted to be. At the same time Paul has
made it clear that he believes I have valuable skills that can and
should be developed further, and has helped me along immesnely with that.

My committe members have all
endured my overly grand first draft ideas and rantings and patiently steered me
towards more practical steps that helped me make real progress on 
tractable pieces of the grand ideas. Teenie Matlock (along with Paul Maglio) first 
took me on when I began the program. I appreciate Teenie's wisdom,
encouragement, and tutelage on metaphor and language, and research in general. 
Her guidance has enabled me to hear and read political discourse in a vastly more
enlightened way, even enabling me to study it scientifically, which I did not
know people did before meeting her. 
I thank Chris Kello for his support, inspiration, and encouragement 
for my grand ideas about timescales in cognitive science even bringing me in
to work with him on a computational project on the subject. I also 
vividly remember Chris giving me a thumbs up from down the hall after the 
Monday brownbag in support of my first modeling results in what became a 
journal article~\cite{Turner2018} and Study 4. I thank Jeff Yoshimi 
for two excellent courses 
(COGS 202 and COGS 269: \emph{Phenomenology})
and for generously tutoring me in the philosophy of science, 
which has been a powerful addition to my theoretical toolkit.

Finally, I thank everyone in the CIS community. These past five years have been a 
stimulating, challenging, and fun experience
that enabled me to learn about and join in the collective project of
cognitive science.
\end{acknowledgements}


%% VITA
%  A brief vita is required in a doctoral thesis. See the OGS
%  Formatting Manual for more information.
\begin{vitapage}
\begin{vita}
\item[2008] B.~S. in Mathematics \& Physics, Syracuse University, Syracuse, New York
\item[2012] M.~S. with Thesis in Applied Physics, Rice University, Houston, Texas
\item[2012-2014] Data Engineer, Economic Modeling Specialists, Int'l, Moscow, Idaho
\item[2014-2016] Research Software Developer II, Northwest Knowledge Network, University of Idaho, Moscow, Idaho
  \item[2016-2020] Graduate Teaching Assistant, University of California, Merced
  \item[2020-2021] Graduate Student Researcher,  University of California, Merced
  \item[2021] Ph.~D. in Cognitive and Information Sciences, University of California, Merced
\end{vita}
\begin{publications}
  \item 
    Turner, M.A., and Smaldino, P.E. (2021). Most group polarization results may be simple conformity. In prep.
  \item
  Turner, M. A., and Smaldino, P. E. (2021). Mechanistic Modeling for the Masses - commentary on Yarkoni, “The generalizability crisis.” \emph{Behavioral and Brain Sciences}, Forthcoming. Retrieved from https://psyarxiv.com/8pj9n.
  \item
    Smaldino, P.E., and Turner, M.A. (2021). 
    Covert signaling is an adaptive communication strategy in diverse populations.
  Under review. Preprint: https://osf.io/preprints/socarxiv/j9wyn/.
  \item 
    Turner, M.A., and Smaldino, P.E. (2020).  Stubborn extremism as a potential pathway to group polarization. 
    In \emph{Proceedings of the 42nd Annual Conference of the Cognitive Science Society}. Online.
  \item
    Smaldino, P.E., Turner, M.A., and Contreras Kallens, P. (2019). Open science and modified funding
    lotteries can impede the natural selection of bad science. \emph{Royal Society Open Science}.
  \item 
    Turner, M.A., and Smaldino, P.E. (2018).  Paths to polarization: how extreme views,
    miscommunication, and random chance drive opinion dynamics.  \emph{Complexity}.
  \item 
    Turner, M.A., Maglio, P.P., and Matlock, T. (2018).  Metaphorical violence
    in political discourse. 
    Under revision. Preprint at \url{https://osf.io/preprints/socarxiv/t8yg9/}.
\end{publications}
\end{vitapage}

%% Abstract
% There does not seem to be a maximum length. From the OGS Formatting
% Manual: ``The abstract may continue on to a second page.''

\begin{abstract}
  Rising extremism and polarization threaten democratic institutions worldwide.
  As opposing factions become more extreme in their opinions, polarization
  widens the chasm between fellow citizens, and common ground erodes, washed
  away down a river of vitriol, bitterness, and hate. What causes increased
  extremism and polarization? Due to the highly complex nature of human
  societies, this problem of explaining polarization must be broken down into
  many sub-problems, which themselves require complex systems thinking to
  address. Simplified models of social systems and rigorous analysis of empirical data,
  are necessary to build a thorough, coherent understanding of social behavior.

  In this dissertation I present my findings from studying three
  sub-problems in explaining why and how extremism and polarization emerge.
  First, I focus narrowly on a communication strategy shown in behavioral
  studies to increase extremism, \emph{metaphorical violence}, such as 
  ``Biden hit Trump over his tax returns in yesterday's debate.'' While we know
  the effects of violence metaphors, we do not understand their distribution
  in the wild, or what causes their usage to increase and decrease. I found
  that metaphorical violence use increased around the time of presidential
  debates and elections in the United States, and was correlated with 
  presidentical candidates' tweets. 

  Second, I show that rising extremism
  in isolated social gropus may be simply explained by the fact that
  extremists are more stubborn than centrists---however existing data on the
  subject is unavailable and behavioral studies on the subject may
  contain ubiquitous false detections of rising extremism. 

  Finally, I developed
  and analyzed an empirically motivated, network theoretic, agent-based model of 
  social influence at the societal level to understand how well we can 
  predict polarization based on the effects of initial conditions, network structure,
  communication noise, and random chance on predictions of polarization.

  Taken together these studies advance our understanding of communicative,
  cognitive, and social factors in the emergence of extremism and polarization.
\end{abstract}
\end{frontmatter}


%% DISSERTATION

% A common strategy here is to include files for each of the chapters. I.e.,
%   \include{chapter1.tex}
%   \include{chapter2.tex}
% etc.  Of course, if you prefer, you can just start with
%   \chapter{My First Chapter Name}
% and start typing away.  
\chapter{Introduction}


High levels of political polarization seem to bring about or go along with
hardening of partisan identites~\cite{Lee2015}.
As society becomes more polarized, political disagreement spills over
and fouls up collaborative social behavior
more generally~\cite{Iyengar2019}, even making violent responses to verbal 
communication more likely~\cite{Kalmoe2014,Kalmoe2018,Mason2018UncivilAgreementBook}.
Why do extremism and polarization increase and decrease over time in different contexts? 
This simple question yields no simple answers. These questions
have been studied for some decades now by researchers across perhaps a dozen
diverse disciplines and sub-disciplines including 
political science~\cite{Mason2018UncivilAgreementBook,Boxell2020}, 
sociology~\cite{Baldassarri2007,Flache2011}, 
economics~\cite{Schelling1971,Dixit2007}, cognitive science~\cite{Rollwage2019}, 
and philosophy~\cite{OConnor2018}.

In this dissertation I focus
on three more specific questions about increasing extremism and polarization.
These questions help us understand rising extremism and polarization, and
predict what situations will foster rising extremism and polarization.
First, what is the prevalence of communication strategies on mass media
known to increase extremism, specifically the use of \emph{violence metaphors}
to describe non-violent political events? Second, what causes observed
increases in extremism among ideologically similar groups over time?
Third, and finally, what are some fundamental cognitive capacities and
social factors that are required for polarization to occur, and what
role do random chance and miscommunication play in the emergence of
polarization? 

To answer these research questions, this dissertation focuses on the communicative, cognitive,
and social factors that provide the human substrate for rising extremism
and polarization~\cite{Jung2019,Rollwage2019}. 
In the course of this work, I identified a statistical
problem that undermines many or possibly most published results on rising
extremism among ideologically biased groups, demonstrated in Study 3.
The Studies I present in this dissertation help delineate and explore points of contact between
the various disciplinary approaches to studying polarization. As such, they
required diverse theoretical, modeling, and computational methods, 
including structured corpus building and analysis for studying metaphor
use on cable TV news, statistical modeling of the opinion generation and
measurement process, and agent-based modeling of social influence processes.

In political communication, people are differentially
influenced depending on what language is used, even down to choice of 
grammar~\cite{Matlock2012}. For example, if the past participle is used
to describe a politicians bad deeds (e.g., he was imbezzling campaign
funds) this worsens people's opinions of the politician compared to 
communicating using the simple past tense (he imbezzled campaign funds).
This dissertation focuses specifically on the choice of metaphor used 
in political discourse, which is a powerful method for framing political 
messages to rally political allies and identify, disparage, and target
political enemies and other out-group members~\cite{OBrien2003,Charteris-Black2009,Landau2010}.

There are also communication-independent cognitive and social factors at work
in social influence that leads to extremism and polarization. Cognitively,
we know that (1) similar individuals tend to find consensus with one another~\cite{French1956,DeGroot1974};
(2) dissimilar individuals can push one another to be even more 
different~\cite{Cikara2014,Bail2018};
(3) social influence between similar others tends to be more attractive the more
similar they are and more repulsive the more dissimilar they are~\cite{Lord1979,Ross2012};
and (4) that extremists tend to be more stubborn than centrists~\cite{Reiss2019,Zmigrod2019a};

Social relationships structure social interactions, determining who interacts
with whom, which is often analyzed using social
networks~\cite{Watts1999}. A major factor that determines an individual's social
network is the tendency towards \emph{homophily}, summarized in the heuristic
that ``birds of a feather flock together''~\cite{McPherson2001}. This results
in social network structures where similar individuals tend to interact more
often compared to dissimilar individuals. This can have major impacts on 
societal opinion structure and dynamics when similar individuals interact more
and more often, and dissimilar individuals sometimes come to not interact at
all~\cite{Axelrod1997,Centola2007,DellaPosta2015}.

\section{Outline}

This dissertation's modest contributions to the expansive literature on extremism
and polarization are in breaking down the complex social influence system of
society into three model sub-systems to understand and predict how communicative,
cognitive, and social factors work together to contribute to extremism and polarization
in different contexts. This has resulted in the four studies presented in this dissertation.

In Study 1 I develop a dynamic model of \emph{violence metaphor} use (e.g.,
``Clinton hit Trump over his tax returns'') on cable TV news to understand how
this changes under the influence of the US presidential debates and elections
in 2012 and 2016. Violence metaphors are important to understand because 
exposure to violence metaphors tends to increase anger towards and dislike of 
political opponents, including increased support of violence to achieve
political goals~\cite{Kalmoe2014,Kalmoe2018}.
I found that violence metaphor usage was more reactive to
the debates in 2016 and 2012, largely influenced by using violence metaphors
to describe Twitter ``attacks'' (i.e.\ tweets) by one political candidate
against the other. 

In Study 2 I focus down onto a group-level process that seems
to increase extremism among like-minded group members known as
\emph{group polarization}~\cite{Brown1986,Isenberg1986,Brown2000,Sunstein2002}.
Existing explanations of group polarization tend to rely on several auxiliary
assumptions that may or may not be well supported, which make the explanations
difficult to evaluate~\cite{Meehl1990}. I use agent-based modeling to
show that group polarization can
be more parsimoniously explained by the empirically motivated assumption 
that people become more stubborn as their opinions become more 
extreme~\cite{Reiss2019,Zmigrod2019a}. 

Unfortunately many published detections
of group polarization are plausibly false, which I demonstrate in
Study 3 using statistical models. False detections may occur in group polarization
data because researchers failed to rigorously account for floor/ceiling effects
introduced when ordinal behavioral data (e.g. Likert scale data) is analyzed
using metric statistical models. Metric statistical models fail to account for
the fact that very extreme opinions can become significantly more moderate
when measured on an ordinal scale. Therefore, metric models can be tricked into
thinking real psychological opinions have become more extreme among a group,
when in reality the ordinal measurement scale failed to detect opinion shifts
towards moderation among extremists, 
which masks a simpler process of consensus around the initial group mean opinion.
I found that 90\% of detections of group polarization across ten journal articles
are plausibly false detections, which throws into question the reality of
the group polarization.

In Study 4 I adapt the same agent-based
model from Study 2 to understand how well we can explain and pedict large-scale
societal polarization under increased 
social network connectivity as might emerge from interacting with dissimilar
others over the Internet~\cite{Bail2018}, in addition to considering the effect
of initial extremism, miscommunication, and path-dependence of social interaction
order (i.e., who interacts with whom, and when). I found that while social network
structure can bias society towards more or less polarization this is highly
path-dependent; furthermore, there are critical values of initial extremism
and communication noise that can override social network structure or path 
dependence to make either high levels of polarization or full consensus 
(i.e., no polarization) inevitable.

In the remainder of this introductory chapter I will introduce the 
twin problems of extremism and polarization, including definitions of both
terms. I then introduce the importance of metaphor and framing more generally
on rising extremism and polarization. After this, I introduce cognitive and
social factors that, together, provide the social influence substrate in which extremism 
and polarization emerge and rise. Next, I explain my strategy for modeling the emergence of 
extremism and polarization since all four studies rely on mechanistic modeling
of social influence processes. To close this chapter I give an overview of the
four studies that comprise the rest of the dissertation.


\section{Rising extremism and polarization}

Among the popular press and politically-involved citizens, it seems obvious
that polarization is increasing, and that this increase is a 
dangerous problem that needs to be solved. 
For example, when in 2014 the Pew Research Center found polarization
in 2014 to be the highest in decades, journalist and Vox founder Ezra Klein 
(2014) found the statement obvious, writing ``(E)veryone already knew that.''
\citeA{Klein2020} later explained further ``Why we're polarized.'' 
However, whether or not polarization occurs depends on how polarization 
is defined and the population being studied. 
In fact, there is a debate among political scientists whether
polarization really occurs, but this is a matter of definition~\cite{Mason2015,Lelkes2016,Kinder2017}.
Extremism and polarization must be well defined to study them scientifically.
In this dissertation, \emph{extremism} is defined by how extreme one's opinion is on some opinion 
scale, which represents how intensely or confidently someone believes in their
own opinion on some topic.  For example, giving a zero on a Likert scale often means that one 
neither agrees or disagrees with some statement. Strong disagreement or agreement
is indicated by indicated the largest negative or positive values on the Likert
scale.  \emph{Polarization} in this dissertation is 
conceptualized as the bimodality in opinions among a population, ignoring 
whether individuals are members of political parties or not. Opinion bimodality
can quantify ideological divergence among all members of society~\cite{Bramson2016,Lelkes2016}. 

At worst, extremism and political
polarization leads to political violence and even civil war~\cite{Epstein2013,Freeman2018}.
This seems true no matter what definition of extremism and polarization one
uses. However, different types of extremism and polarization may have different
effects on society and governance~\cite{Lelkes2016}. \citeA{Lee2015}, for example,
found that although partisan sorting had occurred in recent decades, there
had been little degradation in legislative and other government outcomes.
Alternatives to measuring polarization as bimodality of opinion distributions
start by assuming the existence of political parties and ideologies---the 
Republican and Democratic parties, and conservative and liberal ideologies.
Then, polarization can be measured by the degree of sorting of ideologies
into political parties~\cite{Mason2015}.  In the United States, this means polarization is measured
as the combined degree to which liberals are also Democrats, and to which 
conservatives are also Republicans. Polarization is certainly on the rise,
but this may be a correction to normal from several previous decades of 
unnaturally low levels of political sorting~\cite{Lee2015,Wood2017b}.

The rise of
\emph{affective polarization} between political groups has been found to increase as non-political
preferences align among partisans as well---for example, preferences for
leisure activities and entertainment are becoming increasingly correlated
with ideology and party membership in the United States~\cite{Pew2014PublicPolarization,DellaPosta2015}.
Affective polarization includes the increasing dislike and distrust between opposing political parties, 
which spills over into non-political areas of life~\cite{Iyengar2019}. 

While I focus on extremism and polarization in the United States, similar trends and concerns can be observed 
worldwide~\cite{Borge-Holthoefer2014,Morales2015,Romenskyy2017,Zmigrod2018}. 
We in the United States may have a particularly bad case of rising polarization: 
\citeA{Boxell2020} found that among the United States and eight other OECD countries
the United States had the largest increase in affective polarization over the
past four decades. Further comparative study is necessary to understand the fundamental human factors underlying
extremism and polarization in societies that are not Western or democratic,
with possibly lower standards of living~\cite{Henrich2010}.

Among all the ways in which individual preferences and opinions are sorted and polarized, a
major one one that both reflects and drives rising polarization is the split in
where partisans get their news~\cite{Pew2014PolarizationAndMediaHabits,Martin2017}. 
What is said on cable TV news and other mass media is extremely important, given
the reach of mass media and the way mass media frames the terms of debate~\cite{Chong2007}.
One important communication strategy is the use of different 
metaphors to frame different political
messages, processes, and events. These framings influence the way politics is understood by 
news consumers. One's opinions about immigrants, for example, may depend
on whether one has been exposed to metahpors that cast immigrants as 
``indigestible food, conquering hordes,'' or ``waste materials''~\cite{OBrien2003}.
Those who had been exposed to such metaphors may later tend to favor stricter limits
on immigration and harsher treatment for undocumented immigrants.
In Study 1, I study the change in frequency over time of a 
specific type of metaphor use, violence metaphors, across cable news channels
MSNBC, CNN, and Fox News, around the time of the United States presidential
debates and elections. Violence metaphors are important because they have
been observed to push individuals to more extreme political opinions, even
increasing support for real world political violence~\cite{Kalmoe2014,Kalmoe2018}.
Metaphorical violence is a prime strategy for inflaming partisan passions 
through statements such as ``Trump has been getting \emph{attacked} by the liberal
democrats on Capital Hill,'' which one might hear by a commentator or anchor on Fox News.

Mass media frame the terms of political discourse, which spreads through
interpersonal influence among ordinary citizens~\cite{Katz1955}. But how does 
interpersonal social influence work, and how do we know which social 
influence processes are essential for rising extremism and polarization to
emerge? We break down interpersonal influence into formal computational
models to see if polarization emerges from that model, without assuming anything
about polarization itself. 

Interpersonal influence of opinions can be broken down into
four important cognitive factors: (1) attractive and repulsive influence of
opinions~\cite{French1956,Cikara2014,Bail2018}; (2) homophily, i.e., preferential assortment with like others~\cite{McPherson2001};
(3) biased assimilation, i.e., heightened influence by similar others~\cite{Dandekar2013}; and
(4) a correlation between stubbornness and extremity of opinions~\cite{Reiss2019,Zmigrod2019a}. 
Other cognitive factors may include, e.g., personality traits that may be predictive of 
ideological or other opinion, attitude, belief, etc., preference~\cite{Zmigrod2018}.

Social factors that modulate social influence processes are: (1) social
networks, theoretical entities that represent a person's social relationships
that structure who in a society interacts, when; and (2) the stochasticity
of interpersonal influence, e.g., three people may frequent a certain bar
and talk regularly on Fridays, but who attends varies depending on essentially
random factors like other obligations or obstacles to attending. 
In Study 2 I incorporated these cognitive and social factors into a 
computational model, which led to the simulated emergence of 
``group polarization'', the empirical 
observation that socially isolated, initially biased groups tend to become
more extreme in their opinions over time. In Study 4 I examine critical
tipping points of the model that might guarantee polarization or consensus and
explore the limits of using this model (or any model) for predicting 
rising extremism and polarization.

I conceptualize polarization as an emergent property or phenomenon of society. 
Like the field of psychology generally, understanding and predicting extremism
and polarization requires cross-disciplinary understanding that sometimes
involves scientists working focusing on one system component and sometimes has 
scientists exploring the interfaces between system components~\cite{Brewer2013,Rollwage2019}.
By assuming extremely general things about how social influence works in society, it is
possible to encapsulate many of the dimensions along which individuals are
separated and how social influence regarding opinions (or beliefs, attitudes, etc.) on one dimension
is correlated with social influence along other dimensions. We can add 
details to such a general model as necessary to understand, for example, 
how framing strategies change over time on cable TV news (Study 1) or how
extremism rises in initially biased, socially isolated groups (Study 2). 
In Study 4, I use a general model of social influence to 
investigate the role of social network structure, initial extremism/polarization,
communication noise, and random ``path dependence'' on the order of 
interpersonal interactions on the emergence of polarization.


\section{Metaphor and framing in politics and polarization}

In the first study of this dissertation, I analyze violence metaphor use on
cable news around the times of the 2012 and 2016 United States presidential
debates and elections. But what is metaphor? How is it used and what are the
effects of metaphor use in political communication, especially regarding
political polarization? Metaphor has long been recognized as an important 
element in the study of political 
communication at least since Aristotle's time if not before.
Aristotle saw metaphor as a special 
feature of especially talented orators' rhetoric~\cite{Aristotle1965,Kirby1997}. 
In contrast, modern cognitive understanding of metaphor recognizes the
ubiquity of metaphor as a critical cognitive tool evolved for 
conceptual scaffolding used for abstract 
thought~\cite{Lakoff1980,Heyes2018a,HeyesCognitiveGadgets}.  

The word metaphor
comes from the ancient Greek word \emph{metaphora} 
($\mu \epsilon \tau \alpha \phi o \rho \acute{\alpha}$), meaning
\emph{transferrence}.  Metaphor works by ``transferring'', or mapping in
the mathematical sense, conceptual entailments
from a more concrete concept, such as a fight, onto a more abstract concept,
such as politics~\cite{Regier1996,Kovecses2010a,Lakoff2014}. 
Politics is an abstract concept because it can describe many
different situations, events, and processes. One never directly sees or feels
politics. The outcomes of political decision are only felt indirectly in terms
of increased or restricted liberty, or economic effects such as tax breaks or
an improved economy.  On the other hand, either being engaged in or observing physical conflict
results in a cascade of immediate bodily effects, including body-to-body contact and
possibly injury for fight participants. The conceptual entailments of this
violence metaphors include the
fact that politics generates similar feelings to being in a fight, for example, 
including the physical sensations of elation or depression following a political win
or a political defeat, and the adrenaline (and other biophysical repsonses)
of the fight itself~\cite{Gallese2005,David2016}.

Metaphor is one of several forms of linguistic \emph{framing} that can 
powerfully influence our understanding of political
events through strategic, pragmatic choice of 
words~\cite{Fillmore1982,Chong2007,Lakoff2008,Charteris-Black2009,Fausey2011,Matlock2012,Sagi2013a,Cacciatore2016}.
Embodied metaphors enable people to gain intuition about many different
abstract concepts beyond politics. 
For example, we talk about ``navigating'' the internet, but this is 
really just typing and clicking links or buttons~\cite{Matlock2014}.
We often describe the passage of time in terms of physical motion, as in,
``my dissertation defense date is fast 
approaching''~\cite{Matlock2005,Nunez2012,Flusberg2017a}.
Embodied concepts such as rotations and extrusions permeate the abstract realm
of mathematics~\cite{Lakoff1997,Marghetis2013}. 

Politicians and commentators have long used metaphor to motivate
supporters and villify opponents~\cite{Charteris-Black2009}. 
To take a current example, Fox News has
recently been covering what they call ``Classroom Warfare'' over
Critical Race Theory~\footnote{Humorously summarized by The Daily Show here:\url{https://www.youtube.com/watch?v=7sGK33uTOpU}}.
Anchors and commentators on Fox News variously cast anti-Critical Race Theory
protestors as ``an army of moms and parents'' waging war ``on the front lines
of this fight.'' To understand the purpose and effectiveness of this metaphorical framing, 
we have to consider the \emph{entailments} that go
along with the \textsc{war} conceptual frame. Wars have at least two combatants---
soldiers for each side are literally mortal enemies. 
In the context of American politics and to Fox News viewers, there
are conservatives on the Fox News side and liberals on the other side. 
Patterns in news consumption reflect this, with Donald Trump
voters watching Fox News far more than any other oultet in 2016, and 
Hillary Clinton voters watching MSNBC and CNN more than any other 
outlet~\cite{Prior2013,Pew2014PolarizationAndMediaHabits,Pew2017TrumpClinton}.

Beyond the partisan ``wars'' that play out in the minds of American citizens
based on what they see on American cable news, there are many
other political issues and events in the USA and abroad are 
described and understood using metaphorical language. For instance, metaphors were
used to cast Saddam Hussein as a madman and the United States as ``givers''
of freedom to Kuaiti citizens, which was metaphorically ``taken away''
with the Iraqi invasion~\cite{Lakoff1991}. 
This is a metaphor since freedom
is not a thing one can give, receive, or take away, 
as one would give another person water or food.
War metaphors for addressing the climate crisis seem to foster a greater
sense of urgency for finding solutions to the crisis~\cite{Flusberg2018}. 
War metaphors have also been ubiquitous in attempts to mobilize public 
responses to mitigate the spread of COVID-19~\cite{CastroSeixas2021}.

In the French and UK presses, metaphor use was observed to vary depending on the
political context: when Obama won the 2008 US election,
the UK and French presses framed Obama's victory as something predestined,
casting Obama as a sort of savior ushering in a new era of US politics, 
saying things like ``Obama walked on water.'' However, reporting on politics in Pakistan,
when former Pakistani General and retiring President Pervez Musharaf's party 
lost in Pakistan's presidential elections, UK and French news outlets 
called it a ``knockout'' and generally used other violent and disparaging
metaphors against the former president, despite him not even being a 
candidate~\cite{Burnes2011}. Similarly, the metaphor casting Washington, D.C.,
as a swamp has been used over the years
by both liberals and conservatives to cast the other side 
as dirty and corrupt~\cite{Burgers2019}.

Empirical data from behavioral studies supports the inference
that violence metaphors could contribute to readers and 
listeners' to resort to real world violence to attain their political goals.
In a series of studies, \citeA{Kalmoe2014} and \citeA{Kalmoe2018} showed that
exposure to violent metaphors drove partisans further apart in terms of opinions, and 
exacerbated aggressive tendencies towards one's political out-group. These effects
were most pronounced for the most aggressive members of society. Clearly violence
metaphors need to be understood due to their possibly detrimental 
effects on political and social stability. This need is a major motivation for
Study 1 that measures the dynamics of metaphorical violence usage on cable
TV news and finds it to be correlated with candidate Twitter activity---this
correlation is amplified in 2016 compared to 2012.


\section{Cognitive and social factors in extremism and polarization}

Extremism and polarization are emergent phenomena of social systems composed
of individuals. Human beings are the most fundamental components in the
models of social systems I use in this dissertation. 
But humans are complex themselves~\cite{Kello2007,Spivey2020},
a composite of simpler cells properly organized to have the capacity for
social influence of and by others, among many other capacities. 
So, how can humans be treated as fundamental?
The modeling strategy that solves this is to assume humans 
have only a small set of critical
capacities essential for the social interaction and 
influence that leads to extremism and polarization~\cite{Cartwright1989,Smaldino2017}. 
Understanding the capacities for social interaction and influence requires multi-method,
investigation spanning several disciplines in the form of computational,
behavioral, and neurobiological studies. Also important for understanding
the emergence of extremism and polarization are social factors, such as
the effect of social relationship networks on emergent
social phenomena. While communication is essential to increasing extremism and
polarization, much can be understood about cognitive and social factors
in polarization, indpendent of specific details about interpersonal 
communication.



\subsection{Cognitive factors in polarization}

Cognitive factors in polarization I focus on here are some essential
individual- and dyad-level capacities and processes that
enable social influence of one individual by others. The first essential capacity
is the capacity for one's opinions to become more similar to others' opinions.
The second essential capacity is the ability to become more different from those
with whom we disagree. Whether two individuals are attracted to or repulsed
from one anothers' opinions is often determined by their group membership---people
tend to be attracted to in-group members' opinions and repulsed by out-group
members' opinions. Therefore determining one's own and others' group membership is also an essential
cognitive capacity. Social influence can be modulated by one's degree of
similarity or dissimilarity to in-group or out-group members---more similar
views may be more attractive, e.g., or more different opinions more repulsive.
Individuals may also vary in their susceptibility to social influence---for
example in the model used in Studies 2 and 4 I assume those with more
extreme opinions are less susceptible to social influence, i.e., 
they are more stubborn.

We know that humans tend to find agreement with one another and consensus often 
emerges within groups~\cite{Festinger1954,Cartwright1956,French1956}. 
Consensus with (or conformity to) others' opinions has been shown to emerge 
even when direct evidence contradicts those opinions, as \citeA{Asch1955,Asch1956}
found in his classic studies in which participants were fooled by confederates
into going along with the crowd despite their own direct perception that
the crowd was obviously wrong. Consensus can be problematic when
consensus occurs around, e.g., 
false scientific beliefs and 
misinformation~\cite{Zollman2007,Zollman2013,OConnor2018,OConnor2019e}.

Often in intergroup social influence, members from different groups
develop more different opinions over time when they interact, instead of becoming more 
similar~\cite{Tajfel1979,Sherif1988,Flache2011,Bail2018}.
Group membership may be determined by observable traits such as race, language,
or style of dress, but it need not be. The ``minimal group'' experimental design has 
been used to design experiments that revealed that novel group membership
specified by experimenters can almost immediately override observable
indicators of group membership~\cite{Tajfel1971,Billig1973,Tajfel1982}.
These quick changes in behavior are reflected by equally quick changes in 
brain activity, showing that neural responses to group membership are
extremely plastic~\cite{Cikara2014,Cikara2017}. This is both a problem and
an opportunity---it is a problem because people can be quickly hijacked to
see their neighbors as ``other'', but an opportunity because people can be
equally quickly converted to more prosocial behaviors, such as mitigating
climate change, if they feel they are part of a group.

People often are more strongly attracted to others' opinions the more similar
they are. That is, we tend to adopt our friends' opinions more readily than
strangers' opinions because we know we agree with our friends on several other
issues or topics. Conversely, individuals are often
more repulsed by opposing views the more different other views are. For 
example, if someone we dislike buys a car, we will maybe be less likely to buy the
car brand in the future. This is
known as \emph{biased assimilation}~\cite{Lord1979}.
In politics, individuals have been observed to be more influenced by presidential
candidates in a debate who are perceived as similar to themselves.
On large scales, it has been observed that food, hobby, and other
preferences are becoming increasingly correlated with political ideologies such
as conservativism and liberalism~\cite{DellaPosta2015}.
\citeA{Suhay2018} found that emotions may be critical: they found that
anger especially, along with other emotional states, ``fuel(ed) biased reactions
to issue arguments'' in an online behavioral study. 

The final cognitive factor in social influence I consider is that those with
more extreme opinions tend to be more stubborn, i.e.\ less susceptible to
social influence, than centrists. Evidence from several fields supports
this assumption. On the one hand, longitudinal survey studies
have found that a large portion of the population are centrists demonstrating
low ``opinion stability over time''~\cite{Converse1964,Zaller1992,Kinder2017}. 
Extremists in the United States and United Kingdom were observed to be 
more cognitively inflexible than their centrist counterparts~\cite{Zmigrod2019a}.
Centrist opinions tend to be more susceptible to framing effects~\cite{Chong2007}
and to question ordering~\cite{Zaller1992}. Extremism has also been electrophysiologically linked
to differences in responses to stimuli. In an EEG study, \citeA{Reiss2019} found that ERP
responses to anomolies in experimental stimuli were muted among participants
with more extreme socio-political opinions compared to centrist participants.

Other approaches to studying the cognitive factors in extremism and polarization
include analyzing the correlation between personality traits and ideological
alignment~\cite{Rollwage2019}, and considering cognitive factors of social influence 
in the context of exchanging information, instead of influencing opinions~\cite{Carley1990,Carley1991,Bala1998}.
In the UK, for instance, \citeA{Zmigrod2018} found that dependence on routines was
positively correlated with subscribing to conservativism, nationalisim, and
authoritarianism, which in turn were positively correlated with support for
Brexit from the European Union. My work complements these approaches in that
it considers simpler cognitive factors than personality traits, which I see
as a composite of opinions and beliefs. Due to the complexity of the
personality trait construct, it is difficult to tell whether personality traits
and their relationship to national-scale ideologies and policies
are biologically or culturally determined~\cite{Claidiere2012c,Smaldino2019d,Falandays2021},
and so possibly subject to change along with cultural context.
Some approaches to social influence of knowledge assume mechanisms for generating and
sharing knowledge, possibly under an assumption of biased 
assimilation~\cite{Mark1998,Mark1998a,Mark2003}. In the social epistemology approach, 
knowledge that one behavior is better than another is influenced over two
channels: one channel is the observation of stochastic payoffs 
received from taking one action or another; the other influence channel is 
social, where individuals influence one anothers' beliefs about which behavior is
more beneficial~\cite{Zollman2007,OConnor2019e}.


\subsection{Social factors contributing to polarization}

If we want to understand how opinions change under social influence in the
media and societal system outlined above, we must also consider
social relationships. Social polarization and its opposite, consensus, strongly
depend on who interacts with whom, and when~\cite{Flache2008,Turner2018}. 
But what determines these social relationships and how do these relationships
change over time? How do we model these relationships 
scientifically? 

Who we interact with is somewhat
random and out of our control: it depends on our family membership, geographic location, 
participation in social activities (e.g.\ attending school, getting
groceries, going to a restaurant), etc. In addition to these random factors, 
we also adjust our social relationships based on interpersonal affinity and similarity, 
i.e., we tend to prefer to interact with people we like and avoid people we dislike.
Thinking of these
evolving relationships as a social network, modeled by a mathematical graph,
enables us to formally represent social relationships and harness graph
theory to calculate and predict social facts and behavior. For example, graph
theory can help us predict how quickly information~\cite{Milgram1967,Travers1969}, 
disease~\cite{Salathe2010,Block2020}, violence~\cite{Epstein2002}, and
innovations~\cite{Deroiain2002,Acemoglu2011a,Kreindler2014}
spread in groups and in society~\cite{Milgram1967,Travers1969,Watts1999,Palla2007,Backstrom2012,Wohlgemuth2014}.

In social systems people tend to choose social interaction partners who 
are similar to themselves, a tendency known as \emph{homophily}. As homophily
increases among a population, this
increases the chance that individuals interact with similar others, 
and decreases the chance that individuals will act with dissimilar
others~\cite{McPherson2001}. Homophily, then, amplifies the cognitive factor of biased assimilation,
since increased homophily tends to further insulate individuals from
exposure to opposing viewpoints as biased assimilation causes individuals to 
ignore or reflexively dislike opposing viewpoints and uncritically incorporate
information that supports their pre-existing opinions~\cite{Mark2003,Dandekar2013}.
Another social factor that affects social outcomes are power structures
in which some people have a greater social influence than others.
This may be represented as having a greater number of relationships
with others, so that their opinions are more widely shared~\cite{French1956,Friedkin1986},
or due to social status, or both~\cite{DeGroot1974}.

To analyze social influence and social relationships, we can use a social 
network to represent the relationships between individuals
in groups or societies. 
Social networks are based on mathematical graph theory. Study 4 uses a network-theoretic
model of social influence to understand how social networks contribute to
the emergence of extremism and polarization---the model is formally 
introduced there. Individuals in a social network are represented by \emph{nodes},
often drawn as dots or some other marker. Nodes can encapsulate an individual's
identity in addition to traits such as group
membership, accumulated resources (i.e. ``payoffs''), etc. Sometimes these traits are visualized
by changing the marker size, color, or shape of the node in network visualizations.  Relationships
between individuals are represented by \emph{edges}, drawn as lines that connect
individual nodes.  Homophily and power differentials between individuals 
may also be represented in terms of edge \emph{weights} on the graph. Any graph
may have weighted edges which could stand for many different things; in
navigation applications edge weight might represent the time it takes to reach 
one location from another. \emph{Temporal} or \emph{dynamic} graphs are graphs that change 
over time, which in social situations could result from changing affinities,
geographic relocation, or new communication technologies~\cite{Li2017}. 
In social influence
social networks change when, for example, social ties are abandoned when interpersonal
similarity drops below some threshold~\cite{Axelrod1997,Hegselmann2002,Centola2007,Kossinets2009}.

% Whabout these? \cite{Watts1999,Macy2003,Baldassarri2007, Flache2011,DellaPosta2015,Turner2018,Stewart2020b}

\section{Mechanistic models of emergent social phenomena}

This dissertation relies on theory driven, empirically motivated mechanistic
models to simplify the complex system of human social influence. But how do
mechanistic models work, specifically for rigorous scientific investigation of
emergent phenomena such as extremism and polarization?
A hallmark of the scientific explanation of some phenomenon is that the explanation
only posits the existence of theoretical entities, entity capacities, 
and relationships between entities~\cite{Kauffman1970,Cartwright1989,Craver2006,Turner2021}. 
If the phenomenon of interest 
emerges from system dynamics specified by the entities and their capacities,
then the model and its theoretical basis have some explanatory power. 
A phenomenon \emph{emerges} when a statistical pattern is detected that is 
associated with that phenomenon, e.g., polarization is often measured as the
bi-modality of individual opinions (i.e.\ attitudes, beliefs, etc.) in a society.
The patterns of interest in this dissertation are static (polarization at a given
point in time) and dynamic~\cite{Kelso1995}, e.g.\ rising extremism and collective changes in 
violence metaphor use on cable news.
It is not valid to assume in advance the existence of the phenomenon. 

In this dissertation I use a mechanistic model-based theoretical
approach designed to (Study 1) explain observed patterns in mass media metaphorical violence use;
(Study 2) explain rising extremism in socially isolated groups; (Study 3) 
demonstrate that many or perhaps most detections of
rising extremism in socially isolated groups are false detections; and 
(Study 4) to identify critical determinants social polarization.


\subsection{Emergent social phenomena}

In this dissertation I focus on the emergence of rising extremism and polarization,
which is theoretically influenced by the emergent dynamics of metaphorical violence
use on cable news (one of many influential mass media communication strategies). 
Emergent social phenomena are identified by finding patterns in
the distributions of individual-level behaviors, opinions, traits, etc., 
among a population~\cite{Blau1974,Schelling2006}. It is challenging to
explain, with scientific rigor, how emergent social phenomena such as
rising extremism and political polarization actually 
emerge from repeated instances of social influence~\cite{Watts2011}. Social systems are
complex systems of groups of various sizes, and individual humans themselves
are complex emergent phenomena~\cite{Kello2007,Lazer2009}. 

In this work we have assumed that individuals ``have'' opinions. 
Polarization is calculated 
as the distributional variance (or similar measures of bimodality) 
of individual opinions~\cite{Bramson2016}. A totally polarized society has
exactly half of the population holding one of two extreme opinions, and the
other half holding the opposing view. 
Other behaviors that lead to different emergent social phenomena 
include choosing where to live based on racial preferences (not racial animosity), 
which can result in emergent racial segregation~\cite{Schelling1971}; publishing journal 
articles of differing validity which leads to systemic scientific problems~\cite{Smaldino2019};
or writing statements and documents online that together form a system of 
cultural frames including harmful ethnic, gender, and racial biases and 
stereotypes~\cite{Caliskan2017,Garg2018}. 

There are also emergent phenomena
at smaller social scales, such as dyads and other small groups~\cite{Abney2014a}. 
For example, dyads were found to synchronize with one another when working
together on collaborative tasks, and ``asynchronize'' when in an adversarial 
relationship~\cite{Abney2014,Ramirez-Aristizabal2018,Schloesser2019,Schneider2020,Abney2021}.
In turn, individual humans are emergent properties of a complex electrochemical
interaction of individual, differentiated cells~\cite{Schrodinger2012,Kello2007,Lazer2009}.
It is for this reason that I believe it is best to avoid thinking about
``micro'' and ``macro'' scales as seems to be popular among 
sociologists~\cite{Macy2002}. I conceptualize the assumptions
we must make about individual cognition and social interaction between dyads
as ``individual-level'' assumptions instead of ``micro-motives''~\cite{Schelling2006}.
Similarly I prefer the concept of an 
\emph{emergent phenomenon} to Schelling's concept of ``macrobehavior''.


\subsubsection{Collective violence metaphor usage on cable TV news}

The first emergent phenomenon I study is the frequency of violence metaphor
usage across cable TV news outlets. I hypothesize and show in Study 1 that
violence metaphor use varies depending on how soon there will
be or how recently there has been a presidential debate or the presidential 
election. This approach
complements similar approaches to studying time series of semantic content in
mass media and social media in order to understand how cognitive, cultural, and
communicative frames covary with historical 
events~\cite{Nunn2012,Klingenstein2014,Hamilton2016c,Caliskan2017,Barron2018,Garg2018}. 
Partisan polarization can be identified by analyzing semantic differences in partisan
communications~\cite{Gentzkow2019}.

From the complex systems perspective, ``pragmatic choice'', i.e.\ what words
to use when, is the result of many ongoing subprocesses, which occur within
different contexts~\cite{Gibbs2012a}. The collective attention of society becomes entrained on
shared cultural events~\cite{Fusaroli2015}. The utterances of news anchors, commentators, and 
pundits cannot be separated from their pragmatic purpose and societal context~\cite{Kovecses2010}.
Collective violence metaphor use, then, can be considered an emergent property of 
social systems since it depends on complex interactions between individuals
at varying time and population scales. 


\subsubsection{``Group polarization'': rising extremism in small, socially isolated groups}

\emph{Group polarization} is the name given by social psychologists to the observation
that novel, socailly isolated, collectively biased groups become more extreme
in their opinions after deliberating on some 
topic~\cite{Brown1986,Isenberg1986,Brown2000,Sunstein2002}.
In group polarization, the emergent phenomenon is the rising extremism among the group.
Furthermore, there is a higher-level emergent pattern of the magnitude of the group polarization
effect---extremsim has been observed to rise more when the group is already
relatively extreme~\cite{Myers1982}. Because of the complex interplay between individual-level
cognition and social power dynamics~\cite{Friedkin1999a}, we can identify group
polarization as an emergent social phenomenon as well. 

\subsubsection{Polarization}

Polarization may be arrived at through a varity of cognitive and social mechanisms, 
though of course communication details can exacerbate polarization as already discussed.
Polarization can increase through repulsive 
influence~\cite{Baldassarri2007,Flache2011,Bail2018,Turner2018}. When dissimilar
individuals repulsively influence one another, their opinions become more
extreme in opposite directions, marginally increasing opinion distribution
bimodality. Polarization can also increase through attractive influence only~\cite{Mas2013,Turner2020},
e.g., through group polarization.
In group polarization, isolated groups become more extreme as they find consensus. 
If one group becomes more extreme, polarization also increases
since bimodality will have increased. 

\subsection{Model-based theoretical approach}

The Studies I present in this dissertation all develop
mechanistic models of social influence and mass communications. Studies 1 and
3 implement mechanistic models as generative, computational statistical models, while
Studies 2 and 4 implement their social influence models as agent-based models
of social influence incorporating the cognitive and social factors listed above,
theorized to be important in the emergence of rising extremism and polarization.
The dissertation studies are organized based on their approach
and findings in the study of how rising extremism and polarization emerge
under mass communications and social influence, not their modeling approach.

Scientific models are simplified versions of reality used to identify which components of
complex systems are most important in the emergence of collective larger-scale
phenomena~\cite{Kauffman1970,Wimsatt1972,Wimsatt1997,Machamer2000,Wimsatt2007,Smaldino2017}.
The most explanatory models are \emph{mechanistic models}, ones that explicitly identify the atomic theoretical 
entities in a system and how those entities influence one another~\cite{Machamer2000,Craver2006,Turner2021}. 
In our case mechanistic models of societal and group systems would explain that
human individuals communicate with and influence those with whom they share
social connections, with social connections represented as social network neighbors. 
Above I listed
assumptions about how individuals process social influence and how social
interactions are structured, which are further details incorporated in the
model of social influence used in Study 2 and Study 4. 
Mechanistic models are stronger still when they are 
formalized into mathematical notation and implemented computationally to
make quantitative predictions of how different social phenomena emerge based
on model assumptions. 


\subsubsection{Models in the dissertation studies}

All four studies presented here use some form of mechanistic modeling to 
represent system dynamics that give rise to emergent phenomena. Mechanistic
models may be expressed and implemented in a variety ways. In addition to 
developing detailed verbal models of how social influence and mass
communication work, Studies 1 and 3 implement statistical models and 
fitting procedures to empirically
determine inflection points in violence metaphor dynamics (Study 1) and 
to demonstrate that a high rate of experimental detections of rising extremism
are plausibly false (Study 3). Studies 2 and 4 use 
agent-based models to understand which cognitive and social factors best explain
and predict rising extremism and polarization, respectively.

Study 1 and Study 3 both use statistical models---in Study 1 the model is fit
to observations, and in Study 3 the model generates simulated counterfactual
data. In Study 1, to partially explain the dynamics of metaphorical violence use on cable TV news,
I developed a dynamical model expressed as a statistical regression
model where each news channel is in either a normal state or a 
transient excited or depressed state.  
In Study 3, I used a generative 
statistical model to simulate experimental group polarization data where
pre- and post-deliberation opinions are drawn from distributions with the
same mean. By simulating the measurement of these opinions, I show
that floor/ceiling effects lead to a false detection of an opinion shift due to
the process of consensus that reduces group opinion variance from
pre- to post-deliberation.

Study 2 and Study 4 model different systems using the same underlying
agent-based social influence model that incorporates the cognitive and social 
factors outlined above. Agent-based models of social systems start 
by defining a computational representation of a person, called
an \emph{agent}. To implement these models, 
I wrote computer code that created a world in which computational agents
were brought to life, made to interact with other agents according to rules and
assumptions based on the cognitive and social factors outlined above. 
After thousands or millions of rounds of simulated social interaction I
measured the distribution of opinions to calculate either a rise in extremism or
increased polarization.
     

\section{Overview}

Now we have reviewed the overarching problems of extremism polarization that
motivated this work, the
theoretical foundations I draw on to address specific subproblems, and
the analytical approach I take to studying different emergent social
phenomena. I will now give an overview of the four studies presented in this
dissertation.

First, Study 1  
calculates the influence of political events on metaphorical violence use across
three cable TV news channels in 2012 and 2016. I found that significant changes
in metaphorical violence usage occur across cable news outlets MSNBC, CNN, and
Fox News in both 2012 and 2016 around the time of the presidential debates. 
In 2012, changes were significant, but rather small and there was no discernable
pattern across news outlets. In 2016, however, metaphor use increased dramatically
across networks in the run-up to the 2016 elections.  
I hypothesized that Twitter usage was an important driver of this difference due to 
a close reading of cable news violence metaphors that seemed to cast candidate
tweets as metaphorical violence. This was indeed the case, as revealed through a
correlational analysis between Twitter activity by Republican
and Democratic candidates and violence metaphor usage on cable news in both
election cycles. Correlations
in 2016 were more significant overall than in 2012. Candidate Twitter use
accounted for over 30\% of variance in overall violence metaphor use in 2016, compared to
about 8\% of variance in 2012. By understanding the
dynamics of violence metaphors on cable news, we can better understand and 
predict downstream effects of violence metaphor use, including reduced capacity
for rational thinking due to emotional overwhelm~\cite{Suhay2018} and heightened
risks of political violence~\cite{Kalmoe2014,Kalmoe2018}.

Study 2 focuses down from
mass communications to group-level social influence and shifts in extremism.
In Study 2 I use agent-based modeling to show that group polarization 
may be driven by the cognitive factor of ``stubborn extremism'', where
extremists are less susceptible to social influence~\cite{Reiss2019,Zmigrod2019a}.
To detect group polarization, researchers survey participant
opinions, assemble participant groups that are biased in one direction or another, and then
re-survey participants. If the mean of the group opinions changes, then
a group polarization opinion shift is said to occur. Existing explanations of
group polarization include potentially
problematic axuiliary assumptions that seem to lack robust empirical support~\cite{Meehl1990}.
Existing explanations also fail to explicitly account for the observation that
the magnitude of opinion shifts are positively correlated with initial group
extremism~\cite{Myers1982}. However, in Study 2 I show that the stubborn extremism explanation
also predicts this correlation between initial extremity and opinion shift.
The stubborn extremism explanation seems more parsimonious in that
it makes a single, simple, empirically valid assumption that is more explanatory
than any alternatives.

In the course of Study 2 I found that many behavioral studies of
group polarization used a problematic method for measuring group polarization.
Specifically, many group polarization studies used metric statistical models
on ordinal data, which is known to sometimes yield false inferences~\cite{Liddell2018}.
Study 3 uses a generative statistical model to show that, indeed, over 90\% of 
published detections of group polarization over ten journal articles
are plausibly false detections. 
False detections occur due to the fact that extreme opinions are mapped to
relatively moderate opinions, but this is not accounted for when using 
metric statistical models designed to detect whether two distributions of
\emph{continuous} variables are different (e.g., a \emph{t}-test). Metric
models can be tricked into detecting a difference between two distributions' means when
the data are ordinal measurements of continuous \emph{latent} psychological opinions
from two distributions with the same mean but different variances. We know that
opinion variance changes within groups as groups discuss a topic: group members
find consensus, i.e., variance decreases.

Finally, in Study 4, I explore how well the
model from Study 2 could be used to explain and predict society-level
polarization, as opposed to increased extremism in small groups in Study 2.
I found that (1) greater initial polarization often led to greater long-term polarization; 
(2) realistic small-world networks tend to result in higher levels of
polarization than common alternative configurations; (3) more miscommunication
leads to greater polarization; and (4) polarization outcomes are
highly stochastic, i.e., the same initial configurations and
parameter settings can result in a range of predictions from low to high
levels of polarization solely due to the path dependence of interpersonal
interactions over many time steps.




\chapter{Metaphorical violence in political discourse on US cable TV news}

Metaphor is far more than a literary device. It is a
fundamental cognitive ability that drives the human capacity for reasoning about
states, situations, and actions in the world~\cite{Gibbs1994,Lakoff1980}.
Metaphor---which involves understanding of abstract concepts in
terms of relatively more basic ones---permeates political
discourse~\cite{Lakoff2008,Matlock2012}. Its ubiquity in everyday discourse
is evident in the frequent use of statements such as ``It's time
to drain the swamp'', ``Obama sprinted toward victory on Election Day'', and
``Trump attacks Jeff Sessions over Russian probe methods''. No one is releasing
water. No one is running. No one is causing physical harm. How is metaphorical
violence expressed, for instance, expressions with words such as ``attack'',
``slaughter'', and ``hit'', and how does such language influence political
thought and communication? Here, we describe novel time-resolved 
observations and explanatory dynamical models of the use of metaphorical 
violence language in political discourse on U.S. cable television news in the 
period leading up to the two most recent presidential elections. 
Our results quantify the details and dynamics of the use of these metaphors, 
revealing how cable news shows act as reporters, promoters, 
expectation-setters, and ideological agents in different degrees in 
response to differing cultural situations. Our work has implications for 
shaping political discourse and influencing political attitudes. 	

\section{Introduction}

Conceptual metaphor theory holds that linguistic
metaphors, such as ``Costs are rising,'' reflect a process whereby one concept is
structured in terms of another; in this case, costs are conceptualized in terms
of physical verticality. In this way, metaphor is not just language; it is a way
of thinking~\cite{Gibbs1994,Lakoff1980,Thibodeau2011}  and it is intimately
linked to emotions \cite{Kovecses2010} and grounded in bodily
experience~\cite{Gallese2005}. Because metaphor is so pervasive and because many
people care about political matters, it is useful to consider how it is used and
how it might shape public opinion on matters of national or international
importance, such as climate change~\cite{Flusberg2017} and
politics~\cite{Lakoff2008,Lakoff2012}. 

We are especially interested in violence
metaphors in the context of political discourse. We define \emph{violence
metaphors} as those that portray political concepts in terms of physical
violence. Consider two statements from cable TV news in 2012 that both feature the
word ``attack'':
\begin{exe} \ex Because we want you to pay for your own birth control,
  that's an \emph{attack} on your womb like we're flying a predator drone over your
  fallopian tubes and calling in a strike?\footnote{Adam Carolla on \emph{The
  O'Reilly Factor}, FOX News, September 10, 2012; \url{https://goo.gl/jVBsqH}}
  \label{ex:carolla} 
  \ex John McCain and his allies have been trying to turn the
  Benghazi attacks into a political scandal for the president since
  September.\footnote{Chris Matthews on \emph{Hardball with Chris Matthews},
  MSNBC, November 15, 2012; \url{https://goo.gl/Pfs4Sc}}
\label{ex:benghazi-attacks} \end{exe} The first statement refers to political
efforts to force employers to provide insurance that covers birth control for
women.  It is metaphorical because the
womb is not physically assaulted. Here there is a mapping from a \emph{source domain} of  violence,
associated with bodily harm, wars, battles, etc., to  a \emph{target domain} of
argumentation, in this case, about who should pay for birth control. The second
refers literally to a terrorist attack in the town of Benghazi;
clearly, it is not metaphorical. 

Metaphor can heighten emotions in political
communication~\cite{Charteris-Black2009}. Reporters seem to understand this
well.  They use metaphor to draw attention and create a reaction in readers and
listeners~\cite{Lakoff2008}. Americans have long been fascinated by the
political theater afforded by television, and, over time, the media has come to
frame debates as violent events~\cite{Schroeder2008}. The trend toward increased
spectacle and competitive framing continues; 
for instance, political campaigns are often portrayed as
military campaigns (see Burnes, 2011, and Kalmoe, 2014). In the U.S. and
elsewhere, political contests are now routinely conceptualized in terms of
physical actions, often taken against another, such as footraces
(see Matlock, 2013) or battles (see Flusberg, Matlock, and Thibodeau, 2018). 
Importantly, using metaphorical violence in political discourse has 
real consequences on reasoning; for instance, it can increase the tendency to polarize
(see Kalmoe, Gubler, and Wood, 2018).

The influence and diverse range of ideological perspectives of U.S. cable
television news make it an important system to understand.  Interested in how
metaphorically violent language would vary in reportage around debates leading
up to a U.S. presidential election, we analyzed language used on the
most-watched cable television news networks MSNBC, CNN, and Fox
News~\cite{OConnell2017}. CNN and MSNBC are on the progressive end of the
ideological spectrum, and Fox News, the conservative end~\cite{Pew2014}. Right
before the 2016 presidential election, 40\% of Trump voters said Fox News was
their primary source of news, whereas 27\% of Clinton voters said theirs was 
MSNBC or CNN~\cite{Pew2017TrumpClinton}. For our analysis, we analyzed the use of metaphorical violence language on two
different shows from each of these three networks during September 1 to November 30 in 2012
and 2016, periods in which four major political events occurred: three
presidential debates and election day.

The main questions of interest concern how metaphorical violence was used leading
up to election day: Which networks produce the most
metaphorical violence language? Is this consistent across years? What is the
contribution of each show to total use?  What is the difference in how often
metaphorical violence language is used, and does this change across networks or
years? Who is conceptualized more often as attacking and being attacked by
metaphorical violence, and does this change across networks and time? In
addition to revealing details of the use of metaphorical violence language on
cable television news, informing the study of political communication and
action, our results provide data for understanding a deep question in cognitive
linguistics: to what extent and how does the cultural context influence which
metaphors are used~\cite{Gibbs1997,Kovecses2010}? 

Our main results are a series of observations about the use of metaphorical
violence language across different cultural situations and by different cultural
actors. We expected metaphors to change over time in response to, or in
anticipation of, the cultural events of the presidential debates and election
day, and on the specific actions taken and language used by the candidates
themselves. We also expected metaphor use to differ across the three networks
given their differing ideologies~\cite{Lakoff2008}, though we also expected some
similarities across networks, because of shared cultural frames \cite{Kovecses2010}. 

To address these questions, we collected data
from the Internet Archive's TV News Archive (TVNA), a curated
library containing millions of short video clips from cable television news
shows from the last decade. We collected data from the two most highly rated
news shows on each network in each of the two study years. We relied on closed caption data provided by the TVNA
to create textual transcripts of each show, and searched each transcript for
words that signal, or instantiate, the source domain of violence, the
\emph{violence signal}. We considered only phrases that use one of three
violence signals---\emph{attack}, \emph{beat}, or \emph{hit}. If a violence
signal was found in an episode of a show, a human reviewer then manually decided
whether it represented metaphorical violence based on the context, annotating
the text to identify subject, verb, and object of the phrase for all uses of
metaphorical violence.  Analyzing subject and object allowed us to determine who
was portrayed as the aggressor and who was portrayed as the victim.

\section{Methods}
\input{/Users/mt/workspace/Papers/metvi/methods.tex}

\section{Analysis}

Overall, we observed 758 uses of metaphorical violence language in 2012, and 583
in 2016. In 2012, the MSNBC show \emph{Hardball} alone contained 208
metaphorical violence uses, whereas other MSNBC shows ranged from 60 to 120.
Shows on CNN were more consistent, ranging from 99 to 118, as were shows on Fox
News, ranging from 130 to 150. The distribution of specific violence signals
across networks and shows was similar in both 2012 and 2016: \emph{attack} was
used most, \emph{beat} next most, and \emph{hit} least. Interestingly, in 2012
MSNBC led in total metaphorical violence language use, and \emph{hit} and
\emph{beat} were used more often than \emph{attack} on that network.
In both 2012 and 2016, the
Republican candidate was both the aggressor and victim of metaphorical violence
more often than the Democratic candidate. In 2016, Trump was characterized as
doing metaphorical violence 102 times by Fox News, compared to 30 times for
Clinton. This finding is consistent with other research that suggests
conservatives more often conceptualize interpersonal relationships in terms of
violence or are more likely to resort to violence in interpersonal relationships
than progressives~\cite{Lakoff1996, Cohen1996}. 

To compare the dynamics and time-course of metaphorical violence use across the
different networks, we modeled change in frequency of use as an impulse function
with two states (Equation 1). We fit our dynamical
model for six time series, one for each network in each study year. Bayesian
multi-model inference allowed us to identify the best-fit model and to quantify
the relative likelihood of other parameterizations being better (all best-fits were
significant). We next calculated change in relative frequency of metaphorical
violence use, or \emph{delta}, across networks, violence signals, and clausal
subject and object (Equation 2). We found both positive and negative values for \emph{delta},
meaning that metaphorical violence language did not increase uniformly within
the study period across networks and years. Fox News and CNN had negative
\emph{deltas} in 2012. In the case of Fox News in 2012, metaphorical violence
language decreased starting September 9 and ending September 25, the days
leading up to the first presidential debate on October 3. CNN's use dipped after
November 6, election day. In 2012, MSNBC was the only network with a positive
change, starting on September 13 and ending September 27, just before the first
debate. In 2016, \emph{delta} was positive and larger in magnitude for all three
networks, with the start date of the elevated state overlapping to a much
greater degree (see Figure~\ref{fig:ModelFits}). This reflects the differences in cable news
viewership between 2012 and 2016: 67.2 million watched the first
Obama-Romney debate in 2012 compared with 84 million for the first Clinton-Trump
debate in 2016 \cite{Perlberg2016}.  Part of this broad, synchronized excitement about 
the election 	may have been because of the big personalities of the two main contenders: 
Clinton was the first woman candidate and a controversial
first-lady.  Candidate Trump was a rich, controversial television star.

\begin{figure}[H]
  \vspace{.2in}
  \centering
  \begin{subfigure}{0.7\linewidth}
    \centering
    \includegraphics[width=\textwidth]{/Users/mt/workspace/Papers/metvi/Figures/ModelFits-2012.pdf}
   \caption{\quad2012}
    \label{fig:ModelFits-2012}
  \end{subfigure}
  \begin{subfigure}{0.7\linewidth}
    \centering
    \includegraphics[width=\textwidth]{/Users/mt/workspace/Papers/metvi/Figures/ModelFits-2016.pdf}
   \caption{\quad2016}
    \label{fig:ModelFits-2016}
  \end{subfigure}

  \caption{Observed daily frequencies (markers) and best-fit models (lines).
    The dynamical impulse model is given in 
    Equation 1. In four of the six network-year pairs, 
    there is an increase in the frequency of metaphorical violence language in the
    three-month study period: MSNBC in 2012 and all three networks in 2016. 
    However, two of the six network-year pairs showed decreases in frequency
    of metaphorical violence language use in one three-month period: CNN and Fox News
    in 2012. 
  } 
  \label{fig:ModelFits}
\end{figure}

MSNBC's positive \emph{delta} in 2012 resulted mainly
from an increased use of the signal \emph{attack}. There was no change in use of
the signal \emph{hit} on MSNBC. CNN's use of \emph{hit} and \emph{attack}
decreased by about 80\%. On Fox News in 2012, most of the decrease in overall
metaphorical violence use resulted from decreases in the use of \emph{hit} and
\emph{beat}, with \emph{attack} use remaining nearly constant. In 2016,
\emph{deltas} were positive for all networks. All \emph{deltas} were positive
for violence signals as well, with one exception: MSNBC's use of \emph{hit} fell
by 63\%. MSNBC's use of \emph{beat} and \emph{hit} increased by a factor of
almost 2. In 2016, CNN's use of \emph{attack} accounted for most of its overall
increase in metaphorical violence language use, and for Fox News, use of
\emph{attack} increased by nearly 300\%. 

\begin{table}
  \centering
  \bgroup
    \begin{subtable}{\textwidth}
      \centering
      \begin{tabular}{lr}
        \toprule
        Show (Network) & Total Uses \\
        \midrule
        The Rachel Maddow Show (MSNBC) & 93 \\
        Hardball With Chris Matthews (MSNBC) & 208 \\
        Anderson Cooper 360 (CNN) & 99 \\
        Piers Morgan Tonight (CNN) & 118 \\
        The O'Reilly Factor (Fox News) & 141 \\
        Hannity (Fox News) & 133 \\
        \bottomrule
      \end{tabular}
      \caption{Total number of uses metaphorical violence language by news show in 2012}
      \label{tab:by-show-2012}
    \end{subtable} \\  \vspace{1.5em}
    \begin{subtable}{\textwidth}
      \centering
      \begin{tabular}{lr}
        \toprule
        Show (Network) & Total Uses \\
        \midrule
        The Rachel Maddow Show (MSNBC) & 66 \\
        The Last Word with Lawrence O'Donnel (MSNBC) & 80 \\
        Anderson Cooper 360 (CNN) & 100 \\
        Erin Burnett OutFront (CNN) & 118 \\
        The O'Reilly Factor (Fox News) & 146 \\
        The Kelly File (Fox News) & 148 \\
        \bottomrule
      \end{tabular} \quad
      \caption{Total number of uses metaphorical violence language by news show in 2016}
      \label{tab:by-show-2016}
    \end{subtable}
  \egroup
  \caption{Total uses by show in each of the two study years}
  \label{tab:by-show}
\end{table}

In 2012, the candidates were involved in less of the metaphorical violence than
in 2016 (Table \ref{tab:subjobj}). Two of the three networks showed a decrease in overall metaphorical
violence use at some point in the three-month study period in 2012. Even the
increase in use on MSNBC was not as pronounced in 2012 as it was in 2016, with a
\emph{delta} of  0.57 in 2012 and 1.40 in 2016. Changes in frequency of
metaphorical violence language use were uniformly positive and larger in
magnitude in 2016, beginning before the first presidential debate (September 26)
and ending soon after the last debate (October 19). CNN and Fox News showed
increased frequency on the same day, September 9, decreasing a few days apart,
October 27 for CNN and October 22 for Fox News. MSNBC's frequency of the use of
metaphorical violence language rose later, on October 8, but decreased around
the same time as the other networks, on October 26. 

% \begin{longtable}
\begin{table}
  % \vspace{.25in}
  \centering
  \begin{subtable}{\linewidth}
    \centering
    \input{/Users/mt/workspace/Papers/metvi/Table3-2012.tex}
    % \input{Table3-2012.tex}
    \caption{\quad 2012}
    \label{tab:subjobj-2012}
  \end{subtable}
  
  % \vspace{.25in}

  \begin{subtable}{\linewidth}
    \centering
    \input{/Users/mt/workspace/Papers/metvi/Table3-2016.tex}
    % \input{Table3-2016.tex}
    \caption{\quad 2016}
    \label{tab:subjobj-2016}
  \end{subtable}

  \caption{Uses and \emph{delta} for Republican and Democratic candidates as
  subject and object of metaphorical violence.}
  \label{tab:subjobj}
\end{table}
% \end{longtable}


\section{Discussion}

What might have caused the difference in timing and magnitude of changes in
the level of metaphorical violence usage between 2012 and 2016? Fox News' 
decrease in metaphorical violence usage preceding the first debate seems to
fit with the traditional role of lowering passions and expectations, and 
casting one's preferred candidate as the underdog \cite{Schroeder2008}. 
This explanation is supported by the content of the metaphors themselves. For example, on
the September 17 episode of \emph{Hannity}, Sean Hannity said, 
\begin{exe}
  \ex I want to see Romney {\em hit} harder. I want to see him \ldots take it right to (Obama).
\end{exe}
A panelist followed up, telling Hannity, ``If Romney had your passion, if
Romney had your intelligence, he would have a shot.''\footnote{https://goo.gl/mc8aXk}
Then the next day, a contributor on \emph{The O'Reilly Factor} said,
\begin{exe}
  \ex Romney is not projecting strength. He put out a statement, he got {\em attacked}, 
   and he crawled into a hole. He should have kept moving forward with what he was saying.\footnote{https://goo.gl/HJ6BWu} 
\end{exe}
The underdog strategy worked: Romney enjoyed a boost in the polls
after the first debate. Before the debate, only 29\% of survey respondents
expected Romney would ``do a better job'' in the debate. After the debate,
72\% of those who watched it thought Romney did a better job \cite{Pew2012}.

The increase in MSNBC's usage of metaphorical violence began September
13 and continued for two weeks in response to a statement made
by Mitt Romney when he was criticizing the Obama administration's response to terrorist 
attacks on a U.S. compound in Benghazi, Libya. 
Romney called the administration's response ``disgraceful'' and 
claimed they ``sympathize with those who waged the attacks.''
On September 13, Rachel Maddow described this statement as both ``\emph{attacking}''
President Obama and 
\begin{exe}
  \ex {\em attacking} U.S. diplomatic personnel in the places that were
being attacked.\footnote{\url{https://goo.gl/QD2SFv}} 
\end{exe}
Maddow went on to quote a 
``senior republican foreign policy adviser'' who said the Romney campaign was
``just trying to score a cheap news cycle \emph{hit} based on the embassy statement
and now it's just completely blown up.'' On the same day, Chris Matthews
wondered on \emph{Hardball}, 
\begin{exe}
  \ex who is pushing that and saying, `release the
statement, \emph{attack, attack, attack}'?\footnote{\url{https://goo.gl/DQULSG}} 
\end{exe}

Later on MSNBC,
Rachel Maddow covered the controversial Massachusetts
senate race between Republican Scott Brown and Democrat Elizabeth Warren. 
Maddow cast Warren, the Democrat, as the victim of metaphorical attacks. 
The controversy began in the first debate when Brown noted 
that Warren identified herself as a 
Native American on school applications, but, Brown said, ``You can see that she's not''\footnote{\url{https://www.c-span.org/video/?c4722477/attack-referred-msnbc}}. 
Maddow first addressed this event in an interview with Rep. Barney 
Frank (D-Massachusetts), when she asked him his thoughts about 
\begin{exe}
  \ex personal \emph{attacks} by Senator Brown against Elizabeth Warren.
\end{exe}
Frank said it was a 
\begin{exe}
  \ex silly \emph{attack} on the fact that she once said she was of Native American 
  ancestry\footnote{\url{https://archive.org/details/MSNBCW_20120921_040000_The_Rachel_Maddow_Show/start/2400/end/2460}}.
\end{exe}
%The \emph{attack} was not too silly for Maddow to continue to cover using
%metaphorical violence. 
Over the next seven days, Maddow or a Maddow guest used metaphorical
violence twelve times to describe Brown's comments on Warren's race at the 
debate. 

As described, 2016 differed from 2012 in quantity and
dynamics of use of metaphorical violence on cable television news. 
We now consider specific examples of metaphor use in
2016.  Donald Trump made aggression an explicit character
feature early on in the primary election campaign, 
claiming in January 2016, ``I could stand in the middle of Fifth Avenue
and shoot somebody and I wouldn't lose any voters, OK?''\footnote{\url{https://www.realclearpolitics.com/video/2016/01/23/trump_i_could_stand_in_the_middle_of_fifth_avenue_and_shoot_somebody_and_i_wouldnt_lose_any_voters.html}}
Many of Trump's statements against others either originated on Twitter 
or were echoed on Twitter. These statements were reported in 
cable news as metaphorical violence. 
Below we first give some examples demonstrating the themes of metaphorical
violence usage that made up the higher use for the three cable 
news channels we studied. As tweeting seemed to show up
regularly in 2016, we end by calculating models relating
metaphorical violence frequency to candidate tweeting.

In playing its role as cheerleaders and expectation-setters, 
CNN and Fox News anticipated a first debate where much metaphorical violence
would be done by each candidate. This anticipation partly caused increased 
metaphorical violence usage.
Fox News and CNN increased metaphorical violence usage on 
September 24 and September 25, respectively, just before the first 
presidential debate on September 26. 
It seems news outlets were expecting debates that resembled violence, not
taking the underdog strategy for either candidate as in 2012.
Both CNN and Fox News noted the novelty of having a debate between 
candidates of different genders, and the new experience for Trump of debating
a woman ``of his same generation.''
On \emph{Anderson Cooper 360} on CNN, various commenters said the 
following\footnote{\url{https://archive.org/details/CNNW_20160926_000000_Anderson_Cooper_360}}

\begin{exe}
  \ex Is he going to \emph{hit} back if \emph{attacked} tomorrow, or even if not \emph{attacked}?
  \ex There's a gender dynamic going on here.  It'll be interesting to see 
    whether he \emph{attacks} her the way he \emph{attacked} ``Little'' Marco 
    (Rubio).
  \ex (Trump) thrives on the \emph{attack} \ldots how that will work out when 
    it's a woman of his same generation \ldots that will be dramatic.
\end{exe}

The same broadcast mentions a Trump tweet that referenced Gennifer Flowers'
affair with Bill Clinton\footnote{\url{https://archive.org/details/CNNW_20160926_000000_Anderson_Cooper_360/start/120/end/180}}. A commentator goes on to quote
Jane Goodall, who said ``Trump debates like a chimp in a dominance ritual.''
Cooper's guest explained that Trump ``is not just arguing, but intimidating'' his 
opponents\footnote{\url{https://archive.org/details/CNNW_20160926_000000_Anderson_Cooper_360/start/3180/end/3240}}.''
On Fox news, September 25, host Megyn Kelly and guests on \emph{The Kelly File} 
weighed in on debate 
strategy\footnote{\url{https://archive.org/details/FOXNEWSW_20160926_010000_The_Kelly_File}}:
\begin{exe}
  \ex You have a column out saying she should get in his face and stay in his
    face \ldots put him in the pain locker and shake it around.
    You think she should \emph{attack, attack, and attack} some more. Doesn't she
    have to worry about people saying \ldots sexist terms like she is a shrew, 
    she's shrill?
  \ex What she needs to do is \emph{attack} him on many points calmly one after
    another.
  \ex If I was Donald Trump I would really stay away from \emph{attacking} Hillary
    Clinton.
\end{exe}
During a man-on-the-street segment on \emph{The O'Reilly Factor}, one 
passerby said Clinton ``\emph{beat} the [bleep] out of Donald Trump. It was
like a boxing match, Hillary hit him 1, 2, bing bing.''\footnote{\url{https://archive.org/details/FOXNEWSW_20160928_030000_The_OReilly_Factor/start/2940/end/3000}} 

In the closing minutes of the first 2016 debate, Hillary Clinton introduced the 
story of former Miss Universe
Alicia Machado. Machado won Miss Universe when Trump owned the competition
in 1996. Clinton said Trump called Machado ``Miss Piggy'' because
of Machado gained too much weight and ``Miss Housekeeping'' 
because Machado was born in Venezuela. 
This controversy reverberated throughout the rest of the presidential race.
Trump spoke out on the Fox News morning show Fox and Friends the 
next morning, and on Twitter over the 
next few days, to defend his negative view of Machado. A reporter on 
\emph{Anderson Cooper 360} cast Clinton's strategy as metaphorical violence 
on September 27
\begin{exe}
  \ex The Clinton campaign had an ad ready to \emph{hit} Trump\footnote{\url{https://archive.org/details/CNNW_20160928_040000_Anderson_Cooper_360/start/1800/end/1860}}.
\end{exe}
On the morning of September 30, in the third of
a series of three tweets about Machado, Trump called 
Machado ``disgusting'' and told readers to ``check out sex tape.''\footnote{\url{https://twitter.com/realdonaldtrump/status/781788223055994880}} 
The following quotes from the September 30 episode
of \emph{Erin Burnett OutFront}\footnote{\url{https://archive.org/details/CNNW_20160930_230000_Erin_Burnett_OutFront}} 
demonstrate how metaphorical violence was used
to describe the exchange of words on this issue, and the candidates' 
reactions and counter-reactions:

\begin{exe}
  \ex Did the debate hurt Donald Trump and are his \emph{attacks} on a former
    Miss Universe taking a toll?
  \ex In a statement (Machado) says Trump's latest \emph{attacks} are cheap lies 
    with bad intentions.
  \ex Trump is also \emph{attacking} the media.
  \ex Tonight Hillary Clinton hammering Donald Trump for his \emph{attacks} on 
    former Miss Universe Alicia Machado.
\end{exe}
Regarding Trump's tweets, Clinton herself asked at a campaign rally, ``Who
gets up at three o'clock in the morning to engage in a Twitter \emph{attack} against
a former Miss Universe?''\footnote{\url{https://archive.org/details/CNNW_20160930_230000_Erin_Burnett_OutFront/start/1020/end/1080}} 


Two days before the second debate, October 7, the Washington Post published a video
in which Trump brags that being famous enables him to 
sexually assault women\footnote{\url{https://goo.gl/tk3ZNf}}. 
Trump apologized for those words in a video posted to Twitter that night
\footnote{\url{https://twitter.com/realDonaldTrump/status/784609194234306560}} 
Along with the apology in the same video, Trump accused, ``Bill Clinton has actually
abused women, and Hillary has bullied, attacked, shamed, and 
intimidated his victims,'' and
foreshadowed ``we will discuss this more in the coming days. See you at the
debate on Sunday.'' 
Two quotes from a special edition of \emph{Last Word} illustrate the coverage
of this threat using metaphorical violence\footnote{\url{https://archive.org/details/MSNBCW_20161008_100000_The_Rachel_Maddow_Show}; although the show
ID says it's \emph{The Rachel Maddow Show}, but it is in fact 
\emph{The Last Word with Lawrence O'Donnell}}. This is also further evidence that
cable news uses metaphorical violence in their coverage of debate preparation
and in expectation-setting.  

\begin{exe}
  \ex Clinton \ldots has already been practicing for these \emph{attacks} from 
    Donald Trump \ldots she already has her playbook.
  \ex (Clinton's) team has been preparing for Donald Trump to throw every 
    possible \emph{attack} at her.
\end{exe}

% David Axelrod, as a guest on the October 14 episode of
% \emph{Anderson Cooper 360}, explained
% Trump's strategy using metaphorical violence:

% \begin{exe}
%   \ex What he's employing now is a deny and 
%     \emph{attack} strategy \ldots He's going to \emph{attack} 
%     hillary clinton as hard and as fiercely as he can, hoping to 
%     drag her numbers down, drive people to the third party, 
%     depress her turnout, and make his 41 or 42\% stand up.
% \end{exe}

On October 9, the day of the second presidential debate, 
all three channels were in an elevated state of 
metaphorical violence usage. In pre-debate coverage on \emph{The O'Reilly Factor},
Fox News anchor Tucker Carlson used metaphorical violence to 
describe his understanding of Trump's strategy
\begin{exe}
  \ex (Donald Trump) has decided not simply to \emph{attack} Hillary Clinton \ldots but to 
  \emph{attack} basically the entire American establishment, the press \ldots and
  basically the keepers of American standards.
\end{exe}
Here Donald Trump is framed as a herculean aggressor,
with the specific victims of his attacks being Clinton, the American establishment,
the press, and those who value prototypical American norms of behavior. 

% October 9 debate: ``In a notable exchange that was rehashed often 
%     on cable news...'' Trump attacked on the Benghazi issue as a counterattack
%     against Clinton's criticism that Trump was up at 3AM tweeting about 
%     Ms. Universe. \footnote{\url{https://archive.org/details/FOXNEWSW_20161010_030000_The_Kelly_File}}

% October 10: Hillary says ``Donald Trump spent his time last night
%     attacking me when he should have been apologizing.'' But even his 
%     apologies don't appear to be real apologies, at least according to 
%     Fox News. Another instance in explaining 
%     this\footnote{\url{https://archive.org/details/FOXNEWSW_20161010_000000_The_OReilly_Factor}}


On October 10, the day after the second debate, Donald Trump began criticizing Paul 
Ryan, the Speaker of the House, on Twitter. Trump 
wrote, ``Paul Ryan should spend more time on balancing the budget, jobs and 
illegal immigration and not waste his time on fighting the Republican nominee.''
Ryan had said he was ``sickened'' by Trump's comments and decided to cancel a
scheduled joint appearance with Trump \cite{Fahrentold2016}.
Trump would criticize the Speaker five more times in
the next six days on Twitter\footnote{\url{https://www.thetrumparchive.com/?searchbox=\%22Paul+Ryan\%22&dates=\%5B\%222016-07-31\%22\%2C\%222016-11-30\%22\%5D}}.
All three networks reported on this exchange using metaphorical
violence. Juan Williams said this on October 15 on \emph{The O'Reilly Factor}\footnote{\url{https://archive.org/details/FOXNEWSW_20161016_030000_The_OReilly_Factor/start/420/end/480}}:

\begin{exe}
  \ex That is something that Donald Trump is spending time on, attacking Paul
  Ryan because Paul Ryan is distancing himself, but he's attacking a fellow
  Republican instead of broadening or shoring up his base with republicans.
\end{exe}

In the following weeks, there were many more contentous issues which caused a
series of ``attacks'' and counter-attacks. Among these was the Al Smith 
fundraising dinner, a tradition where each candidate is invited and expected
to make light-hearted jokes at the other candidate's expense. 
Voices on MSNBC, and on the other two networks, felt
Trump's jokes were mean-spirited and described the jokes as attacks.
Here is one example from MSNBC's in which 
Senator Al Franken described some of Trump's jokes as attacks, 
with Clinton as the victim, at that dinner on the October 20 episode of
\emph{The Rachel Maddow Show}

\begin{exe}
  \ex It takes skill to write a joke. And there were some where he just \emph{attacked} her.
\end{exe}

Many of these instances of metaphorical violence involve statements on Twitter. 
To understand the link between candidate tweeting and metaphorical violence usage,
we fit a series of linear regressions with classes of metaphorical violence
usage as the dependent variable and daily tweets issued by major candidates 
as the independent variable.  This analysis also provides a further quantification of how 
broader cultural trends affect the timing and amount of metaphorical violence
usage. This analysis demonstrates that metaphorical violence can be used as an
indicator of communicative efficacy. In this case, metaphorical violence use
provides a yardstick for the
impact of candidate Twitter use in both election years. Our
analysis confirms that, compared to 2012, 2016 was the year of the ``Twitter Election''
\cite{Heller2016}, using metaphorical violence as a measure of Twitter
impact. 

In 2016, all linear model fits were significant across categories of 
metaphorical violence (Table~\ref{tab:twitterRegrResults}). In 2012, there
were still a number of statistically significant fits, but much less variance
could be explained through Twitter use. About 1/3 of metaphorical violence use across all
categories can be predicted from either Hillary Clinton's or Donald Trump's
Twitter use in 2016. Across both years and all candidates, 
CNN was most reactive to Twitter use. Candidate Twitter use explained between
14\% and 23\% of the variance
in metaphorical violence where the candidates were either the subject or object
of metaphorical violence in both years. 

  \begin{table}
    \centering
    \vspace{.45in}
    \input{/Users/mt/workspace/Papers/metvi/TweetCoeffTable.tex}
% \caption{yo}
    \caption{Regression coefficients, %$r^2$, 
      and significance indicators for
      linear models of metaphorical violence usage as a function of the number of
      tweets from individual candidates. The regression
      coefficient represents the additional metaphorical violence uses that 
      occur with each message the candidate tweets. % $r^2$ represents the fraction
      of variance that is represented through a linear relationship with candidate
      tweets. The 2016 candidates's Twitter use had a greater impact on metaphorical
      violence usage than the 2012 candidates's. In both years, Twitter use 
      had a strong effect on metaphorical violence use where the tweeting candidate was
      cast as the subject of metaphorical violence, or where the other candidate
      was the object of metaphorical violence.}
    \label{tab:twitterRegrResults}
    \vspace{.25in}
  \end{table}
% Illustrations of the linear models with significance
% values are shown in Figures~\ref{fig:regressions-all}-\ref{fig:2016-subjobj}.

% \subsubsection*{Wikileaks}

% \url{https://archive.org/details/CNNW_20161012_230000_Erin_Burnett_OutFront}
% As if to summarize all the issues on which each candidate ``attacked'' 
% each other, \emph{OutFront} closed its October 10, 2016, episode 
% discussing the Clinton campaign emails published by WikiLeaks. See
% \href{http://www.bbc.com/news/world-us-canada-37639370}{this BBC article}
% for what those emails revealed.

% \subsubsection*{Al Smith Dinner}

% On an same episode of \emph{Erin Burnett OutFront}, 
% it was reported that the Trump campaign was very upset with the 
% \emph{New York Times} for ``launching
% an attack like this so late in the game,'' referring to a story in the
% \emph{Times} that told the stories of two women who claim they were 
% sexually assaulted by Donald Trump?\footnote{\url{https://archive.org/details/CNNW_20161012_230000_Erin_Burnett_OutFront}}
% As if to summarize all the issues on which each candidate ``attacked'' 
% each other, \emph{OutFront} closed its October 10, 2016, episode 
% discussing the Clinton campaign emails published by WikiLeaks. See
% \href{http://www.bbc.com/news/world-us-canada-37639370}{this BBC article}
% for what those emails revealed.

% \subsubsection*{Obamacare}

\section{Conclusion}

Our efficient and effective approach to data collection and annotation enables
new experiments aimed at understanding the dynamic relationship of language use
in the media and voter attitudes. Consider that in one large scale study, online
news agencies selected which news topics would be published when, and results
showed that discussion of the chosen topics on social media correlated with
publication of news stories~\cite{King2017}. Whereas that study took years to
implement, we believe many more natural experiments can be done using the
approach we have outlined here (see also Fusaroli, et al., 2015). To understand the
impact of metaphorical violence language---or any specific sort of language---we
can record data from the Internet TV News Archive, concurrently polling test
subjects to record, for instance,
their recent TV viewing history, political opinions, use of metaphorical violence
in prompts, and support for political
violence to identify correlations (as in Kalmoe, 2014). 

In summary, our data and analyses revealed similarities and differences in the
use of metaphorical violence language on U.S. cable television news across
networks and presidential election years. There were differences in how much
metaphorical violence language was used and in the relative changes and timing
of use across networks and years; for instance, in some cases, metaphorical
violence language use increased around presidential debates, and in others, it
decreased. There were similarities in the details of use of specific violence
signals and in which party's candidate was most involved in metaphorical
violence; for instance, \emph{attack} was used most often, and Republican
candidates were represented as either the aggressor or the victim of
metaphorical violence more than Democratic candidates.  Thus, our study has
provided detail and perspective on the workings and dynamics of metaphorical
violence in political discourse. Previously, metaphorical violence was known to
be a feature of political communication, but its extent and dynamics were not
known. We have shown that use of metaphorical violence language can change
substantially over a short period, both in amount and in kind, in response to
external actions and cultural events. We have shown that different political
perspectives make different use of metaphorical violence language. Yet there is
still a lot more we do not know.  We know little about the relationship between
metaphorical violence language used on television and actual violent actions.
Some may infer cause and effect, as the suggestion that observing violence in
video games leads to tolerance for and actions of violence in the real
world~\cite{Calvert2017}. Others may simply see the use of specific metaphorical
language primarily for purposes of political
persuasion~\cite{Charteris-Black2009,Mio1997}. In a time of ever-more
optimization and automation, we must consider carefully how to shape political
discourse to create the desired outcomes. Our results are one step in the
direction of understanding how use of specific language influences political
attitudes.  

\section*{Acknowledgements}

We thank Oana David and Jamin Pelke for helpful and inspiring discussions. We also thank
Isabella Methot, Gloria Quintana, and Amy Tang for assistance with annotating


\chapter{Stubborn extremism explains and predicts group polarization}


Group polarization is the widely-observed phenomenon in which the opinions held by members of a small group become more extreme after the group discusses a topic. 
For example, conservative individuals become even more conservative, while liberal individuals become even more liberal. 
Social psychologists have offered competing explanations for this phenomenon. These typically require questionable assumptions about human psychology. 
Here, we posit a more parsimonious explanation: the stubbornness of extreme opinions. Using agent-based modeling, we demonstrate that such ``stubborn extremism'' 
gives rise to group polarization as observed across the literature on 
polarization.  We conclude with an evaluation of stubborn 
extremism and existing explanations
to identify opportunities for theoretical integration. 

% \textbf{Keywords:} 
% opinion dynamics; polarization; social influence; agent-based modeling


\section{Introduction}
\label{sec:introduction}

\emph{Group polarization} is a phenomenon in which 
the opinions held by members of a small group become more extreme after the group 
discusses a topic \cite{Myers1982,Brown1986,Isenberg1986,Sunstein2002,Sieber2019}.
This phenomenon is socially important for many reasons. First, 
small groups of advisers often influence executive decisions in government and business. 
At the ``grass roots'' level in politics, individuals discuss 
important issues first in small groups before they vote. 
Second, group polarization at the local level increases overall polarization at the societal level. Polarization, measured as the bimodality of the distribution of
opinions in a group or society, increases whenever either of two opposed 
groups becomes more extreme~\cite{Bramson2016}. 
Many studies of political polarization frame the issue in terms of intergroup 
conflict~\cite{Mason2018UncivilAgreementBook,Klein2020}. 
However, we also must understand how
group polarization can exacerbate political polarization through increased
in-group extremism without an explicit out-group. 
Understanding the cognitive mechanisms supporting group 
polarization is therefore a matter of concern. 

Social psychologists have offered several explanations for group polarization, 
but four are considered acceptable today~\cite{Sieber2019}. 
First, \emph{social comparison theory} posits that individuals' privately-held 
opinions tend be more extreme than those they express publicly, and exposure to consonant 
opinions gives them confidence to express their true opinions openly~\cite{Myers1982}. 
It is not clear, however, when or if people really do hold more extreme views
than they tend to express.
%In contrast, 
Second, \emph{persuasive arguments theory} posits that when individuals discuss a topic within an already-biased group, 
they accumulate more persuasive arguments supporting those biases, leading to a more extreme version~\cite{Bishop1974,Vinokur1974}. 
This is problematic because it lacks inclusion of arguments for moderation, explaining that
moderation comes from knowing arguments for each polar opinion.
Third, \emph{self-categorizaztion theory} explains group polarization as emerging from
the desire of individuals to consolidate group membership by expressing more
extreme opinions, which further contrasts individuals from hypothetical out-groups~\cite{Turner1987,Abrams1990,McGarty1992}.
This is problematic because it is not clear how the calculation works that
would determine how much one individual should shift their opinions to more clearly
signal group membership.
Finally, \emph{social decision schemes theory} explains group polarization as the 
result of two main factors, namely the distribution of individual-level traits
that determine individual opinions and the method by which groups make collective decisions~\cite{Zuber1992,Friedkin1999a}.
If extremists are more powerful in groups on average, then group decisions
on collective opinions will become more extreme after group deliberation.
This is problematic because it may be difficult to determine a novel group's
decision scheme \emph{a priori}.
These explanations may explain the empirical phenomenon of group polarization, 
though more formal modeling is required to bring precision to the underlying 
theories~\cite{Smaldino2017StupidModels,Smaldino2019b}. 

We present an alternative explanation for group polarization that, while not mutually exclusive with the other theories discussed, manages to explain the phenomenon of group polarization without assuming anything about the intrinsic distribution of extreme opinions in human groups. We do so by appealing to a property of human psychology we call  
\emph{stubborn extremism}: as a person's opinion on
some topic becomes more extreme, that opinion also becomes more stubborn, 
i.e.\, less susceptible to social influence. 
We support this explanation using a computational model of group polarization. 
Our model was originally developed for explaining how polarization emerges
where two groups become more extremely opposed~\cite{Flache2011,Turner2018}.
The model incorporates both negative, repulsive social influence~\cite{Cikara2014}, assimilative influence,
and the stubborn extremism assumption, though repulsive influence is not at work
in group polarization because all opinions start out similarly valenced.   

Group polarization can emerge computationally by simply 
assuming agents hold binary opinions on a multitude of topics~\cite{Mueller2018,Banisch2019}. 
However, most group polarization studies do not measure participants' binary opinions (e.g., for vs.\ opposed) on a multitude of topics, but rather measure opinions as falling on a range between strongly for and strongly opposed. Furthermore, the assumption of discrete opinions is problematic from a psychological perspective,
since it is rare for quantum leaps in opinion to occur---more often we are
influenced gradually over the course of many interactions~\cite[p.793]{Baldassarri2007}.
Our model is most similar to that of \citeA{Baldassarri2007} in that stubbornness is 
a function of opinion extremity directly. \citeA{Martins2013} allow for agents
to become more or less stubborn, but assume discrete opinions and a separate,
continuous measure of open-mindedness/stubbornness.
Most other opinion dynamics models that link 
stubbornness to extremism assume infinitely stubborn
extreme agents (sometimes called ``zealots'') whose opinions are static and whose existence is specified
\emph{a priori} by the modeler~\cite{Galam2007,Mobilia2007,Arendt2015,Mueller2018}.
\citeA{Baldassarri2007} nearly make the connection between stubborn extremism
and group polarization, but they mischaracterize group polarization and discuss it in terms of negative influence, writing
``interaction with dissimilar others may increase distance, leading to group
polarization'' (p. 792). Group polarization experiments are designed so that
this never occurs. Instead, it is only interaction among relatively like-minded
individuals that leads to the group polarization opinion shift.


Current empirical support for the stubborn extremism explanation is positive, though not uniformly so.
\citeA{Zaller1992} and \citeA{Converse1964} established that, at least at the time of their studies, most of the United States electorate, for example, were relatively ignorant of real political issues
and easily swayed by momentary predilections and the framing of questions. \citeA{Guazzini2015} found that stubborn extremists drove the opinions in groups discussing the use of animals in laboratory experiments, and 
\citeA{Lewandowsky2019} found that stubborn extremists have an
outsized influence in the perpetuation of scientific misinformation 
regarding climate change. 
%ps: This paragraph below seems disconnected from what is described above, and ends with something that is perhaps better suited for the Discussion (proposing future research on political group polarization). Perhaps these should simply be put in the Introduction where the Lewandowsky experiment is discussed? 
Group polarization opinion shifts have been observed to increase with the group's initial 
extremity~\cite{Teger1967,Myers1972,Myers1982,Brown1986}. This has only been
tested in detail by \citeA{Teger1967} and \citeA{Myers1972}, apparently, and
has not been established for political opinions. This 
could cause acceleration of political polarization.  
Some researchers have suggested that stubbornness is an attribute found generally among people, and is not limited to those with extreme opinions. However, support for this view often comes from studies in which opinions are operationalized as answers to general knowledge tests (such as found in a pub quiz), and not on opinions with political or ethical components in which subjective judgment plays a larger role~ \cite{Moussaid2013,Chacoma2015}. More direct empirical tests of the stubborn extremism explanation for group polarization are needed. 

The rest of the paper is organized as follows. We first review evidence and explanations
for group polarization in more detail. We will then introduce an agent-based model of opinion dynamics with stubborn extremists, which is adapted from previous work by Flache and Macy (2011), and we will demonstrate how the model supports the stubborn extremism hypothesis. We will then compare our model to the persuasive arguments model of \citeA{Mas2013}, and show how our model can yield a fit to the empirical dataset they test that is at least as congruent. We conclude with limitations of our model's assumptions, and suggestions for future work. 


\section{Group polarization theory, methods, and results}

Initially, the group polarization effect was thought only to apply to opinions about how much
risk would be appropriate to take given some life 
decision~\cite{Wallach1965,Teger1967,Stoner1968}---the so-called ``risky shift''. 
\citeA{Moscovici1969} then showed that deliberation about political opinions also led to group 
polarization. Motivated by this and by Cartwright's critiques~\cite{Cartwright1971,Cartwright1973},
new explanatory mechanisms were proposed. The two explanations that survived to today 
are the social comparisons theory~\cite{Brown1974,Sanders1977,Myers1978} 
and persuasive arguments theory~\cite{Burnstein1973,Vinokur1974,Burnstein1975}. 
Around the time of Isenberg's
(1986) review, a self-categorization explanation~\cite{Turner1987} of
group polarization was developed and supported with new empirical 
studies~\cite{Turner1989,Abrams1990,Hogg1990,McGarty1992,Krizan2007}. 
Following the development of social comparisons, persuasive arguments, and
self-categorization theories, 
social decisions scheme theory that identifies social power structures
as a dominant factor in the emergence of group polarization~\cite{Zuber1992,Friedkin1999a}. 
More recently, focus on the correlation between stubbornness and extremism has emerged as
a simple, empirically-motivated explanation of group 
polarization~\cite{Mueller2018,Banisch2019}. However, existing studies do not
allow extremism to emerge naturally, but instead posit the problematic existence of
infinitely stubborn extremists who are totally unsusceptible to social influence.
Because it is unlikely that anyone is totally unsusceptible to social influence,
we allow individuals to become more stubborn as they become more extreme.


\subsection{Common experimental design elements}

Group polarization studies all follow the same general experimental paradigm,
with slight variations to test particular theoretical explanations or 
real world situations. In this paradigm, participants first answer questionnaire
items or somehow give their opinions or positions on some situation. Small groups
typically of 2-6 participants are formed such that the mean opinion or position
of group members is non-neutral, baised towards one or the other extreme of 
the measurement scale. Group formation is sometimes based on initial participant answers to
the questionnaire, but sometimes uses some other method such as a 
different questionnaire \cite{Myers1970} or geographic location that is correlated with individual
opinion~\cite{Schkade2010}. Political questionnaires are common choices.
For example, \citeA{Moscovici1969} asked Parisian lycée students about their 
opinions of then-president Charles de Gaulle and of American foreign policy.
More recently, \citeA{Schkade2010} asked US residents of Colorado
about affirmative action, same-sex civil unions, and global warming. 

Many studies using questionnaires
prompt participants to give their responses on an ordinal, Likert-type scale.
Stoner's (1961; 1968) choice dilemma questionnaire was a 10-point ordinal
scale, with 1 representing the most risk acceptance and 10 representing the
least risk acceptance. When French students answered ``American economic aid
is always used for political pressure'', they marked a whole number on
a seven-point scale from
-3 (strongly disagree) to +3 (strongly agree), with zero representing 
neutral or no opinion. These scales do not always include 0 as the neutral
point. \citeA{Schkade2010} used a ten-point scale from 1 (disagree very strongly)
to 10 (agree very strongly). 

Non-questionnaire group polarization studies have used a variety of 
methods. In one approach, researchers simulate jury deliberations
for an experimental design where participants give either opinions
on whether a defendant is guilty or how much money for damages should be
awarded~\cite{Kaplan1977,Kaplan1977a,Schkade2000,Schkade2007,Sunstein2000}\footnote{
Schkade, et al., (2000), entitled ``Deliberating about Dollars: The Severity Shift'',
was funded by Exxon Company, U.S.A., who have a clear interest in understanding
what causes individuals to raise or lower the amount of damages they believe
a responsible party should pay.}. Another approach studied group
polarization in the context of gambling behavior in the game of 
blackjack~\cite{Blascovich1974,Blascovich1975,Blascovich1976}, which
found that participants demonstrated opinion shifts to be more risky merely
when exposed to other group members' bets. 

In this study, we are only interested in the effect of explanatory assumptions
and can ignore details of the measurement schemes. Therefore, 
we assume that we can directly observe people's opinions as we do in our
computational models and analyses. Our model simulates the three stages
of group polarization experiments identified above: (1) administer survey
to participants to poll pre-deliberation opinions; (2) participants deliberate
about their opinion in small groups; and (3) poll participants' post-deliberation
opinions.


\subsection{Theoretical explanations of group polarization}

Below we review the four explanations or theories of group polarization 
assumed or evaluated by the case studies we investigated for false detections. 
We also review select empirical support for each explanation.
The explanation of group polarization as due to the stubbornness of 
extremists comes from empirically motivated modeling 
projects that have yet to be verified empirically in group polarization settings.
Therefore, we do not review that here. Future work will use the results of
the present paper to devise more appropriate measurement and statistical 
procedures that will help ensure the validity of future empirical studies.

Following our theoretical review, we identify and explain common experimental design
elements and statistical methods used commonly across group polarization research
independent of theoretical aims and assumptions. Then in this section
we review findings from these decades of research, which overwhemingly support 
group polarization in general---each theory can boast supporting empirical
evidence as well. This sets up the following section where
we explain our model in mathematical detail that we will use to
show that we should be highly skeptical of the broadly supportive evidence for group polarization. 

\subsubsection{Social comparisons}

When researchers began searching for an explanation of group polarization
in response to Cartwright's (1971; 1973) critiques of the ``risky shift'' literature, 
some adapted the extant ``theory of social comparison processes''~\cite{Festinger1954} of group-level
social influence as an explanation. This theory assumes that when people
interact in group settings, each individual infers what the prevailing
social norms are, compares their own opinion to the social norm, and adjusts 
one's own opinions or behaviors so they are more socially accepted or celebrated. 
One testable corollary of this explanation is that no deliberation is required, \emph{per se}.
All that is required is ``mere exposure'' to others' 
opinions~\cite{Zajonc1968,Burgess1971,Bornstein1990,Montoya2017}. Several studies
have shown that when a group polarization experiment is run as explained above,
but without group deliberation, non-verbal displays of individual opinions 
to the group is alone sufficient social influence to foster group 
polarization~\cite{Teger1967,Blascovich1973,Blascovich1975,Blascovich1976,Sanders1977,Myers1978,Myers1982}.

Just because mere exposure to others' opinions tends to lead to group polarization
does not necessarily support all auxiliary assumptions made by the
social comparisons explanation~\cite{Meehl1990}. It is not clear what the
mechanism is by which individuals infer the group norm if it is not just the
average. How is it, exactly, that individuals infer this more extreme than
average group norm? \citeA{Festinger1954} assumes first that ``there is a 
universal human drive to evaluate our opinions and abilities'' \cite[p. 78]{Brown2000}.
But how ubiquitous is this drive to distinguish oneself through conformity?
Clearly individuals vary in their drive to conform to social norms in general---how
does this affect group polarization opinion shifts?
Furthermore, achieving distinctiveness through conformity may have counterintuitive
effects~\cite{Smaldino2015a}. Social comparisons theory fails to make contact
with extensive literature on norms and norm change, which should be accounted
for~\cite{Bicchieri2006,Bicchieri2014,Bicchieri2017}.

These are important questions to answer. Perhaps social comparisons offers a
good starting point for a partial explanation of group polarization, but 
its epistemological status is shaky. It is therefore important that we
understand how to properly measure opinion shifts to either support,
refute, or revise and incorporate the social comparisons account into a 
broader explanatory model of group polarization.

\subsubsection{Persuasive arguments}

Persuasive arguments theory explains that opinion change is determined by the number and persuasiveness of 
arguments that support different poles of the opinion scale. Arguments, then,
are central theoretical entities in this model alongside opinions. If there are more
arguments favoring one polar opinion (disagree/agree) over another~\cite{Ebbesen1974}, or if
arguments that exist for one polar opinion are more persuasive 
then the group will collectively move towards that 
polar opinion~\cite{Vinokur1974,Burnstein1977}. This theory assumes that for an argument to have an effect
on a participant, that participants must not have heard the argument 
before~\cite[see Equation on p. 96]{Bishop1974}. Furthermore, the validity,
or informativeness, is hypothesized to be the primary auxiliary factor in 
determining the magnitude of influence for a given argument~\cite{Vinokur1978}. 

One problem with the persuasive arguments explanation is that only arguments 
are persuasive, not people. Perhaps, for example, there is a simple 
consistency in that more extreme individuals tend to be more persuasive than moderates, 
perhaps due to their confidence in their opinions. This assumption would actually 
explain observations made by \citeA{Burnstein1973} who 
found that insincere arguments are not influential.
Another related problem is underspecified
psycholinguistic mechanisms of social influence. Perhaps novelty and 
informativeness are two important factors in what makes an argument persuasive.
Surely, though, there are other factors.  

\subsubsection{Self-categorization}

The self-categorization explanation of group polarization 
posits that people conform to others' attitudes,
opinions, or beliefs, by considering how best to ``contrast'' themselves
with members of an out-group so as to consolidate their membership with
an in-group~\cite{Tajfel1971,Tajfel1979,Turner1987}. 
Experiments testing the self-categorization hypothesis use 
the minimal group paradigm approach to
understand differences in social influence (that leads to extremism) 
between in-group members versus out-group members. 
In one interesting counter-example to
the persuasive arguments theory, the basic experimental design was used, but
participants did not interact with a group---instead they were listened to 
tape recordings of arguments for or against some statement. Participants
were told they would either be joining the group or that they were listening
to members of an out-group. This changed whether opinion shifts were to a greater
extreme they were already bised towards (in-group) or if participant 
opinions tended to shift away from their initial bias (out-group)~\cite{McGarty1992}. 
Persuasive arguments theory does not account
for group membership, so it could not have predicted this result. 
The minimal group approach continues
to be applied today across cognitive sciences, especially in understanding
the neuroscience of emotions towards novel in- and 
out-groups~\cite{Cikara2014,Molenberghs2014}.

To explain
group polarization, where there no explicit out-group, self-categorization
theorists proposed that people engaging in social interaction mentally
calculate the ``metacontrast ratio'', which is defined as a person's average
distance in opinion space from all out-group members divided by that person's
average opinion distance from all in-group members~\cite[p. 3]{McGarty1992}.
This requires them to infer their average distance to the imagined outgroup.
A person is then hypothesized to update their opinions to match the prototypical
opinion, which is defined as ``the pre-test mean where the mean is at the mid-point
of the comparative context\ldots.'' This supposedly leads to group polarization, since
``(a)s in-group responses shift\ldots towards a more extreme position, 
then it becomes more likely that the prototype will tend to be more extreme than
the mean in the same direction'' (p. 4, \emph{ibid}). 

While neuroscientific studies implementing the minimal group paradigm support
the assumption that differential social influence depends 
on whether an individual interacts
with in-group or out-group members~\cite{Cikara2014}, it is not clear that it operates
as hypothesized in self-categorization explanations of group polarization. 
Specifically, the assumption
that people calculate meta-contrast ratios and hypothetical in-group
prototype opinions does not seem to be empirically supported. 
It is not clear to us how such a claim could be empirically supported. 
Another possible critique is that this reasoning seems to be circular: the 
in-group prototype begins as the pre-deliberation
mean, but changes once opinions begin to change. This seems to sidestep the
problem of how opinions change in the first place and why the average opinion
tends to become more extreme. Finally, it seems that perhaps ``prototype'' in
the self-categorization explanation is homologous in form and function to
a ``norm'' in social comparisons theory. Future work should explore this connection
in more detail to understand exactly how the two theories substantively differ.


\subsubsection{Social decision schemes}

Social decision schemes generally considers the social structure of groups to
account to determine what opinions or behaviors group members will 
take in the course of group interaction~\cite{Davis1973}. In the main branch
of social decision schemes, individual-level interaction strategies are 
hypothesized and specified. To understand social decision schemes, 
consider the following example adapted from 
\citeA[p. 195]{Brown2000}. Assume a group is trying to solve some problem.
The group may be composed of three types of people: 
(1) people who are able to solve the problem, (2) people
who can recognize a solution but not solve the problem themeselves, and
(3) people who cannot solve the problem or recognize a correct solution.
The group may adopt different decision rules, such as ``Truth wins'' 
(as long as one member has the solution, the group solves the problem),
``Majority rule'' (a majority of group members must know or recognize the
solution), or ``Unanimous'' (all group members must know or recognize the solution).
If we know the composition of the group in terms of these three types, then
we can calculate the probability that a group solves the problem. 
According to the social decision schemes
framework, if we observe how often a group solves a
problem and we know the distribution of strategies, we can infer the
decision rule used by the group.

In the context of group polarization, instead of recognizing solutions to
problems, people are assumed to adopt a strategy of ``risk wins'',
``conservatism wins'', or ``majority wins'' in the context of the
choice dilemma questionnaire \cite{Laughlin1982,Zuber1992}. 
\citeA{Friedkin1999a} developed a network theoretic model that aligned with the
social decision schemes approach, but focused on power structures that 
determine relative social influence. When extremists are more powerful, one
would expect group polarization to emerge. 
Friedkin ran behavioral experiments to support his explanation, but
unfortunately, several of Friedkin's results are \emph{prima facie} 
null, since several of the confidence intervals around the opinion
shift measurements include zero. 

One issue with the social decision schemes approach seems to be that the
emergence of distribution of strategies, and the strategies themselves, is
not accounted for. How does such a norm as ``risk wins'' emerge? How is this
not a ``norm'' or ``prototype'' as could be found in either the social comparisons
or self-categorization explanations, respectively? Because norms may indeed be
important for group polarization, future theorizing should consider
how norms emerge and culturally 
evolve~\cite{Bicchieri2006,Bicchieri2014,Bicchieri2017}.


\section{The model}

We developed an agent-based model to demonstrate the stubborn extremism model
predicts group polarization patterns reviewed above. Our goal is to demonstrate
that the relatively minimal assumption of stubborn extremism can predict
observed patterns group polarization opinion shifts.
This model allows for both positive
and negative influence, wherein initially similar agents become more similar 
after interacting, while initially dissimilar agents become more polarized. 
The model is identical to that studied previously in~\citeA{Flache2011} 
and \citeA{Turner2018}, but is analyzed here with a different focus than 
was used in those studies. 

We consider a population of $N$ agents, who each have opinions
on one topic. This model can account for social influence across multiple
opinion topics, but one suffices for our purposes. Future work could consider
the effect of deliberation on multiple opinions, which has been shown to foster
cultural fragmentation~\cite{DellaPosta2015}. Agent $i$'s opinion at time
$t$ is written $o_{i,t} \in (-1, 1)$ and changes after $i$ has interacted with its
$N_i$ network neighbors. The weight of social influence with each neighbor $j$  is $w_{ij,t}$, 
with zero direct influence over non-neighbors. Weights 
depend on the Manhattan distance between agents $i$ and $j$: $d_{ij,t} = |o_{i,t} - o_{j,t}|$. 
The specific operation of these social influence 
mechanisms is defined by the following dynamical equation
\begin{equation}
  o_{i,t} = o_{i,t-1} + \Delta o_{i,t}(1 - |o_{i,t-1}|^{\alpha})
  \label{eq:basicDynamics}
\end{equation}
\noindent
where
\begin{equation}
  \Delta o_{i,t} = 
    \frac{1}{2N_i} \sum_{j} w_{ij,t} (o_{j,t} - o_{i,t})
\end{equation}
\noindent
and
\begin{equation}
  w_{ij,t} = 1 - d_{ij,t}.
\end{equation}
Our model includes both positive and negative influence. 
Positive influence is when agents become increasingly similar to their dyad partner 
if the pair are sufficiently similar to begin with ($d_{ij} < 1$). Negative influence
is when interaction causes a dyad to become more different, to be
repulsed away from one another toward more extreme regions of opinion space
if the pair are sufficiently dissimilar to begin with ($d_{ij} > 1$). 
This is important for group polarzation because while a group overall may
be biased towards one extreme, in general there may be group members who lean
towards the opposite opinion pole---in these situations sometimes dyads become
more different when they interact instead of more similar~\cite{Bail2018}.
The parameter $\alpha$ determines the degree to which extreme opinions are stubborn. 
In the analyses presented here, we use $\alpha=1$.
Stubborn extremism emerges in our model due to the 
smoothing factor $(1 - |o_{i,t-1}|)$, which is smaller when $|o_{i,t-1}|$ is 
larger. Therefore, more extreme opinions (larger
$|o_{i,t-1}|$) are less susceptible to social influence than less extreme opinions 
(smaller $|o_{i,t-1}|$).

Our model generates a number of empirically-observed outcomes. 
First, we show that our model 
yields group polarization in an idealized generic case that 
resembles the studies of  \citeA{Moscovici1969}, \citeA{Myers1970}, and \citeA{Myers1975}.  For our computational experiments, we set the number of agents in the 
population to $N=25$\footnote{This is much larger than real group polarization experiments, but
served to generate group polarization shifts in a shorter number of time steps
for a proof of concept. This will need to be made realistic for the journal article.}. 
The social network for this first experiment was fully connected, meaning all agents could 
potentially influence all other agents. Second, we represent the \citeA{Mas2013} 
empirical experiment with our model and show our model predicts their 
empirical observations as accurately as their computational model of 
persuasive arguments theory.


\subsection{Computational experiments}

Our first experiment examined the correlation between initial mean opinion 
and shift magnitude. This also establishes that our model
generates group polarization. 
Initial agent opinions were drawn from a normal distribution 
with $\sigma=0.25$. In order to demonstrate that our model predicts the correlation
between opinion shift and initial opinion extremity, we ran the model with
seven different experimental conditions. Each of the seven conditions 
specified a different mean for the normal distribution from which initial opinions were drawn,
$\mu \in \{0.2, 0.3, \ldots, 0.8\}$. For each condition we ran 100 trials. %MT: prob needs a rewrite.
%ps: Not sure how to interpret this method above. Were opinions drawn from one of these 4 distributions at random?
Since opinions are bounded between $\pm1$ and group polarization experiments
force group members to have opinions of the same valence,
we re-mapped any drawn opinions greater than 1 to be $+1$ if
the drawn opinion was greater than 1, and 0 if the drawn value was less than 0.
Each model run consisted of 100 rounds of agent interactions. 
In one round of agent interaction, $N$ agents are
selected at random to update their opinions according to 
Equation~\ref{eq:basicDynamics}. To model a typical group polarization experiment
with open discussion, we assume a fully-connected network, so all agents
influence one another.


% \subsubsection*{Flache and Macy model for Mäs and Flache's (2013) study}
Our second experiment was designed to 
generate the results of \citeA{Mas2013}. Here we utilized the multidimensionality of 
opinions to represent different ``persuasive arguments'' that participants held.
To do this, we set $K=12$, the total number of persuasive arguments available
to each agent in M\"{a}s and Flache's study, and initialized three of the twelve
opinions to be non-zero.  Recall that in their study, M\"{a}s and Flache provided
individuals with one of twelve pre-defined ``arguments'' they were to share with 
others to advocate for their opinion. Six of the twelve were chosen as pro-A arguments 
and six of the twelve were chosen as pro-B arguments. The pro-A arguments
were given initial values of $-1/3$ and pro-B arguments given initial
values of $1/3$. In our adaptation of this experimental setup, we are using
each of $K$ elements of agent $i$'s opinion vector to represent the presence
or absence of an argument. 
%ps: It's not clear how "arguments" are represented in this model, and how they differ from opinions.
As in the M\"{a}s and Flache study, group ``A" members all received
the same initial pro-B argument, and vice versa. 
To calculate each agent's scalar opinion based on its $K=12$
``persuasive argument'' components, we first normalize opinions so their absolute
values sum to 1, and then averaged over all opinions. This is similar to the
persuasive argument model that assumes an individual's opinion is an aggregate
of the arguments they know for their position. 
This computational experiment mirrors M\"{a}s and Flache's persuasive arguments model, 
but includes stubborn extremism. Furthermore, in our formulation, agents can 
partially agree or disagree
with a given argument, unlike persuasive arguments which assumes an agent either
knows an argument or not.
For our computational experiment's outcome measure, we calculated the 
average over all agent opinions in each group at each timestep, and then
averaged those averages across 100 trials at each timestep, identical to
M\"{a}s and Flache's procedure for obtaining their results (Figures 5 and 6 of their paper).

\subsection{Implementation}

The model was implemented as an agent-based model written in plain Python with 
user-defined \texttt{Agent}, \texttt{Model}, and \texttt{Experiment} classes.
We use NumPy and SciPy for numerical and scientific routines and functions.
For full implementation details including instructions for installing and
running model code and reproducing our results, please visit the GitHub
repository, \url{https://github.com/mt-digital/group-polarization}.
Our computational experiments easily run on a laptop. 


\section{Analysis}

Our model predicts that more extreme initial group opinion
results in larger shifts up to a certain extremity where the trend 
reverses (Figure~\ref{fig:shiftVsInitial}). 
In terms of stubborn extremism, this general trend is expected 
because there will be more extremists when the initial mean is greater. These
initial extremists exert a greater pull towards extremism when they are more
numerous. However, when many agents are extreme and there are few neutral agents
to be shifted to more extreme views, the shift begins to decrease in magnitude
compared to the maximum shift over initial mean 
(occurs at initial mean of 0.8 in Figure~\ref{fig:shiftVsInitial}).

\begin{figure}[t] %[h]
  \centering
  % \begin{subfigure}{0.5\textwidth}
    % \centering
    \includegraphics[width=\textwidth]{/Users/mt/workspace/Papers/stubex/Figures/ContinuousBoxplot.pdf}
    \caption{Group opinion shift when individuals' initial and final 
      opinions are given on a continuous scale.}
    % \label{fig:continuousVsInitialAverage}
  % \end{subfigure}
  % \begin{subfigure}{0.5\textwidth}
  %   \centering
  %   \includegraphics[width=\textwidth]{/Users/mt/workspace/Papers/stubex/Figures/7pointLikertBoxplot.pdf}
  %   \caption{Group opinion shift when individuals' initial and final opinions
  %     are given on a 7-point Likert scale.}
  %   \label{fig:binnedVsInitialAverage}
  % \end{subfigure}
  \caption{Demonstration of the trend that opinion shift is positively
  correlated with the mean initial group opinion. The trend is distored
  by binning into Likert scale responses. 
  Boxes enclose the first and third quartile of the data. 100 trials shown
  for each condition.}
  \label{fig:shiftVsInitial}
\end{figure}

%ps: Just focus on YOUR results. Don't confuse them with the empirical studies done by others. Once you report the result, you can say something like "..., consistent with the empirical results of Myers (1975)." 
% Binning disrupts the 
% positive linear relationship between opinion shift and initial mean group 
% opinion (Figure~\ref{fig:binnedVsInitialAverage}). This is because, in our
% model, if enough agents are neutral and not too many agents are extreme, 
% then some agents with an opinion of +3 will shift to +2, and enough agents 
% with opinions of +1 or less do not shift their opinions, the sign of the
% shift may be negative, and group polarization will not emerge.

Our model predicts group polarization as
observed by \citeA{Mas2013}, but via the assumption 
of stubborn extremists instead of persuasive arguments. 
Our model predicts the same initial increase in the extremity
of the average group opinion for both A- and B-Type agents as predicted and
observed in \citeA{Mas2013}. Then when A-Types and B-Types interact with one
another, our model predicts consensus emerges, 
as was observed by M\"{a}s and Flache's experiments and predicted by their 
model (Figure~\ref{fig:MasFlacheComparison} above; compare with Figure 6 \citeA{Mas2013}). 
Note that, in our model, no explicit persuasive
arguments are exchanged. Instead, each argument is represented as an opinion
on a certain cultural topic. Influence occurs on all cultural topics, and
similar group members draw one another closer in hypothetical 
12-dimensional opinion space through attractive social influence
and stubborn extremism, resulting in group polarization.

\begin{figure} %[t]  %[H]
    \centering
    \noautomath
  \includegraphics[width=0.885\textwidth]{/Users/mt/workspace/Papers/stubex/Figures/MeanOpinionVsTime_MF2013.pdf}
  \caption{Our model's prediction of group opinions in the M\"{a}s and Flache (2013) study. 
  Within-group interactions are rounds 1-3, intergroup interactions are rounds 4-7.}
  \label{fig:MasFlacheComparison}
  \vspace{-1em}
\end{figure}




\section{Discussion}
%Summarize and discuss the results in terms of explaining group polarization. Remember also that there were two alternate explanations presented: persuasive arguments and social comparison. Talk about these again. At minimum, the stubborn extremism is a competing explanation. It also seems more congruent with the available cognitive science on beliefs and social influence. Discuss limitations and the importance of understanding more about the cognitive processes involved in the maintenance and update of opinions. 

We have shown that stubborn extremists are a feasible explanation for group
polarization. Our model that incorporates this simple mechanism predicts
behavior observed in a number of empirical studies. These empirical studies have
often considered two alternative pathways to group polarization: 
\emph{persuasive arguments} and \emph{social comparisons}. The persuasive
argument theory explains that group polarization occurs because individuals
are exposed to more arguments supporting their initial position in contrast with the opposing opinions, thereby strengthening that opinion. 
At the group level, this leads the average opinion to shift towards
an extreme. Alternatively, social comparison theory posits that group 
polarization is due to group members calculating some optimal opinion to express publicly that takes into account both their private opinion and the perceived social consequences of expressing that opinion. The theory posits that, following group discussion, this optimal public opinion is usually judged to be more extreme than individuals' initially stated opinions.

First, to address persuasive arguments theory, 
it certainly matters what language and communication strategies are used.  % (\cite{Hart2005,Druckman2007,Kalmoe2014,Flusberg2017,Kalmoe2018}). 
Linguistic frames modulate the perceived meanings of words and sentences~\cite{Fillmore1982,Chong2007,Cacciatore2016}.
These frames often become norms that are shared, repeated, and 
modified by group members.
In this process linguistic frames co-evolve with the meanings of words~\cite{Hamilton2016,Garg2018,Hawkins2020}.
Metaphorical framing provides a particularly strong example of how language can
lead to extremism. \citeA{Kalmoe2014,Kalmoe2018} found that using violence
metaphors to describe political issues and events (e.g., ``EPA regulation
is \emph{strangling} the economy'') led participants to increase their support
for real world violence to reach political goals---this effect was even more
pronounced among the most trait aggressive participants.

Self-categorization theory is correct to assume that it is a fundamental human capacity 
to evaluate one's own and others' group membership status~\cite{Cikara2014,Cikara2017}. 
The desire to clearly belong
to one's in-group may well motivate individuals to increase their extremism 
in such a way as to lead to more clear signals of group membership, whether that is
from being drawn towards the direction others are tending, or to be more
clearly different from a perceived out-group.
Whether this is achieved through a calculation of the 
hypothesized ``meta-contrast ratio''~\cite{Turner1987} is less clear.
Using the meta-contrast ratio as a theoretical variable
calculated in the brain lacks the sort of mechanical explanation of behavior
as Bayesian cognitive models. To ensure the validity of the meta-contrast ratio,
or any other theoretical psychological calculation, one must co-develop a 
mechanistic model of how the value is calculated~\cite{Jones2011}, which does
not seem to be developed in self-categorization explanations of group polarization.

Social decision schemes models of group polarization posit that there exist
individual-level decision making traits (e.g., the ability to find or identify
a solution to some problem) and group-level decision making schemes (e.g.,
the group must unanimously vote to choose an opinion or behavior)~\cite{Brown2000}. 
Power dynamics are an important component for determining the 
social decision scheme used by a group~\cite{Friedkin1999a}.
If it is the case that that one can enumerate individual-level traits and group-level
decision schemes and power structures, then the social decision scheme model 
can theoretically be used to predict group decisions, opinions, and 
resulting behavior~\cite{Zuber1992,Friedkin1999a}. If the social decision
scheme model encodes or evolves extremists to be more powerful, then group
polarization will emerge. If extremists dominate the conversation, which seems
like it may plausibly occur often, then group polarization will emerge.
One issue here is the introduction of the social decision scheme construct, 
which itself would be subject to cultural evolutionary pressures depending
on group constitution and estimated payoffs of different strategies~\cite{King-Casas2005}. 
The idea of payoffs in a group polarization context is potentially problematic
as well since there is no tangible benefit to finding consensus, becoming
more extreme, etc. It can only be understood as emotionally beneficial.

We believe that the stubborn extremism explanation of group polarization is
a more appropriate starting point since it seems more parsimonious and
robustly supported than alternative explanations~\cite{Kinder2017,Reiss2019,Zmigrod2019}.
The stubborn extremism explanation makes one simple assumption, which could be
complemented by certain elements of existing explanations outlined above.
Even if stubborn extremism explains group polarization in some contexts, 
it is not clear which contexts. Our work does not address this 
important outstanding question directly. Likely it will take multiple methods
and approaches to understanding the subtleties of the effect of context
on group polarization. Although there is evidence supporting the hypothesis that extreme opinions are more stubbornly held, 
we are aware of no research specifically investigating the relationship between stubbornly held opinions and group polarization.
Future empirical work should evaluate the stubborn extremism 
hypothesis using a statistical model to detect correlation between 
opinion extremity and stubbornness. 

Models of opinion dynamics should be able 
to explain a number of empirical phenomena, including but not limited to group polarization. 
Another program of future work, then, could be to perform similar
computational experiments shown here using alternative, 
influential models of political polarization,
such as Bayesian/information-theoretic models (e.g.~\citeA{Dixit2007}) or 
algorithmic models (e.g.~\citeA{Dandekar2013}).
    



\chapter{Most group polarization results may be simple conformity}



  ``Group polarization'' is said to occur when socially isolated groups become more
  extreme following deliberation on some topic. This has clear implications for
  politics and other social organizations since extremism tears at the fabric
  of society. The goal of the current paper is to raise an alarm that
  many published results may plausibly be false detections of group polarization.
  These false detections are 
  caused by failing to account for how opinions are represented psychologically
  and measured in the physical world. Group polarization studies implicitly assume latent psychological opinions are continuous
  when they use \emph{t}-tests to detect group polarization, as many or most do.
  We demonstrate that if we assume 
  participant opinions are drawn from a continous distribution but reported
  on an ordinal scale, then common group polarization experiments could be
  reporting group polarization when groups really just converged to the 
  pre-deliberation average. This may be masking interesting differences 
  in social dynamics when the group is more moderate versus more extreme.
  Our analysis revealed other problems including
  a lack of specificity in process models of group polarization and a failure
  to account for important sources of variance (e.g., group membership 
  and survey item) in statistical models. To ensure reliable group polarization
  results, appropriate statistical designs must be adopted.

\newpage

\begin{quote}
In our introductory social psychology course, 
we have for many years used the [group polarization experimental paradigm] as
a laboratory exercise. The exercise works beautifully, but one must be
careful to forewarn a class that [group polarization] does not occur with every 
group\ldots and that the effect is not large. 
\par\raggedleft\cite[p. 205]{Brown1986}
\end{quote}

\begin{quote}
One of the most robust findings in social psychology is that of attitude polarization 
following discussion with like-minded others.
\par\raggedleft\cite[p. 267]{Cooper2001}
\end{quote}


\section{Introduction}
\label{sec:intro}

Social and political extremism and polarization threaten democratic 
institutions worldwide. If we could explain how and predict when extremism
emerges, we could brace for its ill effects and perhaps devise interventions
to counter it. To explain how extremism emerges, we need to focus on specific
instances where extremism does emerge, since social systems are complex systems.
Social psychologists, sociologists, political scientists, and legal
scholars for decades have tried to explain ``group polarization'', 
the name given to the specific phenomenon where small, 
socially isolated groups tend become even more extreme in their opinions if their
initial opinions centered around some non-neutral mean opinion.  Several 
theories explaining group polarization have emerged, supported by extensive
empirical evidence demonstrating that these groups reliably shift their opinions
to become more extreme after group 
deliberation~\cite{Brown1986,Brown2000,Schkade2010,Sieber2019}. The scientific consensus seems
to be that different theories are potentially valid since there exists 
supporting evidence for all~\cite{Brown2000}. However, we show in this paper
that the evidence for group polarization is weak at best, meaning that these
theories may be explaining a non-effect.

We demonstrate in this paper that a potentially 
large fraction of these group polarization detections are
plausibly false. This means that the decades of theorizing about 
group polarization may be for naught as there is no value explaining something
that does not really exist.. 
This is highly concerning given that the regularity of detecting
group polarization makes the phenomenon a celebrated social 
psychological result~\cite{Brown1986,Brown2000} that has essentially never
been seriously questioned as a real effect. One prominent author has even elevated it
to a scientific ``law''~\cite{Sunstein2002}. Litte work has been done
on group polarization in the past two decades, apparently because researchers 
thought it was real and explained well enough.  
Understanding group polarization, if it really exists, has broad impacts for 
society at large. We must have a solid empirical foundation to trust theoretical 
explanations of group polarization---our study suggests that foundation is
cracked at best.  
% Our work uncovered other potential issues with group polarization 
% studies, including a lack of theoretical specificity and 
% overgeneralization~\cite{Gervais2021,Yarkoni2021}. 
Mechanistic modeling and appropriate Bayesian statistics can be used
to elminate the problems we identify and explain here~\cite{Kruschke2018,Kruschke2018a,Turner2021}.

Many group polarization studies' findings are plausibly false due to their
use of metric models (e.g. $t-$tests) to detect group opinion shifts measured with ordinal valued
survey instruments. False detections plausibly occur due to the simpler process of consensus where
shifts in more extreme values are masked by ceiling effects, but shifts
from less extreme to more extreme opinions are detected~\cite{Liddell2018}. 
It is a common observation among group polarization researchers
that consensus occurs just as would be expected, i.e., the pre-deliberation
variance in opinions is greater than the post-deliberation variance~\cite{Asch1951,Asch1955,French1956,DeGroot1974}.
What makes group polarization special is that the consensus (mean) opinion
has increased in extremity compared to the pre-deliberation opinion.
However, using an ordinal scale introduces ceiling/floor effects so that 
those in, say, the 80th percentile and the 99.99th percentile opinions
report the same opinion, or worse. When simple consensus occurs we 
expect the less extreme opinions tend to become more extreme and the more extreme opinions tend to
be less extreme.  If simple consensus occurs, but only the less extreme opinions' shifts are detected, 
then the average will apparently increase. If 
participants' internal, ``latent'' psychological opinions are measured either
directly on a continuous scale (a minority of group polarization studies do) 
or indirectly somehow, then perhaps such problems could be avoided. 

We are not the
first to point out serious problems in the group polarization paradigm---\citeA{Cartwright1973} 
was concerned about poor theory and methods in group polarization from the start
when researchers believed the phenomenon only occurred in situations of
risk determination, and claimed their research should be used for studying
how critical decisions about, e.g., nuclear deterrence should be made. We
hope our work here further pushes group polarization researchers, and
social psychologists and others who measure opinion change, to develop sound
theories supported by valid statistical inferences.

To understand the technical and theoretical importance of this work, it is
necessary to first explain how group polarization experiments work.
Below we review 
common methods of inducing and detecting group polarization among groups 
of participants and how individual opinions and group polarization opinion shifts 
are typically measured.  
After introducing group polarization methods, 
we will explain in detail how these
methods plausibly lead to false group polarization detections.
Then we develop our
formal, generative statistical model that simulates false group polarization
detections. We then present our results
showing that over 90\% of published detections of group polarization
opinion shifts are plausibly false. We close with a discussion of whether
group polarization is real and how to improve group polarization research going 
forward.


\section{Group polarization theory, methods, and results}

Initially, the group polarization effect was thought only to apply to opinions about how much
risk would be appropriate to take given some life 
decision~\cite{Wallach1965,Teger1967,Stoner1968}. \citeA{Moscovici1969} then
showed that deliberation about political opinions also led to group 
polarization. At this point, motivated by \citeA{Cartwright1971} and
\citeA{Cartwright1973},
new explanatory mechanisms were proposed. The two explanations that survived to today 
are the social comparisons theory~\cite{Brown1974,Sanders1977,Myers1978} 
and persuasive arguments theory~\cite{Burnstein1973,Vinokur1974,Burnstein1975,Vinokur1978}. 
Around the time of Isenberg's
(1986) review, a self-categorization explanation~\cite{Turner1987} of
group polarization was developed and supposedly 
supported empirically~\cite{Turner1989,Abrams1990,Hogg1990,McGarty1992,Krizan2007}. 
There is also a
``social decisions scheme'' theory that identifies social power structures
as a dominant factor in the emergence of group polarization~\cite{Zuber1992,Friedkin1999a}. 
More recently, focus on the correlation between stubbornness and extremism has emerged as
a simple, empirically-motivated explanation of group 
polarization~\cite{Baldassarri2007,Mueller2018,Banisch2019,Turner2020}.
The results we present here, while damning for many group polarization studies,
will enable real progress to be made untangling this theoretical boondoggle.

\subsection{Common experimental design elements}

Group polarization studies tend to follow the same general experimental paradigm,
with slight variations to test particular theoretical explanations or 
real world situations. Participants first answer questionnaire
items or otherwise give their opinions or positions on some topic. Small groups
typically of 2-6 participants are formed such that the mean opinion or position
of group members is baised towards one or the other extreme of 
the measurement scale. Group formation is often based on initial participant
answers to the questionnaire. Sometimes researchers use a different, but similar, 
questionnaire to make like-minded groups. For example, \citeA{Myers1970} 
examined group polarization in the context of racial attitudes. To create groups
with different levels of mean tolerance or racism, Myers and Bishop used a
survey instrument to assess racial attitudes generally. Then they used a different
questionnaire on racial policy opinions for deliberation topics, where pre- and post-deliberation 
survey responses were used not to pick groups, but to measure group polarization.
In a different approach altogether, \citeA{Schkade2010} relied on a correlation between 
geographic location and political opinions to create novel groups that were
reliably biased towards liberal or conservative bias. 

In the most common paradigm participants first answer one or
several questionnaire items to determine their initial opinions on some 
deliberation topic. One widely used questionnaire is the choice dilemma
questionnaire first used by~\citeA{Stoner1961} to induce group polarization.
The questionnaire prompts participants for their opinions on how much risk would be 
acceptable for certain life decisions, such as whether or not to pursue
riskier research projects with higher payoffs compared to lower risk projects
with lower payoffs. Political questionnaires are also common. For example,
\citeA{Moscovici1969} asked Parisian lycée students about their 
opinions of then-president
Charles de Gaulle and of American foreign policy; \citeA{Myers1970} asked about
racial attitudes; \citeA{Schkade2010} asked about affirmative action, 
same-sex civil unions, and global warming. 

Most studies using questionnaires
prompt participants to give their responses on an ordinal, Likert-type scale.
Stoner's (1961; 1968) choice dilemma questionnaire was a 10-point ordinal
scale, with 1 representing the most risk acceptance and 10 representing the
least risk acceptance.  More generally common Likert scales typically 
have participants rate (un)favorability of some entity
in the world or degree of (dis)agreement with some statement of opinion or
belief. For example, when French students answered ``American economic aid
is always used for political pressure'', they marked a whole number on
a seven-point scale from
-3 (strongly disagree) to +3 (strongly agree), with zero representing 
neutral or no opinion. These scales do not always include 0 as the neutral
point. \citeA{Schkade2010} used a ten-point scale from 1 (disagree very strongly)
to 10 (agree very strongly).

Non-questionnaire group polarization studies have used a variety of 
methods. In one approach, researchers simulate jury deliberations
for an experimental design where participants give either opinions
on whether a defendant is guilty or how much money for damages should be
awarded~\cite{Kaplan1977,Kaplan1977a,Schkade2000,Schkade2007,Sunstein2000}\footnote{
Schkade, et al., (2000), entitled ``Deliberating about Dollars: The Severity Shift'',
was funded by Exxon Company, U.S.A., who have a clear interest in understanding
what causes individuals to raise or lower the amount of damages they believe
a responsible party should pay.}. Another approach studied group
polarization in the context of gambling behavior in the game of 
blackjack~\cite{Blascovich1974,Blascovich1975,Blascovich1976}, which
found that participants demonstrated opinion shifts to be more risky merely
when exposed to other group members' bets. Another odd example of questionable
\emph{prima facie} validity is an
experimental design that used an ``autokinetic situation'' where participants
watched a flashlight move in a darkened room, then deliberated about how
far the light moved after being told that longer measurements were more socially
desirable~\cite{Baron1976}. Our model does not apply to these studies, but there
are many more studies that use ordinal scales. Furthermore, other problems
such as not accounting for the multilevel structure of the data may 
subvert the validity of these studies.


\subsection{Common statistical procedures and implicit assumptions}

All group polarization studies we reviewed that used an ordinal opinion
measurement scale also used a $t$-test (or similar) to detect group polarization opinion
shifts. Statistical tests like $t$-tests are used to determine the probability
that two datasets were drawn or generated from the same distribution. These
tests assume that individual opinions are continuous
and normally distributed. To determine whether two datasets came from the
same distribution, a normal distribution is fit to each dataset. Then, the
probability that the two datasets were drawn from the same distribution
is proportional to the degree of overlap between the fitted distributions.
In the studies we reviewed, pre- and post-deliberation
distributions are always pooled over groups, and often by 
pooling over several items within one topic. For example, \citeA{Moscovici1969}
pool over 11 items in the topic about Charles de Gaulle and 12 items in the
topic and deliberation about American policy. 
\citeA{Schkade2010} provide a counterexample to this, where there is only
one item per topic. 

When $t$-tests are used to detect group polarization with ordinal observations, 
they are susceptible to false positives due to ignoring the effects of the
measurement process~\cite{Liddell2018}. 
The problem is that no matter how extreme a participant's
latent opinion is, it will be reported as the maximal ordinal value. This
means that an opinion in the 99th percentile of extremity may be mapped to the
same value as an opinion in the 80th extremity percentile. This means that if,
for example, an extremist shifted their opinion towards moderation, the 
measurement scheme could not detect this---it would appear as if the opinion
did not change at all. 

% Consider qualitatively what happens when a participant gives their opinion
% on an ordinal scale. If participants' opinions are continuous and there
% is perfect fidelity between the ability of a participant to convert their 
% opinion to a numerical value, then participants will convert their continuous
% opinion to the ordinal opinion bin value closest to their opinion. 
% For example,
% consider the 10-point choice dilemma questionnaire response scale 
% that runs from 1 to 10. If a participant's latent opinion is 9.2 they will
% report an ordinal opinion of 9. If their latent opinion is 9.6, 10.2, or 100,
% their reported ordinal opinion will be 10.



% \subsection{Theoretical explanations of group polarization}

% Below we review the four explanations or theories of group polarization 
% assumed or evaluated by the case studies we investigated for false detections. 
% We also review select empirical support for each explanation.
% The explanation of group polarization as due to the stubbornness of 
% extremists comes from empirically motivated modeling 
% projects that have yet to be verified empirically in group polarization settings.
% Therefore, we do not review that here. Future work will use the results of
% the present paper to devise more appropriate measurement and statistical 
% procedures that will help ensure the validity of future empirical studies.

% Following our theoretical review, we identify and explain common experimental design
% elements and statistical methods used commonly across group polarization research
% independent of theoretical aims and assumptions. Then in this section
% we review findings from these decades of research, which overwhemingly support 
% group polarization in general---each theory can boast supporting empirical
% evidence as well. This sets up the following section where
% we explain our model in mathematical detail that we will use to
% show that we should be highly skeptical of the broadly supportive evidence for group polarization. 

% \subsubsection{Social comparisons}

% When researchers began searching for an explanation of group polarization
% in response to Cartwright's (1971; 1973) critiques of the ``risky shift'' literature, 
% some adapted the extant ``theory of social comparison processes''~\cite{Festinger1954} of group-level
% social influence as an explanation. This theory assumes that when people
% interact in group settings, each individual infers what the prevailing
% social norms are, compares their own opinion to the social norm, and adjusts 
% one's own opinions or behaviors so they are more socially accepted or celebrated. 
% One testable corollary of this explanation is that no deliberation is required, \emph{per se}.
% All that is required is ``mere exposure'' to others' 
% opinions~\cite{Zajonc1968,Burgess1971,Bornstein1990,Montoya2017}. Several studies
% have shown that when a group polarization experiment is run as explained above,
% but without group deliberation, non-verbal displays of individual opinions 
% to the group is alone sufficient social influence to foster group 
% polarization~\cite{Teger1967,Blascovich1973,Blascovich1975,Blascovich1976,Sanders1977,Myers1978,Myers1982}.

% Just because mere exposure to others' opinions tends to lead to group polarization
% does not necessarily support all auxiliary assumptions made by the
% social comparisons explanation~\cite{Meehl1990}. It is not clear what the
% mechanism is by which individuals infer the group norm if it is not just the
% average. How is it, exactly, that individuals infer this more extreme than
% average group norm? \citeA{Festinger1954} assumes first that ``there is a 
% universal human drive to evaluate our opinions and abilities'' \cite[p. 78]{Brown2000}.
% But how ubiquitous is this drive to distinguish oneself through conformity?
% Clearly individuals vary in their drive to conform to social norms in general---how
% does this affect group polarization opinion shifts?
% Furthermore, achieving distinctiveness through conformity may have counterintuitive
% effects~\cite{Smaldino2015a}. Social comparisons theory fails to make contact
% with extensive literature on norms and norm change, which should be accounted
% for~\cite{Bicchieri2006,Bicchieri2014,Bicchieri2017}.

% These are important questions to answer. Perhaps social comparisons offers a
% good starting point for a partial explanation of group polarization, but 
% its epistemological status is shaky. It is therefore important that we
% understand how to properly measure opinion shifts to either support,
% refute, or revise and incorporate the social comparisons account into a 
% broader explanatory model of group polarization.

% \subsubsection{Persuasive arguments}

% Persuasive arguments theory explains that opinion change is determined by the number and persuasiveness of 
% arguments that support different poles of the opinion scale. Arguments, then,
% are central theoretical entities in this model alongside opinions. If there are more
% arguments favoring one polar opinion (disagree/agree) over another~\cite{Ebbesen1974}, or if
% arguments that exist for one polar opinion are more persuasive 
% then the group will collectively move towards that 
% polar opinion~\cite{Vinokur1974,Burnstein1977}. This theory assumes that for an argument to have an effect
% on a participant, that participants must not have heard the argument 
% before~\cite[see Equation on p. 96]{Bishop1974}. Furthermore, the validity,
% or informativeness, is hypothesized to be the primary auxiliary factor in 
% determining the magnitude of influence for a given argument~\cite{}. These
% assertions are supported by statistical inference using $t$-tests (or similar)
% in experimental manipulations designed to measure novelty and informativeness
% by controlling for other factors. 
% We show in our Analysis, however, that all experimental conditions in 
% two influential persuasive arguments studies are plausibly false.

% One problem with the persuasive arguments explanation is that only arguments 
% are persuasive, not people. Perhaps, for example, there is a simple 
% consistency in that more extreme individuals tend to be more persuasive than moderates, 
% perhaps due to their confidence in their opinions. This assumption would actually 
% explain observations made by \citeA{Burnstein1973} who 
% found that insincere arguments are not influential.
% Another related problem is underspecified
% psycholinguistic mechanisms of social influence. Perhaps novelty and 
% informativeness are two important factors in what makes an argument persuasive.
% Surely, though, there are other factors.  

% Note that based on these theoretical assumptions, 
% study data must be analyzed with multilevel
% statistical models that account for group membership 
% (ignoring whether the assumptions are well-founded)~\cite{GelmanHillRegression,Yarkoni2021}. 
% This is because the number of arguments is a key model factor in predicting opinion shifts,
% but the number of arguments available in each group surely varies by group.
% Unfortunately, $t$-tests (and similar) lack the capacity for multilevel structure.


% \subsubsection{Self-categorization}

% Self-categorization accounts for theory posits that people conform on what attitudes,
% opinions, or beliefs to hold, by considering how best to ``contrast'' themselves
% with members of an outgroup so as to solidify their membership with
% an ingroup~\cite{Tajfel1971,Tajfel1979,Turner1987}. 
% Experiments testing the self-categorization hypothesis use 
% the minimal group paradigm approach to
% understand differences in social influence (that leads to extremism) 
% between in-group members versus out-group members. 
% In one interesting counter-example to
% the persuasive arguments theory, the basic experimental design was used, but
% participants did not interact with a group---instead they were listened to 
% tape recordings of arguments for or against some statement. Participants
% were told they would either be joining the group or that they were listening
% to members of an out-group. This changed whether opinion shifts were to a greater
% extreme they were already bised towards (in-group) or if participant 
% opinions tended to shift away from their initial bias (out-group). 
% Persuasive arguments theory does not account
% for group membership, so it could not have predicted this result. 
% The minimal group approach continues
% to be applied today across cognitive sciences, especially in understanding
% the neuroscience of emotions towards novel in- and 
% out-groups~\cite{Cikara2014,Molenberghs2014}.

% To explain
% group polarization, where there no explicit out-group, self-categorization
% theorists proposed that people engaging in social interaction mentally
% calculate the ``metacontrast ratio'', which is defined as a person's average
% distance in opinion space from all out-group members divided by that person's
% average opinion distance from all in-group members~\cite[p. 3]{McGarty1992}.
% This requires them to infer their average distance to the imagined outgroup.
% A person is then hypothesized to update their opinions to match the prototypical
% opinion, which is defined as ``the pre-test mean where the mean is at the mid-point
% of the comparative context\ldots.'' This leads to group polarization, since
% ``(a)s in-group responses shift\ldots towards a more extreme position, 
% then it becomes more likely that the prototype will tend to be more extreme than
% the mean in the same direction'' (p. 4, \emph{ibid}). 

% While neuroscientific studies implementing the minimal group paradigm support
% the assumption that differential social influence depends 
% on whether an individual interacts
% with in-group or out-group members, it is not clear that it operates
% as hypothesized in self-categorization explanations of group polarization. 
% Specifically, the assumption
% that people calculate meta-contrast ratios and hypothetical in-group
% prototype opinions does not seem to be empirically supported. 
% It is not clear to us how such a claim could be empirically supported. 
% Another possible critique is that this reasoning seems to be circular: the 
% in-group prototype begins as the pre-deliberation
% mean, but changes once opinions begin to change. This seems to sidestep the
% problem of how opinions change in the first place and why the average opinion
% tends to become more extreme. Finally, it seems that perhaps ``prototype'' in
% the self-categorization explanation is homologous in form and function to
% a ``norm'' in social comparisons theory. Future work should explore this connection
% in more detail to understand exactly how the two theories substantively differ.


% \subsubsection{Social decision schemes}

% Social decision schemes generally considers the social structure of groups to
% account to determine what opinions or behaviors group members will 
% take in the course of group interaction~\cite{Davis1973}. In the main branch
% of social decision schemes, individual-level interaction strategies are 
% hypothesized and specified. To understand social decision schemes, 
% consider the following example adapted from 
% \citeA[p. 195]{Brown2000}. Assume a group is trying to solve some problem.
% The group may be composed of three types of people: 
% (1) people who are able to solve the problem, (2) people
% who can recognize a solution but not solve the problem themeselves, and
% (3) people who cannot solve the problem or recognize a correct solution.
% The group may adopt different decision rules, such as ``Truth wins'' 
% (as long as one member has the solution, the group solves the problem),
% ``Majority rule'' (a majority of group members must know or recognize the
% solution), or ``Unanimous'' (all group members must know or recognize the solution).
% If we know the composition of the group in terms of these three types, then
% we can calculate the probability a group solves the problem. 
% According to the social decision schemes
% framework, if we observe how often a group solves a
% problem and we know the distribution of strategies, we can infer the
% decision rule used by the group.

% In the context of group polarization, instead of recognizing solutions to
% problems, people are assumed to adopt a strategy of ``risk wins'',
% ``conservatism wins'', or ``majority wins'' in the context of the
% choice dilemma questionnaire \cite{Laughlin1982,Zuber1992}. 
% Unsurprisingly, when group polarization was found to shift towards greater
% risk in these studies, \citeA{Laughlin1982} found that the best-fitting model
% was the situation where 
% \citeA{Friedkin1999} counts his network theoretic model as aligned with the
% social decision schemes approach, especially over and above alternative 
% explanations of social comparisons, persuasive arguments, and self-categorization.
% Friedkin found support for his hypothesis that such power structures are
% predictive of group polarization. 
% Unfortunately, several of Friedkin's results are \emph{prima facie} 
% plausibly null since several of the confidence intervals around the opinion
% shift measurements include zero. Furthermore, we found that even when the
% confidence intervals do not include zero, our model demonstrates Friedkin's
% detections of group polarization are plausibly false.

% One issue with the social decision schemes approach seems to be that the
% emergence of distribution of strategies, and the strategies themselves, is
% not accounted for. How does such a norm as ``risk wins'' emerge? How is this
% not a ``norm'' or ``prototype'' as could be found in either the social comparisons
% or self-categorization explanations, respectively? 


% % \subsection{Review}

% % We now have reviewed the outstanding theoretical questions in group polarization
% % studies, how ordinal measurements of opinions work, how group polarization
% % opinion shifts are commonly measured and the implicit assumptions that 
% % imposes, and the empirical support for the four extant theoretical explanations
% % of group polarization. 
% % It is rare indeed that supposedly significant results are not found 
% % when expected in group polarization experiments. When there are surprising results
% % these are waved away like a pesky fly. For example, on noticing that
% % there was a significant shift for a Choice Dilemma Questionnaire item when
% % none was expected, \citeA{Burnstein1973} called their finding ``curious''
% % and that it defied ``straightforward explanation'', as if that is something
% % we should expect when doing science. They decided to ``merely note'' the
% % ``occurrence and (did) not venture to speculate as to (the) cause''.

% % \citeA{Isenberg1986} found through a meta-analysis that effect sizes are
% % significantly higher for persuasive arguments accounts of group polarization,
% % as compared with social comparisons theory. Given the practical and 
% % theoretical problems we have identified, 
% % No such meta-analysis has been done to compare the predictions. But, given our analysis that follows,
% % there would be no point because several of these studies present mostly 
% % plausibly false detections of group polarization opinion shifts.



% % \section{Metric models of ordinal data lead to false detections 
% % (and general confusion)}



\section{Model}

Our primary goal in this paper is to evaluate whether published positive
detections of group polarization are reliably true, or, equivalently, plausibly false. 
We do this by first developing a generative model of group polarization 
experiments that simulates how opinions are reported and change, and
how standard analytical techniques can generate the appearance of group
polarization where none exists.
Our model is based on the assumptions that (1) a 
participant's internal ``latent'' opinion on some topic can be represented as a real number
varying continuously; (2) when a participant reports their 
opinion on an ordinal scale, the formulation
of their latent opinion can be represented as a draw from a latent opinion distribution; 
and (3) participants faithfully convert their continuous latent
opinion into whatever ordinal ratings scale (e.g.\ a Likert scale) 
the experimenters present them with. 
Note that these assumptions assume there are there are two forms of opinions. 
There are \emph{latent} opinions that are somehow represented and formulated in a 
person's mind, but never directly observed. Then there are \emph{observed}
opinions that participants report on an ordinal scale. We also then have two
distributions of opinions that do not in general have the same summary statistics (mean
and variance).

We make these assumptions for the sake of consistency with 
the implicit assumptions made in psychological and social science studies of
opinions. When someone gives their opinion
on some topic it is the result of a complex psychological process that is
sensitive to personal beliefs and experiences, and cultural and contextual factors.
Because of this complexity, opinions may not in fact be readily mapped onto a 
unidirectional scale, continuous or ordinal. For our purposes we can ignore
this possibility because our goal is to show that, under common assumptions 
of group polarization studies, many detections of group polarization 
may plausibly be false detections.

Because simple conformity to the mean can be masked by ordinal measurements,
a change in pre- and post-deliberation opinion variance can masquerade as 
group polarization, i.e., a change in \emph{mean} from pre- to post-deliberation.
Theoretically, we expect variance to decrease from pre- to post-deliberation
as participants feel pressure to 
conform~\cite{Asch1951,Asch1955,French1956,DeGroot1974,Lorenz2009}. 
Conformity has been observed across group polarizaiton studies, with
many containing explicit instructions to find consensus with group
members as part of the experimental design.


\subsection{Formal model}

Formal models are important to develop because in doing so 
we specify which social influence components 
are important, and which are not~\cite{Kauffman1970,Cartwright1999}. 
Our formal model incorporates three 
main features we review now. First, we
formalize our assumptions about what opinions are and how they are generated
``internally'' in model participants. Next, we formalize the measurement
process where participants transform their internal, \emph{latent} opinions to
their reported opinions in one of several ordinal scale bins, e.g., a Likert
scale. Finally, we develop a statistical model that can generate plausibly
false detections of group polarization if that is possible, or fail if it is
not possible, which instead would support a positive finding of group 
polarization. 

All model calculations are done in the large $N$ limit. This enables us to 
perform exact calculations to directly find what pre- and post-deliberation variances
could have generated false detections of group polarization. 
Theoretically, effect sizes calculated with finite $N$ will be less reliable, if
anything, so demonstrating that a false discovery occurs even in the large-$N$
limit is a sort of formal proof that there exists a plausible
combination of parameters that gives rise to a false group polarization discovery.

Each experimental condition that claims to detect group polarization
is a ``possible false detection'' (Table~\ref{tab:resultsSummary}). 
When we determine that a false detection is plausible, that means we have no
data to decisively say whether or not the published result is reliable, 
meaning we cannot count it as evidence of group polarization.

After we formally introduce the psychological representation of opinions, we
will consider how a large collection of opinions becomes a distribution of
observed ordinal scale opinion ratings, which in turn are used to calculate
mean pre- and post-deliberation opinions and which, in experimental analyses,
are tested against one another to detect a significant opinion shift due to
group polarization. We will attempt to generate pre- and post-deliberation
observed opinion distributions with different means, but that were 
generated from two latent distributions with the \emph{same} mean. 
Different pre- and post-deliberation latent standard deviations are what
cause different observed mean opinions to be generated, 
even though latent means are identical.

\subsubsection{Opinions}

We assume that participant $i$'s internal, latent psychological opinion at 
time of reporting ($t \in \{pre, post\}$) is drawn from a normal distribution
with mean $\mu_t$ and standard deviation $\sigma_t$,
\begin{equation}
  o_{i,t} \sim \mathcal{N}(\mu_t, \sigma_t).
  \label{eq:opinionDistribution}
\end{equation}
\noindent
All group polarization studies we have reviewed, using both ordinal and
continuous measures of opinion, make this same assumption, which implicitly
pools participant data over groups, even though it is well-known that, e.g.,
the initial extremity of the group predicts the magnitude of the group
polarization opinion shift~\cite{Myers1982}. Studies such as
\citeA{Moscovici1969,Myers1970} also implicitly pool over opinion items, on which
participants give several opinions, but these shifts are given only as an
average over all items (and groups). Future work should examine the impact of
this practice, which has been shown to lead to overgeneralizations and
overestimations of other psychological effects~\cite{Clark1973,Yarkoni2021}.


\subsubsection{Experiment model}

\begin{figure}
  \centering

    \includegraphics[width=1.05\textwidth]{/Users/mt/workspace/Papers/gp-stat/Figures/Model/ExperimentModel.pdf}

  \caption{
    Schematic diagram of our model of a group polarization experiment.
    Many experiments add additional complexity, but this simple model suffices
    for studying the effect of measurement and statistical procedures on 
    empirical results. For each of the ten case studies presented here 
    In Step 1, participants have not yet met one another and
    so report their opinions independently of any experimental social influence.
    We denote $t=pre$ at this stage, referring to \emph{pre}-deliberation. 
    In Step 2, a discussion
    group is formed that has an overall bias in one direction or another.
    It is through discussion that opinions are hypothesized to change, i.e.,
    group polarization occurs. At the third and final step, post-deliberation
    ($t=post$), participants again report their opinions, which, if group 
    polarization has occurred, have increased in extremity overall. 
  }
  \label{fig:modelSchematic}
\end{figure}

There are many versions of the group polarization experiment, however they all
share three main steps, which constitute our model here~\cite[p. 143]{Turner1987Book}
(Figure~\ref{fig:modelSchematic}).
First, typically before small deliberation groups are formed, participants
are given a questionnaire on which the indicate their initial opinions on the
item(s) on the experiment's topic(s) of discussion. 
% Extant empirical studies
% assume, as we do here, that participants opinions are being drawn from a
% latent, normal opinion distribution with latent mean $\mu_{t=pre}$ and
% latent standard deviation $\sigma_{pre}$. 
We generate pre-deliberation
data by first drawing a latent opinion from this distribution, then 
binning participant opinions into an ordinal opinion scale, which is 
described in more detail in the next subsection on the Measurement Model.

Next, participants are placed with a small discussion group with all
or mostly others who share their bias, e.g., towards -3 or +3 on a seven-point
Likert scale, and then the participants deliberate in these biased 
groups---though in some conditions
participants may only display their opinion to others or some other sort of
twist on communicating individual opinions.
To form bias groups in our model
we simply assume participant opinions are drawn from a non-neutral latent mean.
Deliberation is simulated in the aggregate, with its effects modeled as a possible change
in mean (if group polarization does indeed occur) and as a decrease in variance
due to consensus/conformity processes.
After deliberation, participants again report their opinions. 
To generate a false detection, we assume that the pre- and post-deliberation
means are identical ($\mu_{pre} = \mu_{post}$), but their variances are not. 


\subsubsection{Measurement model}

\input{/Users/mt/workspace/Papers/gp-stat/ModelFigure.tex}

Our measurement model transforms a distribution of
pre- or post-deliberation latent opinions into an ordinal-valued
distribution of ordinal scale opinion measurements. 
This simulates the three step group polarization experimental design where
participants do not directly report their continuous latent opinions, but instead 
report their opinions in terms of a finite set of ordinal bins.
Formally, this is achieved by integrating over the probability density function
(Equation~\ref{eq:opinionDistribution}) of opinions for each ordinal scale 
bin (Figure~\ref{fig:NoChangeHypothesisIllustration}). 

We assume that participants give their opinions in terms of one of $K$ opinion
bins, with each bin value denoted $b_k$ and indexed by $k=1,\ldots,K$. 
The array of all bin values a participant may choose is 
simply $b$. In the popular choice dilemma questionnaires the
ten opinion bins are $b = \{1, 2, \ldots, 10\}$.
In this case, we happen to have $b_k = k$.
A seven-point Likert scale (e.g., -3 strongly disagree, 0 neutral, and +3 
strongly agree) has $K=7$ bins, $b = \{-3, -2,\ldots,3\}$, i.e.\ 
$b_1 = -3$ and $b_{K=7} = 3$.  

An individual reports an opinion in bin $b_k$ if their latent opinion is
within bin thresholds $\theta_{k-1}$ and $\theta_k$. There are
$K+1$ thresholds, starting from $\theta_0 = -\infty$. Similarly, $\theta_K = \infty$.
Other than $k=\{0,K\}$, $\theta_k = b_k + 0.5$. Taking the example of a seven-bin
Likert scale, if $o_{i,t} = 1.4$, then participant $i$ would report a 
binned opinion of $b_5 = 1$. In this case
we assume for simplicity that except for thresholds at $\pm \infty$, 
thresholds are separated by 1 in ``opinion space''---for more on 
the Cartesian representation of opinions see \citeA{Blau1974}.

We model measurement of $N \to \infty$ participant opinions,
which results in a histogram of frequency of opinions
in each bin, i.e., $o_{i,t} = b_k$. The frequency of responses in each bin is 
the integral over the continuous normal probability density function from 
one bin threshold to another. This transforms the probability density 
function from $p(o_{i,t};~\mu_t, \sigma_t)$ to the probability
of observing a reported opinion of each bin value. 
The probability of observing an opinion in bin
$b_k$ is calculated by integrating the normal probability density 
function over the range of $\theta_{k-1}$ to $\theta_{k}$. 
Formally, we write the 
probability of observing an opinion in bin $k$ at time $t$ as
\begin{equation}
\begin{aligned}
  p(o = b_k;\mu_t, \sigma_t, \theta) 
    &= \int_{\theta_{k-1}}^{\theta_k} p(o;\mu_t, \sigma_t) do \\
    &= \Phi \left( \frac{o - \mu_t}{\sigma_t} \right)\Big|_{o=\theta_{k-1}}^{\theta_k} \\
    &= \Phi \left( \frac{\theta_k - \mu_t}{\sigma_t} \right) - 
       \Phi \left( \frac{\theta_{k-1} - \mu_t}{\sigma_t} \right)
\end{aligned}
\label{eq:binFrequency}
\end{equation}
\noindent
where $\Phi(\frac{o - \mu}{\sigma})$ is the normalized 
normal cumulative distribution function over opinions $o$, shifted by an amount $\mu$ with
standard deviation $\sigma$.

From this we can calculate the simulated expected value of observed
opinions at time $t \in \{pre,post\}$, written
\begin{equation}
  \langle o_t \rangle = 
        \sum_{k=1}^K b_k \cdot p(o = b_k ; \mu_t, \sigma_t, \theta).
  \label{eq:expectedOpinion}
\end{equation}
\noindent
We differentiate this from the mean opinion observed in a particular 
experimental condition, which we write $\bar{o}_t$. Importantly for performing
our investigation of whether published findings may be false detections,
it is vanishingly rare that the latent mean and expected observed value
are identical, i.e.\ it is rare to find $\mu_t = \langle o_t \rangle$.
This occurs only for $\mu_t$ at the exact midpoint of the ordinal
opinion measurement scale. This fact underlies our method for generating
false detections explained in the following subsection.



\subsubsection{False detection model}


We use our model to re-evaluate the reported results to see if, in fact,
the null hypothesis is plausible, i.e.\ there is plausibly no difference 
between pre- and post-deliberation means.
To do this, we test data from published experiments assuming the null
hypothesis, i.e.  $\mu_{pre} = \mu_{post}$. 
We demonstrate that the null hypothesis 
is often plausible, i.e., that there was in fact no shift in group opinions.
We do this by first finding a latent mean that generates the observed pre- and 
post-deliberation means ($\langle o_{i,pre} \rangle$ and $\langle o_{i,post} \rangle$) 
reported in published studies, 
for certain pre- and post-deliberation latent standard deviations 
($\sigma_{pre}$ and $\sigma_{post}$). The challenge is to identify which
$\sigma_t$ generate false detections.

To find $\sigma_t$ we might first think to simply set the observed mean
equal to the calculated mean, i.e., $\bar{o}_t$. However, it is not clear if
this is tractable to solve directly.
Therefore, we solve for $\sigma_{t}$ numerically by finding the 
the $\sigma_t$ that minimizes 
the squared error between the observed mean, $\bar{o}_t$, and the
simulated observed mean, $\langle o_t \rangle$, i.e.
\begin{equation}
  \begin{aligned}
     \sigma_t & = \argmin_{\sigma}~(\bar{o}_t   - \langle o_t \rangle)^2 
        = \argmin_{\sigma}~(\bar{o}_t   - \sum_{k=1}^K b_k \cdot p(o = b_k ; \mu_t, \sigma_t, \theta)^2  \\
       & \text{s.t. } |\bar{o}_t - \langle o_t \rangle| < \epsilon 
  \end{aligned}
    \label{eq:argmin}
\end{equation}
\noindent
where $\epsilon$ is the error tolerance. We will find that different studies
allow for finding $\sigma_t$ with larger or smaller $\epsilon$. Note that since 
all bins are unit distance from one another, it is reasonable to set 
$\epsilon \sim 0.1$, especially since we are comparing large-$N$ simulations
with finite-$N$ observations. However, we use the smallest possible
$\epsilon$ we can as long as it is at most on the order of 0.1; the values of
$\epsilon$ used for each condition is available in an Excel spreadsheet
we have provided as supplemental information.  

There are two ways for this search to fail. First, there could be total
failure, i.e.\ no
$\sigma$ that generate both pre- and post-deliberation 
distributions whose means match those reported in a given experimental 
condition. Second, a solution to Equation~\ref{eq:argmin} may be found, 
but the solution $\sigma$ is too large, which yields a highly ``bi-polarized''
distribution that is not feasible in most group polarization studies,
which do not contain groups of opposing viewpoints.


\subsection{Model implementation and analysis}

We implemented the model in R using primarily built-in or open source
packages~\cite{RLang}. We programmed a simple hillclimbing algorithm
to solve the optimization problem in Equation~\ref{eq:argmin} to find 
which latent standard deviations $\sigma_{pre}$ and $\sigma_{post}$
generate the observed data for a given observed group polarization
opinion shift.

To facilitate the use and re-use of this code, we also developed
a Shiny web application\footnote{https://mt-digital.shinyapps.io/grouppolarizationstatmod/} 
to specify observed pre- and post-deliberation
mean opinions, a hypothesized latent mean, the measurement scale, and
to vary hillclimbing parameters (step size and stopping condition). This
app will display theoretical histograms of responses for the binned
latent pre- and post-deliberation opinion distributions.  

In each study, we tabulate the number of plausible false discoveries made out
of the number of potential false discoveries, which is equal to the number of
experimental conditions in each study. 
For example, in the case study of \citeA{Schkade2010} 
we analyze below, they calculate group polarization opinion shifts in
six experimental conditions. The conditions arise from two 
geographical locations where groups were assembled (Boulder, CO and Colorado
Springs, CO) and three deliberation topics (affirmative action, 
civil unions, and global warming). For this case study,
we inspect each of the six conditions to determine if the observed shift
reported for each condition is plausibly a false detection arising from 
simple a consensus process (reduction in opinion variance from pre- to 
post-deliberation) instead of a group polarization process. For each
of the 62 experimental conditions we inspected across ten case studies,
we made this determination of plausibility of the null hypothesis and recorded
the latent mean and latent pre- and post-deliberation standard deviations,
and hillclimbing step size parameter,
that gave rise to our counterexample data supporting our assertion of a 
plausibly false detection. 

With all case studies inspected this way,
we obtained a table with columns Study, Experimental Condition, and
whether the detection was Plausibly False. We then calculated the worst-case
false detection rate for individual studies and a global worst case
false discovery rate across all ten original published studies (files with
original data and analyses will be available in a supplement). 

We developed our model to analyze published results demonstrating 
group polarization and other opinion shifts to determine whether shift
detections are actually plausibly false. We evaluated 62 experimental
conditions across ten influential published studies. Our approach can and should be
applied to more studies. This can be achieved by the following 
strategy that we used to perform our Analysis presented in the next section.
We can only provide the worst case false detection rate because we do not, 
and can not in any of the case studies, 
know if plausibly false detections are false or not. This is not a comfort,
since this means that plausibly false detections are unreliable, i.e.,
of no practical scientific value. If original source data had been provided
then the data could have been re-analyzed with proper statistical methods,
and perhaps discoveries of group polarization could be confirmed.

More specific information about how our model and hillclimbing algorithm
for solving Equation~\ref{eq:argmin} were implemented can be found in the 
Appendix.



\section{Analysis}

We now show our results of applying our model to analyze whether published
detections of group polarization are false. We chose ten influential 
group polarization studies published between 1969 and
2010.  Each case study has one or more experimental conditions in which a 
group polarization opinion shift was hypothesized to occur. The studies
reported experimental data and possibly associated statistical tests
to support their hypothesis that group polarization occurred in groups
subjected to these conditions.
  
Across the ten case studies, we found that 92\% of group polarization
detections are plausibly false according to our model. No studies
had a false detection rate below 50\% (Table~\ref{tab:resultsSummary}),
This means that for each published group polarization paper, at least half of
their group polarization detections are explained by simple conformity.
\citeA{Myers1970} and~\citeA{Myers1975} had the lowest plausibly 
false detection rates (67\% and 50\%, respectively), 
possibly due to their use of 18-point Likert scales and highly
charged deliberation topics, including racism and gender roles and
relations in the United States.

\vspace{2em}

\input{/Users/mt/workspace/Papers/gp-stat/FalseDiscoveryRatesCustomized.tex}

\section{Discussion}

In this paper we showed that many published studies presented plausibly 
false detections of group polarization that could be equivalently described
as conformity to the initial group mean, not to a more extreme mean. 
This was enabled by the use
of metric statistical models to make inferences about ordinal valued data,
which induces ceiling effects that mask changes in opinions among the most
extreme group members.  Because the original data is not
available, we cannot, can show that the observed effects are 
true or false detections, nor can anyone else. 
Unfortunately for the authors of these studies and
for psychological science in general, this means that the results cannot be
used to support the theoretical explanations of group polarization they were
meant to. Furthermore, it causes us to doubt whether there is a group polarization
effect at all. 

The immediate solution is clear: use appropriate statistical procedures 
for group polarization research that uses ordinal opinion measurement
scales, but assumes opinions are countinuous. 
This means we must expand our statistical models to incorporate opinion binning.
This can be achieved through the use of ordered probit models, or any other
statistical model that treats observed data as ordinal, and generated from 
binning continuous opinions into categorical bins. In the course of our study
we also found that statistical models of group polarization failed to account
for the multilevel, and sometimes hierarchical, structure of group polarization
data. This must also be accounted for in the design of valid, robust 
statistical models of group polarization. 


\subsection{Is group polarization real?}

It may seem that we have little justification left for asserting the reality of
group polarization. We have demonstrated many detections of group polarization
are plausibly false. Furthermore, in the course of this study, 
we observed that significant sources of variance are regularly not 
accounted for in statistical models used in group polarization research. 
This results in a lack of multilevel structure in statistical models that
tends to lead to overestimates of effect sizes and underestimates of 
confidence interval widths~\cite{Clark1973,Yarkoni2021}.
The theoretical weaknesses identified earlier and these facts may seem to 
kill off any potential reality of group polarization. No research data is available
from previous studies, so the data cannot be re-analyzed.
Although we may no longer count group polarization as an empirical reality
(pending new work using appropriate statistical methods), 
there are several theoretical reasons to believe that properly designed studies
will find, in certain cases, that group
opinions become more extreme following deliberation. 

% Existing theories and explanations of group polarization contain 
% some valid, empirically supported points that should be incorporated
% in group polarization theory going forward. Persuasive arguments theory is right to hypothesize a
% relationship between language and extremism. Self-categorization theory 
% correctly identifies the emotional importance of unambiguous group
% membership. The social decision schemes explanation predicts that if extremists are
% more powerful or influential, then group polarization will occur. 

% (REMOVED REVIEW OF EXISTING THEORIES OF GROUP POLARIZATION --- STILL NEED TO
% FINISH THIS SUBSECTION, BUT MAYBE JUST CONDENSE ALL THESE INTO ONE PARAGRAPH)
% First, to address persuasive arguments theory, 
% it certainly matters what language and communication strategies are used.  % (\cite{Hart2005,Druckman2007,Kalmoe2014,Flusberg2017,Kalmoe2018}). 
% Linguistic frames modulate the perceived meanings of words and sentences~\cite{Fillmore1982,Chong2007,Cacciatore2016}.
% These frames often become norms that are shared, repeated, and 
% modified by group members.
% In this process linguistic frames co-evolve with the meanings of words~\cite{Hamilton2016,Garg2018,Hawkins2020}.
% Metaphorical framing provides a particularly strong example of how language can
% lead to extremism. \citeA{Kalmoe2014,Kalmoe2018} found that using violence
% metaphors to describe political issues and events (e.g., ``EPA regulation
% is \emph{strangling} the economy'') led participants to increase their support
% for real world violence to reach political goals---this effect was even more
% pronounced among the most trait aggressive participants.

% Self-categorization theory is right that it is a fundamental human capacity 
% to evaluate one's own and others' group membership status~\cite{Cikara2014,Cikara2017}. 
% The desire to clearly belong
% to one's in-group may well motivate individuals to increase their extremism 
% in such a way as to lead to more clear signals of group membership, whether that is
% from being drawn towards the direction others are tending, or to be more
% clearly different from a perceived out-group.
% Whether this is achieved through a calculation of the 
% hypothesized ``meta-contrast ratio''~\cite{Turner1987} is less clear.
% Using the meta-contrast ratio as a theoretical variable
% calculated in the brain lacks the sort of mechanical explanation of behavior
% as Bayesian cognitive models. To ensure the validity of the meta-contrast ratio,
% or any other theoretical psychological calculation, one must co-develop a 
% mechanistic model of how the value is calculated~\cite{Jones2011}.

% Social decision schemes models of group polarization posit that there exist
% individual-level decision making traits (e.g., the ability to find or identify
% a solution to some problem) and group-level decision making schemes (e.g.,
% the group must unanimously vote to choose an opinion or behavior)~\cite{Brown2000}. 
% Power dynamics are an important component for determining the 
% social decision scheme used by a group~\cite{Friedkin1999a}.
% If it is the case that that one can enumerate individual-level traits and group-level
% decision schemes and power structures, then the social decision scheme model 
% can theoretically be used to predict group decisions, opinions, and 
% resulting behavior~\cite{Zuber1992,Friedkin1999a}. If the social decision
% scheme model encodes or evolves extremists to be more powerful, then group
% polarization will emerge. If extremists dominate the conversation, which seems
% like it may plausibly occur often, then group polarization will emerge.
% One issue here is the introduction of the social decision scheme construct, 
% which itself would be subject to cultural evolutionary pressures depending
% on group constitution and estimated payoffs of different strategies~\cite{King-Casas2005}. 
% The idea of payoffs in a group polarization context is potentially problematic
% as well since there is no tangible benefit to finding consensus, becoming
% more extreme, etc. It can only be understood as emotionally beneficial.

% Finally, we believe group polarization will again be considered a real
% empirical phenomenon due to its emergence
% in different agent-based computational models of 
% social influence. These social influence
% models make various empirically supported assumptions about 
% individual- and dyad-level social influence capacities and
% behavior~\cite{Baldassarri2007,Mueller2018,Banisch2019}, including the
% observation that group polarization effect size increases with initial
% group extremism~\cite{Myers1982,Turner2020}. 
% The common factor that leads to simulated group polarization in these models
% is the assumption that more extreme individuals are also more stubborn.
% In other words, these modeling results have showed that if extremists
% are more stubborn than centrists, then group polarization will emerge.
% Empirical evidence suggests extremists may 
% indeed be more stubborn than centrists~\cite{Kinder2017,Reiss2019}, 
% so by \emph{modus ponens} 
% group polarization should tend to emerge, as long as auxiliary
% theoretical criteria are met~\cite{Meehl1990}.
% Note the similarity to social decision schemes theory that predicts
% group polarization will occur when extremists are more powerful, i.e., exert
% a greater influence. This seems homologous to the case where extremists are
% less influenced than centrists due to greater stubbornness.

Even if we expect to observe group polarization in some
contexts with more rigorous methods, it is not clear which contexts. 
As Brown (1986) observed (quoted in the epigraph to this paper), 
group polarization often occurs, 
but inconsistently, and the effect is not 
always large. Explaining this context-dependence of group polarization 
is, in our opinion, the next step for group polarization research. Valid 
statistical models are necessary to reliably move forward.


\subsection{Statistical model features and implementation for valid group polarization measurement}

Future research on group polarization needs a valid statistical measurement
procedure for quantifying group polarization. As we have demonstrated, one requirement
for a valid statistical procedure is that the data must be represented as
ordinal measurements, not continuous and normally distributed.  
In the course of our work, we also observed that each group should have its own
mean and variance in pre- and post-deliberation opinions, and similarly
different items have been observed to vary in their response distributions.
Failing to account for this multilevel structure is known to lead to overconfident
overestimates of effect sizes. Therefore
group polarization statistical models must include this multilevel structure
to be valid. One model that can meet these needs is the ordered probit model
that combines a normal model of latent opinions with an ordinal model of
ordinal measurement data. 

% One must use Bayesian statistical methods to 
% calculate parameters in a multilevel model. In the following paragraphs we
% explain the requirements for statistical models group polarization and 
% their implementation in more detail.

% A statistical model for ordinal observations needs to encode the ordinality
% and non-normality of categorical survey responses, e.g., opinions given on a Likert
% scale. In our Analysis above, we showed the dire consequences of failing to
% account for the non-normality of ordinal measurement data. We showed that
% a decrease in pre- to post-deliberation variance can produce two datasets 
% that appear to have different means when they were in fact generated from
% latent distributions with the same mean. This exact phenomenon of
% consensus formation, where opinion variance decreases as consensus is formed,
% has been observed in group polarization studies, as plotted in Figure 2 by
% \citeA{Schkade2010}. 

A statistical model must also account for the multi-level structure of
group polarization data. Literally all published articles we reviewed failed
to account for this structure. This probablem is totally perpendicular to the
one analyzed in this current paper, but it is equally critical for ensuring
valid inferences. Failure to account for multilevel structure can inflate effect sizes and
shrink confidence intervals if groups vary in their standard deviation of
opinions, which could falsely indicate significant
pre- and post-deliberation differences~\cite{GelmanHillRegression,Yarkoni2021}. 
We would expect systematic inter-group and inter-item variation in group polarization data, 
as well as other factors depending on the exact experimental design.

One statistical model that represents ordinal measurements of metric
data is the ordered probit model~\cite[Ch. 23]{Kruschke2015}. 
The ordered probit model combines a normal
model of latent psychological opinions (equivalently beliefs, attitudes, etc.)
with an ordinal model of observed data.  In addition to latent normal opinion 
distribution parameters mean and variance, there are additional parameters that represent
the binning of opinions into ordinal survey responses. 
These are the thresholds, $\theta_k$, which we were free 
to set constant in our generative model.
To fit a multilevel ordered probit model, one must fit the model
using Bayesian methods, which, unlike frequentist methods, can account for several,
even hundreds, of groups across multiple levels~\cite{Liddell2018}.


\subsection{Open science to improve group polarization research}

This current paper and project could have provided much stronger conclusions
about the validity of published results if group polarization researchers had 
followed current open science best-practices.
Open science practices, including open data sharing, data and metadata standards,
and publishing analysis code, can improve scientific outcomes 
generally~\cite{Hart2016,Smaldino2019,Samuel2021}.
Our study would have been further streamlined if group polarization research data was
stored in a central database, accessible through an API for automated gathering
and analysis. If we had access to the original data formats, our paper would not have 
simply shown existing findings are plausibly false or not. Instead we could have
re-analyzed the existing data with more appropriate ordinal statistical 
models~\cite{Liddell2018}.  However, if that data was haphazardly stored in
disparate personal websites, or even just in separate Open Science Foundation 
data repositories, then the process would be extremely tedious, and 
analyses of additional datasets would be needlessly time consuming. 



% Current theoretical explanations of
% group polarization rely on auxiliary assumptions that are not empirically
% supported, e.g., the existence of norms and prototypes regarding extremism
% in the social comparisons and the implicit assumption that group membership
% itself has no effect on informational influence processes. While the data
% might appear to support a particular theoretical explanation, it may be the 
% case that auxiliary assumptions are not supported, and thus the theory is 
% \emph{a priori} invalid, and consequentially so are any explanations deriving
% from that theory~\cite{Meehl1990}. 
% Going forward, problematic auxiliary assumptions can be avoided by 
% developing mechanistic models of 
% theoretical entities such as opinions and norms, and processes such as 
% calculating group prototypes, been highlighted as placeholders which
% we do not yet understand well~\cite{Machamer2000,Craver2006}. 




% In our analysis of whether or not group
% polarization exists, we identified some simple core assumptions made by
% existing theories of group polarization, stripped of many of the potentially
% unnecessary and problematic auxilliary assumptions. Identifying
% these core assumptions is just one step in a larger project of 
% advancing group polarization theory. This work should be
% continued to try to integrate existing group polarization theories into a
% more parsimonious form that makes as few auxiliary assumptions as possible.

% Then perhaps the lack of
% empirical support for these assumptions could be noted, with plans or calls
% to add empirical support. Better yet, those auxiliary assumptions should be
% supported separately.  At the least it is necessary to identify and explain
% these placeholders in order to make honest scientific progress on
% explaining group polarization. Methodological problems with statistical
% procedures identified here may well be remedied with improved, mechanistic
% approaches to modeling group-level behavior~\cite{Yarkoni2021,Turner2021}.

% \subsubsection{Representation and measurement of opinions}

% Another important theoretical and modeling point is that it is perhaps time
% to more seriously critique the use of numbers to represent opinions, and
% explore changes in opinions using alternatives to survey methods. For instance,
% one might measure opinions implicitly, as in the implicit attitude test (IAT)
% that measures participant reaction time in a task requiring the identification
% of revealing items. Does IAT reaction time change following group discussion? 

% Or, in an alternative approach to ordinal data, why not just use continuous 
% opinion measures? This may be worth exploring in more detail after some of
% the other statistical issues are addressed, such as implementing multi-level
% models that account for group-level variation in group polarization 
% magnitude.  One reason to not use continuous opinion measures is that there
% are more pressing theoretical issues with group polarization experiments that
% should be addressed first.  It seems just as easy to transition studies of 
% group polarization to use the appropriate statistical model for detecting group 
% polarization~\cite{Liddell2018} as it would be to transition to adopt
% continuous measures of group polarization.
% Finally, Likert measurements may be more reliable than continuous measurements,
% especially in the context of online experiments~\cite{Toepoel2018}.



\subsection{Conclusion}

We developed a measurement and statistical model of group polarization that
invalidated the results of several published studies when we analyzed those
studies' supporting data.
While not all observations of the group polarization effect
are invalidated by our model, many of the ones we studied are
widely referenced and high profile---even though some are decades old, they
continue to motivate new work~\cite{Mas2013,Keating2016,Sieber2019,Pallavicini2021}.
Even the literature that does not apply continuous statistical models to
ordinal data has separate problems, including possible theoretical 
inconsistencies, overgeneralizations from underrepresentative sampling and
failing to account for important sources of variance, and a lack of publicly
available data.

By examining the effects of measurement and statistics in detail, we 
demonstrate that future work on group polarization must use ordinal 
statistical models to analyze ordinal data. This effort will be further supported 
by the adoption of open science practices for the further refinement of
research methods and new analyses and theorizing.





\chapter{Paths to polarization: extreme views, miscommunication, and random chance}


\noindent Understanding the social conditions that tend to increase or decrease polarization is
important for many reasons. 
%Polarization is often identified as a cultural problem that co-occurs with failure of governments and fracturing of societies. But how and when does polarization emerge? How do different levels of polarization emerge? To address these questions, 
We study a network-structured agent-based model of opinion dynamics, extending a model previously introduced by Flache and Macy (2011), who found that polarization appeared to increased with the introduction of long-range ties but decrease with the number of salient opinions, which they called the population's ``cultural complexity.'' 
We find the following. 
First, polarization is strongly path dependent and sensitive to stochastic variation. 
%almost always a probabilistic occurrence.  
Second, polarization depends strongly on the initial
distribution of opinions in the population. In the absence of extremists,
polarization may be mitigated. 
Third, noisy communication can drive a population toward more extreme opinions and even cause acute polarization. 
Finally, the apparent reduction in polarization under increased ``cultural complexity'' arises via a particular property of the polarization measurement, under which a population containing a wider diversity of extreme views is deemed less polarized. 
This work has implications for understanding the population dynamics of beliefs, opinions, and polarization, as well as broader implications for the analysis of agent-based models of social phenomena. 
%o performed a number of computational experiments using an extension of an existing agent-based model of cultural polarization. 
% We identified three confounding factors affecting opinion polarization: the order in which agents update their opinions, the magnitude of the initial extremism in a population, and communication noise. These are in addition to previous findings that network structure and cultural complexity are also confounding factors, which we reproduce in this paper. We found there are critical levels of initial extremism and communication noise that lead to either inevitable consensus or polarization. In many cases, however, limiting initial extremism or communication noise, selecting a particular network structure, or achieving a certain cultural complexity does not lead to deterministic results, but stochastic results that depend on the order of agent opinion updating. We found that, in general, these confounding factors bias outcomes towards greater or lesser polarization levels. We quantify the effect of each of these factors. 

% \small{\textbf{Keywords:} polarization; opinion dynamics; small-world networks; cultural complexity; agent-based models}
% INTRODUCTION
% %%%%%%%%%%%%%%%%%%%%%%%%%%%%%%%%%%%%%%%%%%%%%%%%%%%%%%%%%%%%%%%%%
\section{Introduction}

Diversity of opinions in a community is often difficult to maintain. Iterative exposure, norm
enforcement, and psychological biases for conformity can drive consensus within
a group \cite{DeGroot1974,deffuant2000mixing,henrich1998evolution,smaldino2015social,efferson2008conformists,muthukrishna2016and}.
%(DeGroot 1974; Deffuant et al. 2000; Henrich and Boyd 1998; Smaldino and Epstein 2015; Efferson et al. 2008; Muthukrishna et al. 2016). 
On the other hand, in-group bias, outgroup aversion, and the
tendency to further differentiate ourselves from those deemed different may lead to the
emergence of strong inter-group differences \cite{tajfel1971social,Lord1979,carley1990group,Axelrod1997,mark1998beyond,mcelreath2003shared,Dandekar2013,gray2014emergence,smaldino2017adoption}.
%(Tajfel et al. 1971; Lord et al. 1979; Dandekar et al. 2013; McElreath et al. 2003; Mark 1998; Carley 1990; Axelrod 1997; Gray et al. 2014; Smaldino et al. 2017). 
Such differences
can lead to polarization in opinions under certain conditions.  Understanding
the social conditions that tend to increase or decrease polarization is
important for many reasons. Primary among these is that a functioning 
democratic society
depends on clear communication among the citizenry, which is impeded by the
mismatch in norms, the differential interpretation of facts, and the
dehumanization that polarization can engender (see \citeA{PewResearchCenter2017}
for a current analysis of these dynamics in the United States). 
The maintenance of social differences in the form of cliques and clubs 
may be inevitable, but cooperation depends on transcending differences. 

We take a network theoretic approach to studying the conditions for polarization in an agent-based model of opinion dynamics. Empirical research on the population dynamics of opinions is challenging and must be supplemented by formal modeling 
\cite{flache2017models}. 
Models reduce complex systems to ones that are tractable using mathematical or computational analysis, and allow for the exploration of replicate and counterfactual scenarios. Of course, the conclusions we draw from our models depend essentially on the assumptions of those models, and so caution must be taken when using model results to make inferences about empirical phenomenon. For example, \citeA{smaldino2012human} analyzed models of human mate choice and showed that very different individual decision rules could be fit to almost any empirical outcome by modulating assumptions about the population structure that had been ignored in prior analyses. When considering an important phenomena such as polarization, similar caution must be exercised, as we will demonstrate. 

Our analysis extends the work of \citeA{Flache2011}, who used a
network-structured model of opinions and biased influence (hereafter
the FM model) to study polarization. Network ties in this model exist between individuals as an indicator of social influence. Like several other models of opinions and beliefs, they
operationalized the well-known phenomena of {\em biased assimilation} \cite{Lord1979,Dandekar2013}, the
tendency for an individual to become more similar to those to whom they are
similar, and to become more distinct from those with whom they already differ.
Some empirical studies support the assumption of both positive and negative biased assimilation
\cite[e.g.]{Adams2005, Hart2012}. Other empirical studies failed to find evidence of negative
biased assimilation at work where computational studies suggested it would be 
\cite[e.g.]{Takacs2016, Boxell2017a}. 
Of course, if further empirical research turns out to invalidate that assumption, then our model conclusions must also be re-examined, as with any theoretical model \cite{Smaldino2017}.
Flache and Macy found that, when compared with a highly clustered population structure, the addition of
long-range ties could dramatically increase polarization. 
When individuals were clustered into relatively isolated groups, they tended to converge to local consensus while maintaining diversity in the population at large. However, the addition of long-range ties increased exposure to substantially different opinions. 
Whether by attractive or repulsive forces, these long-range ties tended to
drive opinions more toward their extreme values, resulting in increased
polarization.  Another important result was that the extent of ``cultural
complexity''---the number of orthogonal traits that are important to individuals
in assessing their similarities and differences with others---mitigated
polarization. When the number of traits was large, polarization was
reduced. \citeA{DellaPosta2015} used a variant of the FM model to explain data from the General Social Survey indicating that arbitrary traits tend to become associated with polarized identity groups, leading to
often-puzzling stereotypes such as ``latte-drinking liberals" and ``bird-hunting conservatives."  

If we take the results of \citeA{Flache2011} at face value, two possible recommendations for the reduction of polarization readily emerge.  First, we might try to reduce the
number of long-range ties in our social network. This is made difficult 
due to the pervasive
influence of internet social media~\cite{PewResearchCenter2016,PewResearchCenter2018}. 
Second, we might attempt to broaden the
number of domains in the public discussion, so that points of agreement are
easier to discover. This is also challenging, due to the increasingly fractured
media landscape in which niche interests are increasing and common knowledge
diminishing \cite{Pew2014}. 
However, challenging is not the same thing as impossible. We must ask, then: 
How seriously should we take these recommendations? Might there be
other solutions available?

To address these questions we perform new analyses of the FM model and reveal several additional factors
influencing polarization. First, polarization is almost always a probabilistic
occurrence. Even when parameter exploration appears to reveal regularities in
polarization, specific outcomes are strongly path dependent. Indeed, there is
often a wide range of possible outcomes even given identically repeatable
starting conditions, due to stochasticity in the dynamics of interactions. This
result highlights potential limits of our ability to make reliable
predictions about polarization in any particular social system. Complex systems are
often stochastic, and something that increases or decreases average polarization
in a simulation is not guaranteed to do so in reality.
Second, resultant polarization depends strongly on the initial
distribution of opinions in the population. In the absence of extremists,
polarization may be mitigated. This highlights the well-known danger of
extremists and suggests new routes to avoiding polarization. More broadly, we
show that too much diversity
of extreme opinions makes polarization more likely.
Third, noisy communication can drive a population toward more extreme opinions and even cause acute polarization. Cooperation and consensus-building depend on individuals finding common
ground, which can be jeopardized even in the presence of unbiased error \cite{Clark1996}. 
Finally, we show that the apparent reduction in polarization under increased ``cultural complexity'' arises via a particular property of the polarization measurement, 
under which a population containing a wider diversity of extreme views is 
deemed less polarized. Although this may often be a reasonable assumption, 
it highlights the need for caution in our measurement of complex social phenomena. 



% MODEL %%%%%%%%%%%%%%%%%%%%%%%%%%%%%%%%%%%%%%%%%%%%%%%%%%%%%%%%%%%%%%%%%
\section{Model}

\subsection{Modeling individuals and their opinions} 

Our model is an extension
of one presented by \citeA{Flache2011}, and shares many general features with
other models of opinion dynamics in structured populations \cite{Nowak1990,carley1990group,Axelrod1997,mark1998beyond,mark2003culture,Dandekar2013,DellaPosta2015,battiston2017layered}.  The population is modeled as a network of individuals (or agents), each of whom is defined by a vector of opinions. The
size of this vector, $K$, is called the ``cultural complexity,'' 
and may be more descriptively explained as the number of  opinions that are
important to individuals in assessing their similarities and differences with
others. Opinions can present political views, religious or moral values,
artistic tastes, or myriad other beliefs. The opinion of agent $i$ on issue $k$
$(1 \leq k \leq K)$, $s_{ik}$, is operationalized as a real number implicitly bounded in
$[-1, 1]$ by smoothing (Equation \ref{eq:smoothed-update}). 
%
In Flache and Macy's original analysis, all opinions were initialized as random
draws from the uniform distribution $U(-1,1)$. In order to study the importance
of initially extreme opinions, each initial opinion is here drawn instead from
$U(-S, S)$, where $0 < S \leq 1$.   

\subsection{Modeling social influence} 

The aggregation of the $K$ opinions held by an agent 
%has on each cultural feature
determines its coordinates in opinion space.
%, sometimes called Blau space \cite{McPherson2004}. 
%PS: My understanding of Blau space is that it includes all sociodemographic dimensions, including where people live, their income, etc. This means that many positions in Blau space are not malleable as the opinions in the model. 
We adopt the FM model's measure of distance between agents $i$ and $j$,

\begin{equation}
  d_{ij} = \frac{1}{K}\sum_{k=1}^{K} |s_{jk,t} - s_{ik,t}|.
\end{equation}
\noindent
Distance thus defined measures the average absolute difference across 
opinion coordinates.  Agents are nodes in a network, 
with an edge between agents reflecting a
relationship and an opportunity for the agents to influence one another. The
magnitude and direction of that influence is characterized by the {\em weight}
of each edge. Weights are determined by the relative opinions of the
two agents, as measured by their distance, and so can change dynamically. Positive weights represent positive
influence, in which agents become closer in their opinions, while negative
weights represent the tendency toward differentiation. For descriptive convenience, if two agents are
connected with a positive weight, they could be considered ``friends'' and if
the weight is negative they could be considered ``enemies.'' In reality, no assumptions about such clear social roles are necessary. The weight of an edge
between agents $i$ and $j$ is given by 

\begin{equation} 
  w_{ij,t+1} = 1 - d_{ij,t}.  
\end{equation} 
\noindent
So, if the opinions of agents $i$ and $j$ are separated by $d_{ij} < 1$, 
the agents are friends and will harmonize their opinions. If $d_{ij} > 1$, 
the agents are enemies, and will drive each other's opinions to more extreme levels.
This weighting rule embodies the psychological phenomena of {\em biased assimilation},
in which similar individuals grow more similar and dissimilar individuals grow
further apart after interacting \cite{Lord1979}. This is a common assumption in models
of social influence \cite{Hegselmann2002,Flache2011,Dandekar2013}). 
It should be noted that while the
empirical evidence for biased assimilation is quite strong, and spans almost four decades, it is less clear how coherence on various opinions or beliefs affects influence on orthogonal opinions or beliefs. The assumption in this model is that it is only average distance in opinions that matters.  

At time $t+1$, agents update their opinions by adding the average 
influence from all neighbor agents.  For each opinion $k$, 
agent $i$ uses the following update rule: 

\begin{equation} 
  s_{ik,t+1} = s_{ik,t} + \Delta s_{ik,t} \left(1 - \text{sgn}(s_{ik,t}) s_{ik,t} \right), 
  \label{eq:smoothed-update}
\end{equation} 
\noindent
where

\begin{equation}
  \Delta s_{ik,t} = \frac{1}{2N_i} \sum_{j \neq i} w_{ij,t} (s_{jk,t} - s_{ik,t}) + \epsilon.  
\end{equation} 
\noindent
Here, $N_i$ is the number of agents
with which agent $i$ shares an edge, and $\epsilon$ is a noise term that
reflects errors in the communication of opinions. This term is in each
instance drawn at random from a normal distribution with a mean of zero and a
standard deviation of $\sigma$. We conceptualize updating to be the result of
agents sensing the communicated opinions of neighbors. Furthermore, we 
conceptualize this $\sigma$ as representing noise either in an agent 
sensing the opinions of other agents, noise in agents communicating their
opinions, or both. In their original study \citeA{Flache2011} considered only scenarios without noise ($\sigma = 0$).  
%Flache and Macy left out this noise term, and so only considered the case of $\sigma = 0$. 
Time in the model progressed in discrete time steps. At each time step, each agent's opinions were updated asynchronously in random order to avoid well-known artefacts that often accompany simultaneous agent updating. 
%PS: Matt, is this right? 

It is worth noting a few immediate consequences of these update equations.  
%There is much to unpack in these update equations. 
First, agents with extreme opinions in %cultural 
dimension $k$ will 
%require larger $\Delta s_{ik,t}$ to move their opinion on that cultural feature 
tend to make smaller changes to those opinions
because of the smoothing factor $(1 - \text{sgn}(s_{ik,t})s_{ik,t})$. In other words, extreme opinions will be harder to change. 
Second, there are two opposing factors that modulate the magnitude of 
%the ``force'' 
influence between two agents. On the one hand, edge weight is maximal when agents' opinions are very similar. 
%If the agents are closer together with $d_{ij,t} < 1$, the edge weight increases, and so influence each other more strongly. 
On the other hand, $\Delta s_{ik,t}$ (which Flache and Macy refer to as the ``raw" state change) increases the more agents' opinions differ, presumably because larger distances provide larger room for change, with a mathematical form drawn from psychological models of reinforcement learning \cite{rescorlaw72,sutton1998reinforcement}. Influence will therefore be maximal for agents who are an intermediate distance apart in opinion space. 
%in magnitude with the difference between opinions. However, $\Delta s_{ik,t}$ decreases in magnitude with the difference between opinions because of the $(s_{jk,t} - s_{ik,t})$ term. agents will be most strong influenced by those who are close in opinion space, but 
To facilitate an intuitive understanding of dyadic interactions, we illustrate the strength of influence on
agent opinions in $K=2$ opinion space 
%are illustrated 
\noautomath
in Figure~\ref{fig:dyadInfluence}. 
We see that an agent with opinions at the origin of opinion space has only 
a moderate, attractive influence on other agent opinions in the opinion space.
Agents at the corners of opinion space are barely influenced by a central 
opinion vector. When we consider the influence of an agent opinion nearer
to the corner, at $\vec s_i = (0.9, 0.9)$, we see that there is a clear
line where relationships switch from friend to enemy ($s_{j2} = s_{j1} - 0.2$). 
Due to the co-mingling of effects described above, 
there is a varied and non-monotonic landscape of influence.

\begin{figure}[H]
  \centering
  \begin{subfigure}[t]{\textwidth}
    \centering
    \includegraphics[width=.5\textwidth]{/Users/mt/workspace/Papers/Archive/complexity-overleaf-git/Figures/quiver_0_0.pdf}
    \caption{Influence of agent at origin.}
  \end{subfigure}\\[2em]
  \begin{subfigure}[t]{\textwidth}
    \centering
    \includegraphics[width=.5\textwidth]{/Users/mt/workspace/Papers/Archive/complexity-overleaf-git/Figures/quiver_0p9_0p9.pdf}
    \caption{Influence of agent at (0.9, 0.9).}
  \end{subfigure}
  \caption{Influence by one agent on another changes 
    depending on the location of each agent. This illustrates the influence
    exerted by a central agent (white circle) on another agent at different
    locations in opinion space. 
  }
  \noautomath
  \label{fig:dyadInfluence}
\end{figure}

\subsection{Measuring Polarization}

There are a multitude of measures for polarization \cite{Bramson2016} and no single measure is widely agreed upon. We follow \citeA{Flache2011} and define polarization at 
time $t$ to be the variance of all distances between agents,

\begin{equation}
  P_t = \text{var}(d_{ij,t})
\end{equation}
\noindent
This metric has the advantage of simple interpretation. If half of all agents are
in one corner of opinion space and the other half of agents are in the
opposite corner, then the population is maximally polarized. As agent opinions
spread to other corners and to other regions of opinion space, polarization
will decrease. One disadvantage is that more general patterns of clustering,
as would be detected using various machine learning clustering algorithms, will go undetected. In the final subsection of our Results, we illustrate another limitation of this metric. Nonetheless, we generally find that it is a useful and suitable operationalization for the concept of polarization. 


\subsection{Network structure}

Our network structures are taken from Flache and
Macy's (2011) Experiment 2. We begin with the connected caveman network
structure introduced by \citeA{Watts1999}. Specifically, we consider a network of
$N = 100$ agents, grouped into 20 fully connected clusters (caves) of five
agents each. These caves are arranged on a circle, and for each cave one edge
is selected at random and rewired to connect to a random agent in the cave
immediately to the right of the focal cave. This network has the appearance of tight-knit communities with weak ties to neighboring communities. The connected caveman network is highly clustered, meaning that if two agents are both neighbors of another single agent, there is a high probability that those two agents are also neighbors. However, relative path length is considerably greater in a connected caveman graph
than for a totally random graph. 
%In a random graph, there is a low probability of having to traverse all connections between caves and within caves to get to the other ``side'' of the connected caveman graph. 
%Low average path length is known as the ``small-world'' condition \cite{Milgram1967} and is highly likely for random graphs \cite{Bollobas2001}.

To assess the influence of adding long-range ties, we then
consider a network for which 20 additional edges are added between randomly
selected pairs of agents from across the entire network (Figure~\ref{fig:network}). 
Long-range ties are added at $t = 2000$ to give the local communities (caves) 
time to yield enclaves of conformity that differ slightly from their neighboring enclaves,
following \citeA{Flache2011}. The long range ties reduce the average path length of the network while retaining high clustering, yielding networks with ``small-world'' properties \cite{Watts1999}.

\begin{figure}[H]
  \centering
  \begin{subfigure}[t]{0.45\textwidth}
    \centering
    \includegraphics[width=\textwidth]{/Users/mt/workspace/Papers/Archive/complexity-overleaf-git/Figures/connected_caveman_vis.pdf}
    \caption{Connected caveman graph before long-range ties added.}
  \end{subfigure}
  \begin{subfigure}[t]{0.45\textwidth}
    \centering
    \includegraphics[width=\textwidth]{/Users/mt/workspace/Papers/Archive/complexity-overleaf-git/Figures/randomized_caveman.pdf}
    \caption{After long-range ties added.}
  \end{subfigure}
  \caption{Connected caveman network with and without twenty long-range ties.
    Colors represent cave membership.
  }
  \label{fig:network}
\end{figure}

Finally, as a way to control for the effect of simply adding additional ties, we also consider the connected caveman network with {\em short-range} ties. In this case a randomly selected agent from each cave (who is not already connected to another cave) is
connected to a random agent in the cave immediately to the right of the focal cave. Unless stated otherwise, all of our analyses were restricted to the connected caveman network with long-range ties, as this was the network structure found by \citeA{Flache2011} to maximize polarization. 
%PS: It might be a good idea to provide an illustration of the network structure to help readers unfamiliar with this work understand (also good to have for presentations). Might do one with simple connected caveman and then with added ties. 

\subsection{Computational experiments} 
Below we present the results of our computational experiments. For all parameter combinations we ran 100 simulations of the model, with data collected after $10^4$ time steps. This was always sufficient time for the system to settle down into a relatively stable pattern (true equilibria were not always reached due to the stochasticity inherent in the model). By calculating the difference in polarization on the final timestep for all simulations and finding all to be sufficiently small, we confirmed that 10$^4$ timesteps was sufficient to achieve stable behavior across all simulations. 
We first replicate the major result of \citeA{Flache2011} that polarization increases with the addition of long-range ties but decreases with increasing cultural complexity, $K$. We then perform three sets of experiments: 
\begin{enumerate}
\item {\em Quantifying variation.} We take a closer look at the variation among simulation runs, and explore path dependence on the road to polarization. 
\item {\em Reducing extremism.} We investigate values of $S < 1$, in which the initial distribution of opinions is less extreme.  
\item {\em Adding noise.} We investigate values of $\sigma > 0$, in which communication about opinions is noisy and influence is therefore more stochastic.  
\end{enumerate}
Unless stated otherwise, all simulations used a connected caveman network with random long-range ties, $S = 1$, and $\sigma = 0$. 
Model and analysis code is available on GitHub at \url{https://github.com/mt-digital/polarization}.


% RESULTS %%%%%%%%%%%%%%%%%%%%%%%%%%%%%%%%%%%%%%%%%%%%%%%%%%%%%%%%%%%%%%%%%
\section{Results}

%Through a series of computational experiments, we explore the details of the FM model dynamics. 
In their original analysis of the FM model, \citeA{Flache2011} found two main causes of polarization. First, random long-range ties decreased the average path length of the network and increased the average polarization of the system across trials. Second, 
average polarization across trials decreased with increasing cultural complexity, $K$.
We replicated these results, as illustrated in Figure~\ref{fig:pVsKFM}. 
The remainder of this section is dedicated to novel results. The first three subsections show results of new analyses of the original FM model. The final subsection shows our analysis of the FM model modified to include communication noise.

\begin{figure}[H]
  \centering
    \includegraphics[width=0.6\textwidth]{/Users/mt/workspace/Papers/Archive/complexity-overleaf-git/Figures/p_vs_K_fm.pdf}
  \caption{Reproduction of Figure 12b of Flache and Macy (2011). Average
    polarization decreases with $K$. However, as shown in subsequent figures,
    this does not mean trials with high polarization never obtain for large $K$. Average
    taken over 100 trials.
  }
  \label{fig:pVsKFM}
\end{figure}

%The  connected caveman (CC) network, without any added ties, was {\em on average} associated with states of lower polarization. How common were high polarization states on the CC network? If there existed a wide range of outcomes, which factors caused differences in final polarization? Initial conditions? Update path dependence? Both? Do large polarizations obtain for larger $K$, even though the average appears to goes to 0? The answers to these questions set the stage for two new experiments we ran. We present the results of those below. First, we ask what are the effects of limiting the maximum initial opinion magnitude through the parameter, $S$, which sets the extent of the  uniform distribution of initial opinions. We also consider different levels of communication noise, which in general cause an agent not to move in the direction of maximum opinion gradient with respect to its neighbors.


\subsection{Polarization is probabilistic and path-dependent}

Averages do not carry information about variation between trials. Here we explore that variation. Figure~\ref{fig:single-experiments-over-k} shows the polarization for each of the individual trials averaged in Figure~\ref{fig:pVsKFM}. We see a lot of variation around those averages, and that although polarization was low in all cases for large $K$, there are still individual trials for which polarization was high across all three network structures. 

\begin{figure}[H]
  \centering
    \begin{subfigure}[t]{\textwidth}
      \centering
      \includegraphics[width=.7\textwidth]{/Users/mt/workspace/Papers/Archive/complexity-overleaf-git/Figures/connected-caveman-over-K.pdf}
      \caption{Non-random connected caveman network.}
      \label{fig:connected-caveman-trials}
    \end{subfigure}
    \begin{subfigure}[t]{\textwidth}
      \centering
      \includegraphics[width=.7\textwidth]{/Users/mt/workspace/Papers/Archive/complexity-overleaf-git/Figures/random-long-range-over-K.pdf}
      \caption{Randomized connected caveman network with long-range random ties added at iteration 2000.}
      \label{fig:random-anyrange-trials}
    \end{subfigure}
  \caption{Results of individual model runs under different network conditions. 
    The averages of these were shown in Figure~\ref{fig:pVsKFM}.
    Even in the non-random connected caveman structure, there is 
    variation in the final polarization for different values of $K$. Highly
    polarized final states may obtain even for large $K$. 100 trials are shown for each 
    network condition. Solid lines indicate the average across all trials.}
  \label{fig:single-experiments-over-k}
\end{figure}
% Our results agree with \citeA{Flache2011}, showing that average final polarization is greater when there are random long-range ties added to a connected caveman network, but decreases with larger $K$.  
% %for when there are less than 9 cultural features (i.e. $K<9$). 
% For $K\geq9$ average final polarization was near zero for all three network structures. %in %all three cases vanishes. 
% While the average case is zero polarization, non-zero final polarization 
% can and does occur. Even in the non-small world populations on a CC network 
% and on a CC network with short-range random ties added, we observed simulations
% that ended with polarizations of about 0.2 for $K=10$ 
% (Figures~\ref{fig:connected-caveman-trials} and~\ref{fig:random-shortrange-trials}). 
% In the small world case of a CC network with long-range random ties added, 
% one simulation had a final polarization of nearly 1.0 
% (Figure~\ref{fig:random-anyrange-trials}).

%Since we see large non-zero polarizations across $K$, even in the CC configuration,  the variation in final polarization between trials must be not only due to network structure. 
In addition to the demonstrated influence of the overall network structure, three possible sources of  variation in system polarization are (1) the initial distribution of agent opinions, (2) the initial distribution of how agent opinions are clustered on the network, and (3) the update path---the order in which weights or agent opinions are updated. We performed additional analyses to investigate the contributions from each of these three factors, focusing on the initial distribution of agent opinions. We studied the non-random connected caveman network so as to keep network structure constant across trials, and for simplicity we restricted this analysis to $K=2$. 
Due to the nature of our polarization measure, at initialization the system will have some non-zero degree of polarization, which will vary depending on the random draws of agents' initial opinions. Over 100 trials, we compare the initial polarization of the system to the final polarization.    
% We held the network structure 
% constant in the connected caveman condition, analyzed the variability 
% due to initial conditions for $K=2$, and performed an experiment in which we
% used identical initial conditions to run more trials and observed the 
% resulting distribution of simulation outcomes.
% Consider the $K=2$ column of outcomes for the connected caveman condition
% presented in Figure~\ref{fig:connected-caveman-trials}. We will focus on 
% $K=2$ for the following two analyses in this subsection. 
% The final polarization
% of this condition ranges from just under 0.2 to just over 0.8. 
We found a significant, if relatively small, correlation between the initial and final polarization of agent opinions, $r^2=.137$ (Figure~\ref{fig:final-initial-pol-regplot}). 
This means that the level of initial polarization accounts for only about 14\% of the 
variation in final polarizations.
It seems, then, that initial clustering of agent opinions and the stochasticity of the update path account for a large portion of the variability. In order to delineate the contributions of these two remaining factors to the overall variability in polarization, we considered the previously discussed simulations and ran 100 replicate trials with the initial conditions taken from the trials with the lowest and highest initial polarization. In other words, for each of two conditions, we ran replicate simulations with the exact same starting conditions between trials. Any variation in outcomes must therefore be due to stochasticity in the update paths. 
For example, if two opposing extremists influence a disjoint set of moderates disproportionately 
often, polarization will increase. The results are shown in Figure~\ref{fig:highpol-histogram}.  
% set of initial agent opinions, we picked two instances of randomly-generated
% initial opinions with different initial polarizations. We took one of 
% these initial conditions from the trial that resulted in maximum final
% polarization in the $K=2$ CC trials and the other from trial that ended with 
% the smallest final polarization. The maximum final polarization trial began 
% with an initial polarization of 0.120 and the minimum final polarization 
% trial began with an initial polarization of 0.106. We ran 100 trials starting 
% from each of these initial conditions. 
Final polarization was clearly biased by the initial polarization (average final polarization across trials  
was 0.66 for the larger initial polarization, and 0.290 for the smaller initial polarization), but showed considerable variability. 
%.Trials for both initial conditions ended in a wide range of final polarizations, but   
%biased towards greater or lesser final polarizations for the greater or lesser initial polarization conditions, respectively (Figure~\ref{fig:highpol-histogram}). 
In other words, a large proportion of the variation between trials was due to stochasticity not in the initial configuration of the population, but to stochasticity in the transient dynamics of agent interactions.


\begin{figure}[H]
  \centering
    \includegraphics[width=.5\textwidth]{/Users/mt/workspace/Papers/Archive/complexity-overleaf-git/Figures/final-initial-pol-regplot.pdf}
  \caption{Regression of final polarization against initial polarization 
    for $K=2$ in the non-random connected caveman network configuration.
    Final polarizations are same as in the $K=2$ column of 
    Figure~\ref{fig:connected-caveman-trials}. 100 trials are shown. The
    top histogram shows the distribution of initial polarization across
    trials. The right histogram shows the distribution of final polarization
    across trials.}
  \label{fig:final-initial-pol-regplot}
\end{figure}

\begin{figure}[H]
  \centering
    \includegraphics[width=.75\textwidth]{/Users/mt/workspace/Papers/Archive/complexity-overleaf-git/Figures/caveman_extremes_histograms.pdf}
  \caption{Distribution of final polarizations at $t = 10^4$
    starting from initial conditions of either maximum or minimum polarization taken from the 
    the connected caveman trials with $K = 2$.
%    the connected caveman trial with 
%    maximum and minimum final polarization for $K=2$ shown in 
%    Figure~\ref{fig:connected-caveman-trials}.
%     The distribution is skewed towards final polarizations considerably larger
%     or smaller
%     than the mean polarization of 0.41 for the connected caveman experiment
%     with $K=2$ shown in Figure~\ref{fig:pVsKFM}.
%     We used the initial conditions from the greatest and least
%     polarization for $K=2$ shown in Figure~\ref{fig:connected-caveman-trials} 
%     as initial conditions highlights the experimental configuration
%     reused to make Figure~\ref{fig:highpol-histogram}. The trial that ended
%     with the greatest final polarization had the initial polarization of 
%     0.120 and the trial with the least final polarization had the initial
%     polarization of 0.106.
  }
  \label{fig:highpol-histogram}
\end{figure}

\subsection{The absence of initially extreme opinions reduces polarization}

Next we extend our analysis of initial conditions further, by studying the breadth of opinions initially present in the population. 
%the effect on initial conditions on polarization.
%Above we showed that final polarization was correlated with initial polarization.
%In the following experiment 
Specifically, initial opinions were drawn from the uniform distribution $U(-S, S)$. 
%We do this via the \emph{maximum initial opinion magnitude} parameter, $S$. In the previous experiments, $S=1$, and
Figures~\ref{fig:SAverage} and \ref{fig:SMedian} show the mean and median polarization of the population as function of $S$, for $K=2,\ldots,6$. 
In general, the average final polarization decreased with smaller $S$ for all values of $K$. The lines are not perfectly smooth due to the large variation in outcomes described in the previous section (see Figure~\ref{fig:singleSK}). 


\begin{figure}[H]
  \centering
    \includegraphics[width=0.65\textwidth]{/Users/mt/workspace/Papers/Archive/complexity-overleaf-git/Figures/s_k_zoom_2-6_mean.pdf}
  \caption{Average final polarization for different cultural complexities over 
    maximum initial opinion magnitude, $S$. 
    Averages are roughly zero for $S<0.75$ for all cultural complexities.
%     Each data point is the average over 100 trials of the condition
%     where long-range ties were added at iteration 2000. Each trial was run to 
%     10k total iterations. 
  }
  \label{fig:SAverage}
\end{figure}

\begin{figure}[H]
  \centering
    \includegraphics[width=0.65\textwidth]{/Users/mt/workspace/Papers/Archive/complexity-overleaf-git/Figures/s_k_zoom_2-6_median.pdf}
  \caption{Median final polarization for different cultural complexities over
    maximum initial opinion magnitude, $S$.
    Median polarization for $K=5$ and $K=6$ are both flat at zero; $K=5$ 
    data is obscured by $K=6$.  
%     Median taken over 100 trials.
%     Same data as in Figure~\ref{fig:SAverage}, so the condition was
%     long-range ties added at iteration 2000, with each trial run to 10k
%     total iterations.
  }
  \label{fig:SMedian}
\end{figure}

% initial opinions were drawn from the uniform distribution $U(-1, 1)$. In the
% experiment we now describe, initial opinions were drawn from the distribution
% $U(-S, S)$, where $S$ was varied between 0.75 and 1.0, in steps of 0.01, and
% the final polarization at iteration 10000 was recorded for 100 trials of 
% every value of $S$. We perfomed these trials for $K=2,\ldots,6$.

%With a few exceptions, 
% for a given value of $S$
% was larger for smaller values of $K$. While average final polarization does 
% increase as $S$ increases, there is too much noise to determine if there is
% some underlying functional form that would be appropriate to fit to the 
% average polarization versus $S$ (Figure~\ref{fig:SAverage}). Although non-zero
% average polarizations do obtain for all $K$ other than $K=6$ by $S=0.87$,
% the median polarizations are zero for all $K$ but $K=2$ until $S=0.90$, when
% the median polarization for $K=3$ reaches a non-zero value. Median polarizations
% are zero for $K=5$ and $K=6$ for all values of $S$ (Figure~\ref{fig:SMedian}).

% The large discrepancy between average and median polarizations seen in Figures~\ref{fig:SAverage} and \ref{fig:SMedian}, and the within-$K$
% noisiness in the plots of average polarization versus $S$ led us to again 
% inspect the result of individual trials in conjunction with the average. 

We again examined the within-condition variation in final polarization (Figure~\ref{fig:singleSK}). Even when the average polarization was very small, we nevertheless saw instances of strongly polarized outcomes for $S < 1$ across all values of $K$. For small values of $S$, much more polarization occurred with small $K$. 
%The onset of a greater number of highly-polarized outcomes happens for smaller $S$ when $K$ is smaller (Figure~\ref{fig:singleSK}). 
This further highlights the fact that initial conditions, in conjunction with the cultural complexity, bias the system towards larger or smaller levels of polarization, but do not eliminate the possibility of either conformity or extreme polarization.


\begin{figure}[H]
  \centering
  \begin{subfigure}[t]{\textwidth}
    \centering
    \includegraphics[width=0.5\textwidth]{/Users/mt/workspace/Papers/Archive/complexity-overleaf-git/Figures/single_S_K_2.pdf}
  \end{subfigure} \\
  % \begin{subfigure}[t]{0.49\textwidth}
  %     \centering
  %     \includegraphics[width=\textwidth]{/Users/mt/workspace/Papers/Archive/complexity-overleaf-git/Figures/single_S_K_3.pdf}
  %     % \caption{}
  % \end{subfigure}
  ~
  \begin{subfigure}[t]{0.49\textwidth}
      \centering
      \includegraphics[width=\textwidth]{/Users/mt/workspace/Papers/Archive/complexity-overleaf-git/Figures/single_S_K_4.pdf}
      % \caption{}
  \end{subfigure} \\
  % \begin{subfigure}[t]{0.49\textwidth}
  %     \centering
  %     \includegraphics[width=\textwidth]{/Users/mt/workspace/Papers/Archive/complexity-overleaf-git/Figures/single_S_K_5.pdf}
  %     % \caption{}
  % \end{subfigure}
  ~
  \begin{subfigure}[t]{0.49\textwidth}
      \centering
      \includegraphics[width=\textwidth]{/Users/mt/workspace/Papers/Archive/complexity-overleaf-git/Figures/single_S_K_6.pdf}
      % \caption{}
  \end{subfigure}
  \caption{Final polarization of individual trial runs and averages from
    Figure~\ref{fig:SAverage} for a selection of $K$. 
    %Each trial was run in the condition where random long-range ties were added at iteration 2000 and each trial was run to 10k iterations.
  }
  \label{fig:singleSK}
\end{figure}



\subsection{The meaning of polarization in high-dimensional opinion space}
%PS: This seems like the wrong point to make here. 
Clearly extreme positions are important in the FM model. Extremists are
more stubborn (and therefore more influential) than centrists due to smoothing. Our analysis indicates that under a wide range of conditions, all opinions are likely to end up at extreme values. Indeed, the only stable states of the model are complete consensus, which can be at any point in opinion space in the absence of noise, or for all opinions to be at extreme values. This brings us back to a key result of the FM model, which is that increased cultural complexity, $K$, decreases polarization. Recall that polarization is measured as the variance among distances between agent opinions. To what extent is this decrease in polarization with increased cultural complexity driven by the fact that, for larger $K$, there are simply more ``corners'' (extreme opinion values) for agent opinions to settle on? 

We investigated this question 
%importance of extremism 
by comparing polarization emerging from the dynamics of the FM model with
polarization that occurs when agents are artificially placed on a random vertex of 
the $K$-dimensional opinion hypercube. We found the polarization for this
combinatorial condition is $P_{c} \approx 1/K$ via Monte Carlo 
sampling with 100 agents and 1000 trials for each $K \in \{1, \ldots, 12\}$.  
In the Appendix we derive a formal proof that $P_{c} = 1/K$ exactly in the limit as $N \rightarrow \infty$.

When we compare the combinatorial result to the FM model results, 
%connected caveman and random long-range tie conditions, 
we find that observed decrease in polarization with increased $K$ follows the combinatorial results very closely (Figure~\ref{fig:combinatorial-comparison}).
The connected caveman condition results in a lower polarization, on average,
than $P_{c}$ for all $K$ that we tested. The random long-range condition results in an 
average polarization roughly equal to $P_{c}$ for $K=1$, higher average polarization 
than $P_{c}$ from $K=2$ to $K=4$, and lower polarization for $K \geq 5$. The
source of this jump from above-combinatorial to below-combinatorial is not
clear, but is an interesting avenue for future work.


\begin{figure}[H]
  \centering
    \includegraphics[width=0.75\textwidth]{/Users/mt/workspace/Papers/Archive/complexity-overleaf-git/Figures/combinatorial-comparison.pdf}
  \caption{Polarization resulting from FM model simulations under connected caveman and random long-range tie conditions, compared with polarization resulting from agents arbitrarily choosing a corner of opinion space at random. Monte Carlo simulations revealed that polarization goes as $1/K$ if agents
    simply pick a corner at random. Random long-range and connected caveman
    data points are averaged from 100 trials with $10^4$ iterations. 
    Combinatorial condition data points are the average over 1000 trials 
    and $10^4$ iterations. Standard deviation around combinatorial trial averages was
    less than $10^{-2}$.
  }
  \label{fig:combinatorial-comparison}
\end{figure}

\subsection{Noisy communication increases polarization, particularly in the
absence of initially extreme opinions}

Up to this point, we have assumed that agents accurately express their own opinions and accurately receive information concerning the opinions of others. As this assumption is unlikely to fully hold in most cases of human interaction, it is important to assess the model's robustness to noisy communication. 
%Finally, we show the results of experiments designed to understand the effect of noisy communication on the effect of final polarization. 
To do this, we introduced random error into the opinion update equation, so that every cultural feature communication
channel, for every connected dyad, was modulated by a noise term, $\epsilon$, 
drawn from a normal 
distribution with mean 0 and standard deviation $\sigma$. 
%For our purposes, we call $\sigma$ the \emph{communication noise level}, or simply ``noise level.'' 
Let us call $\sigma$ the ``noise level.'' 
We varied the noise level from 0 to 0.2 in increments of 0.02. For each of these
noise levels, we also varied $S$ from 0.5 to 1.0 in steps of 0.05 for a total of
121 parameter pairs for each  $K \in \{2, 3, 4, 5\}$. %, $K=2,3,4,5$. 
Note that we did not explicitly bound opinion components in the presence of noise. This led to us discarding 19 of the 60500 runs due to runaway opinions that diverged to infinity, and this was only for the highest noise levels used. These  (discarded runs had noise levels of .18 or .2). Most parameter settings had only one discarded run if any, with one parameter setting having three discarded runs, lowering the number of samples to 97 from 100 for that parameter setting (K = 5, S = 0.95, and noise level= 0.2).  This lack of smoothing had no effect on non-divergent model runs polarization outcomes, as polarization was less than or equal to 1.0 for all.

These experiments reveal an interesting pattern of results.
% (Figures~\ref{fig:heatmaps} and~\ref{fig:avedist_heatmaps}). 
A sufficiently large amount of noise produced high levels of polarization for low values of $S$, which never produced polarization in the absence of noise. 
Indeed, there appears to be a phase transition point for $\sigma$ under low $S$, 
below which the system collapses to complete conformity and above which 
we see high levels of polarization (Figure~\ref{fig:heatmaps}). Across the values of $K$ we tested, this 
threshold appealed to be around $\sigma=0.8$, below which we never saw any 
polarization for low $S$ (Figure~\ref{fig:single-runs-commnoise}). 
%For example, holding $S=0.5$, high levels of polarization commonly emerge at about $\sigma=0.12$, and high final polarization is rare, but not impossible, for $\sigma=0.1$. 
As $S$ increases, however, the system behavior becomes less sensitive to 
noise, appearing to be completely insensitive to noise close to $S=1$. 




\begin{figure}[H]
  \centering
  \includegraphics[width=\textwidth]{/Users/mt/workspace/Papers/Archive/complexity-overleaf-git/Figures/noise_experiment_heatmaps.pdf}
  \caption{Final average polarization varies with both the width of the
    uniform distribution of initial opinion magnitudes and the noise level in
    the opinion updates. The value in each square of the heatmap is the average of
    100 trials. 
    %Each trial was run in the condition where random long-range
    %ties were added at iteration 2000. Each trial ran to 10k timesteps. 
  }
  \label{fig:heatmaps}
\end{figure}

\begin{figure}[H]
  \centering
      \begin{subfigure}[t]{0.49\textwidth}
          \centering
          \includegraphics[width=\textwidth]{/Users/mt/workspace/Papers/Archive/complexity-overleaf-git/Figures/noisecomm_S_0p5_K_2.pdf}
          % \caption{}
      \end{subfigure}
      ~
      \begin{subfigure}[t]{0.49\textwidth}
          \centering
          \includegraphics[width=\textwidth]{/Users/mt/workspace/Papers/Archive/complexity-overleaf-git/Figures/noisecomm_S_0p5_K_3.pdf}
          % \caption{}
      \end{subfigure} \\
      \begin{subfigure}[t]{0.49\textwidth}
          \centering
          \includegraphics[width=\textwidth]{/Users/mt/workspace/Papers/Archive/complexity-overleaf-git/Figures/noisecomm_S_0p5_K_4.pdf}
          % \caption{}
      \end{subfigure}
      ~
      \begin{subfigure}[t]{0.49\textwidth}
          \centering
          \includegraphics[width=\textwidth]{/Users/mt/workspace/Papers/Archive/complexity-overleaf-git/Figures/noisecomm_S_0p5_K_5.pdf}
          % \caption{}
      \end{subfigure}
      % ~
  \caption{Final polarization of individual trial runs and averages from
    Figure~\ref{fig:heatmaps} for $S=0.5$ as a function of noise level, $\sigma$. 
    %Each trial was run in the condition where
    %random long-range ties were added at iteration 2000 and each trial was
    %run to 10k iterations.  
    As the noise level is increased, the system is increasingly biased towards larger final polarization outcomes.
  }
  \label{fig:single-runs-commnoise}
\end{figure}


Even though polarization is rare at moderate noise levels, extremism is not.
A noise level of over 0.1 was required to reliably drive
the system to polarization in our simulations, but lower noise levels led to consensus around an
extreme location in opinion space rather than at a most centrist position. We infer this because the average agent distance from center increases to the maximum, 1.0, with noise levels of only 0.6
(Figure~\ref{fig:avedistHeatmaps}). Thus, we obtain the interesting result that
even small amounts of communication noise can move the population to extremist
positions. 


\begin{figure}[H]
  \centering
  \includegraphics[width=\textwidth]{/Users/mt/workspace/Papers/Archive/complexity-overleaf-git/Figures/ave_dist_heatmaps.pdf}
  \caption{Noisy communication causes extremism without polarization before
    it causes extremism with polarization.  For all $K$ pictured, the
    average distance from center increases with moderate levels of noise, even
    though polarization has not increased, as shown in Figure~\ref{fig:heatmaps}.
    The value in each square of the heatmap is the average of
    100 trials. 
    %Each trial was run in the condition where random long-range
    %ties were added at iteration 2000. Each trial ran to 10k timesteps. 
  }
  \label{fig:avedistHeatmaps}
\end{figure}

Figures \ref{fig:heatmaps} and \ref{fig:avedistHeatmaps} also illustrate a curious interaction between noise level, $\sigma$, and initial extremism, $S$. For smaller $S$, we observe clear phase transitions from centrist conformity to extremist conformity to polarization. For larger $S$, the populations responses are less clearly delineated.
%Why do we see such different patterns of polarization and response to noise for different initial distributions of opinions, $S$? 
To help explain, we present illustrations of the spatiotemporal dynamics of the model for exemplar trials. 
Consider first a case of very low initial extremism, $S=0.5$  (Figure~\ref{fig:noiseCoordsS0p5}). 
In the absence of noise, the system collapses around the center of opinion space at $t=200$, and by 
$t=3000$ has reached full consensus (Figure~\ref{fig:noiseCoordsS0p5}, top row). At the other extreme, under high levels of noise, $\sigma=0.2$,
agents reach a near-consensus by $t=1000$ and remain there until $t=2000$, when random long-range ties are added. 
At this point, agents are exposed to individuals with very slightly different sets of opinions, and those differences are amplified by the noise, leading to repulsion. This is sufficient to jolt the system away from conformity and into opposing camps moving towards opposing corners 
(Figure~\ref{fig:noiseCoordsS0p5}, bottom row). 

For $\sigma=0.08$ we found most simulations end in extreme consensus. That is, all opinions were at the extremes ($\pm 1$) rather than closer to zero, but these opinions were universally shared so that final polarization was zero. One such trial is shown in the middle row of Figure~\ref{fig:noiseCoordsS0p5}.This occurs because noise is sufficient to move the population toward the extremes (from which it is  difficult to return to center), but agents remain sufficiently clustered so that all forces remain attractive rather than repulsive. 




\begin{figure}[H]
  \centering
    \begin{subfigure}[t]{\textwidth}
      \centering
      \includegraphics[width=.7\textwidth]{/Users/mt/workspace/Papers/Archive/complexity-overleaf-git/Figures/noise_coords_S0p5_n0p0.pdf}
    \end{subfigure} \\
    \begin{subfigure}[t]{\textwidth}
      \centering
      \includegraphics[width=.7\textwidth]{/Users/mt/workspace/Papers/Archive/complexity-overleaf-git/Figures/noise_coords_S0p5_n0p08.pdf}
    \end{subfigure} \\
    \begin{subfigure}[t]{\textwidth}
      \centering
      \includegraphics[width=.7\textwidth]{/Users/mt/workspace/Papers/Archive/complexity-overleaf-git/Figures/noise_coords_S0p5_n0p2.pdf}
    \end{subfigure} \\
  \caption{Exemplar spatiotemporal dynamics of agent opinion coordinates with $K=2$ and 
    $S=0.5$ for $\sigma \in \{0.0,0.08,0.2\}$. There are three regimes. In the first, without
    noise, every simulation 
    ends in centrist consensus (top row). In the presence of noise with $\sigma=0.08$,
    agents find extremist consensus; in this trial agents found consensus around the point $(-1, -1)$.
    The third regime is the high polarization regime at the highest level of communication noise
    we tested, $\sigma=0.2$. In this regime, agents split into 
    opposing camps, led by first-mover extremists.
  }
  \label{fig:noiseCoordsS0p5}
\end{figure}


\begin{figure}[H]
  \centering
    \begin{subfigure}[t]{\textwidth}
      \centering
      \includegraphics[width=.7\textwidth]{/Users/mt/workspace/Papers/Archive/complexity-overleaf-git/Figures/noise_coords_S1p0_n0p0.pdf}
    \end{subfigure} \\
    \begin{subfigure}[t]{\textwidth}
      \centering
      \includegraphics[width=.7\textwidth]{/Users/mt/workspace/Papers/Archive/complexity-overleaf-git/Figures/noise_coords_S1p0_n0p08.pdf}
    \end{subfigure} \\
    \begin{subfigure}[t]{\textwidth}
      \centering
      \includegraphics[width=.7\textwidth]{/Users/mt/workspace/Papers/Archive/complexity-overleaf-git/Figures/noise_coords_S1p0_n0p2.pdf}
    \end{subfigure} \\
  \caption{Exemplar spatiotemporal dynamics of agent opinion coordinates with $K=2$ and
    $S=1.0$ for $\sigma \in \{0.0,0.08,0.2\}$.  Before the random long-range ties are added at 
    $t=2000$, extremists pull centrists to the extremes, but more centrist
    agent caves are balanced between more extreme caves. When long-range ties
    are added, the balance is broken and agents proceed to move to one of the
    extremes. Because at least some extremists held each of the corners, 
    centrist agents do not move only to polar opposite corners, but in many 
    cases to the nearest corner contained a neighboring (in the network sense) agent. 
  }
  \label{fig:noiseCoordsS1p0}
\end{figure}


\begin{figure}[H]
  \centering
    \begin{subfigure}[t]{0.48\textwidth}
      \centering
      \includegraphics[width=.95\textwidth]{/Users/mt/workspace/Papers/Archive/complexity-overleaf-git/Figures/commnoise_0p08_K4.pdf}
      \caption{Moderate noise, extreme consensus}
      % \label{fig:noise_coords_S0p5_K4_noise0p1}
    \end{subfigure}
    \begin{subfigure}[t]{0.48\textwidth}
      \centering
      \includegraphics[width=.95\textwidth]{/Users/mt/workspace/Papers/Archive/complexity-overleaf-git/Figures/commnoise_0p2_K4.pdf}
      \caption{More noise, polarization}
      % \label{fig:noise_coords_S0p5_K4_noise0p2}
    \end{subfigure} 
  \caption{Exemplar parallel coordinate timeseries for $K=4$ and $S=0.5$. 
    Here the x-axis represents a single opinion coordinate, $k_i$, and the 
    y-axis is the location of an agent for that coordinate. Each agent is
    represented by a line, colored by cave membership.
    With $\sigma=0.1$, 
    consensus emerges but at a corner of the opinion space. 
%     This illustrates
%     the phenomenon of moderate levels of noise leading to consensus with 
%     agents at a large average distance away from center, as shown in 
%     Figure~\ref{fig:avedistHeatmaps}. When $\sigma$ is
%     increased to 0.2, polarization emerges, with most agents moving to 
%     extreme antipodal corners of the opinion space.
  }
  \label{fig:noiseCoordsS0p5K4}
\end{figure}

When initial opinions are drawn from the full range of possibilities ($S=1$), the system always achieves some degree of polarization. Because noise only serves to increase the likelihood of extreme opinions, this condition is unaffected by noise. 
 %   Noise does not seem to change the normal course of events for $S=1.0$. 
 Typical cases are shown in Figure~\ref{fig:noiseCoordsS1p0}. The behavior for
$t\leq 2000$ is similar in all three cases: each cave reaches a local consensus,
and the network of caves reaches a stable configuration. 
Some of the caves find consensus values at the corners. When
random ties are added, the stable configuration is broken, and agents are
pulled towards one of the four corners, where some caves have already 
been stably established. The caves in the corners do not move. Recall that a key assumption of the FM model is that extremist opinions influence centrist opinions more than centrists influence extremists. 
The noise is not strong enough to move extremists from extreme positions. In other words, in the presence of extreme opinions, network structure, not noise, dominates the dynamics.
%
We extend the intuition to higher dimensions of opinions space using parallel coordinate plots, visualizing time series of opinion dynamics for $K=4$ (Figure~\ref{fig:noiseCoordsS0p5K4}). 
%An example of extreme consensus for moderate communication noise is shown in Figure~\ref{fig:noise_coords_S0p5_K4_noise0p1}. The combination of random ties and communication noise disrupt centrist consensus after $t=2000$, leading to extremist consensus. Random ties and communication noise lead to polarization, as happened with $K=2$ (Figure~\ref{fig:noise_coords_S0p5_K4_noise0p2}).

\section{Discussion}

Humans are the quintessential cultural species. Our instinct to learn from others is a key reason for our domination of the planet \cite{henrich2015secret,laland2017darwin}. An under-appreciated component of cultural learning concerns exacerbating differences and rejecting opinions when individuals are not likely to share one's current norms and beliefs. When those differences occur within a community, they can lead to discord. Many of us live in multicultural societies requiring cooperation and common ground, and so it natural to ask: when do we expect polarization, and is there anything we can do about it. Any suggestions based on our modeling efforts here should of course be compared with empirical studies. Hopefully these results stimulate further 
empirical work to understand when and why polarization emerges in
real-world situations. One such opportunity for future work is to connect our findings to 
the political science literature on polarization \cite{Sides2015}, 
especially in relation to communication. 
If agents had different roles, such as elite agents (politicians and media) and common agents, 
we could model the effects of ideologically-biased news in political polarization \cite{Prior2013,Pew2014}. 
Our results show that in the presence of sufficiently large communication noise and 
small-world networks, a situation we are arguably in today, a state of polarization 
is the only stable state (Figures~\ref{fig:noiseCoordsS0p5},~\ref{fig:noiseCoordsS1p0}, and~\ref{fig:noiseCoordsS0p5K4}). 
It is interesting to consider this in light of one recent analysis suggesting that the United States Constitution was designed not just to accommodate polarization, but to foster it for the sake of stability 
\cite{Wood2017b}. 

We have highlighted the stochastic nature of the system being modeled. A key conclusion is that empirical results of opinions on social networks may, when taken on a case-by-case basis, exhibit trends that bear little resemblance to those predicted by the model. This is not necessarily an invalidation of the model, but merely a consequence of the variability inherent in complex systems. That said, given enough data, key trends should emerge.  
We have confirmed Flache and Macy's (2011) result that long-range ties increase polarization. As such, we might emphasize the importance of local communities being allowed to reach their own consensus. We have shown that decreasing initial extremism can reduce polarization, as one might expect. Achieving consensus in a community relies heavily on the absence of opinions at the extremes. 
However, this result is quite sensitive to noise in communication. A little bit of noise can shift consensus from centrist or ambivalent positions to more extreme views, while more noise can lead to polarization. 
Even if polarization is to be avoided, what about the intermediate case of 
``extreme consensus''? While it may be natural to view extreme opinions as 
undesirable, an alternative perspective is that they represent a more stable system of 
cultural coherence. Note that these findings contradict computational and mathematical studies of the 
bounded confidence model under the influence of noise, where sufficient noise breaks 
polarization and leads to disordered opinion spreading \cite{Pineda2009,Carro2013,Kurahashi-Nakamura2016}. 
This is because in the bounded confidence model, agents that are too far from 
one another do not interact. In the FM model, connected agents always interact, 
and the further apart they are in opinion space, the more strongly they repel 
one another in opinion space. 

We confirmed Flache and Macy's (2011) result that increased ``cultural complexity''---the number of  opinions that are important to individuals in assessing their similarities and differences with
others---decreased overall polarization. We also showed that this result stems directly from an increase in the number of permutations of extreme opinions individuals can hold when there are more items on which one can hold opinions. This might be viewed as a flaw in the metric of polarization used here. Alternatively, we believe it is reasonable to posit that a community with a wider diversity of views should be considered less polarized than a community with only a few suites of clustered opinions. In any case, this finding highlights the importance of a thorough understanding of one’s distance measure when dealing with multidimensional opinions. 
Our analysis may in fact cast doubt on the interpretation by Flache and Macy (2011) that 
cultural complexity decreases opinion polarization, if one also rejects the interpretation 
that adding arbitrary traits on which actors are indifferent should reduce their opinion 
distance.

As noted, the model we have studied is a simplified abstraction, and does not include many details that are important to the empirical reality of opinion dynamics. In general, theoretical modeling work should start simple, and gradually add heterogeneity as the simpler versions of the system in question become fully described. Future work should explore these sources of heterogeneity. 
First, we did not distinguish between private opinions and public productions
representing those opinions \cite{Nowak1990}. Our operationalization of communication
noise could be interpreted as a modulation of private opinion, 
but communication noise could also be interpreted as misunderstanding of perfectly-reproduced, publicly voiced opinions. People often communicate public opinions that differ from their private opinions 
when incentives for the parties involved are not aligned \cite{crawford1982strategic,pinker2008logic,smaldino2018evolution}.  
Second, we ignored the structural influence of explicit identity groups. It could be argued that clustering of agent opinions implicitly defines an identity group. For example, \citeA{DellaPosta2015}
measured network auto-correlation to explain why people's preferences cluster
together. This data-driven approach was offered as an attempt to explain arbitrary opinion clustering, as indicated by the paper's title, ``Why do liberals drink lattes?''. Nevertheless, explicit identity with groups and roles influences human behavior far beyond homophilic clustering \cite{barth1969ethnic,berger2008drives,smaldino2018social}. 
Third, we ignored individual differences in how individuals influence and are influenced. Some people may be stubborn while others are easily swayed. 
Some prestigious or charismatic individuals may have outsized influence while others are ineffective at communicating their opinions. Relatedly, individuals may also vary in their confidence in their opinions, which will influence the extent of their mutability and persuasion. 
The assumption that as agents become more extreme, their opinions become more stubborn, as formalized in Equation~\ref{eq:smoothed-update}, may not always hold. Indeed, our work highlights the need for additional empirical work on how individuals alter their opinions as a function of how extreme those opinions are.
Finally, the social networks used in our model are simplistic in both dynamics and structure. Ties in many real world networks change with greater frequency than we modeled, providing new opportunities for social influence. Moreover, interactions and opinions are contextual. Individuals are embedded in multilayered social networks, in which the dynamics of opinions may be considerably more nuanced than indicated by our relatively static, single-layer network \cite{battiston2017layered,smaldino2018resilience}. 

In our study of the FM model we have found rich behaviors and theoretical lessons for understanding opinion dynamics. This work highlights the potential for complexity even in a very simple model of individual behavior, because network structure provides for path dependent effects and can be further influenced by initial conditions and noise. Our analytic approach highlights the value of systematic investigation of a model's explicit and tacit assumptions. 

\section*{Appendix: Proof that polarization scales with $1/K$}

We hypothesized that the decrease in polarization with increasing $K$ observed in simulations of the FM model were driven by an increase in the number of permutations of binary vectors of length $K$, in which each element was $-1$ or 1. We supported this hypothesis in the main text with simulations in which agents were randomly initialized at such extreme positions in opinion space. Here we derive a formal proof that polarization in the FM model scales with $1/K$ if we assume that agents are randomly assigned a vector of ``extreme'' opinions, such that $\forall {i,k}$, $s_{ik} \in \{-1, 1\}$. 
% Extreme opinions are more stubborn than centrist opinions in the FM model. 
% Because agents update their opinions via biased assimilation, one might guess that 
% agents are simply assorting their opinions into various ``sticky'' corners
% of opinion space. To better understand to what degree agents are simply assorting
% into corners of opinion space, 
To do this, we exactly calculate the polarization of a population 
where each agent occupies one of the $2^K$ corners of opinion space with $K$
cultural features. 
% The derivation illuminates the cause of decreased polarization
% when of adding more cultural features through building intuition of the
% geometry of opinion space, of the distance measure $d_{ij}$, and our
% chosen polarization measure.

Recall that polarization is defined as the variance in pairwise distances
between all agents. We define the \emph{combinatorial polarization}, $P_c(K)$, 
as the polarization that arises from randomly placing each agent at one of the $2^K$
corners of opinion space with $K$ cultural features, 
which is a $K$-hypercube, denoted $Q_K$. ``Corners'' of opinion space are 
simply vertices in the graph of $Q_K$.
We computed this value numerically for $K \in \{1, \ldots, 12\}$ and found it tracks
closely to $1/K$ (see Figure~\ref{fig:combinatorial-comparison}). 
Here we demonstrate that $P_c = 1/K$ exactly in the large $N$ limit.
To calculate $P_c(K)$, we need three elements. First, we need to
calculate the distance between pairs of agents at different corners of
$Q_K$. Second, we must count the number of agent pairs separated by the distance
from one corner to another. We do this by first counting the number of 
subcubes of dimension $L$, or $L$-subcube.
Then we count the number of maximally separated pairs in a subcube of 
$L$-subcube. Finally, we calculate the distance of maximally separated, 
or \emph{antipodal} pairs, of agents in an $L$-subcube. We can then calculate
the expected value of pairwise distances, $\langle d \rangle$, and the expected
square of pairwise distance, $\langle d^2 \rangle$, from which we will have the
combinatorial polarization 

\begin{equation}
  P_c = \langle d^2 \rangle - \langle d \rangle^2
\end{equation}

We will show that $P_c=\frac{1}{K}$ by showing that $\langle d \rangle = 1$ and
$\langle d^2 \rangle = \frac{K+1}{K}$. Before we do that, we will derive functions
to help us count the number of pairs separated by a particular distance, and
to calculate distances between vertices on subcubes $Q_L \subseteq Q_K$. First,
we denote the total number of pairwise distances as $n=\frac{N(N-1)}{2}$ where $N$ is
the number of agents. The number of $L$-subcubes $Q_L \subseteq Q_K$ is

\begin{equation}
  n_s(L, K) = 2^{K-L} {K \choose L}
  \label{eq:ns}
\end{equation}
\noindent
This results from the fact that at all $2^K$ vertices of $Q_K$, ${K \choose L}$
subcubes can be created by choosing $L$ nodes adjacent to the vertex. This 
gives us $2^K {K \choose L}$ subcubes. This overcounts since each 
generated subcube was generated once for each of its $2^L$ vertices. So we must
divide by a factor of $2^L$, giving us the expression in 
Equation~\ref{eq:ns}.% \cite{Banchoff1996}.

Within $Q_L$, the number of pairwise distances where agents occupy antipodal
vertices is 

\[
  n_a'(L, K) = 2^{L-1} \left(\frac{N}{2^K}\right)^2
\]
\noindent
There are $2^{L-1}$ pairs of antipodal vertices in $Q_L$. In the large
$N$ limit, agents are distributed in equal number to each vertex of $Q_K$.
Then, the number of agents in a single vertex is $\frac{N}{2^K}$, so the number
of pairwise distances between any two antipodal pairs is $\left(\frac{N}{2^K}\right)^2$.
The total number of antipodal pairs across all $Q_L$ is then

\begin{equation}
  n_a(L, K) = n_a'(L, K) n_s(L, K).
\end{equation}
\noindent
Finally, the distance between agent opinions $\vec s_1$ and
$\vec s_2$ in antipodal vertices of $Q_L$ is

\begin{equation}
  d_a(L, K) = \frac{1}{K} \sum_{k=1}^{K} |s_{1k} - s_{2k}| = \frac{2L}{K}
\end{equation}
\noindent
since any antipodal vertices of $Q_L$ share $K-L$ opinion coordinates, and the
maximum magnitude of difference on a single opinion dimension is 2. 

With these quantities we can write the expected value of pairwise distance,

\begin{equation}
  \langle d \rangle = \frac{1}{n} \sum_{L=1}^K n_a(L, K) d_a(L, K).
\end{equation}
\noindent
Simplifying and taking $N \rightarrow \infty$, this becomes
\[
  \langle d \rangle = \frac{(K-1)!}{2^{K-1}} \sum_{L=1}^K \frac{1}{(K-L)!(L-1)!}
\]
\noindent
Using the identity
%%\footnote{\url{http://www.wolframalpha.com/input/?i=Sum%5B1%2F((L-1)!(K-L)!),+%7BL,+1,+K%7D%5D}}  
\[
  \sum_{L=1}^K \frac{1}{(K-L)!(L-1)!} = \frac{2^{K-1}}{(K-1)!},
\]
\noindent
we find $\langle d \rangle = 1$. Calculating $\langle d^2 \rangle$ proceeds 
similarly, beginning with

\begin{equation}
  \langle d^2 \rangle = \frac{1}{n} \sum_{L=1}^K n_a(L, K) d_a(L, K)^2.
\end{equation}
\noindent
Simplifying and taking $N \rightarrow \infty$, this becomes
\[
  \langle d^2 \rangle = \frac{(K-1)!}{2^{K-2}K} \sum_{L=1}^K \frac{L}{(K-L)!(L-1)!}.
\]
\noindent
With the identity
%\footnote{\url{http://www.wolframalpha.com/input/?i=Sum%5BL%2F((L-1)!(K-L)!),+%7BL,+1,+K%7D%5D}} 
\[
  \sum_{L=1}^K \frac{L}{(K-L)!(L-1)!} = \frac{2^{K-2}(K+1)}{(K-1)!}
\]
\noindent
we find $\langle d^2 \rangle = \frac{K+1}{K}$. So

\begin{equation}
  P_c = \langle d^2 \rangle - \langle d \rangle^2 = \frac{K+1}{K} - 1 = \frac{1}{K}.
\end{equation}

% Our results suggest that, in real-world political problems, it is crucial
% to change the opinions of certain actors before others. Movement towards consensus
% appears to depend on which influential leaders Political crises
% such as those in Venezuela (REF), Egypt (REF), or Ukraine (REF) have recently
% been studied for how discourse polarizes . It is not a random number generator
% that determines who we interact with and learn from in the real world. We do
% not have to choose the path that leads to higher polarization, we can choose
% one that leads to an equilibrium with a lower polarization.  We have shown that
% understanding one another well can lead to less polarized outcomes. If we do
% not, it does not matter how politically savvy a society is, it will find itself
% in a state of high tension.

% Our computational experiments showed that the FM model of opinion dynamics yields
% rich behaviors with valuable real-world interpretations. Our study has laid the
% groundwork for further analysis of this model, both empirical and theoretical. 
% On the empirical side, we can do experiments to see if it is correct that
% centrist opinions are more malleable than extremist opinions. It is another
% empirical question to ask, what is the maximum opinion distance two agents
% can be from one another and still have a ``friendly'' relationship? Furthermore,
% in real-world situations, what degree of extremism can a social group have
% and still come to consensus? How far from perfect is communication? Our work
% shows that these parameters have consequences for model behavior. The difficulty
% in answering these empirical questions is sure to be compounded by the
% fact that these parameterizations will not be identical for different groups
% of people, or even for the same group of people at different points in time.

% Our results showed that a state of consensus is less stable than states of
% varying degrees of polarization. This, too, seems to be an empirical 
% question. Reviewing history it seems even in stable governments there are
% groups with opposing ``opinions.''

% Not only do our results guide us to ask more and different questions, but we
% believe our approach can be fruitful for examining the myriad opinion dynamics
% models available now, and into the future. It is a standard procedure to 
% examine the assumptions of a model through testing the sensitivity of a model
% to initial conditions and to noise. Because of the non-deterministic nature
% of the FM model, due to update path dependence, more work should be done to 
% explore a model with synchronous updating. Such a model might be amenable to
% analytic analysis and solutions, which may shed more light on the combinatorial
% and statistical natures of the problem. 

% \subsection{Limitations} 
% \begin{itemize} 
%   \item No difference between public and private opinions---unless we interpret
%     ``noise'' in communication to represent (part of) that difference.
%     \cite{Nowak1990}
%   \item No identity groups---Unless, as \citeA{DellaPosta2015} show can
%     happen on networks, we interpret the homophilous clusters that emerge 
%     as identity groups.
%   \item Individual differences in, e.g., communication fidelity, stubborness,
%     threshold to consider others friends.
% \end{itemize}

% \subsection{Comment on the results of Flache and Macy} 
% Increasing cultural complexity doesn’t so much reduce polarization as increase the number of clusters. It is arguable whether a community evenly split between four extremes should be considered less polarized than one that is evenly split between two extremes. Perhaps this in reality does serve to reduce tension between individuals. It may be easier to unite many disparate bands than two powerful warring factions. 

% \subsection{New recommendations for dealing with polarization} 
% \begin{itemize} 
%  \item Get rid of extremists, or deny them influence. 
%  \item Reduce noise. 
%  \item Caveat: There may be much stochasticity that recommendations are at best highly probabilistic. That said, this is likely true for all social systems, which are highly complex and have many factors, both endogenous and exogenous, that influence results. At the very least, these results show us that sometimes even the best laid plans can go awry, via factors that are intrinsically outside anyone’s control. 
% \end{itemize}



% \clearpage

% \input{figures.tex}
\section*{Data Availability}
Data used for our analyses is available for download ($\sim$14GB) from 
\url{http://mt.digital/static/data/polarization_v0.1-data.tar}. 

\section*{Conflicts of Interest}
There are no conflicts of interest for either author.

\section*{Acknowledgements}
Computational experiments were performed on the MERCED computing cluster,
which is supported by the National Science Foundation [Grant No. ACI-1429783].


\clearpage




% \chapter{}
% \chapter{Title}
% This is only a test.
% \section{A section}

% \appendix
% \chapter{Final notes}
%   Remove me in case of abdominal pain.

%% END MATTER
% \printindex %% Uncomment to display the index
% \nocite{}  %% Put any references that you want to include in the bib 
%               but haven't cited in the braces.

\bibliographystyle{apacite}
\bibliography{/Users/mt/workspace/Papers/library.bib,
    /Users/mt/workspace/Papers/Archive/complexity-overleaf-git/ComplexitySpecialIssueForDissertation.bib
  }
\end{document}

