
High levels of political polarization seem to bring about or go along with
hardening of partisan identites~\cite{Lee2015}.
As society becomes more polarized, political disagreement spills over
and fouls up collaborative social behavior
more generally~\cite{Iyengar2019}, even making violent responses to verbal 
communication more likely~\cite{Kalmoe2014,Kalmoe2018,Mason2018UncivilAgreementBook}.
Why do extremism and polarization increase and decrease over time in different contexts? 
This simple question yields no simple answers. These questions
have been studied scientifically for some decades now by researchers across perhaps a dozen
diverse disciplines and sub-disciplines including 
political science~\cite{Mason2018UncivilAgreementBook,Boxell2020}, 
sociology~\cite{Baldassarri2007,Flache2011}, 
economics~\cite{Schelling1971,Dixit2007}, cognitive science~\cite{Rollwage2019}, 
and philosophy~\cite{OConnor2018}.

In this dissertation I focus
on three more specific questions about increasing extremism and polarization.
These questions help us understand rising extremism and polarization, and
predict what situations will foster rising extremism and polarization.
First, what is the prevalence of communication strategies on mass media
known to increase extremism, specifically the use of \emph{violence metaphors}
to describe non-violent political events? Second, what causes observed
increases in extremism among ideologically similar groups over time?
Third, and finally, what are some fundamental cognitive capacities and
social factors that are required for polarization to occur, and what
role do random chance and miscommunication play in the emergence of
polarization? In the course of this work, I also identified a statistical
problem that undermines many or possibly most published results on rising
extremism among ideologically biased groups.

To answer these research questions, this dissertation studies the communicative, cognitive,
and social factors that provide the human substrate for rising extremism
and polarization~\cite{Jung2019,Rollwage2019}. 

The studies I present in this dissertation help delineate and explore points of contact between
the various disciplinary approaches to studying polarization. As such, they
required diverse theoretical, modeling, and computational methods, 
including structured corpus building and analysis for studying metaphor
use on cable TV news, statistical modeling of the opinion generation and
measurement process, and agent-based modeling of social influence processes.

In political communication, people are differentially
influenced depending on what language is used, even down to choice of 
grammar~\cite{Matlock2012}. For example, if the past participle is used
to describe a politicians bad deeds (e.g., \emph{he was imbezzling campaign
funds}) this worsens people's opinions of the politician compared to 
communicating using the simple past tense (\emph{he imbezzled campaign funds}).
This dissertation focuses specifically on the choice of metaphor used 
in political discourse, which is a powerful method for framing political 
messages to rally political allies and identify, disparage, and target
political enemies and other out-group members~\cite{OBrien2003,Charteris-Black2009,Landau2010}.

There are also communication-independent cognitive and social factors at work
in social influence that leads to extremism and polarization. Cognitively,
we know that (1) similar individuals tend to find consensus with one another~\cite{French1956,DeGroot1974};
(2) dissimilar individuals can push one another to be even more 
different~\cite{Cikara2014,Bail2018};
(3) social influence between similar others tends to be more attractive the more
similar they are and more repulsive the more dissimilar they are~\cite{Lord1979,Ross2012};
and (4) that extremists tend to be more stubborn than centrists~\cite{Reiss2019,Zmigrod2019a};

Social relationships structure social interactions. Social relationships
determine who interacts with whom, which is often analyzed using social
networks~\cite{Watts1999}. A major factor that determines an individual's social
network is the tendency towards \emph{homophily}, summarized in the heuristic
that ``birds of a feather flock together''~\cite{McPherson2001}. This results
in social network structures where similar individuals tend to interact more
often compared to dissimilar individuals. This can have major impacts on 
societal opinion structure and dynamics when similar individuals interact more
and more often, and dissimilar individuals sometimes come to not interact at
all~\cite{Axelrod1997,Centola2007,DellaPosta2015}.

\section{Introduction to the dissertation studies}

This dissertation's modest contributions to the expansive literature on extremism
and polarization are in breaking down the complex social influence system of
society into three model sub-systems to understand and predict how communicative,
cognitive, and social factors work together to contribute to extremism and polarization
in different contexts. This has resulted in the four studies presented in this dissertation.

In Study 1 I develop a dynamic model of \emph{violence metaphor} use (e.g.,
``Clinton hit Trump over his tax returns'') on cable TV news to understand how
this changes under the influence of the US presidential debates and elections
in 2012 and 2016. Violence metaphors are important to understand because 
exposure to violence metaphors tends to increase anger towards and dislike of 
political opponents, including increased support of violence to achieve
political goals~\cite{Kalmoe2014,Kalmoe2018}.
I found that violence metaphor usage was more reactive to
the debates in 2016 and 2012, largely influenced by using violence metaphors
to describe Twitter ``attacks'' (i.e.\ tweets) by one political candidate
against the other. 

In Study 2 I develop a novel, parsimonious explanation of a group-level process that seems
to increase extremism among like-minded group members, known as
\emph{group polarization}~\cite{Brown1986,Isenberg1986,Brown2000,Sunstein2002}.
Existing explanations of group polarization tend to rely on several auxiliary
assumptions that may or may not be well supported, which make the explanations
difficult to evaluate~\cite{Meehl1990}. I use agent-based modeling to
show that group polarization can
be more parsimoniously explained by the empirically motivated assumption 
that people become more stubborn as their opinions become more 
extreme~\cite{Reiss2019,Zmigrod2019a,Kinder2017}. 

Unfortunately many published detections
of group polarization are plausibly false, which I demonstrate in
Study 3 using a statistical model. False detections may occur in group polarization
data because researchers failed to rigorously account for floor/ceiling effects
introduced when ordinal behavioral data (e.g. Likert scale data) is analyzed
using metric statistical models (e.g.\ a $t$-test to detect differences between
normal distributions). Metric statistical models fail to account for
the fact that very extreme opinions can become significantly more moderate
when measured on an ordinal scale. Therefore, metric models can be tricked into
thinking real psychological opinions have become more extreme among a group,
when in reality the ordinal measurement scale failed to detect opinion shifts
towards moderation among extremists, 
which masks a simpler process of consensus around the initial group mean opinion.
I found that 90\% of detections of group polarization across ten journal articles
are plausibly false detections, which throws into question the reality of
the group polarization.

In Study 4 I adapt the same agent-based
model from Study 2 to understand how well we can explain and pedict large-scale
societal polarization under increased 
social network connectivity as might emerge from interacting with dissimilar
others over the Internet~\cite{Bail2018}, in addition to considering the effect
of initial extremism, miscommunication, and path-dependence of social interaction
order (i.e., who interacts with whom, and when). I found that while social network
structure can bias society towards more or less polarization~\cite{Flache2011}, 
this is highly path-dependent. Furthermore, there are critical values of initial extremism
and communication noise that can override social network structure or path 
dependence to make either high levels of polarization or full consensus 
(i.e., no polarization) inevitable.

In the remainder of this introductory chapter I will introduce the 
twin problems of extremism and polarization, including definitions of both
terms. I then introduce the importance of metaphor and framing more generally
on rising extremism and polarization. After this, I introduce cognitive and
social factors that, together, provide the social influence substrate in which extremism 
and polarization emerge and rise. Next, I explain my strategy for modeling the emergence of 
extremism and polarization since all four studies rely on mechanistic modeling
of social influence processes. To close this chapter I give an overview of the
four studies that comprise the rest of the dissertation.


\section{Rising extremism and polarization}

Among the popular press and politically-involved citizens, it seems obvious
that polarization is increasing, and that this increase is a 
dangerous problem that needs to be solved. 
For example, when in 2014 the Pew Research Center found polarization
in 2014 to be the highest in decades, journalist and Vox founder Ezra Klein 
(2014) found the statement obvious, writing ``(E)veryone already knew that.''
\citeA{Klein2020} later explained further in a book for the popular press,
{Why we're polarized}. 
However, whether or not polarization occurs depends on how polarization 
is defined and the population being studied. 
In fact, there is a debate among political scientists whether
polarization really occurs, but this is a matter of definition~\cite{Mason2015,Lelkes2016,Kinder2017}.
Extremism and polarization must be well defined to study them scientifically.

At worst, extremism and political
polarization leads to political violence and even civil war~\cite{Epstein2013,Freeman2018}.
However, different types of extremism and polarization may have different
effects on society and governance~\cite{Lelkes2016}. \citeA{Lee2015}, for example,
found that although partisan sorting had occurred in recent decades, there
had been little degradation in legislative and other government outcomes.
The rise of \emph{affective polarization} between political groups has been found to increase as non-political
preferences align among partisans as well---for example, preferences for
leisure activities and entertainment are becoming increasingly correlated
with ideology and party membership in the United States~\cite{Pew2014PublicPolarization,DellaPosta2015}.
Affective polarization includes the increasing dislike and distrust between opposing political parties, 
which spills over into non-political areas of life~\cite{Iyengar2019}. 
Polarization is certainly on the rise, but this may be a correction to normal from several previous decades of 
unnaturally low levels of political sorting~\cite{Lee2015,Wood2017b}.
Similar trends and concerns can be observed 
worldwide~\cite{Borge-Holthoefer2014,Morales2015,Romenskyy2017,Zmigrod2018}. 
We in the United States may, however, have a particularly bad case of rising polarization: 
\citeA{Boxell2020} found that among the United States and eight other OECD countries
the United States had the largest increase in affective polarization over the
past four decades. Further cross-cultural study is necessary to understand the fundamental human factors underlying
extremism and polarization, especially societies that are not Western or democratic,
with possibly lower standards of living~\cite{Henrich2010}.

In this dissertation I aim to identify fundamental communicative, cognitive, and
social factors that foster extremism and polarization, with limited contextual
details. Of course context is important, but we cannot know how important without
first understanding baseline capacities and processes that lead to changes in
extremism and polarization. Note that the above evidence
for increasing extremism and polarization
assumes the existence of political parties and ideologies---the 
Republican and Democratic parties, and conservative and liberal ideologies.
Polarization in these studies is measured by the degree of sorting of ideologies and preferences
into associated political parties~\cite{Mason2015} and increasing interpersonal 
dislike~\cite{Iyengar2019}. To understand more fundamental factors than 
parties and ideologies, I take a more general approach that does not label 
individual opinions, and defines increased extremism and polarization formally
in terms of opinion distributions. In this dissertation, \emph{extremism} is 
defined by how extreme one's opinion is on some opinion 
scale, which represents how intensely or confidently someone believes in their
own opinion on some topic.  For example, giving a zero on a Likert scale often means that one 
neither agrees or disagrees with some statement. Strong disagreement or agreement
is indicated by indicated the largest negative or positive values on the Likert
scale. \emph{Polarization} in this dissertation is 
conceptualized as the bimodality in opinions among a population, ignoring 
partisanship and political or ideological identities. Opinion bimodality
can quantify ideological divergence among all members of 
society~\cite{Bramson2016,Lelkes2016}. 

Among all the ways in which individual preferences and opinions are sorted and 
polarized in the United States, a
major one one that both reflects and drives rising polarization is the split in
where partisans get their news~\cite{Pew2014PolarizationAndMediaHabits,Martin2017}. 
What is said on cable TV news and other mass media is extremely important, given
the reach of mass media and the way mass media frames the terms of debate~\cite{Chong2007}.
One important communication strategy is the use of different 
metaphors to frame different political
messages, processes, and events. These framings influence the way politics is 
understood and discussed by news consumers. One's opinions about immigrants, for example, may depend
on whether one has been exposed to metahpors that cast immigrants as 
``indigestible food, conquering hordes,'' or ``waste materials''~\cite{OBrien2003}.
Those who had been exposed to such metaphors may later tend to favor stricter limits
on immigration and harsher treatment for undocumented immigrants.

In Study 1, I quantify the change in frequency over time of a 
specific type of metaphor use, violence metaphors, across cable news channels
MSNBC, CNN, and Fox News, around the time of the United States presidential
debates and elections. Violence metaphors are important because they have
been observed to push individuals to more extreme political opinions, even
increasing support for real world political violence~\cite{Kalmoe2014,Kalmoe2018}.
Metaphorical violence is a prime strategy for inflaming partisan passions 
through statements such as ``Trump has been getting \emph{attacked} by the liberal
democrats on Capital Hill,'' which one might hear by a commentator or anchor on Fox News.

Mass media frame the terms of political discourse, which spreads through
interpersonal influence among ordinary citizens~\cite{Katz1955}. But how does 
interpersonal social influence work, and how do we know which social 
influence processes are essential for rising extremism and polarization to
emerge? In this dissertation I break down interpersonal influence into formal computational
models to analyze whether polarization emerges from that model, without assuming anything
about polarization itself. 

Interpersonal influence of opinions can be broken down into
four important cognitive factors: (1) attractive and repulsive influence of
opinions~\cite{French1956,Cikara2014,Bail2018}; (2) homophily, i.e., preferential assortment with like others~\cite{McPherson2001};
(3) biased assimilation, i.e., heightened influence by similar others~\cite{Dandekar2013}; and
(4) a correlation between stubbornness and extremity of opinions~\cite{Reiss2019,Zmigrod2019a}. 
Other cognitive factors may include, e.g., personality traits that may be predictive of 
ideological or other opinion, attitude, belief, etc., preference~\cite{Zmigrod2018}.

Social factors that modulate social influence processes are: (1) social
networks, theoretical entities that represent a person's social relationships
that structure who in a society interacts, when; and (2) the stochasticity
of interpersonal influence, e.g., three people may frequent a certain bar
and talk regularly on Fridays, but who attends varies depending on essentially
random factors like other obligations or obstacles to attending. 

In Study 2 I incorporated these cognitive and social factors into a 
computational model, which led to the simulated emergence of 
``group polarization'', the empirical 
observation that socially isolated, initially biased groups tend to become
more extreme in their opinions over time. In Study 4 I examine critical
tipping points of the model that might guarantee polarization or consensus and
explore the limits of using this model (or any model) for predicting 
rising extremism and polarization.

I conceptualize polarization as an emergent property or phenomenon of society. 
Like the field of psychology generally, understanding and predicting extremism
and polarization requires cross-disciplinary understanding that sometimes
involves scientists working focusing on one system component and sometimes has 
scientists exploring the interfaces between system components~\cite{Brewer2013,Rollwage2019}.
By making general assumptions about how social influence works in society, it is
possible to encapsulate many of the dimensions along which individuals are
separated and how social influence regarding opinions 
(or beliefs, attitudes, etc.) on one dimension
is correlated with social influence along other dimensions. We can add 
details to such a general model as necessary to understand, for example, 
how framing strategies change over time on cable TV news (Study 1) or how
extremism rises in initially biased, socially isolated groups (Study 2). 
In Study 4, I use a general model of social influence to 
investigate the role of social network structure, initial extremism/polarization,
communication noise, and random ``path dependence'' on the order of 
interpersonal interactions on the emergence of polarization.


\section{Metaphor and framing in politics and polarization}

In the first study of this dissertation, I analyze violence metaphor use on
cable news around the times of the 2012 and 2016 United States presidential
debates and elections. But what is metaphor? How is it used and what are the
effects of metaphor use in political communication, especially regarding
political polarization? Metaphor has long been recognized as an important 
element in the study of political 
communication at least since Aristotle's time if not before.
Aristotle saw metaphor as a special 
feature of especially talented orators' rhetoric~\cite{Aristotle1965,Kirby1997}. 
In contrast, modern cognitive understanding of metaphor recognizes the
ubiquity of metaphor as a critical cognitive tool evolved for 
conceptual scaffolding used for abstract 
thought~\cite{Lakoff1980,Heyes2018a,HeyesCognitiveGadgets}.  

The word metaphor
comes from the ancient Greek word \emph{metaphora} 
($\mu \epsilon \tau \alpha \phi o \rho \acute{\alpha}$), meaning
\emph{transferrence}.  Metaphor works by ``transferring'', or mapping in
the mathematical sense, conceptual entailments
from a more concrete concept, such as a fight, onto a more abstract concept,
such as politics~\cite{Regier1996,Kovecses2010a,Lakoff2014}. 
Politics is an abstract concept because it can describe many
different situations, events, and processes. One never directly sees or feels
politics. The outcomes of political decision are only felt indirectly in terms
of increased or restricted liberty, or economic effects such as tax breaks or
an improved economy.  On the other hand, either being engaged in or observing physical conflict
results in a cascade of immediate bodily effects, including body-to-body contact and
possibly injury for fight participants. The conceptual entailments of casting
politics as violence encodes the fact that politics generates similar 
feelings and physiological reactions to being in a fight, for example, 
including the physical sensations of elation or depression following a political win
or a political defeat, and the adrenaline and other biophysical repsonses
of the fight itself~\cite{Gallese2005,David2016}.

Metaphor is one of several forms of linguistic \emph{framing} that can 
powerfully influence our understanding of political
events through strategic, pragmatic choice of 
words~\cite{Fillmore1982,Chong2007,Lakoff2008,Charteris-Black2009,Fausey2011,Matlock2012,Sagi2013a,Cacciatore2016}.
Embodied metaphors enable people to gain intuition about many different
abstract concepts beyond politics. 
For example, we talk about ``navigating'' the internet, but this is 
really just typing and clicking links or buttons~\cite{Matlock2014}.
We often describe the passage of time in terms of physical motion, as in,
``my dissertation defense date is fast 
approaching''~\cite{Matlock2005,Nunez2012,Flusberg2017a}.
Embodied concepts such as rotations and extrusions permeate the abstract realm
of mathematics~\cite{Lakoff1997,Marghetis2013}. 

Politicians and commentators have long used metaphor to motivate
supporters and villify opponents~\cite{Charteris-Black2009}. 
To take a current example, Fox News has
recently been covering what they call ``Classroom Warfare'' over
Critical Race Theory~\footnote{Humorously summarized by The Daily Show here: \\
https://www.youtube.com/watch?v=7sGK33uTOpU}.
Anchors and commentators on Fox News variously cast anti-Critical Race Theory
protestors as ``an army of moms and parents'' waging war ``on the front lines
of this fight.'' To understand the purpose and effectiveness of this metaphorical framing, 
we have to consider the \emph{entailments} that go
along with the \textsc{war} conceptual frame. Wars have at least two opposing
belligerent groups---
soldiers for each side are literally mortal enemies. 
In the context of American politics and to Fox News viewers, there
are conservatives on the Fox News side and liberals on the other side. 
Patterns in news consumption reflect this, with Donald Trump
voters watching Fox News far more than any other oultet in 2016, and 
Hillary Clinton voters watching MSNBC and CNN more than any other 
outlet~\cite{Prior2013,Pew2014PolarizationAndMediaHabits,Pew2017TrumpClinton}.

Beyond the partisan ``wars'' that play out in the minds of American citizens
based on what they see on American cable news, there are many
other political issues and events in the USA and abroad are 
described and understood using metaphorical language. For instance, metaphors were
used to cast Saddam Hussein as a madman and the United States as ``givers''
of freedom to Kuaiti citizens, which was metaphorically ``taken away''
with the Iraqi invasion~\cite{Lakoff1991}. 
This is a metaphor since freedom
is not a thing one can give, receive, or take away, 
as one would give another person water or food.
War metaphors for addressing the climate crisis seem to foster a greater
sense of urgency for finding solutions to the crisis~\cite{Flusberg2018}. 
War metaphors have also been ubiquitous in attempts to mobilize public 
responses to mitigate the spread of COVID-19~\cite{CastroSeixas2021}.

In the French and UK presses, metaphor use was observed to vary depending on the
political context: when Obama won the 2008 US election,
the UK and French presses framed Obama's victory as something predestined,
casting Obama as a sort of savior ushering in a new era of US politics, 
saying things like ``Obama walked on water.'' However, reporting on politics in Pakistan,
when former Pakistani General and retiring President Pervez Musharaf's party 
lost in Pakistan's presidential elections, UK and French news outlets 
called it a ``knockout'' and generally used other violent and disparaging
metaphors against the former president, despite him not even being a 
candidate~\cite{Burnes2011}. Similarly, the metaphor casting Washington, D.C.,
as a swamp has been used over the years
by both liberals and conservatives to cast the other side 
as dirty and corrupt~\cite{Burgers2019}.

Empirical data from behavioral studies supports the inference
that violence metaphors could contribute to readers and 
listeners' to resort to real world violence to attain their political goals.
In a series of studies, \citeA{Kalmoe2014} and \citeA{Kalmoe2018} showed that
exposure to violent metaphors drove partisans further apart in terms of opinions, and 
exacerbated aggressive tendencies towards one's political out-group. These effects
were most pronounced for the most aggressive members of society. Clearly violence
metaphors need to be understood due to their possibly detrimental 
effects on political and social stability. This need is a major motivation for
Study 1 that measures the dynamics of metaphorical violence usage on cable
TV news and finds it to be correlated with candidate Twitter activity---this
correlation is amplified in 2016 compared to 2012.


\section{Cognitive and social factors in extremism and polarization}

Extremism and polarization are emergent phenomena of social systems composed
of individuals. Human beings are the most fundamental components in the
models of social systems I use in this dissertation. 
But humans are complex themselves~\cite{Kello2007,Spivey2020},
a composite of simpler cells properly organized to have the capacity for
social influence of and by others, among many other capacities. 
So, how can humans be treated as fundamental?
The modeling strategy that solves this is to assume humans 
have only a small set of critical
capacities essential for the social interaction and 
influence that leads to extremism and polarization~\cite{Cartwright1989,Smaldino2017}. 
Understanding the capacities for social interaction and influence requires multi-method,
investigation spanning several disciplines in the form of computational,
behavioral, and neurobiological studies. Also important for understanding
the emergence of extremism and polarization are social factors, such as
the effect of social relationship networks on emergent
social phenomena. While communication is essential to increasing extremism and
polarization, much can be understood about cognitive and social factors
in polarization, indpendent of specific details about interpersonal 
communication.



\subsection{Cognitive factors in polarization}

Cognitive factors in polarization I focus on here are some essential
individual- and dyad-level capacities and processes that
enable social influence of one individual by others. The first essential capacity
is the capacity for one's opinions to become more similar to others' opinions.
The second essential capacity is the ability to become more different from those
with whom we disagree. Whether two individuals are attracted to or repulsed
from one anothers' opinions is often determined by their group membership---people
tend to be attracted to in-group members' opinions and repulsed by out-group
members' opinions. Therefore determining one's own and others' group membership is also an essential
cognitive capacity. Social influence can be modulated by one's degree of
similarity or dissimilarity to in-group or out-group members---more similar
views may be more attractive, e.g., or more different opinions more repulsive.
Individuals may also vary in their susceptibility to social influence---for
example in the model used in Studies 2 and 4 I assume those with more
extreme opinions are less susceptible to social influence, i.e., 
they are more stubborn.

We know that humans tend to find agreement with one another and consensus often 
emerges within groups~\cite{Festinger1954,Cartwright1956,French1956}. 
Consensus with (or conformity to) others' opinions has been shown to emerge 
even when direct evidence contradicts those opinions, as \citeA{Asch1955,Asch1956}
found in his classic studies in which participants were fooled by confederates
into going along with the crowd despite their own direct perception that
the crowd was obviously wrong. Consensus can be problematic when
consensus occurs around, e.g., 
false scientific beliefs and 
misinformation~\cite{Zollman2007,Zollman2013,OConnor2018,OConnor2019e}.

Often in intergroup social influence, members from different groups
develop more different opinions over time when they interact, instead of becoming more 
similar~\cite{Tajfel1979,Sherif1988,Flache2011,Bail2018}.
Group membership may be determined by observable traits such as race, language,
or style of dress, but it need not be. The ``minimal group'' experimental design has 
been used to design experiments that revealed that novel group membership
specified by experimenters can almost immediately override observable
indicators of group membership~\cite{Tajfel1971,Billig1973,Tajfel1982}.
These quick changes in behavior are reflected by equally quick changes in 
brain activity, showing that neural responses to group membership are
extremely plastic~\cite{Cikara2014,Cikara2017}. This is both a problem and
an opportunity---it is a problem because people can be quickly hijacked to
see their neighbors as ``other'', but an opportunity because people can be
equally quickly converted to more prosocial behaviors, such as mitigating
climate change, if they feel they are part of a group.

People often are more strongly attracted to others' opinions the more similar
they are. That is, we tend to adopt our friends' opinions more readily than
strangers' opinions because we know we agree with our friends on several other
issues or topics. Conversely, individuals are often
more repulsed by opposing views the more different other views are. For 
example, if someone we dislike buys a car, we will maybe be less likely to buy the
car brand in the future. This is
known as \emph{biased assimilation}~\cite{Lord1979}.
In politics, individuals have been observed to be more influenced by presidential
candidates in a debate who are perceived as similar to themselves.
On large scales, it has been observed that food, hobby, and other
preferences are becoming increasingly correlated with political ideologies such
as conservativism and liberalism~\cite{DellaPosta2015}.
\citeA{Suhay2018} found that emotions may be critical: they found that
anger especially, along with other emotional states, ``fuel(ed) biased reactions
to issue arguments'' in an online behavioral study. 

The final cognitive factor in social influence I consider is that those with
more extreme opinions tend to be more stubborn, i.e.\ less susceptible to
social influence, than centrists. Evidence from several fields supports
this assumption. On the one hand, longitudinal survey studies
have found that a large portion of the population are centrists demonstrating
low ``opinion stability over time''~\cite{Converse1964,Zaller1992,Kinder2017}. 
Extremists in the United States and United Kingdom were observed to be 
more cognitively inflexible than their centrist counterparts~\cite{Zmigrod2019a}.
Centrist opinions tend to be more susceptible to framing effects~\cite{Chong2007}
and to question ordering~\cite{Zaller1992}. Extremism has also been electrophysiologically linked
to differences in responses to stimuli. In an EEG study, \citeA{Reiss2019} found that ERP
responses to anomolies in experimental stimuli were muted among participants
with more extreme socio-political opinions compared to centrist participants.

Other approaches to studying the cognitive factors in extremism and polarization
include analyzing the correlation between personality traits and ideological
alignment~\cite{Rollwage2019}, and considering cognitive factors of social influence 
in the context of exchanging information, instead of influencing opinions~\cite{Carley1990,Carley1991,Bala1998}.
In the UK, for instance, \citeA{Zmigrod2018} found that dependence on routines was
positively correlated with subscribing to conservativism, nationalisim, and
authoritarianism, which in turn were positively correlated with support for
Brexit from the European Union. My work complements these approaches in that
it considers simpler cognitive factors than personality traits, which I see
as a composite of opinions and beliefs. Due to the complexity of the
personality trait construct, it is difficult to tell whether personality traits
and their relationship to national-scale ideologies and policies
are biologically or culturally determined~\cite{Claidiere2012c,Smaldino2019d,Falandays2021},
and so possibly subject to change along with cultural context.
Some approaches to social influence of knowledge assume mechanisms for generating and
sharing knowledge, possibly under an assumption of biased 
assimilation~\cite{Mark1998,Mark1998a,Mark2003}. In the social epistemology approach, 
knowledge that one behavior is better than another is influenced over two
channels: one channel is the observation of stochastic payoffs 
received from taking one action or another; the other influence channel is 
social, where individuals influence one anothers' beliefs about which behavior is
more beneficial~\cite{Zollman2007,OConnor2019e}.


\subsection{Social factors contributing to polarization}

If we want to understand how opinions change under social influence in the
media and societal system outlined above, we must also consider
social relationships. Social polarization and its opposite, consensus, strongly
depend on who interacts with whom, and when~\cite{Flache2008,Turner2018}. 
But what determines these social relationships and how do these relationships
change over time? How do we model these relationships 
scientifically? 

Who we interact with is somewhat
random and out of our control: it depends on our family membership, geographic location, 
participation in social activities (e.g.\ attending school, getting
groceries, going to a restaurant), etc. In addition to these random factors, 
we also adjust our social relationships based on interpersonal affinity and similarity, 
i.e., we tend to prefer to interact with people we like and avoid people we dislike.
Thinking of these
evolving relationships as a social network, modeled by a mathematical graph,
enables us to formally represent social relationships and harness graph
theory to calculate and predict social facts and behavior. For example, graph
theory can help us predict how quickly information~\cite{Milgram1967,Travers1969}, 
disease~\cite{Salathe2010,Block2020}, violence~\cite{Epstein2002}, and
innovations~\cite{Deroiain2002,Acemoglu2011a,Kreindler2014}
spread in groups and in society~\cite{Milgram1967,Travers1969,Watts1999,Palla2007,Backstrom2012,Wohlgemuth2014}.

In social systems people tend to choose social interaction partners who 
are similar to themselves, a tendency known as \emph{homophily}. As homophily
increases among a population, this
increases the chance that individuals interact with similar others, 
and decreases the chance that individuals will act with dissimilar
others~\cite{McPherson2001}. Homophily, then, amplifies the cognitive factor of biased assimilation,
since increased homophily tends to further insulate individuals from
exposure to opposing viewpoints as biased assimilation causes individuals to 
ignore or reflexively dislike opposing viewpoints and uncritically incorporate
information that supports their pre-existing opinions~\cite{Mark2003,Dandekar2013}.
Another social factor that affects social outcomes are power structures
in which some people have a greater social influence than others.
This may be represented as having a greater number of relationships
with others, so that their opinions are more widely shared~\cite{French1956,Friedkin1986},
or due to social status, or both~\cite{DeGroot1974}.

To analyze social influence and social relationships, we can use a social 
network to represent the relationships between individuals
in groups or societies. 
Social networks are based on mathematical graph theory. Study 4 uses a network-theoretic
model of social influence to understand how social networks contribute to
the emergence of extremism and polarization---the model is formally 
introduced there. Individuals in a social network are represented by \emph{nodes},
often drawn as dots or some other marker. Nodes can encapsulate an individual's
identity in addition to traits such as group
membership, accumulated resources (i.e. ``payoffs''), etc. Sometimes these traits are visualized
by changing the marker size, color, or shape of the node in network visualizations.  Relationships
between individuals are represented by \emph{edges}, drawn as lines that connect
individual nodes.  Homophily and power differentials between individuals 
may also be represented in terms of edge \emph{weights} on the graph. Any graph
may have weighted edges which could stand for many different things; in
navigation applications edge weight might represent the time it takes to reach 
one location from another. \emph{Temporal} or \emph{dynamic} graphs are graphs that change 
over time, which in social situations could result from changing affinities,
geographic relocation, or new communication technologies~\cite{Li2017}. 
In social influence
social networks change when, for example, social ties are abandoned when interpersonal
similarity drops below some threshold~\cite{Axelrod1997,Hegselmann2002,Centola2007,Kossinets2009}.

% Whabout these? \cite{Watts1999,Macy2003,Baldassarri2007, Flache2011,DellaPosta2015,Turner2018,Stewart2020b}

\section{Mechanistic models of emergent social phenomena}

This dissertation relies on theory driven, empirically motivated mechanistic
models to simplify the complex system of human social influence. But how do
mechanistic models work, specifically for rigorous scientific investigation of
emergent phenomena such as extremism and polarization?
A hallmark of the scientific explanation of some phenomenon is that the explanation
only posits the existence of theoretical entities, entity capacities, 
and relationships between entities~\cite{Kauffman1970,Cartwright1989,Craver2006,Turner2021}. 
If the phenomenon of interest 
emerges from system dynamics specified by the entities and their capacities,
then the model and its theoretical basis have some explanatory power. 
A phenomenon \emph{emerges} when a statistical pattern is detected that is 
associated with that phenomenon, e.g., polarization is often measured as the
bi-modality of individual opinions (i.e.\ attitudes, beliefs, etc.) in a society.
The patterns of interest in this dissertation are static (polarization at a given
point in time) and dynamic~\cite{Kelso1995}, e.g.\ rising extremism and collective changes in 
violence metaphor use on cable news.
It is not valid to assume in advance the existence of the phenomenon. 

In this dissertation I use a mechanistic model-based theoretical
approach designed to (Study 1) explain observed patterns in mass media metaphorical violence use;
(Study 2) explain rising extremism in socially isolated groups; (Study 3) 
demonstrate that many or perhaps most detections of
rising extremism in socially isolated groups are false detections; and 
(Study 4) to identify critical determinants social polarization.


\subsection{Emergent social phenomena}

In this dissertation I focus on the emergence of rising extremism and polarization,
which is theoretically influenced by the emergent dynamics of metaphorical violence
use on cable news (one of many influential mass media communication strategies). 
Emergent social phenomena are identified by finding patterns in
the distributions of individual-level behaviors, opinions, traits, etc., 
among a population~\cite{Blau1974,Schelling2006}. It is challenging to
explain, with scientific rigor, how emergent social phenomena such as
rising extremism and political polarization actually 
emerge from repeated instances of social influence~\cite{Watts2011}. Social systems are
complex systems of groups of various sizes, and individual humans themselves
are complex emergent phenomena~\cite{Kello2007,Lazer2009}. 

In this work we have assumed that individuals ``have'' opinions. 
Polarization is calculated 
as the distributional variance (or similar measures of bimodality) 
of individual opinions~\cite{Bramson2016}. A totally polarized society has
exactly half of the population holding one of two extreme opinions, and the
other half holding the opposing view. 
Other behaviors that lead to different emergent social phenomena 
include choosing where to live based on racial preferences (not racial animosity), 
which can result in emergent racial segregation~\cite{Schelling1971}; publishing journal 
articles of differing validity which leads to systemic scientific problems~\cite{Smaldino2019};
or writing statements and documents online that together form a system of 
cultural frames including harmful ethnic, gender, and racial biases and 
stereotypes~\cite{Caliskan2017,Garg2018}. 

There are also emergent phenomena
at smaller social scales, such as dyads and other small groups~\cite{Abney2014a}. 
For example, dyads were found to synchronize with one another when working
together on collaborative tasks, and ``asynchronize'' when in an adversarial 
relationship~\cite{Abney2014,Ramirez-Aristizabal2018,Schloesser2019,Schneider2020,Abney2021}.
In turn, individual humans are emergent properties of a complex electrochemical
interaction of individual, differentiated cells~\cite{Schrodinger2012,Kello2007,Lazer2009}.
It is for this reason that I believe it is best to avoid thinking about
``micro'' and ``macro'' scales as seems to be popular among 
sociologists~\cite{Macy2002}. I conceptualize the assumptions
we must make about individual cognition and social interaction between dyads
as ``individual-level'' assumptions instead of ``micro-motives''~\cite{Schelling2006}.
Similarly I prefer the concept of an 
\emph{emergent phenomenon} to Schelling's concept of ``macrobehavior''.


\subsubsection{Collective violence metaphor usage on cable TV news}

The first emergent phenomenon I study is the frequency of violence metaphor
usage across cable TV news outlets. I hypothesize and show in Study 1 that
violence metaphor use varies depending on how soon there will
be or how recently there has been a presidential debate or the presidential 
election. This approach
complements similar approaches to studying time series of semantic content in
mass media and social media in order to understand how cognitive, cultural, and
communicative frames covary with historical 
events~\cite{Nunn2012,Klingenstein2014,Hamilton2016c,Caliskan2017,Barron2018,Garg2018}. 
Partisan polarization can be identified by analyzing semantic differences in partisan
communications~\cite{Gentzkow2019}.

From the complex systems perspective, ``pragmatic choice'', i.e.\ what words
to use when, is the result of many ongoing subprocesses, which occur within
different contexts~\cite{Gibbs2012a}. The collective attention of society becomes entrained on
shared cultural events~\cite{Fusaroli2015}. The utterances of news anchors, commentators, and 
pundits cannot be separated from their pragmatic purpose and societal context~\cite{Kovecses2010}.
Collective violence metaphor use, then, can be considered an emergent property of 
social systems since it depends on complex interactions between individuals
at varying time and population scales. 


\subsubsection{``Group polarization'': rising extremism in small, socially isolated groups}

\emph{Group polarization} is the name given by social psychologists to the observation
that novel, socailly isolated, collectively biased groups become more extreme
in their opinions after deliberating on some 
topic~\cite{Brown1986,Isenberg1986,Brown2000,Sunstein2002}.
In group polarization, the emergent phenomenon is the rising extremism among the group.
Furthermore, there is a higher-level emergent pattern of the magnitude of the group polarization
effect---extremsim has been observed to rise more when the group is already
relatively extreme~\cite{Myers1982}. Because of the complex interplay between individual-level
cognition and social power dynamics~\cite{Friedkin1999a}, we can identify group
polarization as an emergent social phenomenon as well. 

\subsubsection{Polarization}

Polarization may be arrived at through a varity of cognitive and social mechanisms, 
though of course communication details can exacerbate polarization as already discussed.
Polarization can increase through repulsive 
influence~\cite{Baldassarri2007,Flache2011,Bail2018,Turner2018}. When dissimilar
individuals repulsively influence one another, their opinions become more
extreme in opposite directions, marginally increasing opinion distribution
bimodality. Polarization can also increase through attractive influence only~\cite{Mas2013,Turner2020},
e.g., through group polarization.
In group polarization, isolated groups become more extreme as they find consensus. 
If one group becomes more extreme, polarization also increases
since bimodality will have increased. 

\subsection{Model-based theoretical approach}

The Studies I present in this dissertation all develop
mechanistic models of social influence and mass communications. Studies 1 and
3 implement mechanistic models as generative, computational statistical models, while
Studies 2 and 4 implement their social influence models as agent-based models
of social influence incorporating the cognitive and social factors listed above,
theorized to be important in the emergence of rising extremism and polarization.
The dissertation studies are organized based on their approach
and findings in the study of how rising extremism and polarization emerge
under mass communications and social influence, not their modeling approach.

Scientific models are simplified versions of reality used to identify which components of
complex systems are most important in the emergence of collective larger-scale
phenomena~\cite{Kauffman1970,Wimsatt1972,Wimsatt1997,Machamer2000,Wimsatt2007,Smaldino2017}.
The most explanatory models are \emph{mechanistic models}, ones that explicitly identify the atomic theoretical 
entities in a system and how those entities influence one another~\cite{Machamer2000,Craver2006,Turner2021}. 
In our case mechanistic models of societal and group systems would explain that
human individuals communicate with and influence those with whom they share
social connections, with social connections represented as social network neighbors. 
Above I listed
assumptions about how individuals process social influence and how social
interactions are structured, which are further details incorporated in the
model of social influence used in Study 2 and Study 4. 
Mechanistic models are stronger still when they are 
formalized into mathematical notation and implemented computationally to
make quantitative predictions of how different social phenomena emerge based
on model assumptions. 


\subsubsection{Models in the dissertation studies}

All four studies presented here use some form of mechanistic modeling to 
represent system dynamics that give rise to emergent phenomena. Mechanistic
models may be expressed and implemented in a variety ways. In addition to 
developing detailed verbal models of how social influence and mass
communication work, Studies 1 and 3 implement statistical models and 
fitting procedures to empirically
determine inflection points in violence metaphor dynamics (Study 1) and 
to demonstrate that a high rate of experimental detections of rising extremism
are plausibly false (Study 3). Studies 2 and 4 use 
agent-based models to understand which cognitive and social factors best explain
and predict rising extremism and polarization, respectively.

Study 1 and Study 3 both use statistical models---in Study 1 the model is fit
to observations, and in Study 3 the model generates simulated counterfactual
data. In Study 1, to partially explain the dynamics of metaphorical violence use on cable TV news,
I developed a dynamical model expressed as a statistical regression
model where each news channel is in either a normal state or a 
transient excited or depressed state.  
In Study 3, I used a generative 
statistical model to simulate experimental group polarization data where
pre- and post-deliberation opinions are drawn from distributions with the
same mean. By simulating the measurement of these opinions, I show
that floor/ceiling effects lead to a false detection of an opinion shift due to
the process of consensus that reduces group opinion variance from
pre- to post-deliberation.

Study 2 and Study 4 model different systems using the same underlying
agent-based social influence model that incorporates the cognitive and social 
factors outlined above. Agent-based models of social systems start 
by defining a computational representation of a person, called
an \emph{agent}. To implement these models, 
I wrote computer code that created a world in which computational agents
were brought to life, made to interact with other agents according to rules and
assumptions based on the cognitive and social factors outlined above. 
After thousands or millions of rounds of simulated social interaction I
measured the distribution of opinions to calculate either a rise in extremism or
increased polarization.
     

\section{Overview}

Now we have reviewed the overarching problems of extremism polarization that
motivated this work, the
theoretical foundations I draw on to address specific subproblems, and
the analytical approach I take to studying different emergent social
phenomena. I will now give an overview of the four studies presented in this
dissertation.

First, Study 1  
calculates the influence of political events on metaphorical violence use across
three cable TV news channels in 2012 and 2016. I found that significant changes
in metaphorical violence usage occur across cable news outlets MSNBC, CNN, and
Fox News in both 2012 and 2016 around the time of the presidential debates. 
In 2012, changes were significant, but rather small and there was no discernable
pattern across news outlets. In 2016, however, metaphor use increased dramatically
across networks in the run-up to the 2016 elections.  
I hypothesized that Twitter usage was an important driver of this difference due to 
a close reading of cable news violence metaphors that seemed to cast candidate
tweets as metaphorical violence. This was indeed the case, as revealed through a
correlational analysis between Twitter activity by Republican
and Democratic candidates and violence metaphor usage on cable news in both
election cycles. Correlations
in 2016 were more significant overall than in 2012. Candidate Twitter use
accounted for over 30\% of variance in overall violence metaphor use in 2016, compared to
about 8\% of variance in 2012. By understanding the
dynamics of violence metaphors on cable news, we can better understand and 
predict downstream effects of violence metaphor use, including reduced capacity
for rational thinking due to emotional overwhelm~\cite{Suhay2018} and heightened
risks of political violence~\cite{Kalmoe2014,Kalmoe2018}.

Study 2 focuses down from
mass communications to group-level social influence and shifts in extremism.
In Study 2 I use agent-based modeling to show that group polarization 
may be driven by the cognitive factor of ``stubborn extremism'', where
extremists are less susceptible to social influence~\cite{Reiss2019,Zmigrod2019a}.
To detect group polarization, researchers survey participant
opinions, assemble participant groups that are biased in one direction or another, and then
re-survey participants. If the mean of the group opinions changes, then
a group polarization opinion shift is said to occur. Existing explanations of
group polarization include potentially
problematic axuiliary assumptions that seem to lack robust empirical support~\cite{Meehl1990}.
Existing explanations also fail to explicitly account for the observation that
the magnitude of opinion shifts are positively correlated with initial group
extremism~\cite{Myers1982}. However, in Study 2 I show that the stubborn extremism explanation
also predicts this correlation between initial extremity and opinion shift.
The stubborn extremism explanation seems more parsimonious in that
it makes a single, simple, empirically valid assumption that is more explanatory
than any alternatives.

In the course of Study 2 I found that many behavioral studies of
group polarization used a problematic method for measuring group polarization.
Specifically, many group polarization studies used metric statistical models
on ordinal data, which is known to sometimes yield false inferences~\cite{Liddell2018}.
Study 3 uses a generative statistical model to show that, indeed, over 90\% of 
published detections of group polarization over ten journal articles
are plausibly false detections. 
False detections occur due to the fact that extreme opinions are mapped to
relatively moderate opinions, but this is not accounted for when using 
metric statistical models designed to detect whether two distributions of
\emph{continuous} variables are different (e.g., a \emph{t}-test). Metric
models can be tricked into detecting a difference between two distributions' means when
the data are ordinal measurements of continuous \emph{latent} psychological opinions
from two distributions with the same mean but different variances. We know that
opinion variance changes within groups as groups discuss a topic: group members
find consensus, i.e., variance decreases.

Finally, in Study 4, I explore how well the
model from Study 2 could be used to explain and predict society-level
polarization, as opposed to increased extremism in small groups in Study 2.
I found that (1) greater initial polarization often led to greater long-term polarization; 
(2) realistic small-world networks tend to result in higher levels of
polarization than common alternative configurations; (3) more miscommunication
leads to greater polarization; and (4) polarization outcomes are
highly stochastic, i.e., the same initial configurations and
parameter settings can result in a range of predictions from low to high
levels of polarization solely due to the path dependence of interpersonal
interactions over many time steps.


