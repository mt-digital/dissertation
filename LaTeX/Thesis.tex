% UC Merced  PhD Dissertation Template
% (UCSD Mathematics Dissertation Template is modified according to the UC Merced guidelines by Lasith Adhikari. Not responsible for any issues regarding modifications. You are welcome to add/update if there is any missing requirernment)
%
% Please read the comments in this file and make appropriate edits.
% NOTE: Always refer to the ``Preperation and Submission Manual for 
% Doctoral Dissertations and Masters Theses for 20**'', where 20** is 
% the year of your graduation, for officiation preparations guidelines.
%
% If you desire more control, please see the attached files:
%   * ucsd.cls -- Class file
%   * uct10.clo, uct11.clo, uct12.clo -- Configuration files for font sizes 10pt,11pt,12pt
%
% CHANGELOG:
%   * Original file adapted from brockman.tex by JRB and RMR
%     to work with ucsd.cls


\documentclass[12pt,chapterheads]{UCMerced}
% documentclass options: default is 11pt, oneside, final.
% fonts: 10pt, 11pt, 12pt -- are valid for UCSD dissertations (now UC Merced).
% sides: oneside, twoside -- note that two-sided theses are not accepted by OGS
% mode: draft, final -- draft mode switches to single spacing, removes hyperlinks,
%                       and places a black box at every overfull hbox (check these before submission).
% chapterheads -- include this if you want your chapters to read:
% Chapter 1
% Title of Chapter
%
% instead of
%
% 1 Title of Chapter


% Include all packages you need here.  Some standard options are suggested below.

% GEOMETRY - This will force the use of Letter paper.
% Many TeX installations default to A4 paper.  The formatting
% of the thesis class file requires Letter, else the margins
% will be wrong when you go to print it (and OGS will complain).
% If your TeX implementation is not setup for Letter paper, and
% you cannot change it, uncommenting the following line may fix 
% problem.
% \usepackage[paper=letterpaper]{geometry}


%% AMS PACKAGES - Chances are you will want some or all of these if writing a math dissertation.
% \usepackage{amsmath, amscd, amssymb, amsthm}

%% GRAPHICX - This is the standard package for including graphics for latex/pdflatex.
\usepackage{graphicx}

%% LATIN MODERN FONTS (replacements for Computer Modern)
\usepackage{lmodern}
\usepackage[T1]{fontenc}

%% INDEX
% Uncomment the following two lines to create an index: 
% \usepackage{makeidx}
% \makeindex
% You will need to uncomment the \printindex line near the
% bibliography to display the index.  Use the command
% \index{keyword} within the text to create an entry in the index
% for keyword.

%% HYPERLINKS
% To create a PDF with hyperlinks, you need to include the hyperref package.
% THIS HAS TO BE THE LAST PACKAGE INCLUDED!
% Note that the options plainpages=false and pdfpagelabels exist
% to fix indexing associated with having both (ii) and (2) as pages.
% Also, all links must be black according to OGS.
% See: http://www.tex.ac.uk/cgi-bin/texfaq2html?label=hyperdupdest
% Note: This may not work correctly with all DVI viewers (i.e. Yap breaks).
\usepackage[colorlinks=true, pdfstartview=FitV, linkcolor=black, citecolor=black, urlcolor=black,plainpages=false,pdfpagelabels]{hyperref}
\hypersetup{ pdfauthor = {Your Name Here}, pdftitle = {The Title of The Dissertation}, pdfkeywords = {Keywords for Searching}, pdfcreator = {pdfLaTeX with hyperref package}, pdfproducer = {pdfLaTeX}}

\begin{document}

%% REQUIRED FIELDS -- Replace with the values appropriate to you
\title{Social, cognitive, and communicative drivers of extremism and polarization}
% No symbols, formulas, superscripts, or Greek letters are allowed
% in your title.

\author{Matthew A. Turner}
\degreeyear{2021}
\degree{Doctor of Philosophy} 
% Master's Degree theses will NOT be formatted properly with this
% file.

\field{Cognitive and Information Sciences}

\numberofmembers{4} % |chair|  + |othermembers| (do not count co-chair) % change here

\chair{Paul E. Smaldino}
% Uncomment the next line iff you have a Co-Chair
%\cochair{Professor Cochair Semimaster} 

\memberone{Teenie Matlock}
\membertwo{Christopher T. Kello}
\memberthree{Jeffrey Yoshimi}


\begin{frontmatter}
\makefrontmatter % The title, copyright, and signature pages.

%% DEDICATION
% You have three choices here:
%   1. Use the ``dedication'' environment.   Put in the text you want,
%   and you'll get a perfectly respectable dedication page.
%
%   2. Use the ``mydedication'' environment.  If you don't like the
%   formatting of option 1, use this environment and format things
%   however you wish.
%
%   3. If you don't want a dedication, it's not required.


\begin{dedication} % The style file will format this for you.
  To Phoebe Ruth Turner. As Daniel Tiger and his parents sing, ``We gotta look a little
  closer to see just how things go.''
\end{dedication}

% \begin{mydedication} % You are responsible for formatting here.
%   \vspace{1in}
%   \begin{flushleft}
% 	To me.
%   \end{flushleft}
%   
%   \vspace{2in}
%   \begin{center}
% 	And you.
%   \end{center}
% 
%   \vspace{2in}
%   \begin{flushright}
% 	Which equals us.
%   \end{flushright}
% \end{mydedication}


%% EPIGRAPH
%  The same choices that applied to the dedication apply here.

\begin{epigraph} % The style file will position the text for you.
  \emph{A careful quotation\\
  conveys brilliance.}\\
  ---Smarty Pants
\end{epigraph}

% \begin{myepigraph} % You position the text yourself.
%   \vfil
%   \begin{center}
%     {\bf Think! It ain't illegal yet.}
% 
% 	\emph{---George Clinton}
%   \end{center}
% \end{myepigraph}

\tableofcontents
\listoffigures  % Uncomment if you have any figures
\listoftables   % Uncomment if you have any tables


%% ACKNOWLEDGEMENTS
%  While technically optional, you probably have someone to thank.
%  Also, a paragraph acknowledging all coauthors and publishers (if
%  you have any) is required in the acknowledgements page and as the
%  last paragraph of text at the end of each respective chapter. See
%  the OGS Formatting Manual for more information.

\begin{acknowledgements} 
  \begin{enumerate}
    \item 
      Paul Smaldino for pushing me to improve with more patience than could
      be reasonably expected. 
    \item
      My parents for providing me with many rich experiences through childhood
      and for their help and encouragement while completing this doctorate.
    \item
      The CIS community that has been such a stimulating, eye-opening experience
      that enabled me to learn about and join in one of the most exciting
      and important scientific fields today---cognitive science---uh whatever
      that is exactly.
    \item
      I want to thank the UC Merced undergraduate students who I've gotten to 
      interact with, practice teaching with, and learn from in my five years
      at the University. This is a truly special university. I am excited to
      watch it grow into the future.
  \end{enumerate}
\end{acknowledgements}


%% VITA
%  A brief vita is required in a doctoral thesis. See the OGS
%  Formatting Manual for more information.
\begin{vitapage}
\begin{vita}
\item[2008] B.~S. in Mathematics \& Physics, Syracuse University, Syracuse, New York
\item[2012] M.~S. with Thesis in Applied Physics, Rice University, Houston, Texas
\item[2012-2014] Data Engineer, Economic Modeling Specialists, Int'l, Moscow, Idaho
\item[2014-2016] Research Software Developer, Northwest Knowledge Network, University of Idaho, Moscow, Idaho
  \item[2016-2020] Graduate Teaching Assistant, University of California, Merced
  \item[2020-2021] Graduate Student Researcher,  University of California, Merced
  \item[2021] Ph.~D. in Cognitive and Information Sciences, University of California, Merced
\end{vita}
\begin{publications}
  \item 
    Turner, M.A., and Smaldino, P.E. (2021). Is group polarization Real? \emph{Under review.}
  \item
    Smaldino, P.E., and Turner, M.A. (2021). 
    Covert signaling is an adaptive communication strategy in diverse populations.
    \emph{Psychological Review} (Forthcoming; preprint: https://osf.io/preprints/socarxiv/j9wyn/).
  \item 
    Turner, M.A., and Smaldino, P.E. (2020).  Stubborn extremism as a potential pathway to group polarization. 
    In \emph{Proceedings of the 42nd Annual Conference of the Cognitive Science Society}. Online.
  \item
    Smaldino, P.E., Turner, M.A., and Contreras Kallens, P. (2019). Open science and modified funding
    lotteries can impede the natural selection of bad science. \emph{Royal Society Open Science}.
  \item 
    Turner, M.A., and Smaldino, P.E. (2018).  Paths to polarization: how extreme views,
    miscommunication, and random chance drive opinion dynamics.  \emph{Complexity}.
  \item 
    Turner, M.A., Maglio, P.P., and Matlock, T. (2018).  Stubborn extremism as a potential pathway to group polarization. 
    \emph{Under revision}. Preprint online: \url{https://osf.io/preprints/socarxiv/t8yg9/}.
\end{publications}
\end{vitapage}

%% Abstract
% There does not seem to be a maximum length. From the OGS Formatting
% Manual: ``The abstract may continue on to a second page.''

\begin{abstract}

Polarization and extremism are two of the most pressing problems of our time
for social and cognitive scientists, among others. In some sense, the causes
are “obvious”: members of different political parties make increasingly
aggressive statements broadcast to a wider and wider audience, increasing
animosity and decreasing common ground. At the same time, new online echo
chambers make it possible to exclude opposing views from discourse,
hypothetically leaving extremism to increase unchecked. While on the surface
many causes seem obvious, when we drill down, we see many questions remain, a
few of which I address in this dissertation: how predictable is extremism and
polarization? Does social isolation always produce increased extremism? How do
we know, and can we trust existing experimental measurements of this
phenomenon? How does the current communication environment, specifically
violent rhetoric, exacerbate extremism and polarization? Which political
factions in the United States are most affected by violent rhetoric?  

To answer questions about the predictability of polarization and the inevitability of
echo chamber extremism I develop both agent-based and generative statistical
models of social influence and opinion measurement and use them to perform
computational experiments that show (1) polarization and extremism may be
highly unpredictable; and (2) that at this time it is not clear when echo
chambers lead to extremism partially due to incomplete theory and partly due to
inappropriate measurement and statistical procedures in empirical studies on
the subject. To understand how language might be driving different sectors of
the public to extremism at different important times, I developed new
cyberinfrastructure and models to collect and analyze cable TV news uses of
metaphorical violence (e.g., “Biden hit back at Trump”). Metaphorical violence
has been shown to increase political extremity in the form of support for
violence to achieve political goals. Worryingly, in 2016, I found that
metaphorical violence significantly increased around the time of the
presidential election, and this increase was most pronounced on Fox News
compared to MSNBC and CNN.  

These efforts provide several new
cyberinfrastructure and modeling tools and approaches for studying the social,
cognitive, and communicative foundations of extremism and polarization.
\end{abstract}
\end{frontmatter}


%% DISSERTATION

% A common strategy here is to include files for each of the chapters. I.e.,
%   \include{chapter1.tex}
%   \include{chapter2.tex}
% etc.  Of course, if you prefer, you can just start with
%   \chapter{My First Chapter Name}
% and start typing away.  
\chapter{Introduction}
This is only a test.
\section{A section}
Lorem ipsum dolor sit amet, consectetuer adipiscing elit. Nulla odio
sem, bibendum ut, aliquam ac, facilisis id, tellus. Nam posuere pede
sit amet ipsum. Etiam dolor. In sodales eros quis pede.  Quisque sed
nulla et ligula vulputate lacinia. In venenatis, ligula id semper
feugiat, ligula odio adipiscing libero, eget mollis nunc erat id orci.
Nullam ante dolor, rutrum eget, vestibulum euismod, pulvinar at, nibh.
In sapien. Quisque ut arcu. Suspendisse potenti. Cras consequat cursus
nulla \cite{Goodman}.
\subsection{More Stuff}
Blah

\begin{figure}[h] 
  \begin{center}*\end{center}
    \caption{A figure of Vonnegut.\index{Vonnegut}} 
\end{figure}

\chapter{Title}
This is only a test.
\section{A section}

\chapter{Title}
This is only a test.
\section{A section}

\appendix
\chapter{Final notes}
  Remove me in case of abdominal pain.

%% END MATTER
% \printindex %% Uncomment to display the index
% \nocite{}  %% Put any references that you want to include in the bib 
%               but haven't cited in the braces.
\bibliographystyle{plain}  %% This is just my personal favorite style. 
%                              There are many others.
\bibliography{mybib}  %% This looks for the bibliography in myrefs.bib 
%                          which should be formatted as a bibtex file.
\end{document}

